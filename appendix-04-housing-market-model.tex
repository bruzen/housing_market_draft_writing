\chapter{Salvaging BID PRICES BY AGENT TYPE?}

{\color{blue}
\section{Odd things to save}
unless there are limits on the size of capital flows. For our simulation, we implement such limits. 


%  GET?? Case, Karl E., and Robert J. Shiller. 1988. ``The Behavior of Home Buyers in Boom and Postboom Markets.'' New England Economic Review (November-- December), pp. 29--46.



 % 
%  The agent purchases a house for a price, $P_0$ %a down payment, $D$, 
%  and receives the increased price $P_T = (1 + \dot P)P_0$, back after a period $T$. 
% % The value of the investment is the net present value of buying and then selling after one period:
% The value of the investment is the capital gain, $\mathcal{C}$ plus the rents, net of operating costs and taxes, $\mathcal{R}_N$.

% minus the mortgage, repaid with interest, plus rents, minus any operating costs and taxes,\footnote{We have applied this model to explore the effect of a vacancy tax in Beirut.  For that analysis we  added a use-value, $U$ in place of rent for expatriate owners to represent using the property - say one month a year - when they are not renting the property and a \textbf{vacancy tax}, $T$ at rate $t$ to affect the speculator's  decision.} %\cite{Al-Shihabi}

% OLD equations from working out the above. Could change symbols
% \begin{eqnarray}
% V  	&=& capital\ gain - Interest\ due  	+ Rent  - operating\ cost -taxes \\
% 	&=& \delta P_T-D \qquad \qquad \quad - (1+\delta r)M \quad	 + R  	-C\\
% 	&=& \delta P _T \qquad-(P_0-M) \quad- (1+\delta r)M 	 + R  	-C\\
% 	&=& \delta (1+\dot P)  P_0 -(P_O -M)  -(1+\delta r)mP_0  + R  -C\\
% 	&=& \delta (1+\dot P)  P_0 -P_O + M \qquad -(1+\delta r)mP_0  + R -C\\
% 	&=&( \delta (1+\dot P)-1)  P_0  + mP_0 \quad -(1+ \delta r)mP_0  + (\rho-\kappa)P_0 \\	
% 	&=& \left(  \delta (1+\dot P)-1    + m \quad - m(1+\delta r)  + (\rho-\kappa)\right)P_0 \\'
% 	&=& \left(  \delta (1+\dot P)-1    + m \quad - m-\delta rm  + (\rho-\kappa)\right)P_0
% \end{eqnarray}
 % Where $\mathcal{R}$, $\mathcal{O}$, $\mathcal{T}$, and $M$ are total rent, operating costs, tax payments, and mortgage borrowed, as net present values at the end of the period. The interest rate is $r$, and the discount factor $\delta$. % I don't like this. I think it makes sense for them to be total payments over the period, but I now think it would be more intuitive to compute them as net present values at the start of the period, since that is when the mortgage is borrowed and the decision is made.

 %For ease of calculation, ..To get the return on investment, 
%  This may be written in terms of shares of the purchase price:
% \begin{eqnarray}
% V &=& \delta \left((1+\dot P) P_0 - (1+r)mP_0\right) + \rho P_0 - \theta P_0 - \tau P_0 \nonumber \\
%   &=& \left(\delta \left(1+\dot P - (1+r)m   \right) + \rho     - \theta     - \tau\right) P_0.
% \label{eqn-property-investment-value2}
% \end{eqnarray}
%  Where $\rho$, $\theta$, $\tau$, and $m$ are rent, operating costs, taxes, and mortgage shares, respectively. % Where $\phi$ is a fraction that takes into account taxes and operating costs. 
 % of price for rents, operating costs, and taxes. %The discount factor is $delta$, $P_0$ is the property price at the time of sale. %$r$ is the interest rate paid, and $m$ is the share of the pr
Agents borrow a share of the purchase price, $P$. The amount borrowed is the mortgage, $M$. This is a share of the total purchase price $mP = M$. The \gls{mortgage term}, $T$, is the period it takes to pay down the mortgage.


%which we define as the ``bid price,''% $P_B$ $P^{max}_{bid}$. 
%We return to Equation~\ref{eqn-property-investment-value3} %
%We start by writing the value of the investment, from Equation~\ref{eqn-property-investment-value1}, with the net rent. %, which can be written a share, $\phi$ of total rent $\mathcal{R}_N = \phi \mathcal{R}$, %The net rent can also be written as a share, $\phi$, of the rent 
%: %(originally from Chapter~\ref{chapter-financialization} on Financialization. Since the rent is a known quantity at the beginning of the term, independent of the current, so we reformulate net rent,  $\mathcal{R}$, $\mathcal{O}$, $\mathcal{T}$ as a fraction of total rent: 

% \begin{eqnarray}
% V &=& \delta \left(P_T - (1+r)M\right) +      \mathcal{R}_N   \nonumber \\
%   % &=& \delta \left(P_T - (1+r)M\right) + \mathcal{R}_N   \nonumber \\
%   % &=& \delta \left(P_T - (1+r)m P_0\right) + \mathcal{R}_N \nonumber \\
%   &=& \delta \left((1+\dot P)P_0 - (1+r)m P_0\right)     + \mathcal{R}_N \nonumber \\
%   &=& \delta \left((1+\dot P)    - (1+r)m    \right) P_0 + \mathcal{R}_N
% \label{eqn-property-investment-value3}
% \end{eqnarray}


% Dividing by the down payment, $D = (1-m)P_0$:
% \begin{eqnarray}
% r_{return} = \frac{\delta \left(1 + \dot P - (1+r)m\right)}{1-m} + \frac{\mathcal{R}_N}{(1-m)P_0}
% \end{eqnarray}

% +++++++++++++++ JUST REPLACED BETWEEN THESE WITH THE ABOVE. CHECK IT'S OKAY
% % We also have to assume that $P_B = P_M$, which we can justify as an equilibrium condition - investors believe they are paying the market price. The result is 
% % \[V= \delta(P^T- (1+r)M)   +\phi r P_B\]

% % The price at the end of the term $T$, $P_T$, is a predicted value for the investor \textit{ex-ante}, so we can replace $P^T$ in Equation~\ref{eqn-property-investment-value3} with the predictor, $(1+\dot P)P_B$. 
% % We can replace $P^T$ in Equation~\ref{eqn-property-investment-value3} with $(1+\dot P)P_0$.
% \[V= \delta \left((1+\dot P)P_0 - (1+r)m P_0\right) +\phi \mathcal{R}\] 
% Then we replace $\dot P$ with an estimate, $L(P)$, representing an estimated function of the lagged values of $P$ and any  other relevant data. We imagine the potential investor informed estimates by information from  real estate agents or analysts.  The result is 
% \[V= \delta \left((1+L(P))P_B- (1+r)mP_B\right) +\phi \mathcal{R}\]
% Combining terms:
% \[V= \delta \left((1+L(P))- (1+r)m \right) P_B +\phi \mathcal{R}\]

% WHY BIG BRACKETS NOT SHOWING UP

% The rate of return $r_{return}=V/D=V/(1-m)P_B$ is then

% \[r_{return}= \frac{\delta \left((1+L(P) - (1+r)m\right)}{1-m} + \frac{\phi \mathcal{R}}{(1-m)P_B}\]

% %++++++++++++++++

% \begin{align}  
%   v  & =  \frac{\delta ((1+\dot P)  - (1+r)m)\  + \psi r}{1-m} \\
%   & =  \frac{\delta ((1+\dot P)  - (1+r)m)\  + \psi r}{1-m} \label{eqn-rule}
% \end{align}

% \begin{eqnarray}
% v %&=& \delta(P^T- (1+r)M) \qquad \qquad \qquad 	 + \mathcal{R}_N \nonumber\\
% % &=&\delta\left( (1+\dot P)P_B - (1+r)mP_B \right)  + \mathcal{R}_N  \nonumber\\
%   &=&\delta\left( (1+L(p)) - (1+r)m \right) P_B + \mathcal{R}_N  \nonumber
% \end{eqnarray}

% \subsection{Maximum bid} % given the rate of return

% Agents bid if the RHS is larger than the target rate of return, as stated in Equation~\ref{eqn-property-investment-return2}.  $P_0$ with $P_B$, and $\dot P$ with an estimator for $\dot P$, we get: %TO DO REPLACE PDOT - MAYBE WITH TILDE AND DOT..: 

% \begin{eqnarray}
% r_{target} \le \frac{\delta \left((1+ \dot P_M^e - (1+r)m\right)}{1-m} + \frac{\mathcal{R}_N}{(1-m)P_B}
% \end{eqnarray}

% Solving for $P_{bid}$:

% \begin{align}
% r^{target} &\le \frac{\delta \left(1 + \dot P_M^e - (1+r)m\right)}{1-m}   +\frac{\mathcal{R}_N}{(1-m)P_B}. \nonumber \\
% (1-m)r^{target} &\le \ \ \delta \left(1 + \dot P_M^e - (1+r)m\right) + \frac{\mathcal{R}_N}{P_B} \nonumber \\ %\delta(1+L(P))- (1+r)m%
% (1-m)r^{target} - \delta \left(1 + \dot P_M^e - (1+r)m\right)  &\le  \frac{\mathcal{R}_N}{P_B} \nonumber\\
% P_B & \le    \frac{\mathcal{R}_N}{(1-m)r^{target}-\delta \left(1 + \dot P_M^e - (1+r)m\right)} 
% % P_B & \le    \frac{\mathcal{R}_N}{(1-m)r^{target}-\left[ \delta(1+L(P)- (1+r)m\right]} \nonumber 
% \label{eqn-bid-price1}
% % \end{align}


%Each agent has their own interest rates, discount rates, mortgage share, information, and expectations, so individual bids can differ.

% \begin{align}
% P_i^{bid} \le   \frac{\mathcal{R}_N}{(1-m_i)r_i^{target} - \delta_i \left(1 + L(P) - (1+r_i)m_i \right)}.
% \label{eqn-bid-price2}
% \end{align}

% \begin{eqnarray}%. OLD VERSION: WRONG
% %r^{target}&=& \delta\left( (1+L(p)) - (1+r)m \right) P^{max}_{bid} + \mathcal{R}_N  \nonumber\\
%    P_{max_bid} &=&\frac{r^{target} - \mathcal{R}_N}{\delta\left((1+L(p)) - (1+r)m \right)} %\label{eqn-bid-price2} 
% \end{eqnarray}

% \section{Finding bid price}
% We start with Equation~\ref{eqn-b2}. for convenience, replace $\rho -\kappa - \sigma $ with $\mathcal{R}_N$ (net Rent). 

% Replace $P^T$ with $(1+\dot P)P_B$ assuming that the bidder is bidding the equilibrium market price for the period.

% Then replace   $\dot P$ with $L(p)$ representing some (estimated function ($\tilde{\dot P}$)) of the lagged values of $P$ that incorporates other data. 

% \begin{eqnarray}
% v&=& \delta(P^T- (1+r)M) \qquad \qquad \qquad 	 + \mathcal{R}_N \nonumber\\
%  &=&\delta\left( (1+\dot P)P_B - (1+r)mP_B \right)  + \mathcal{R}_N  \nonumber\\
%   &=&\delta\left( (1+L(p)) - (1+r)m \right) P_B + \mathcal{R}_N  \nonumber
% \end{eqnarray}

% So I want to use this relationship to find the maximum bid price for the bank. The rule is, ``Bid if the RHS is larger than the target rate of return, $r^{target}$, and do not bid if it is smaller.''  The maximum bid  is the bid that makes the two sides equal. 

% {\color{red}
% \begin{eqnarray}
% r^{target}&=& \delta\left( (1+L(p)) - (1+r)m \right) P^{max}_{bid} + \mathcal{R}_N  \nonumber\\
%    P^{max}_{bid} &=&\frac{r^{target}-\mathcal{R}_N}{\delta\left( (1+L(p)) - (1+r)m \right)} \label{eqn-bid-price} 
% \end{eqnarray}}
% %(What makes. this work is that I do not use an identity to get $\dot P$, which made the system of equations singular.)
% \newpage




\section{WHERE DOES THIS GO?}


\subsubsection{CUT Maximum bid second house buyer}

CUT People are limited by the limits  that the bank imposes borrowing capacity. Purchasing a revenue house is only attractive if the rate return on a rental house exceeds the cost of capital $r_i$. Call this the \textbf{profitability constraint} The maximum bid for this investor is given by Equation~\ref{eqn-max-investment-bid} using one period values for $r_i^{target},\ m_i,\ \delta_i$, and $r_i$.

CUT \begin{eqnarray}\label{eqn-max-second-bid}
P_{second}^{max\_bid} & = \frac{\mathcal{R}_N}{(1-m_i^{max\_permitted})r_i^{target}-\delta_i \left(1 + \dot P_M^e - (1+r_i)m_i^{max\_permitted}\right)} 
\end{eqnarray}
Notice that this simplifies if any of $r_i^{target},\ \delta_i$, and $r_i$ are set equal. It is possible the cost of capital and the target rate will be the same.

CUT Combining the investment \textbf{profitability constraint} and mortgage constraints


MAKE A FOOTNOTE. Note if it was a second buyer, they would have a couple of constraints. The price can't exceed these values.
\begin{eqnarray}
P_{second}^{max\_bid} & = min \left\{P_{second}^{max\_bid},\ \frac{\mathrm{savings}_i}{1-m_i^{max\_permitted}},\ \mathrm{saving}s_i + M_i^{max\_permitted}  \right\}  \nonumber
\end{eqnarray}

\subsubsection{CUT Reservation prices for investment owners}
If the current owner of an investment  property gets a high offer higher than their own maximum bid, they will always sell.  


\subsection{New buyer maximum bids and reservation prices }
A new buyer chooses to move if the income in the city, including the wage amenities exceed the rural income after housing costs, $\psi+A^{Rural}$. 

There are two cases - moving as a buyer and moving as a renter.

\subsubsection{KEEP Moving as a renter}
The gain for a potential renter is 
\begin{align}
gain^{rent}=&net^{city}-net^{Rural}\nonumber\\
=&\left[\psi+\omega-cd+\mathbb{A}^{city}-\mathcal{R}\right]-\left[\psi+\mathbb{A}^{Rural}-a\psi\right] \nonumber\\
=&\ \omega-cd+\mathbb{A}^{net}-\mathcal{R}
\label{eq-move-to-rent}
\end{align}
If rent $\mathcal{R}$ exactly capture housing cost and locational rent, the gain reduces to $\mathbb{A}^{city}-\mathbb{A}^{Rural}$. Normally we would expect property rents to capture this component  as well. 

\subsubsection{KEEP Moving as a buyer}
The gain for a buyer is 
\begin{align}
gain^{buy}=&\ net^{city}-net^{Rural}\nonumber\\
=&\left[\psi+\omega-cd+\mathbb{A}^{city}+(\dot p-r_im_i)P-a\psi\right]-\left[\psi+\mathbb{A}^{Rural}-a\psi\right] \nonumber\\
=&\ \omega-cd+\mathbb{A}^{net}+(\dot p-r_im_i)P  \label{eq-move-to-buy}
\end{align}
This requires that the sum of  locational rent, net amenity and net capital gain be greater than or equal to zero for the in-migrant to buy.

The two decisions  are 
\begin{enumerate}
    \item Move only if $max\{gain^{buy},\ gain^{rent}\} \ge 0$, otherwise do not move to the city.
    
    \item Buy a house if $(\dot p-r_im_i)P\ge  \mathcal{R}$, otherwise rent.
\end{enumerate}

The buyer is subject to the mortgage constraints in Equation~\ref{eqn-max-mortgage-combined}.

We will assume that a first-home buyer always takes the maximum mortgage available to maximize the financial return on the transaction. 

The maximum price a new buyer can bid is also given the bank imposed liquidity constraint is given by Equation~\ref{eqn-max-mortgage-combined}, which we repeat here


\begin{align}
P_i^{max\ bid}= min \left\{\frac{\mathrm{savings}_i}{1-m_i^{max\_permitted}},\  M_i^{max\_permitted} + \mathrm{saving}s_i  \right\}   \nonumber  
\end{align}


\section{Reservation Price}



\subsection{Reservation price for retirees}
Both owners and tenants retire. Homes only become available if they move to the country. 



\subsubsection{Why would a renter move to the country?} 
If the rent is greater than the cost of housing in the country and the lost amenity
\[\mathcal{R} > \ a\psi+ \mathbb{A}^{net}0\]

\subsubsection{Why would an owner move to the country?}
Why would an owner sell a home and move to the country? It would be economically advantageous if the interest on income from the sale exceeds the cost of housing in the country and the lost amenity.
\[r_i(P^{realized}-M) >\ a\psi+ \mathbb{A}^{net}\label{eq:movers-gainA}\]

This value provides a \textbf{reservation price}: 
\[P^{reservation} =\ \frac{a\psi+ \mathbb{A}^{net}}{r_i}+M \label{eq:movers-gainB}\]
This just says that the lowest prices one would sensibly accept is the cost of a house in the country plus enough to cover the mortgage, plus capitalized net amenity.\footnote{transaction costs on the sale are omitted. Add a parameter or variable.}

Should a retiree rent the house out? only if the rent exceeds the  interest on income from the sale:
\[\mathcal{R}^{net}>r_i(P^{realized}-M)\]
This value provides a \textbf{another reservation price}: 
\[P^{reservation} =\ \frac{\mathcal{R}^{net}}{r_i}+M \label{eq:movers-gainC}\]


Sellers should use the investment rule calculated before  do determine a  reservation price in the bargaining process. If they can't get a higher price they  can keep the home as a rental  investment and use the home as collateral for a mortgage on a house in the country.
% For a homeowner, the reservation price is tricky. 
% At this point we only have retirement as a hard boundary to consider. That means that the householder wants to sell in the current period. The gain from moving to the country is 
% \begin{equation}
% Gain=P_{ij}^{expected}-M-\frac{a\psi}{r\_prime} - transaction\ costs - Urban Amenity\  
% \end{equation}\label{eq:movers-gain}
% In the initial model  there are no urban amenities and no transaction costs. We could subtract an urban amenity term, $U_{ij}$ and an transaction cost from equation~\ref{eq:movers-gain}. The amenity term is likely to have little effect, since a buyer will pay extra for the amenity. the transaction cost is actually large and is likely to significantly affect land allocation efficiency.  

The reservation price in in Equation~\ref {eq:movers-gain} is the minimum provides an incentive to sell a home immediately on retiring from the workforce. 



If the sale is delayed by one period, however, the gain is deferred, so the present discounted cost of waiting is, $Gain*\frac{\delta}{1+\delta}$.\footnote{This is simply (1 minus the discount factor) times the deferred amount.} A seller would accept a lower price to avoid this cost of waiting. 

My suggestion is that $\frac{\delta}{1+\delta}P_{ij}^{expected}$ is the \textbf{initial reservation price} for a homeowner on retirement. If the seller does not get this she reduces the reservation price for the next period by (say) 5\% .


\subsubsection{for the Investor-owner}
\textbf{For financial sector in general including those holding investment properties} the problem is easier if anyone offers more than your maximum bid, you should sell the property. 
\begin{eqnarray}
P_{bank}^{reservation} & =    \frac{\mathcal{R}_N}{(1-m)r^{target}-\delta \left(1 + \dot P_M^e - (1+r)m\right)} \label{eqn-res-price-B} \end{eqnarray}

% P_B & \le    \frac{\mathcal{R}_N}{(1-m)r^{target}-\left[ \delta(1+L(P)- (1+r)m\right]}
We call this  $i's$ maximum bid and compute it for all potential buyers and sellers. In each sale the highest $P_B$ will make the purchase. The denominator can be seen as an adjusted rate of return for capitalizing net rents, analogous to the value of $r$ in  the standard capitalization formula. 



\subsubsection{expected price}
So what is the \textbf{expected price}? this should be the expected price from a weighted price regression.\footnote{Wheaton \cite{wheatonVacancySearchPrices1990} suggests ``The combination of price and expected sales time determines the ``expected price'' for a house: market price discounted by expected sales time.'' In his model, vacancy, matching, sales time, and prices with positive vacancy, matching, sales time, and prices are all jointly determined.} Crudely it might be
\[P_{d,t}^e=\beta_1 d + \beta_2 P_{d,t-1} +\epsilon\]
where $\beta_1$ captures spatial correlation and $\beta_2$ captures serial correlation. In this model $\beta_2=1+\dot P$. This simple relationship would tend to change in each period, so the regression would be repeated  at the end of each period  to be used by everyone in the next period. 

You could add $+ \beta_3 (P_{d,t-1}-P_{d,t-2})$ to capture the changing  price change. 



%\textbf{Alternative approaches to the reservation price}

%\begin{eqnarray}
% P_{person}^{reservation} & =   \frac{\mathcal{R}_N}{(1-m)r^{target}-\delta \left(1 + \dot P_M^e - (1+r)m\right)} \label{eqn-res-price-P} \end{eqnarray}


% \begin{tabular}{p{1.5cm}|p{4.5cm}|p{4.5cm}}
%     & reservation & maximum bid\\\hline
% person-buyer    &   & $min \left\{\frac{\mathrm{savings}_i}{1-m_i^{max\_permitted}},\ \mathrm{saving}s_i + M_i^{max\_permitted}  \right\} $ \\\hline
%    person-new  &   & $min \left\{\frac{\mathrm{savings}_i}{1-m_i^{max\_permitted}},\ \mathrm{saving}s_i + M_i^{max\_permitted}  \right\} $ \\\hline
% Bank    &$ min \left\{\frac{\mathrm{savings}_i}{1-m_i^{max\_permitted}},$$\newline$$\mathrm{saving}s_i + M_i^{max\_permitted}  \right\}  $ & \hline
% \end{tabular}

% ALTERNATIVE, NOT USED CALCULATION
%An alternative way to do this, based on old calculations is to compute the realized mortgage share. The realized mortgage $m_i$

%$m_i$ is follows the wealth based rule, $m_i*$ follows the savings based mortgage rule. The realized mortgage share is whichever of these is chosen by the rule in Equation~\ref{eqn-max-mortgage}.

%$m_i^*$, and get the savings.
% We use $m_i$ in calculating the maximum bid for individuals.
%The \textbf{maximum bid} (price that will be offered) is the minimum of $M^{max_i} +S$ and the maximum bid calculated using $m_i$, , (which is calculated independently) if price
% $P\le \frac{M_i^{max}}{m_i}$ 
% and $m_i^*$ if 
% $P\ge \frac{M_i^{max}}{m_i}$, where 
% \[m_i^*=\frac{M_i^{max}}{P}\]


\section{FILE AGGLOMERATION MODEL}

\renewcommand{\sfdefault}{phv}

\subsection{Initial values for  the agglomeration parameter}
A segment is now redundant and has been commented out

% \vspace{5lines}

% {\Large  $Prefactor = 1506.712$ based on width=height=10 and density=100} and agglomeration\_coefficient= 1.2,
% We need to get the scales of the parameters and the population size roughly consistent.

% I assume the 10X10 grid is full. 

% \section{Computing the prefactor: details}
% We have set the subsistence\_wage to \$40,000.

% The urban wage premium is in the range of 13-20\%  Assume 20\% and we get \$8,000

% Under the neoclassical assumption  \$40,000 is the  marginal productivity of rural worker

% The urban production function is 
% \[Y=AN^\beta\]
% \[Y=prefactor*working\_population**scaling\_exponent\]

% where $working\_population = width*height*density$ 

% so 
% \[Y=prefactorA*(width*height*density)^{\beta}\]

% We have more conditions that this has to satisfy: The marginal product must be consistent with the urban wage and the distribution rule. The wage cannot add up to more than total output output. Workers cannot get 1.2Y/N, for example. With CRS they would get 0.8Y/N.    

% \subsection{approach one: from agglomeration surplus}
% \[urban\_wage= subsistence\_wage + wage\_share * agglomeration\_surlpus\]
% The \textbf{agglomeration surplus} is the excess relative to the CRS case when $\beta=1$:
% \[agglomeration\_surplus= A(N^{1.2} -N^1) \]
% The urban wage premium is then the share for each worker:
% \[\omega= wage\_share * \frac{agglomeration\_surlpus}{N}\]

% and this becomes
% \[\omega= wage\_share * A\left(\frac{N^{1.2}-N^1}{N}\right)=1 * A\left(N^{0.2}-1\right)\]

% To see what this looks like, consider a population of 10,000 when  wage\_share=1
% \[\omega= \$8,000 = A\left( 6.309-1 \right)\]

% {\Large So $A = 1506.712$ based on width=height=10 and density=100}  

% Smaller N makes A bigger

% % Say width=height=15 and density= 200:
% % \[\omega= \$8,000 = A\left(10000-1\right)\]

% \subsection{Approach 2: from Marginal product and subsistance wage}
% % We want the marginal product of labour in the rural economy at the firm level to be at least close to \$40,000.

% % Firm employment $L$ is small relative to  urban employment {N}.  We can create a generic rural firm and then consider an urban firm with agglomeration effects to get the parameters we need. 

% % Assume that the rural p[rooduction function is 
% % \[Y^R=A^R K^\alpha L^\beta\]
% % where $\alpha=0.2$  and $\beta=0.8$. The marginal products are 
% % \[MPL=\beta Y/L=\$40,000\] and\[\ MPK=\alpha Y/K =0.05\]
% % From the first, 

% % \[Y=\frac{L*\$40,000}{0.8}=\$5\ million\]

% % This is firm revenue. From the MPK, 

% % \[ \frac{0.2 \$5\ million}{0.05}=K =\$20\ million \]

% % We now have the capital, labour and output for a model firm with a marginal product of labour  equal to the subsistence wage we have chosen.

% % We now consider that this firm operates in the city and enjoys  urban agglomeration benefits.  If the scale coefficient=1.12$,$ $\omega$ would be as a first appr Say that the marginal product rises to \$48,000. This supports an urban wage premium of $\omega\$8,000$.

% % We need to have a population size or a number of firms with a firm size. Assume the population is 10,000 and firms have 100 workers. All firms will have the same marginal product of labour in a competitive labour market, so size should not matter. 

% % \[Y^U=A^R N^\gamma K^\alpha L^\beta = N^\gamma Y^R\]
% % with a marginal product of 
% % \[MPL^u=\$48,000=N^\gamma \beta Y^R/L=N^\gamma *\$40,000\]
% % so $N^\gamma=1.2$ and $\gamma = .019$. 


% This value is consistent with an empirical  value for $\beta$   1.02. The value is lower than empirical estimates. Furthermore, if we consider the same firm structure and a population of one million the value falls, which suggests that agglomeration effects are not scale-independent, but instead increase with urban size. $\beta$ may itself be a function of $N$.

% A second interesting possible implication of our calculation is that only part of the agglomeration effect appears in wages. Urban rents  are large, but agglomeration effects are much larger.


 

% %\[subsistance_wage= MPL(\beta=.= \]


\section{COMBINED PRICES - FILE}
\subsection{Combined mortgage maximum permitted by the bank for new buyers and second home buyers}

\begin{align} 
M_i^{max} &= min \left\{ m_i^{max\_permitted}*P, \ M^{max\_permitted}_i \right\} 
% &= min \left\{ \left(0.9- \left( \frac{W_i}{\bar W}\right)^{0.1}\right)P, \  \frac{0.28*(\omega+\psi)}{r_i} \right\}, 
\nonumber
\end{align}


\subsection{Maximum bids and reservation price for investors}.\footnote{transaction costs on the sale are omitted. Add a parameter or variable.}
institutional borrowers have no limit on their borrowing capacity ($m_i=1$). Existing owners face mortgage constraints .
maximum bid price:

\begin{eqnarray}\label{eqn-max-investment-bid}
P_{bank}^{max\_bid} & = \frac{\mathcal{R}_N}{(1-m)r^{target}-\delta \left(1 + \dot P_M^e - (1+r)m\right)} 
\end{eqnarray}


\subsubsection{Maximum bid second house buyer}

\begin{eqnarray}\label{eqn-max-second-bid}
P_{second}^{max\_bid} & = \frac{\mathcal{R}_N}{(1-m_i^{max\_permitted})r_i^{target}-\delta_i \left(1 + \dot P_M^e - (1+r_i)m_i^{max\_permitted}\right)}  \nonumber
\end{eqnarray}

Combining the investment \textbf{profitability constraint} and mortgage constraints

\begin{eqnarray}
P_{second}^{max\_bid} & = min \left\{P_{second}^{max\_bid},\ \frac{\mathrm{savings}_i}{1-m_i^{max\_permitted}},\ \mathrm{saving}s_i + M_i^{max\_permitted}  \right\}  \nonumber
\end{eqnarray}

\subsubsection{Reservation prices for investment owners}
 if anyone offers more than your maximum bid, you should sell the property. 
\begin{eqnarray}
P_{investor}^{reservation} & =    \frac{\mathcal{R}_N}{(1-m)r^{target}-\delta \left(1 + \dot P_M^e - (1+r)m\right)}  \nonumber\end{eqnarray}


\subsection{New buyer maximum bids and reservation prices }

There are two cases - moving as a buyer and moving as a renter.

\subsubsection{Moving as a renter}
The gain for a potential renter is 
\begin{align}
gain^{rent}=&]
=&\ \omega-cd+\mathbb{A}^{net}-\mathcal{R}
  \nonumber
\end{align}

\subsection{Moving as a buyer}
The gain for a buyer is 
\begin{align}
=&\ \omega-cd+\mathbb{A}^{net}+(\dot p-r_im_i)P   \nonumber
\end{align}
Combined
\begin{enumerate}
    \item Move only if $max\{gain^{buy},\ gain^{rent}\} \ge 0$, otherwise do not move to the city.
    
    \item Buy a house if $(\dot p-r_im_i)P\ge  \mathcal{R}$, otherwise rent.
\end{enumerate}

\begin{align}
P_i^{max\ bid}= min \left\{\frac{\mathrm{savings}_i}{1-m_i^{max\_permitted}},\  M_i^{max\_permitted} + \mathrm{saving}s_i  \right\}   \nonumber  
\end{align}


\subsection{Reservation price for retirees}
Both owners and tenants retire. Homes only become available if they move to the country. 
\subsubsection{A renter move to the country if} 
\[\mathcal{R} > \ a\psi+ \mathbb{A}^{net}0\]
\subsubsection{An owner sells and moves to the country if}
\[r_i(P^{realized}-M) >\ a\psi+ \mathbb{A}^{net}\]
This value provides a \textbf{reservation price}: 
\[P_{owner-sell}^{reservation} =\ \frac{a\psi+ \mathbb{A}^{net}}{r_i}+M \]

\subsubsection{An owner Rents and moves to the country if} 
\[\mathcal{R}^{net}>r_i(P^{realized}-M)\]
This value provides a \textbf{another reservation price}: 
\[P_{owner-rent}^{reservation} =\ \frac{\mathcal{R}^{net}}{r_i}+M \]

\subsubsection{Retiring owner's reservation price}
\[P_{owner-combined}^{reservation}= max\left\{ P_{owner-sell}^{reservation} ,\ P_{owner-rent}^{reservation}\right\}\]

}


% The three questions I promised to deal with were
% \begin{enumerate}
%     \item should we add interest to the net rent calculation? (yes)
%     \item should I subtract the purchase price from the calculation of the value of a purchase?
%     \item in the bid price, how is the present-value deflator computed?
% \end{enumerate}


% \subsubsection{Defining Net Rent carefully}


% The market rent may include  a fraction of financing costs $F(v, rM)$  that depends on how tight the market would support additional useful analyses:
%  \[\mathcal{R}_M=\mathcal{R}_W + F(vacancy\ rate) rM\]
 

% Net annual rent for investors is then  defined this way 

\begin{align}\mathcal{R}_N &= \mathcal{R}_M - \mathcal{O} - \mathcal{T} + F(v, rM)\nonumber\\
&= (\omega - {dc} + a\psi )+ \mathbb{A} - a b \psi - \tau  \mathcal{P}_{M, t-1}  + F(v, rM)
\end{align}

This term enters into the computation of housing prices. We assume it is deposited each year by an investor and earns interest $r$

% \subsection{Answer to question 1} about whether to add interest to the rent term: YES

% \section{REDUNDANT Financialized capital}
% %Individuals and institutions play a role in the housing market through credit markets and direct investment.Agent's access credit shapes worker's ability to purchase homes. Credit is offered by institutions.
% % Agents may be able to foresee future growth. %They may even over invest if they follow market trends and bubbles form. 
% % They can claim a share of the urban wealth as it grows over time by owning the land. 

% %If the return on housing investments is competitive with alternative investments, capital from institutions and individuals will flow into housing. Institutional investors can purchase housing.  Individual households can also allocate a larger share to housing to capture the returns.
% % If the return on investment in housing is competitive with alternative investments, can purchase housing for it's financial return. They can rent housing and sell the asset with appreciation later. We examine the conditions in which this increased demand can drive up prices in the market. 
% % Capturing future growth of the city, depending on their foresight - how much does it take to block individuals from gains-- Regime.
% % both institutional investors and individual agents can purchase additional housing for it it's return on investment even though they don't need it as a place to live. 
% % Use value vs rent value. 
% %Institutional investors can purchase housing as an investment. Individuals with more wealth may invest 
% Households may, for example purchase a larger house than they need, purchasing additional units to rent out, or keep a house after retiring rather than downsizing.  % and individuals with sufficient means can purchase larger homes than they need to benefit from appreciation, or purchase additional units to rent to others. 
% % Investors can also purchase housing to claim a share of the future productivity of the city. Individuals and groups can put extra money into housing. Institutional housing providers can buy up the housing supply.
% % HYPOTHESIS FEEDBACK LOOP---FINANCIALIZED INVESTMENT --
% %The rise in spending on housing as a proportion of income can be driven by both rising prices (cutting into quality of life) and increasing investment to claim a share of the returns. ---disaggregate and show the geometry -
% % Test how linear is this relationship? 

% % \subsubsection{Size of mortgage available, $m_i$}
% % \[m_i= \frac{0.25Y_i}{r_i}\]
% % where $r_i$ is $i$'s cost of capital, $Y_i$ is $i$'s income.

% % \subsubsection{Cost of capital $r_i$}
% % The cost of capital is known to differ for rich and poor. Say for example, the cost of borrowing, $r_i$ for agent $i$ if the base lending rate is $\bar{r}$
% %  \[ r_i = (A + B \frac{\bar{W}}{W_i})\bar r\]
% % where $\bar{W}$ is mean wealth and $W_i$ is individual wealth. %Figure~\ref{fig-borrowing-cost} illustrates the effect.


% % This has a number of immediate implications. First, if agents discount at their borrowing rate, wealthier agents a lower discount rate and therefore value properties more highly. 

% % Second, given the  common rule that mortgage payments cannot exceed some fraction of disposable income, the wealthy will be able borrow larger amounts and at lower interest rates that the less wealthy. At any distance from the centre they will be able to make a higher bid.
 
% % If the expected return on a property is greater than the individual cost of borrowing, it would pay any agent to borrow as much as possible and purchase properties as they become available.

% \subsubsection{The rate of return on a property purchase $v$}
% To explore the implication of the financialization of the urban land market we need a function to calculate the return on a unit of land that reflects the actual gradient of opportunity in financial markets. We begin with the price appreciation, $\Delta P=P_T-P_0 = (1+\dot p)P_0-P_0 $, where $\dot p$ is the rate of price appreciation over the period $T$. Rates will all be specified for the period $T$. Transaction costs, including real estate fees, take a fraction from the value of the final sale.

%  The speculator invests a down payment, $D$, and gets back at time $T$ the  increased price $(1+\dot p)P_0$, plus rents, minus any costs and minus the mortgage with interest.
% %footnote{We can include a use value, $U$ in place of rent for expatriate owners to represent using the property - say one month a year - when they are not renting the property and a \textbf{vacancy tax},
% %$T$ at rate $t$ to affect the speculator's  decision.
 
% The rate of return is the value of the gain, $V$,  over the size of the downpayment, $D$, where
% \begin{equation}https://quoteinvestigator.com/2015/08/30/practice/
% V =capital\ gain - Interest\ due  	+ Rent  - operating\ cost\    
% \end{equation}

% The rate of return is $v = \frac{V}{D}$. 

% Both the  share of the price  that can be mortgaged, $m$, and the interest rate  and $r$ may be functions of the agent's wealth. $\delta$ represents the net capital gains tax. It makes it possible to capture the capital gains kept. If it is set to one, it simplifies the equations, all is kept. Keeping the variable offers a policy variable to control the return on financial capital.

% \begin{eqnarray*}
% V  %	&=& capital\ gain - Interest\ due  	+ Rent  - operating\ cost\\
% % 	&=& \delta P_T-D \qquad \qquad \quad - (1+\delta r)M \quad	 + R  	-C\\
% % 	&=& \delta P _T \qquad-(P_0-M) \quad- (1+\delta r)M 	 + R  	-C\\
% %	&=& \delta (1+\dot p)  P_0 -(P_O -M)  -(1+\delta r)mP_0  + R  -C\\
% %	&=& \delta (1+\dot p)  P_0 -P_O + M \qquad -(1+\delta r)mP_0  + R -C\\
% %	&=&( \delta (1+\dot p)-1)  P_0  + mP_0 \quad -(1+ \delta r)mP_0  + (\rho-\kappa)P_0\\	
% %	&=& \left(  \delta (1+\dot p)-1    + m \quad - m(1+\delta r)  + (\rho-\kappa)\right)P_0\\'
% %	&=& \left(  \delta (1+\dot p)-1    + m \quad - m-\delta rm  + (\rho-\kappa)\right)P_0\\
% &=& \delta(P_T- (1+r)M) \qquad \qquad 	 + R  	-C   - T\\
% &=& \delta((1+\dot p)  P_0- (1+r)mP_0)   + \rho P_0  	-\kappa P_0 - tP_0\\
% &=&( \delta((1+\dot p)  - (1+r)m) \ + \rho   	-\kappa -t) P_0
% \end{eqnarray*}

% This is the  net present value of buying, and selling after one period. \textbf{It has  6 exogenous parameters}. Operating revenue and costs $ \rho -\kappa - t$ a present value. 

% The rate of return is $v = \frac{V}{D}$. For expat investors, we get a \textbf{decision rule}:\begin{enumerate}
% \item  if $v \geq a$ (with some private use?) with no rent,  don't bother renting. 
% \item If $v(no\ rent\ and\ tax) < a\leq v(with\ rent)$,  then  rent. 
% \item If $ v(with\ rent) \le a $,  then sell 
% \end{enumerate}

% We can, with some simplifications, write
% \begin{eqnarray}
% \frac{V}{D}&=&( \delta((1+\dot p)  - (1+r)m) \ + \rho   	-\kappa - t ) \frac{P_0}{D}   \nonumber\\
% 		&=&( \delta((1+\dot p)  - (1+r)m) \ + \rho   	-\kappa - t ) \frac{P_0}{P_0-mP_0}   \nonumber\\
% 		&=&\frac{ \delta(1+\dot p  - (1+r)m) \ + \rho   	-\kappa - t } {1-m} \label{eqn-decision-rule}
% \end{eqnarray}


\section{COMMENTED OUT STUFF - ANYTHING IMPORTANT TO ADD BACK HERE?}

%$P_M^{e}$ expectation based on  either rational expectations or another set of explicitly stated rules,  $P_M^{\epsilon}$ is an agent's expectation. The agent's expectation can  etc.

%In the analysis that follows, we work with an \gls{expected market price} based on the theoretical model developed in this dissertation, $P_M^{e}$.

% \subsection{Price forecast approximation} \label{section-price-forecast}
% $L(p)$

% $p$ is all the price data plus any exogenous information (e.g. policy knowledge?). $L(p)$ is an estimation function that produces a `common knowledge' value for the rate of price increase. Later you can add idiosyncratic extra knowledge or extra ignorance.

% Some spatial regression references:
% 'taxonomy of spatial econometric model specifications that incorporate spatial externalities in various ways'
% \cite{anselinSpatialExternalitiesSpatial2003},  
% overviews of econometric methods and computational tools \cite{anselinModernSpat
 
%and the potential for financialized ownership to extract that value, pushing those who contribute to creating growing wealth, to a subsistence frontier. Formalizing this relationship is part of the core conceptual contribution of this work. %\footnote{***WHERE IS MAIN DISCUSSION OF THIS? ** RENT CHAPTER? .. %METHOD/model/intro section - ref section COULD ELABORATE ON BENEFITS. THIS IS A CORE CONTRIBUTION, WE ARGUE IT IS  WORTHWHILE TO EXPLORE THE LINKAGES BETWEEN PRODUCTION AND EXTRACTION. TO OUR KNOWLEDGE, IT HAS NOT BEEN DONE ELSEWHERE. 
%Future work can explore the way in which this locks those who earn bellow average incomes to fall behind, no matter how much wealth they create, the smaller urban concentration of value may increase the total value produced, but also makes it easier to monopolize land and concentrate in a financialized ownership class, that amplifies whatever advantage it begins with, however small or path dependant, (shared ownership is in a sense an unstable equilibrium, under the conditions of combining urban scaling laws and a land tenure models that allows finacialization)  REFERENCE HERE AND  MOVE/EXPAND DISCUSSION IN FUTURE WORK}.
 

% It is an important assumption in the analysis because it connects the work with the formal analytic tradition and is in an important sense what makes this a classical model MOVE DISCUSSION HERE \dots
 %METHODOLOGY CHAPTER? warranted rent is a frontier .  



% *** LOTS OF NOTES MOSTLY ON AMENITY AND OTHER DIFFERENCES BETWEEN WARRANTED AND MARKET RENTS - NEED A CLEAN PARAGRAPH OR TWO EXPLAINING
%\footnote{The assumption $\mathcal{R}_W = \mathcal{R}_M$ is easy to vary in future experiments, 
% Rents can diverge from warranted rents. In practice, some people may pay more or less than the warranted rent. They may draw on family assets, assume debt, spend more on amenity, etc. Rents are, however, limited by what tenants can pay, and that, in turn, is limited by urban wages. WHERE DO WE MAKE THAT ASSUMPTION.
%including by adding a range of decision rules. 
% Our model of income is incomplete, and people don't only care about the income they earn. Some are paying for the amenity of urban life. In real life, many other factors cans shape amenity value including features of the housing, desire to live near clubs, sports, cafe's libraries, hospitals, neighbours, and family members, the quality of nearby walk, of transit to places they may wish to travel to etc. Appendix~\ref{appendix-amenity} includes a theoretical treatment of amenity in the context of the model. People also make choices with imperfect information, some are investing in living in a city
% ADD DISTINCTION BETWEEN REALIZED MARKET RENT AND DIFFERENT KINDS OF EXPECTATIONS --- THEY WONT ACTUALLY BE THE PREDICTION VALUES. THE PREDICTION VALUES ARE WHAT ARE USED. 
% THEY PREDICTION VALUES CAN INCLUDE ERRORS, BIASES, ETC.
% WE USE THE EQUILIBRIUM TO EXPLORE A SET OF FRONTIERS RIGOROUSLY.
% }.
% ADD A FOOTNOTE ABOUT THE DISTINCTION BETWEEN THE VALUE WHICH INCLUDES AMENITY AND THE PRICE PAID?  e.g. if you own the services you capture it - the value experienced can differ from the value paid. - can be subjective, variable, etc, 

% Talk about unobserved variables, and 
% They really care about what they will get- use decision rules to forcast the actual returns realized. Might guess about the biases of others
% The market rent is what the agent actually pays in rent the property. - they can own a shre and captre some


% The following sections detail the model and explain how each of these terms is derived.
% **TODO ADD NOTES ON TIME IN THE FOLLOWING SECTIONS E.G. ALL RENTS REPRESENTED ON AN ANNUAL BASIS, PRESENT DISCOUNTED VALUE, ETC

% \subsection{Warranted rent} \label{section-warranted-rent}


% \subsubsection{Locational services}
% Since people live in homes inside or outside the city, it's a share of the subsistence wage,  OWNERS CAN ALSO CAPTURE? Maybe move some of the discussion here.

% It is the combined annual services a property, at a distance $d$, offers to a worker in the city. It is thus the maximum that they may be charged.
% CUT? This quantity is a locational rent. The capitalized value of the locational rent is:  $w-{dc}$. This is because a worker, located at a distance $d$ from work, paying as much as $w-{dc}$ in rent would still choose to work.  %Workers could pay that much in rent and additional costs and it would still worthwhile to commute. 
% \subsubsection{Land and building services}
% Annual rent charged initially should be  equal to the annual value of services, $h$:
% and the warranted price is 
% \[P_W=\frac{\omega - {dc} + a\psi}{r}\]
% We will use this value as the price for the first period 
% $P_W=\frac{\omega- {dc} + a\psi}{r}$

%$\mathcal{R}_N$ is the value that investors consider when assessing the \gls{financial return} on property, since it's the maximum that it's worth paying to capture the value renters are wiling to pay to live in a property, given the \gls{urban wage premium}. 
% consider in investment decisions. % of services and costs
% For an investor, it is $\mathcal{R}_N$  that is relevant in investment decisions. % ( - fees, cost of money etc.) decision making. 

% All of these are in annual values. We will use the present values  for the appropriate period  T in computations. (See notes on the present value calculations to use)

% \subsection{NEW WORK STARTS - How does this affect the bid price?}

% I now return to Equations B6 to compute the value of a proposed purchase


% \begin{align*}
% V &= \delta \left(P_T - (1+r)M\right) +      \mathcal{R}^w_N  \tag{B6} \\
%   &= \delta \left((1+\dot P)    - (1+r)m    \right) P_B + \mathcal{R}^w_N% \label{eqn-property-investment-value1}
% \end{align*}
% Similarly, Equation B7 becomes

% \begin{align*}
% r_{return} = \frac{\delta \left(1 + \dot P - (1+r)m\right)}{1-m} + \frac{\mathcal{R}^w_N}{(1-m)^{bid}}\tag{B7}
% \end{align*}
% And following the same sequence of steps to derive the bid price 

% \begin{align*}
% P_i^{bid} \le &   \frac{\mathcal{R}^w_N}{(1-m_i)r_i^{target} - \delta_i \left(1 + L(P) - (1+r_i)m_i \right)} \tag{B10}\\
% 		 \le &   \frac{ \omega - dc + a\psi -   ba\psi - ca\psi}{(1-m_i)r_i^{target} - \delta_i \left(1 + L(P) - (1+r_i)m_i \right)}	\\	
%    		 \le &   \frac{ \omega - dc + a(1-b-c)\psi}{(1-m_i)r_i^{target} - \delta_i \left(1 + L(P) - (1+r_i)m_i \right)}	
% \end{align*}

%Our conclusion about the advantage that the bank and the wealthy have can be read out of this equation. 
%That may let us simplify the exposition in the financialization chapter.


}