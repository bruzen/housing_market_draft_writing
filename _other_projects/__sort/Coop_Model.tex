\chapter{A financial model for making housing affordable}

\epigraph{ We considered the dynamics and impacts of a publicly supported coop model designed to produce
two million housing units over five years}

\section{Solutions}
\subsection{Why it is necessary to intervene in the Housing market }
Canada is projected to attract at least 1.3 million immigrants over three years.
\footnote{The  
%\href{https://www.canada.ca/en/immigration-refugees-citizenship/corporate/transparency/committees/cimm-feb-15-17-2022/2022-2024-multi-year-levels-plan.html}
{022-2024 Immigration Levels Plan, tabled on February 14, 2022} specifies   431,645 in 2022 (range: 360,000-445,000),     447,055 in 2023 (range: 380,000-465,000),     451,000 in 2024 (range: 390,000-475,000).} 
At that rate population will rise by 4-5 million by 2030. To put these numbers in perspective, Canadian cities with a population of 100,000 to 1 million are considered medium-sized, 

CMHC has conservatively 
% \href{https://www.cmhc-schl.gc.ca/en/blog/2022/canadas-housing-supply-shortage-restoring-affordability-2030}{estimated} 
estimated that If the current rates of new construction continue, the housing stock will increase by only 2.3 million units between 2021 and 2030. 
 
 To restore affordability, an additional 3.5M affordable housing units are needed by 2030 3.5M affordable housing units will be needed by 2030, bringing the total required to 5.8 million units.  CMHC therefore  calls for ``a drastic transformation of the housing sector, including government policies and processes, and an ‘all-hands-on-deck’ approach to increasing the supply of housing to meet demand.''  The challenge is to produce approximately 4 million affordable units that the housing market as currently organized will not provide. 

The analysis in this thesis concludes that, given the ongoing financialization of the housing market, which is not considered in the CMHC analysis,

\begin{enumerate}
\item the financial system will eventually extract all net urban land rents through investment in urban property
\item housing accessibility will become increasingly challenging for disadvantaged groups
\item housing will be largely eliminated as a saving mechanism and asset fr middle income Canadians,  resulting in a systematic decline in the `middle class'
\item that the quality of urban life will decline
\item the economic growth and development of cities is threatened by this financialization
\end{enumerate}



\subsection{An institutional approach to the inevitable failure of the existing market mechanisms}

 In this note/section/chapter we describe a financial model with several desirable properties
 
 \begin{enumerate}
     \item land rents will be shared with a growing fraction of residents through cooperative housing structures, rather than captured by a declining number of homeowners and asset holders
     \item social housing and land ownership will expand
     \item the rising cost of housing will be ameliorated
     \item economically disadvantaged groups will enjoy increase access to affordable housing.
     \item the quality of housing will improve
     \item society's ability to respond to climate change and to engage in land-use planning will improve     
 \end{enumerate}


 It is our contention that all of these objectives can be achieved with existing institutions and relatively simple, though dramatic, public interventions.

 The instrument that has most potential for solving the growing housing crisis is the cooperative movement. Co-operative housing represents an important part of the housing market in many countries in Europe and clearly have the potential to operate at the necessary scale in Canada. . For example, housing co-operatives currently manage over 3.5 million dwellings in Poland (about 27\% of the total housing stock in the country in 2009), about 17\% of the total housing stock in the Czech Republic and Sweden, 15\% in Norway.
 \footnote{\href{https://coopseurope.coop/cooperative-housing-key-model-sustainable-housing-europe-organised-cecodhas-housing/}{Cooperatives Europe}}

 A housing cooperative is a housing business in the form of a consumer cooperative mutually owned by its members, which operates in accordance with the Cooperative Principles and Values. 
 \footnote{https://www.ica.coop/en/cooperatives/cooperative-identity}
  It is centered around a cooperative of inhabitants that collectively develops, finances, maintains and operates multi-resident projects. This make the cooperative structure  especially appropriate for what we have called the `shoulder' of the urban core - the area of low density, usually single-family housing on the edge of the high-density, multi-unit areas on modern cities. 
 
 There is wide agreement that raising the density in these areas is necessary part of any housing strategy. The bulk of new housing over the next 40 years will be created in these `shoulder' areas. That should makes this class of multi-family housing the main target of Canadian housing policy. 
 
 Because it controls  does not need to make profit, coop housing can be much more affordable. Challenges remain if coops are to be a large-scale instrument of housing policy however. 
 
 \begin{itemize}
     \item Coops are a corporate form of housing and land ownership. As a result speculative gains accrue to the membership, making the conventional housing cooperate useless as a mechanism for capturing the socially produced capital gains for the community as a whole.
     \item Coops are not inherently inclusive although they may seek to expand their membership. They are designed to benefit only members. 

     \item Coops, like privately owned housing with rent-control, may inhibit mobility, reducing community productivity.
     \item Coops, like rental housing, generally do not allow residents to accumulate    `sweat equity', an important savings mechanism for  households.
     \item Individual coops are generally small scale, very local organizations, limiting their impact.
     \item Although credit risk is assumed by the cooperative  (a more robust approach than individual financing), coops do not have access to funds at preferential public sector rates
    % \item 
 \end{itemize}

 \hrule
\vspace{1cm}
\begin{quotation}
\section*{The MOBA* housing model}

The \href{https://moba.coop/}{MOBA housing model}  is centered around a cooperative of inhabitants that collectively develops, finances, maintains and operates a multi-apartment building. Because it controls the entire trajectory (and does not need to make profit), the resulting apartments are much more affordable for the inhabitants. 


The cooperative owns the real-estate as well as takes on the necessary loans to pay for its construction. Participating households or individuals (the members of the cooperative) thus collectively own their building. Individual members or households cannot speculate with their apartment or their stake in the land – in that way it is not just a safe and affordable option for the first generation, but for many generation of its inhabitants to come.  


\tiny * A network of housing cooperatives from Belgrade (Pametnija Zgrada / Ko Gradi Grad), Budapest (R\'ak\'oczi Collective),  Ljubljana (Zadrugator), Prague (Sd\'ilen\'e domy / První Vlaštovka) and Zagreb (Cooperative Open Architecture)  with support from the Cooperative for Ethical Financing (ZEF), urbaMonde, World Habitat, Socialni inovatori, FairCoop and Heinrich B\"oll Foundation.\normalsize

 \end{quotation}
 \vspace{1cm}
 \hrule
 \color{black}

\newpage
\section{A model}
 \textbf{We will examine the dynamics and impacts of a publicly supported coop model designed to produce two million housing units over five years} In our view this is the appropriate design scale for hte project. 

 We begin by specifying the features of a National Housing Cooperative needed to make the project work.

 \begin{enumerate}
     \item The coop is national, and members have the right to apply to transfer to vacant or new units anywhere in the country. This provides \textbf{improved labour mobility}, clear market signals for development and improved household freedom, and reduced transaction costs for households.
     \item individual housing payments can be linked to local market rents, encouraging efficient use of the housing stock. 

     \item  as  individuals join they contribute equity to the coop for the construction of new housing. The coop thus mobilizes household saving effectively. 
 \item individual housing payments are offset by a fair return on the  housholder's equity. 
    \item individual equity may be augmented through social programs based on the right to housing and a national housing-first strategy. 
    
     \item total coop equity is divided between individual member equity and common equity. The coop owns all equity in the land and is a co-investor in each housing unit. 
     \item The coop captures all capital gains on land a part of the capital gains on buildings. The gains are directed to expansion of the housing stock, with some used to keep the cost of housing low.
     
 
     \item public sector land is made available to the cooperative (ideally no public sector land that might eventually be used for housing is allowed to pass into private hands) on the condition that it remain in the non-profit housing cooperative forever be converted to public housing. 
     \item to finance rapid and large-scale coop housing governments lend to the cooperative at the lowest possible public borrowing rate taking as security the assets of the cooperative. In effect there is no addition to net debt for the government. The government does take on some financial risk, as it does with, for example student loans.
     \item all public housing funds, including all CMHC lending  are channeled thought the cooperative. The result is that the public sector ceases entirely to subsidize private. ownership of housing and the concomitant speculative gains for individually.
     \item Since the government has historically subsidized housing, the governments of Canada commit to a universal basic housing grant which may only be used to purchase coop membership and housing. Home-owners would be excluded from this grant on the grounds that they have benefited from prior subsidies and unearned captial gains.
     \item  
     
     \end{enumerate}

     
\section{Advantages of the model}

A feature of this model is that it can combine the advantages of scale in financial management and project development while including local design and responsiveness to member needs.

The model will gradually build a non-market housing  sector that will provide an attractive alternative to pure market housing and will eventually moderate rent increases across the syste.

The model will produce the ``missing middle'' types oh housing because members will participate in design and financing. at t he same time it will be able to exploit  all of the economies of multifamily projects. 

The model will provide a platform for ecological planning, since residents will participate in development and management. This link is almost always absent in developer-built housing.

Public land and, for example, church lands, can legitimately be committed to the cooperative because the land remains a community asset.

The model is able to get `public license' for developments - \begin{itemize}
    \item to use public land
    \item to receive subsidies for the homeless, the poor, and the young
    \item enter into neighbourhood. The price of the risk of failing to get social license for a project is a significant cost (10\%) that these projects can avoid even when they have the same building costs.  Government investment, participation of local residents, guaranteed  tenure structure. This can allow development in the areas where it is most valuable. Existing residents can participate in planning that improves their own property values
\end{itemize}

\section{The moment}
with rising interest rates, a public intervention. can give the entire coop sector a relative advantage in financing. 






\begin{tikzpicture}[
roundnode/.style={circle, draw=green!60, fill=green!5, very thick, minimum size=7mm},
squarednode/.style={rectangle, draw=red!60, fill=red!5, very thick, minimum size=5mm},
]
%Nodes
\node[squarednode, text width= 2cm] (Coop)          {Coop movement};
\node[roundnode, text width= 2cm, text width= 2cm]  (uppercircle)       [above=of Coop] {Rule changes};
\node[roundnode, text width= 2cm, text width= 2cm]  (leftcircle)       [above=of Coop] {Rule changes};

\node[roundnode, text width= 2cm, text width= 2cm]  (leftcircle)       [left=of Coop] {Finance};

\node[squarednode]      (rightsquare)       [right=of Coop] {3};
\node[roundnode]        (lowercircle)       [below=of Coop] {4};

%Lines
\draw[->] (uppercircle.south) -- (Coop.north);

\draw[->] (leftcircle.east) -- (Coop.west);

\draw[->] (Coop.east) -- (rightsquare.west);
\draw[->] (rightsquare.south) .. controls +(down:7mm) and +(right:7mm) .. (lowercircle.east);
\end{tikzpicture}