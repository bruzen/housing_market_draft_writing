\chapter{Lifeboat Canada}

\epigraph{ Canada must develop a humane housing strategy that supports rapid population growth, social integration and economic productiveness.  This is clearly a large-scale system design problem.}

\section{External forces}
Canada's current housing crisis and the projections of the previous chapter are mild compared to what we may face as the globe warms. Canada has committed to admitting  1.3 million immigrants over three years.   %Meanwhile housing  affordability has been declining since 2003-4, especially in Ontario, Alberta and BC.
If external world conditions remain as they are, the number of immigrants is likely to rise through the coming century. An annual increase of just  1\% each year (slightly less than the  rate of population growth for the last 50 years)  will raise immigration to nearly one million per year by the end of the century. The current growth is equivalent to adding one to four medium sized cities per year. That number will rise to as many as nine medium sized cities or one large city per year at the end of the century.   

External world conditions will not remain as they are, however.  According to a \href{https://www.pnas.org/doi/10.1073/pnas.1910114117}{study in the journal Proceedings of the National Academy of Sciences}\footnote{Future of the human climate niche, Chi Xu, Timothy A. Kohler, Timothy M. Lenton  Jens-Christian Svenning, and Marten Scheffer, May 4, 2020, 117 (21) 11350-11355.  %https://doi.org/10.1073/pnas.1910114117
}, the planet could see a greater temperature increase in the next 50 years than it did in the last 6,000 years combined. By 2070, the kind of extremely hot zones, like in the Sahara, that now cover less than 1 percent of the earth’s land surface could cover nearly a fifth of the land, potentially placing one of every three people alive outside the climate niche where humans have thrived for thousands of years. The result will be large scale migrations to the shrinking band of habitable lands.% A change in the geographical distribution of human populations  is  likely part of the spontaneous or managed adaptive response of humanity to a changing climate

Temperature will not be the only source of migration pressure. Kulp1 and Strauss\footnote{Kulp, S.A., Strauss, B.H. New elevation data triple estimates of global vulnerability to sea-level rise and coastal flooding. Nat Commun 10, 4844 (2019). %https://doi.org/10.1038/s41467-019-12808-z
} show – employing NASA’s nes Digital Elevation Model, CoastalDEM, that 190 M people (150–250 M, 90\% CI) currently occupy global land below projected high tide lines for 2100 under low carbon emissions, up from 110 M today. Under high emissions, CoastalDEM indicates up to 630 M people live on land below projected annual flood levels for 2100, and up to 340 M for mid-century.

Extreme weather events and conflict are the top two drivers of forced displacement in the short run globally. Since 2008, according to the United Nations High Commissioner for Refugees (UNHCR), an annual average of 21.5 million people have been forcibly displaced by weather-related events – such as floods, storms, wildfires and extreme temperatures.\footnote{
%https://www.unhcr.org/uk/news/latest/2016/11/581f52dc4/frequently-asked-questions-climate-change-disaster-displacement.html.  
See also Gaudry Haynie, J., Balagna, J., clark-Ginsberg, A. (2021) “Climate Change Migration: Developing a Security Strategy for All,” (https://www.rand.org/blog/2021/03/climate-change-migration-developing-a-security-strategy.html) who give suggest that nearly 30 million people  are displaced by weather and  conflict each year. Cited in the White House REPORT ON THE IMPACT OF CLIMATE CHANGE ON MIGRATION, October 2021. } The Institute for Economics \& Peace (IEP)  predicts that 1.2 billion people could be displaced globally by 2050 due to climate change and natural disasters.\footnote{
%https://www.prnewswire.com/ae/news-releases/iep-over-one-billion-people-at-threat-of-being-displaced-by-2050-due-to-environmental-change-conflict-and-civil-unrest-301125350.html
}. It is difficult to find anyone willing to make forecasts for the rest of the 21st century.\footnote{Projections assume that there will not be a mass die-off in this century as forecast by the 12972 Club of Rome Report, Limits to Growth", if society continued, as it has, on the ``Business as Usual'' path. }  



Canada, with its huge land-mass, relatively mild climate, stable government, high standard of living, and extensive agricultural capacity will be a preferred destination for many migrants. With perhaps one quarter of the most attractive land on earth at the end of the century, Canada is likely to receive or to be under  pressure to receive, a disproportionate share of global cross-border migration. It is unclear how many of those affected will look to Canada. An article in the UN Chronicle argues that, for several reasons,  it is ``improbable that there would be long-distance mass population movements even in a situation of systemic climate change.''\footnote{
%https://www.un.org/en/chronicle/article/will-there-be-climate-migrants-en-masse
}  Displaced people tend to stay near their origins, refugee camps and shelter villages are typically set up not far from the site of the calamity, countries resist cross-border migration, and various forms of adaptation are possible.  

Nonetheless, there is little doubt that Canada will have the opportunity as well as pressure to settle significantly larger numbers that currently envisioned. 
The country will face \textbf{the classic lifeboat problem}  described by by Garrett Hardin:\footnote{\textbf Garrett Hardin, BioScience Vol. 24, No. 10 (Oct., 1974), pp. 561-568 } how many are allowed aboard the lifeboat?  How many can the lifeboat sustain? how many will be left to die?

Faced with this dilemma, Canada can treat the situation as an opportunity to draw in far more talented and productive immigrants  than currently planned. 
 %Furthermore,  humanitarian considerations provide another motive for accepting far larger numbers of migrants.
The carrying capacity of \textbf{Lifeboat Canada}, however, depends on actions taken now - Will Canada have the capacity to house larger number of  immigrants given its difficulty in housing its current population?  Will Canada be able to integrate larger numbers into its economy? Will the country be prepared to quickly mobilize  the creative power of immigrants.

\section{The design challenge}
To prepare \textbf{Lifeboat Canada} for the opportunity and 
potential threat of mass migration and greatly increased immigration, Canada must develop a housing strategy that supports rapid population growth, social integration and economic productiveness. The strategy must be humane and environmentally sound. It has to draw on the pre-existing capacities of immigrants. This is clearly a large-scale system design problem.

\subsection{Features of the required system}

\begin{enumerate}
    \item High-density housing.
    \item Urban access
    \item High quality accommodation for families, including extended families  
    \item Environmental sustainability.
    \item Environmental quality for residents.
    \item Maintenance of the immigrant community structures and support for cultural continuity. 
    (Historically, much of the support received by immigrants has come comes from previously settle migrants.) 
    \item Connection to the resident community
    \item Supportive of commercial activities that generate income 
    and contact with others to accelerate economic integration.
\end{enumerate}



\section{The larger setting of this analysis for Canada}

\subsection{Rent extraction}

This thesis demonstrates the very basic principle of increasing urban rent extraction through the financialization of the housing market. 
{\color{red}(One of my key research findings/methods/outcomes in recent years.)}
 
The institutional basis is the ownership structure that allows the social surplus generated by agglomeration to be privately appropriated. Even in the recent past, this process contributed to the development of a home-owning middle class and to widely distributed wealth. Financialization is now narrowing the distribution of wealth and shrinking the equity-owning middle class. {\color{red}(One of my key research findings/methods/outcomes in recent years.)}

Taken together the two points above describe  one of the deep causes of the current and continuing housing crisis.

It is our contention that the process we describe will also undermine the productivity of cities and the quality of urban life. {\color{red}(The most pressing incomplete research issues of my research to date.)} 

% \documentclass[tikz,minimum size=20cm]{standalone}
% \usepackage{tikz}
% Four Themes for  housing research April 21

% \usetikzlibrary{shadings, shadows, shapes, arrows, calc, positioning, shapes.geometric}
% \usepackage{pgfplots}
% \pgfplotsset{compat=1.16}
% \usepackage{mathtools,amssymb}
% %\input{Preamble}
% %\input{(SpecialDistanceCourse}
% %\input{FRAMES}
% \begin{document} 
 
 \begin{tikzpicture}%[scale=.8]
    \tikzstyle{every node}=[font=\small]
%\draw[help lines,step=.5] (0,-11) grid (11,11);

%  LOCATION OF ALL NODES
\coordinate (Extraction) at (0,3);%PREFACE
\coordinate (Tenure) at (5,3);%
 \coordinate (Corridor) at (5,0); %history
\coordinate (Lifeboat) at (0,0); %
%\coordinate (d) at (9,9);%


%.  CIRCLE NODES
% \fill[red, fill opacity=.8] (0,0) circle (4cm);
%\fill [gray, fill opacity=0.2] (m) node [text width=2cm, black, opacity=1] 				(community)	{} circle (3cm);
%\fill [gray, fill opacity=0.1] (oo)  node [text width=2cm, align=left, black, opacity=1] 		(econ) {} circle (3.5cm);
% \node at (e) [ ] 		(comLtabel) {Data};
% \node at (g) [ ] 		(ecLabel) {THEORY};
	
%\node [circle, draw,  fill=gray, opacity=.5,, text width=1.5cm] at			 (aa) 		(preface)		{HYPOTHESIS};
%\node [] at			 (f) 		(plan)		{\Huge PLAN};
			%\draw [fill=blue, fill opacity=0.35] (b)node [text width=2cm, align=center, black, opacity=1] 			(history){History: New is Old} circle (1.2cm);

\node[fill=red!30, regular polygon, regular polygon sides=6, draw, align=center]at (Extraction) (ExtractionP) 	{\color{black}Rent  \&\\Finance};%Extraction \\and \\Financialization} ;

\node[fill=yellow!30, regular polygon, regular polygon sides=6, draw, align=center] at (Tenure) (TenureP) {Tenure\\ and\\ Wealth} ;

\node[regular polygon, regular polygon sides=6, draw, align=center] at (Corridor) (CorridorP)	{Socal\\ Wealth \\Corridor};

\node[fill=green!30, regular polygon, regular polygon sides=6, draw, align=center] at (Lifeboat) (LifeboatP)	 {Lifeboat\\Canada};

% LINKS
%\draw (J) node [text width=6cm, text centered] {Economic Development };
%\draw ()--();
\draw [gray, line width=2mm,-> ](ExtractionP)to [bend left=35](TenureP);
\draw  [gray, line width=2mm,->](TenureP)to [bend left=60](CorridorP);
\draw  [gray, line width=2mm,-> ](CorridorP)to [bend left=35](LifeboatP);
\draw  [gray, line width=2mm,-> ](LifeboatP)to [bend left=60](ExtractionP);

% \draw [gray, line width=2mm,-> ](whatis)--(joint);
% \draw [gray, line width=1mm,-> ](joint) to [bend right=25](efficiency);
% \draw  [gray, line width=1mm,-> ] (joint) to [bend left=20](capital);
% \draw [gray, line width=1mm,-> ](joint) to [bend right=20](externalities);
% \draw [gray, line width=1mm,-> ](joint) to [bend left=25](trans);
% \draw  [gray, line width=1mm,-> ](joint)->(pubgoods);
% \draw  [gray, line width=.5mm,-> ](capital) to[bend left=25](small);
% \draw  [gray, line width=.5mm,-> ,dashed](joint) to[bend left=7](forestryEC);
\end{tikzpicture} 

% \end{document}

\subsection{Tenure}
The three points above make  it clear that developing \textbf{tenure structures} that stop or reverse the two big trends we highlight above are part of any durable solution to the housing crises. The productivity of our communities has to be maintained with some form of social equity if the wealth creation of the community is not to be sucked away.     {\color{red}(Research in financing and tenure structures, as well as regulatory support and the social innovations required appears is necessary to lay the groundwork for any robust response to the current housing crisis. It will take  multidisciplinary collaboration  \dots}

\subsection{Community wealth corridors: the Oasis model}
More than half of Canada's population lives within  one hundred kilometers of a line from Windsor to Quebec City. More than half of Canada;'s immigrants come to that strip. Canada is now crawling toward a high-speed rail system in that strip. It is crucial to understand that more than half of the housing that will be built in Canada in the next 30 years will be on that corridor. 

It could be a social and environmental disaster. Or we could be building enormous social wealth for the most environmentally integrated society in the  world. 
    
        {\color{red}(My vision for housing research in Canada?would focus on creating a social wealth corridor in which the housing assets are owned by the people, not outside capital, and the communities are genuinely sustainable. This will take not just building design, but integrated building systems, linked to the evolving transportation system and )}

 \subsection{Lifeboat Canada}
 All of these issues are nested inside a broader set of challenges: climate migration. To protect Canadians it will be necessary for Canada to make its economy and its cities more resilient, carbon free  increasingly self-sufficient while expanding the housing supply and  maintaining the quality of life and of the local environments. 
 
 At the same time, it is very likely that Canada will be pressed to, and will want to, take in a fair share of the climate refugees expected from  other parts of the world this.

 I think about this  set of challenges as the ``Lifeboat Canada'' challenge. We have to be able to stay afloat while rescuing as many as we can. Lifeboat Canada presents a massive design problem  for our  existing cities, for the many linked centers related to cities, for the transportation system and for environmental design. 

 It takes us back to how we can capture and use the social wealth that create to survive the coming storm. 
 

  {\color{red}(One part of my vision for housing research in Canada  is to start planning for ``Lifeboat Canada''that proofices XXX (topics, goals, methods for knowledge transfer, collaboration structures, etc.)}
 
