
\section{Models in Labs}

Models can act as a boundary object, letting people with different perspectives communicate and build understanding within a system.

1. Where the uncertainty, disputed values, and high stakes, and the decisions that must be made are urgent, the field of study is often referred to as \textbf{post-normal science}. While many different disciplinary and non-academic perspectives are needed here, bringing people together is often difficult and necessitates the development of shared languages that people working in the problem area can use to communicate with each other.
 
2. To help develop these shared languages, design can be used to facilitate communication. Design thinking is embedded throughout lab processes in both the work conducted by researchers and the workshops that bring together participants to build small-scale prototype interventions. In our view, the prototypes themselves are \textbf{‘boundary objects’} that enable people from different perspectives to communicate with each other easily. It is in the reduced cost of communication, not in producing preliminary interventions, that prototypes have their primary impact.

3. These reduced communication costs enable participants to coordinate activities with each other. Sometimes these coordinated activities will include most participants in a common strategy, though this is not the primary goal. Instead, overlapping subsets of participants, often simply pairs, can use the opportunities created by the lab process to build joint projects or to engage in exchange with each other. These new agreements between participants add value and help build capacity for innovation and adaptation in the problem space.


"For example, the \textbf{developmental impact investing approach} argues that using deep knowledge of a complex social-ecological system like affordable housing to guide impact investment can not only lead to more effective impact investments, it can also build a better understanding of the system that impact investors are working to change. This generates a virtuous cycle that can build multidirectional \textbf{pathways to scaling} social innovation (Geobey, Westley, Weber, 2012).


% DOING THIS IN AN ENGINEERING DEPARTMENT
% Engineering  knowledge for responsibility - just like a bridge, a social choice to have a middle class.
% have housing for everyone, have food for everyone

% social engineering has been a sensitive point- efforts have tended to control.
% Particularly after WWII social experiments, that were horors.

% even building predictive modes is a problem. any intervention that makes your model predictive increases your control.

% Reductionism had some holdouts but largely unchallenged

% %Challenges to reductionism deep within the field too

% Answer is of course to relinquish commitment to state. 

% - syxtems and resilience give  a bridge
% What is the systems compenent - design for the properties- 
% clearly a system

% social engineering- a note on the approach to engineering in social system
% cannot know much about state- have to comite to regime, it is then possible to comit to things like freedom \dots
% to autonomy, relinquish comitent to stat


% RESILIENCE IS CENTRAL TO THE ADVANCE - TO ACTUALLY MAKING AN ENGINEERING OF THE SOCIAL POSSIBLE.


% at the level of capturing and sharing details.

% applied work


% has to do with Social innovation, how societies are transformed,
% gives an ease in relating the model with the action. 

% action and choice
% weak links in hosting, 

% did these models for rockefeller, for the labes-- had a group
% worked iwith the games intstitued ith design \dots


