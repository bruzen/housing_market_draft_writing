
\subsection{Parameter values}\label{sec-param-values}

% OLD TO CUT - JUST CHECK IF WE NEED ANY DETAILS FROM THIS
\renewcommand{\arraystretch}{1.5}
\begin{tabular}{rlrr}\
Symbol         & Name                                 & Value      & Formula  \\ \hline
$\omega$  & Maximum locational rent (wage premium) & 0.012        \\
$a$       & Share of subsistence wage for land and building & 1.0 \\
$\psi$    & Subsistence wage & 10000 \\
$tau$     & Property tax rate &  e.g 1.6\% = 16 mills           & \\
$c$       & Transportation cost & \\
$\mathbb{T}$       & Period & 5 years      \\
$r$       & Individual interest rate & 0.05 \\\
$\dot P $      & Price growth                         & []         & $\frac{P_t-P_{t-1}}{P_{t-1}}$\\
% $P^\mathbb{T}_e$        & Expected price in \mathbb{T} years            &            & $P_0(1+\dot P)^T$ \\ % *** WAS $P^e_\mathbb{T}$ 
\end{tabular}
\renewcommand{\arraystretch}{1.0}





\subsection{Transportation costs}
Transport costs have two parts:
1) fuel and vehicle costs per km
2) time costs per km

\subsubsection{Vehicle related costs}
Use one year as the wage period, converting transportation costs per km to annual cost for consideration in the household budget. Starting with the cost per km, calculate the cost per year:

\textbf{cost per km =$\textit{t}$}:. \$0.59   (from  Ontario data, 2021). sensitive to congestion, use of subways (\$5 /day?), 

 \textbf{work trips per year} 2 way * 5 days/week * 50 weeks work days = 500. [range: 450-550]

\textbf{cost per km-year} = work trips per year*cost per km

=\$0.59/km*500 trips/year  =  \$295/km year 

\subsubsection{Time costs}
\textbf{time per km}. range: 20km/hr -> 3min/km, 40km/hr -> (1.5min/km - 3min/ km)per trip 

(New York rush hour is much slower:  4-9km/hr ->6-15 min/km)

\textbf{time  per km-year} = work trips per year*time t per trip = 500* 3min  = 1500 min/km year = 25 hours= 3-3.5 days/km
 
\textbf{time cost per km-year} =  (days per km-year /work days/year)*wage premium per year  = 3/250 = 0.012 years/km year. ?

\textbf{money cost of time per km year} 

= time cost per km-year* wage(including subsistence) 

= 0.012 year* wage per year

\subsubsection{Total cost per km year of commuting for one agent}
\textbf{money cost of time per km year + \$295/km year * distance} \\
= (0.012 w+ \$295)/km year 
    \begin{quotation}
    \textbf{Example}
    To get a sense of the required wage if we have this annual cost structure, assume city\_extent $d^*$ is 30 km. At this point the transport cost is equal to the wage

\[(0.012 w+ \$295)/km year)*30 =  w\] 
\[.36w+ 8850=w\]
\[w=13828.12\]
\begin{quotation}
\textbf{PLAUSIBILITY CHECK}
This is plausible land rent, but does not include building rent. 
Capitalized at 5\% this house is worth \$ 276,562, a fairly cheap house 30 miles from city centre
\end{quotation}
\end{quotation}



\subsubsection{Value of transportation price to use in model}
\[ {c}=(0.012 w+ \$295)/km year \]






\begin{description}
    \item[hi]

eg r=.05  t=5  $\delta(5)$ =  $(1/(1+.05))^5 = 0.7835262$


% growth rate= rt
% growth factor =($1+r)^t$
% discount rate= r
% discount factor = $1/(1+r)^t$   
\begin{lstlisting}
def get_discount_factor(self):
    """
    The discount factor gives the present value of one dollar received at particular point in the future, given the date of receipt and the discount rate.
    Delta is the subjective individual discount rate for agent
    after one year. This will be close to the prime interest rate, r_i.
    """    
    delta = self.r_prime
    delta_period_1 = 1 / (1 + delta) 
    delta_mortgage_period = delta_period_1**self.mortgage_period
    sum_delta = (1 - delta_mortgage_period)/delta
    return sum_delta
\end{lstlisting}
% #   sum_delta = delta_mortgage_period * (1 - delta_mortgage_period) # Old
Delta could also depend on wealth. For example,  use the bank rate, which is the rational rate but people who are poor typically have higher rates.  It would not change as the central bank changes r-pirme
% delta could be wealth based typically higher for poor.

% sum\_delta is sum of the infinite series minus discounted infinite series after mortgage\_period years
% Here, it is the present value of annual payments from one to mortgage\_period years e.g. of mortgage payments or rent received. 
% delta\_mortgage\_period was previously called   delta\_period\_{\mathbb{T}} 
% # Note delta_mortgage_period is subtracted to subtract the long tail. 1/delta gives the PV of an infineite series of payments

\begin{lstlisting}
# A version with delta depending on wealth
wealth = self.wealth
delta =
\end{lstlisting}
% savings =(sum(0-age)((1+r)**age)*savings_rate*subsistence)


\end{description}

\subsection{Investor}
\begin{description}
\item[mean initial wealth $\bar W$] $= \frac {a\psi}{r_prime}+savings_0$
\end{description}


\section{OLD}

\section{TODO FIX/CUT Table of parameter values}
\renewcommand{\arraystretch}{2}
\begin{longtable}{lp{5cm}cp{2.5cm}}
\caption{Table of parameter values}\label{tab:parameters}\\
\hline 
Symbol         & Name                                 & Value      & Formula  \\ \hline
$a$ replace    & Share of $\psi$ for land and building &   0.3         & \\
$b$ replace    & Share of $\psi$ for maintenance       &   0.2         & \\
$tau$ replace  & Property tax rate &  e.g 1.6\% = 16 mills             & \\
$c$       & Transportation cost & \\
$\mathbb{T}$       & Period & 5 years      \\
$r$       & Individual interest rate & 0.05 \\
$\omega$  & Locational rent & 0.012  \\
$\psi$    & Subsistence wage & 10000 \\
$a$       & Share of subsistence wage for land and building & 1.0 \\
$\tau$       & Tax share & \\

---       &  & \\
$m_i$          & Individual borrowing-ratio           & 0.75-0.85  & $M/P^{ask?}$ \\
$M^{max}_Yi$.  & Maximum mortgage based on income     &            & $\frac{0.28(\omega+w)}{r_i}$ \\
 $M^{max}_P$   & Maximum mortgage based on the price  &            & $0.8*P_0$ \\
$IS$           & Income share for housing debt        & 0.25-0.35  & higher for the poor \\
$\rho$         & Rent ratio                           &            & $\frac{\omega-tau*d_i}{P_0}$ \\
$\kappa $      & Operations ratio                     & 0.1-0.3    & e.g. $ 0.2\frac{\omega-tau*d_i}{P_0}$ \\

$\sigma$       & Tax ratio                            & 0.25-0.35  & e.g. $ 0.3\frac{\omega-tau*d_i}{P_0}$ \\
$\dot P $      & Price growth                         & []         & $\frac{P_t-P_{t-1}}{P_{t-1}}$\\
% $P^T_e$        & Expected price in T years            &            & $P_0(1+\dot P)^T$ \\ % *** WAS $P^e_T$
$\bar r$       & Prime interest rate                  &            & \\
$r^{target}$   & Target interest rate                 &            & $\bar r + margin$ \\
$r_i$          & Individual borrowing-rate            & \multicolumn{2}{c}{ $r^{target}+ K/(W-W_{min}) -K/(\bar W - W_{min})$}\\
$\delta_i$   & Individual discount rate        &  USE $\rho$?          & To assign \\
$\delta_i$     & Discount factor for T                &            & $\left(\frac{1}{1+r_i^\delta}\right)^T$ \\
\hline
\end{longtable}
\renewcommand{\arraystretch}{1.0}

