


Jane Jacobs was the first to define cities as problems of organized complexity. A city is a system with many interrelated variables which are  in constant simultaneous change. 
There is a danger in simplification with complex systems, and our model runs that risk. 

In our model, agglomeration effects run between local productivity  of exporting firms and the population. But the agglomeration of firms and people also drives the agglomeration of other businesses and the creation of a variety of amenities.

Most of the literature of amenities appears s to deal with livability  and the benefits for individual. Amenities would be likely to have a number of other significant effects on the city. For example, it is likely that the presence of amenities will attract higher bids for housing. Since such private amenities are not directly increasing marginal productivity there is o reason to expect that wages will rise to pay for them. Two channels may come into play. Rising costs for housing may drive up wages. The natural adjustment for firms would be to reduce employment by shifting production to cheaper locations or  employing labour-saving technologies.

The other obvious channel is that the amenity induced rise in housing prices absorbs what would otherwise be consumption behaviour.

In either case, there will be distributional effects. Property owners will capture increased land rents. If amenities are funded out of taxes, the burden probably falls roughly equally on tenants and owner occupiers. since property taxes are approximately related to housing consumption, but the land rents are captured by institutional owners as well as owner-occupiers.

Transportation improvements that reduce commuting time would seem to increase effective wages ($\die{\omega}-cd){c}>0$) but not labour costs, making labour more available for the same wage premium. 

Kaufmann et all investigate the  general statistical patterns in the quantity and spatial distribution of different urban amenities including public spaces and institutions as well as businesses, which all provide different services to urban populations, such as  restaurants, parks, or universities. 

They argue that amenities are  central for generating and supporting economic agglomeration effects, attracting investment to ``developing neighborhoods, promoting economic growth, supporting innovation clusters and facilitating businesses linkages''. 

They show that the aggregate quantity of amenities in an urban area  scales sub-linearly  with population size across US metropolitan areas. The model used is exactly the same as the one used to demonstrate that a scaling law holds for  urban GDP. Instead of GDP, however, the dependent variable is a measure of amenity density based on data extracted from a unique new Google Places dataset, Google Places API (2012).

When they disaggregate, however, they find that for approximately 74\% of amenity types, they cannot reject linear scaling. Four percent exhibit super-linear scaling. They list take-away restaurants and travel agents in this range. 

Sub-linear scaling is associated with libraries, universities, and movie theatres.

The authors argue that the scaling features they describe can be used when constructing a new city, planning the development of a growing city, or promoting changes in a city. They may indicate the minimum number of people required to support the introduction of new urban activities. They might also be used in the planning process to identify gaps in the amenity character of a city. In a city undergoing densification, the scaling patterns may be useful in  making land use provisions for new amenities.

\section{application}
I have not figured out how to apply this. 

A simple approach is to assume that the base employment that we consider demands a layer of amenities that represent the additional fraction of  the population needed to provide the amenities - say 10%