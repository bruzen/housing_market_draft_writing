\chapter{Conclusion} \label{chapter-conclusions}

Financialization of the housing market is a flow of money into the housing market, expanding financial ownership and tenancy, while reducing the share of owner-occupied housing. Its goal is the capture of spatial rents, which we will show has implications for both the productivity and the class structure of the city. 
In the urban context financialization is fundamentally \gls{rent-seeking}, and we argue it can have a profound impact on the system, including effects on both distribution and productivity. 

The analysis makes clear that in addition to the recognized distributional consequences, the housing crisis has productivity impacts that should be considered in developing urban and housing policy. Particularly, it centers concern with implication for urban development of growing rent extraction by the financial sector. 

A key insight is that the financialization of the housing sector is a form of \gls{rent-seeking} that must have detrimental effects on urban development and on the well-being of urban residents, that is neither constructive nor productive.

CONCLUSIONS/SUMMARY?
We argue financialization %This dissertation makes the case that it
has worrying implications for the success of cities and the nature of our social fabric. 



The housing crisis raises the question of whether Canadian cities can continue to attract people and accumulate wealth for its residents and industries, and whether they can sustain their growth.
% Our focus is land rents, %but in the context of an urban economy. 


% This appears, at least part of it appears as locational rents. 
% Financialization, is about capturing the surplus generated by the city.  % To model the financialization of land markets.

% To model financialization we need rent because
% To develop a formal model of financialized urban land markets, we introduce rent because rent is % precisely   about extracting and allocating surplus value in a system. % and that is what financialization of the housing market is about/does. 
% The classical approach to rent is a core tool in the development because it brings the extraction of surplus into focus.
% GAP Nobody has linked the rents - linking rents to urban scaling. Beteencourt is talkign about a surplus in the system, wealth, but he hasn't linked to the market/land market for those locational rents.


- hamstrings the whole thing.
--> the whole system as a welfare producing system fails if the value gets sucked out---CONCLUSION TH THESIS---fails from a social point of view-
these are averages---some are structurally below average so some are always behind simply because of the structure of the rents claimed.. that's built in FUTURE WORK- DIFFERENT INCOMES GETS YOU THAT. 


% We could run off a cliff and accidentally destroy the middle class, we should consider the implications, need a language to explore that.
---


We examine the effect of housing on wealth inequality by looking at 

Adding 2 things 1. rent extraction and 2. power law scaling of productivity, we find rent is the breaks on the engine of wealth creation

The economics is clear that this is what's at stake is productivity of cities, the distributive features of the economy and the impact of the middle class.

The result is a simple model in which marginal productivity determines the wage, the wage determines the size of the city, the size of the city determines the labour supply, and labour supply determines marginal productivity. 

There is a market for the urban product produced by firms, and a financial market that agents can invest in.


