\chapter{Conclusion} \label{chapter-conclusions}
% two historic processes are changing society, powerful. 
% Two historic 
% Two powerful processes, urbanization, and financialization, are changing society.
We're concerned with the interaction of two historic processes changing society. One is the continued urbanization, and the other, the process of financialization, 
Our addition is 
articulating the movement of financial capital though the urban land market. 
% This thesis develops a conceptual framework for a model of the housing market and then describes the model and the implications of the analysis of the model. %insights it produces. 

If there are two historic theories of distribution, the classical and neoclassical for two period, we are in a third period centered on human capital. 
We don't actually have a theory for who gets the urban rents. 
% There is an emerging understanding that this transformation may affect the accessibility, security, and affordability of housing.
%, but does not explore who gets those rents. %
There isn't %, however,
a theory of who gets the rents that are produced by the city. 


%  classical theory of who gets the product of land and how that affects the social classes.

% We have built a model  on a model ..  to explore how \dots , we argue that financialization \dots and that it may also have dramatic effects on the productivity of cities.

% There are models of rent in cities---it doesn't link with models of production and of finance.  Models of finance are spaceless.  In order to understand finacialization, and what the urban housing is doing doesn't connect with neoclassical models of production, or of banking and finance, not a link

% claim productivity,---right to be productive in the city.. claiming the rents -the value in cities.. claim with financial instruments \dots once you have that in one model, you can actually do that link..  people think there is a return on investment,  what is it they're getting, it's a return on investment. 

% , and that's important because those rents are capturing some of the productivity of the city. 

% \section{Spaceless finance}
% since financial analysis theorizes objects such as assets, debts, revenue and cost flows, and the changes in and exchange values of their values over time. These are inherently spaceless because they are accounting entities, independent of location. It matters where a worker or a farm is. It does not matter where and dollar or a rouble is. % to spatial rents in an urban system. % to refomulate the 

% The first component, the market model, is illustrated in 
% Figure~\ref{fig-impacts} shows the main effects  that  arise from the growing financial sector participation in the housing market. 


% \section{Linking fields}

% spatial/locational rent is -"Income or payment for the use of location. Locational value is largely created by access to people not by landowners, hence any payment for locational advantages is a rent.
% land rent is "Ricardo payment for the natural productivity of the land, but also considered proximity to markets (locational advantages) as a source of rent. 
% ``The economic  surplus generated in production as a result of differences in the quality of some \gls{factor of production}. Often described as the difference between the opportunity cost of a factor of production and the income it earns. In this thesis we focus on rents generated by \glspl{agglomeration effect}. According to \gls{classical rent theory}, rent is the price paid for the use of land. More generally it is the  surplus generated by any natural resource, up to and including the athletic talents of basketball stars \cite{lackmanClassicalBaseModern1976}. Land, talent, and mineral resources are seen as ``the free gift of nature,'' forms of capital which the owners do not create but do appropriate. Like the productivity of agricultural land in classical theory,  urban \gls{agglomeration effect}s produce land rents that are not created but are appropriated by the landowners.''

% We begin in the region of overlap among the three disciplines with the fundamental linking construct, space. 
% All three fields have central theories built around the value of location, which is essentially the locational rents. 
% MAKE SURE LOCATIONAL VALUE IS DEFINED BEFORE THIS.



% \section{Financialization}
% Financialization operates at different scales. In the following sections, we discuss financialization at the level of finance, microeconomics, and the larger system: the development of particular financial and legal instruments, the growth in the number of individual transactions that create or transfer financial assets, and the set of system-level set of shifts resulting from this process. 

% It is this third aspect, the system-level effects of financialization that is the subject of this work.


% financialization in the housing markets is a multi-faceted process that included the creation of financial instruments like mortgages and REITs and then the trading of these instruments of markets. %This process and it's effect on the housing market are contested.
% Arguably financialization, is playing an important role in this crisis of rising costs of housing. % Because of the significance of housing ownership to wealth distribution, the financialization of ownership may also affect the overall distribution of wealth. Fewer people are able to buy houses and those renting face higher rents and decreased ability to build wealth.


% Financialization happens via the creation of financial instruments represent something of value, and can be bought and sold as an investment. They capture of streams of surplus, converting them to financial assets that can be traded easily. They are legal and ???. When financial instruments are traded on markets this begins the process of financialization WHY \dots Financial instruments allow \dots ?? do something.. that allows financial institutions and investors. to own more of the housing because \dots

% Financialization is an effort to claim these spatial rents. 



% The reason is that city is productive, and financialization is all about capturing the flow of surplus from any process. 
% financialization is just a feature of the capitalist system/expansion of capital. 
% exists land rent, and who gets the rents


 % and explores the potential consequences of financialization in the housing market.  % showing it is a form of \gls{rent-seeking} in the housing market and ?? 
    % \item Chapter~\ref{chapter-background} sketches how this thesis relates to four major fields: classical rent theory, neoclassical production theory and growth theory, the scaling literature, and urban spatial models. % \dots , and the role of space as a unifying factor across three of the fields. % WITH FINANCE IS SPACELESS.
    % *** ADD BACK? This work draws together sub-literatures including rent theory, production functions, the standard urban model, growth theory, urban growth theories, financialization, and the theory of distribution, so the chapters review those areas. % *** link the areas to the chapters better?  %theory for our analysis, 

    %, using an approach similar to that developed in modern growth theory, that we discuss in Chapter~\ref{chapter-rent}.  In that chapter we ground the observation in  \gls{neoclassical growth theory} and recent empirical and theoretical work on \gls{urban scaling}. 
 % We take a step beyond integrating labour markets in a city, to studying the distributional effects: who gets the surplus, what does that mean for the class structure, and ultimately the productivity of cities. 
% We began with the fact that there is growing policy concern about the financialization of  housing. We have produced 

% Financialization is the capture of the streams of surplus from society. %Financialization thus requires not just the existence of instruments of financialization, but that someone takes up these instruments, capturing streams of surplus. 
% financialization requires 
% these instruments are adopted. % that there is a market, and a process of adoption, 
% This has various functions at a systems level, adding liquidity, etc..
% At the microeconomic level, a second aspect of financialization is as 
% In this sense, 
% At the microeconomic level, then, 


% Financialization in this sense may describe
% More generally, financialization describes 
% describing the accelerated 

% CIT

 %arguably roots in early colonial and before, and to the early moetyar sytem.
% By this definition, 
% Analagous to adam's smiths- there is always an incentive eto provide what someone wishes, to serve the interests of another therough the market- the invisible hand, handing desired products whomever wants them and has money, 
%One may not wish to be an input to a compettive market.
% However, financialization is not a new concern.


% Morally neutral, individual and local--. by particular actors. people doing things in their contect, responding to local incentives.
% Financialization is not just the development of instruments. 
% The word also refers the the take up, the adoption as well. % It is the innovation not just the invention.
% it is also the increasingly control derived from the development of these kinds of mechanism. 
% Financialization of the economy is not new

\section{Rent and distribution}

% There's a gap, however, in the formal apparatus in standard economic theory for analyzing the distribution of the enormous value created. 

% The organizing principle in the spatial models of all three disciplines is an economic variable, land rent, % The three disciplines share a simple economic insight.

Financialization of the housing market is a flow of money into the housing market, expanding financial ownership and tenancy, while reducing the share of owner-occupied housing. Its goal is the capture of spatial rents, which we will show has implications for both the productivity and the class structure of the city. 
In the urban context financialization is fundamentally \gls{rent-seeking}, and we argue it can have a profound impact on the system, including effects on both distribution and productivity. 

% A key insight is that the financialization of the housing sector is a form of \gls{rent-seeking} that must have detrimental effects on urban development and on the well-being of urban residents, that is neither constructive nor productive.

% CONCLUSIONS/SUMMARY?
% We argue financialization %This dissertation makes the case that it
% has worrying implications for the success of cities and the nature of our social fabric. 

MOVE? We use what we call an \textbf{\gls{Alonso-Jacobs model}} to explore the source and distribution surplus value. We  work with an extension of the basic Alonso model that incorporates the \gls{agglomeration effects} that Jane Jacobs  described in her book, The Economy of Cities \cite{jacobsEconomyCities1969}. In our model these effects generate the \gls{urban wage premium} central to urban growth. % and the wage premium. 

The analysis suggests that in addition to the recognized distributional consequences, the housing crisis has productivity impacts that should be considered in developing urban and housing policy. Particularly, it centers concern with implication for urban development of growing rent extraction by the financial sector. 
The housing crisis raises the question of whether Canadian cities can continue to attract people and accumulate wealth for its residents and industries, and whether they can sustain their growth.
% Our focus is land rents, %but in the context of an urban economy. 


% This appears, at least part of it appears as locational rents. 
% Financialization, is about capturing the surplus generated by the city.  % To model the financialization of land markets.

% To model financialization we need rent because
% To develop a formal model of financialized urban land markets, we introduce rent because rent is % precisely   about extracting and allocating surplus value in a system. % and that is what financialization of the housing market is about/does. 
% The classical approach to rent is a core tool in the development because it brings the extraction of surplus into focus.
% GAP Nobody has linked the rents - linking rents to urban scaling. Beteencourt is talkign about a surplus in the system, wealth, but he hasn't linked to the market/land market for those locational rents.


- hamstrings the whole thing.
--> the whole system as a welfare producing system fails if the value gets sucked out---CONCLUSION TH THESIS---fails from a social point of view-
these are averages---some are structurally below average so some are always behind simply because of the structure of the rents claimed.. that's built in FUTURE WORK- DIFFERENT INCOMES GETS YOU THAT. 


% We could run off a cliff and accidentally destroy the middle class, we should consider the implications, need a language to explore that.
---


We examine the effect of housing on wealth inequality by looking at 

Adding 2 things 1. rent extraction and 2. power law scaling of productivity, we find rent is the breaks on the engine of wealth creation

The economics is clear that this is what's at stake is productivity of cities, the distributive features of the economy and the impact of the middle class.

The result is a simple model in which marginal productivity determines the wage, the wage determines the size of the city, the size of the city determines the labour supply, and labour supply determines marginal productivity. 

There is a market for the urban product produced by firms, and a financial market that agents can invest in.


