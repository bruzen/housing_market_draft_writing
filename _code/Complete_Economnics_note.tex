\section{the solution to fragmented and biased economic thinking}
I think people are hungry for what you might call complete economics rather than the fragments that they get presented by, for instance, by the neoliberals. 

Neoliberals emphasize one wonderful feature of our economy, which is that free exchange is often efficient. Great. And so that's all we talk about. But in fact free exchange rests on institutions that it sometimes wrecks. It rests on this social capital that is shared. Nobody makes a profit unless there are roads that are provided collectively. 

Complete economics would have to recognize that all the time. Cities don't grow in less there are agglomeration effects, but those are not in any way recognized in the neoliberal model, The neoliberal model intentionally tries to ignore the huge effects of externalities, externalities. They have not assimilated complexity, although as a mathematical form, it's not going to bother them Very much. 

What people really want is complete economics where people are understood to be drawing on  social skills, on social infrastructure, on culture which is costly to build up, and on mores which are costly to enforce, and on the information efficiencies of regulation. All of these things are the economic background of that narrow section. 

The suggestion I'd make is let's talk about complete economics. Not just that little neoclassical piece, neoliberal peace that's the end of that spiel. Well, it's yours. Where does how does this appear? What was that you wanted really wasn't that it's just the so how do you apply that in your meetings? Here? That's an interesting question.

Well, I'm mostly just saying like a sentence to people explaining why they might want to say it would be good to talk to you were starting

I think right now, you might want to take as one of your themes as mapping or understanding the complete economy.  Where do you fit into it? When people talk about capital gains taxes like we are, or the government starts doing it, (just to pick one thing that seems to be at odds with all this neoclassical stuff  but is right down there in the deep real economics.) 

I think one of the things that you've read into your thesis but is almost entirely ignored in neoliberal (and it's intentionally ignored,) is distributional questions. 

That's because so much of the liberal neoliberal economics is drawing on the social assets and then taking full credit for all of the product. Publicly supported foundation, but privately extracted wealth is what neoliberalism is. And the neoliberals push distributional issues off the table. Even though they've always been core in economics.

Because you're not going to change the SEC let me just get this sorted.
