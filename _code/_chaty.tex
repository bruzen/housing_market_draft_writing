axes[1].plot(time, model_out['wage'], label='wage', color='red')
axes[1].set_ylim(0, 110000)  # Replace min_value and max_value with your 

# Plot MPL and wage
axes[1].plot(time, model_out['MPL'], label='MPL', color='blue')
axes[1].plot(time, model_out['wage'], label='wage', color='red')
axes[1].set_xlabel('Time Step')
axes[1].set_ylabel('$')
axes[1].set_title('Marginal product of labour over time')
axes[1].grid(True)
axes[1].legend()
# Set y-axis limits
axes[1].set_ylim(0, 110000)  # Replace min_value and max_value with your desired values



%%%%%%%%%%%%%%%%%%%%%%%%%%
\section{The Equation we have to change}
THIS IS WHAT WE HAVE:
 self.wage_target = self.subsistence_wage + (self.MPL - self.subsistence_wage) / (1 + self.overhead)       


we want COST OF LABOUR = WAGE + OVERHEAD = MPL 
SO 
WAGE = MPL-OVERHEAD
BUT
OVERHEAD =  OVERHEAD_RATIO*WAGE
SO
WAGE = MPL- (OVERHEAD_RATIO * WAGE)
MPL= WAGE+ (OVERHEAD_RATIO * WAGE)
TARGET WAGE = MPL/ (1+ OVERHEAD_RATIO)

THE RATIO SHOULD BE BETWEEN 0.5 AND 1
%%%%%%%%%%%%%%%%%%%%%%%%%%
THIS IS COMMENTED OUT - IT IS CORRECT
 #self.wage_target = self.MPL / (1 + self.overhead) 
 ]# (1+self.overhead) # economic rationality implies intention
        self.wage = (1 - self.adjw) * self.wage + self.adjw * self.wage_target # assume a partial adjustment process
        



Traditional sampling techniques (grid vs random vs sobol vs latin hypercube)
https://www.youtube.com/watch?app=desktop&v=Evua529dAgc


https://salib.readthedocs.io/en/latest/

https://doepy.readthedocs.io/en/latest/#:~:text=At%20its%20heart%2C%20doepy%20is,arbitrary%20range%20of%20input%20variables.



JAN 23
THIS IS INCORRECT
        self.wage_target = self.subsistence_wage + (self.MPL - self.subsistence_wage) / (1 + self.overhead)       #self.wage_target = self.MPL / (1 + self.overhead) # (1+self.overhead) # economic rationality implies intention



- Fast version of the code where we track alternative bids/bid rent curves - actual bids/prices
    - Run on sharcnet
    - Show end grid
- Perturb mid way
- Fix error with crashing values
- Investors can sell

## Hypotheses
The financial sector affects the ownership of housing and the class structure of society, 
There may be dynamic/resilience features of this model that make the effects worse than you might expect
- Boom bust - pump wealth on/out of city on the boom and on the bust. 'There are sharks in the water' 1. Higher bid can amplify up swings - have to compete with speculators 2. can't get it back on down swing since outside finance offers a stable floor- buys up on way down (we expect larger effects as people are 1. displaced on the way up and then 2. evicted on the way down - can't use their spaces)
  - reduce demand for labour temporarily - reduce wages for labour - demand -- cyclic structure - local employment structure - and financial booms that are external.. - cases where they align and where they do not
  - link finance and employment -vary price of output- vary the employment *** 
- Hysteresis - perturb, doesn't come back - e.g. interest rates go up.
- The way it aligns with long run changes in the landscape e.g. tech changing local info changes vulnerability to these shocks -- Depth of the basin changes - er   odes systemic resilience - together these changes bo the capacity to hold value in landscape. (links to the productivity feedback - much will/can they invest in increasing their productivity/supporting kids/good food to grow brains. Education to increase productivity is the feature that makes productivity increases resilient to de-industrialization)
This has implications for landscape- system- class structure- 

This may have implications for urban productivity. (can actually displace productive uses - empty store fronts) - who can/will enter, how , who can rent spaces, speculative value may keep it empty (work spaces or living space - lowering pop), reduces consumption


Amplified effects by extensions
Two city experiments
Different incomes  - inequality
Urban density



## High Priority
imediately do experiments with the price of output - should give a decline in price of labour, may or may not result in changes in the housing market (record housing prices]
Which interest rate is charged to investors
speed/memory, plot and workflow to find parameters, run online - try storing the bid rent curve -- - what is the gap.. 

just keep housing - turn off storage..  cyclic pieces turn off the market. - turn off the production






# List variable_parameters for experiments
experiment_parameters_list = [Jan 5
 {
    #   'density': [1, 10],
        # 'c': [3000, 300.0], 
    #   `'wealth_sensitivity': 0.1,
      'capital_gains_tax_person':   [0.5, 0.6],# 3, 4, 5], # share 0-1
    #   'capital_gains_tax_investor': [0.2, 0.3], # share 0-1
    #   'subsistence_wage': [40000], # [10000, 30000],
    #   'gamma': [0.001, 0.02, 0.7]
        'property_tax_rate': [.04, .08]
    },
    # {
    #     'density': [1, 100],
    #     'gamma': [0.001, 0.02, 0.7]
    # }

    {
    #   'density': [1, 10],
        'c': [3000, 300.0], 
    #   `'wealth_sensitivity': 0.1,
    #   'capital_gains_tax_person':   0.01, # share 0-1
    #   'capital_gains_tax_investor': 0.15, # share 0-1
    #   'subsistence_wage': [40000], # [10000, 30000],
    #   'gamma': [0.001, 0.02, 0.7]
    
    }
]

python(62894) MallocStackLogging: can't turn off malloc stack logging because it was not enabled.

Zeihan on Geopolitics YouTube Channel. Geopolitical Strategist Peter Zeihan is a global energy, demographic and security expert.


class City(Model):
    @property
    def city_extent_calc(self):
        # Compute urban boundary where it is not worthwhile to work
        return self.firm.wage_premium /  self.transport_cost_per_dist
#    @property
#     def populated_extent(self):
#         # Compute urban boundary where it is not worthwhile to work
#         return sqrt(self.N /  (self.density * 4))


pip install mesa


print(mesa.__version__)
https://www.youtube.com/watch?v=WJE7g08NldQ shipping thiefs

(base) david-robinsons-computer-3:housing_app drdavidrobinson$ conda update -n base -c defaults conda
https://salib.readthedocs.io/en/latest/



Sobol Sensitivity Analysis (Sobol 2001, Saltelli 2002, Saltelli et al. 2010)
    Method of Morris, including groups and optimal trajectories (Morris 1991, Campolongo et al. 2007, Ruano et al. 2012)
    extended Fourier Amplitude Sensitivity Test (eFAST) (Cukier et al. 1973, Saltelli et al. 1999, Pujol (2006) in Iooss et al., (2021))
    Random Balance Designs - Fourier Amplitude Sensitivity Test (RBD-FAST) (Tarantola et al. 2006, Plischke 2010, Tissot et al. 2012)
    Delta Moment-Independent Measure (Borgonovo 2007, Plischke et al. 2013)
    Derivative-based Global Sensitivity Measure (DGSM) (Sobol and Kucherenko 2009)
    Fractional Factorial Sensitivity Analysis (Saltelli et al. 2008)
    High-Dimensional Model Representation (HDMR) (Rabitz et al. 1999, Li et al. 2010)
    PAWN (Pianosi and Wagener 2018, Pianosi and Wagener 2015)
    Regional Sensitivity Analysis (based on Hornberger and Spear, 1981, Saltelli et al. 2008, Pianosi et al., 2016)




Collecting package metadata (repodata.json): done
Solving environment: -
The environment is inconsistent, please check the package plan carefully
The following packages are causing the inconsistency:

  - https://repo.continuum.io/pkgs/main/osx-64/osx-64::blaze==0.11.3=py36_0
  - https://repo.continuum.io/pkgs/main/osx-64/osx-64::flask-cors==3.0.7=py36_0
  - defaults/osx-64::_anaconda_depends==2019.10=py36_0
  - defaults/osx-64::anaconda==custom=py36_1
  - defaults/noarch::flask==1.1.2=pyhd3eb1b0_0

  

curl https://bootstrap.pypa.io/get-pip.py -o get-pip.py
python get-pip.py

python -m pip install --upgrade pip

conda update -n base -c defaults conda

 conda env create -f environment.yml -n housing


    
conda deactivate
conda env remove --name housing
conda env create -f environment.yml
conda activate <environment_name>   

https://studiolab.sagemaker.aws/users/kirsten

source activate housing
conda install -c conda-forge mesa
avid-robinsons-computer-3:housing_app drdavidrobinson$ conda env create -f environment.yml -n housing
Collecting package metadata (repodata.json): done
Solving environment: failed

ResolvePackageNotFound:
  - text-unidecode==1.3=pyhd8ed1ab_1
  - markdown-it-py==3.0.0=pyhd8ed1ab_0
  - starlette==0.33.0=pyhd8ed1ab_0
  - mesa-viz-tornado==0.1.3=pyhd8ed1ab_0
  - websockets==12.0=py39ha09f3b3_0
  - ipyvue==1.10.1=pyhd8ed1ab_0
  - pymdown-extensions==10.5=pyhd8ed1ab_0
  - ca-certificates==2023.11.17=h8857fd0_0
  - types-python-dateutil==2.8.19.14=pyhd8ed1ab_0
  - uvicorn==0.24.0.post1=py39h6e9494a_0
  - watchdog==3.0.0=py39h4cd3b3e_1
  - certifi==2023.11.17=pyhd8ed1ab_0
  - filelock==3.13.1=pyhd8ed1ab_0
  - humanize==4.9.0=pyhd8ed1ab_0
  - rich-click==1.7.2=pyhd8ed1ab_0
  - reacton==1.8.1=pyhd8ed1ab_0
  - ipyvuetify==1.8.10=pyhd8ed1ab_0
  - solara==1.25.0=pyhd8ed1ab_0
  - openssl==3.2.0=hd75f5a5_1
  - watchfiles==0.21.0=py39h3f9c672_0
  

https://www.generationatomic.org/
https://www.facebook.com/generationatomic/

k4robins@uwaterloo.ca


Here is the complete list. ones want to see are pulled to the left margin
Example - Bar cross bottom?  This would be easy to crop out. Latex needs us to label figures in the document not in the figure.

Experiment title. 'adjF': 0.02, 'adjw': 0.05,'discount_rate': 0.07,'r_margin': 0.01, CGT_person':   0.0, # share 0-1,  CGT_investor': 1, # share 0-1

'adjF'
'adjw'
'discount_rate'
'r_margin'
'max_mortgage_share'
'capital_gains_tax_person'
'capital_gains_tax_investor'


        # LABOUR MARKET AND FIRM PARAMETERS
            'subsistence_wage': 40000., # psi
            'init_city_extent': 10.,    # CUT OR CHANGE?
            'seed_population': 400,
            'init_wage_premium_ratio': 0.2, # 1.2, ###

            # PARAMETERS MOST LIKELY TO AFFECT SCALE
            'c': 300                             ### .0,                            ###
            'price_of_output': 10,                 ######
            'density':600,                         #####
            'A': 3000,
            'alpha': 0.18,
            'beta':  0.75,
            'gamma': 0.12, ### reduced from .14
            'overhead': 1.5,
            'mult': 1.2,
            'adjN': 0.15,
            'adjk': 0.10,
            'adjn': 0.15,
'adjF': 0.02,
'adjw': 0.05, 
            'dist': 1, 
            'init_F': 100.0,
            'init_k': 500.0,
            'init_n': 100.0,

            # HOUSING AND MORTGAGE MARKET PARAMETERS
            'mortgage_period': 5.0,       # T, in years
            'working_periods': 40,        # in years
            'savings_rate': 0.3,
'discount_rate': 0.07,        # 1/delta    Check THIS
            'r_prime': 0.05,
'r_margin': 0.01,
            'property_tax_rate': 0.04,     # tau, annual rate, was c
            'housing_services_share': 0.3, # a
            'maintenance_share': 0.2,      # b
'max_mortgage_share': 0.9,
            'ability_to_carry_mortgage': 0.28,
            'wealth_sensitivity': 0.1,
'capital_gains_tax_person':   0.0, # share 0-1
'capital_gains_tax_investor': 1, # share 0-1
 

\section{needed experiments Dec 14 DR}
\begin{enumerate}
    \item capital gains tax matters - find if there is a phase boundary - i.i. where in 

            'capital_gains_tax_person': [0.5, 0.01], # share 0-1

            'capital_gains_tax_investor': [0.5, 0.1], # share 0-1

            does investor share begin to rise (defined as ``for a given number of steps - say 70 - where is $Z=share_t - share_{t-1} \ge 0$? ) and where does it fall.

Maybe just store Z and and draw contour lines - there has to be as program for that.
\end{enumerate}
\section{parameter values form DR}
batch_parameters = {
            'data_collection_period': 1,
            'iterations': 1,
            'max_steps': 200
}

# List variable_parameters for experiments
    experiment_parameters_list = [
         {
            #'density': [800, 600],# 400, 200],
            'subsistence_wage':[40000],#[90000],[ 50000, 40000, 30000]
            #'gamma': [0.1, 0.09, 0.08 ]
            #'discount_rate': [.1, .8, .7, .6, .5]
            #'max_mortgage_share': [1.0, 0.9, 0.8, 0.7, 0.6]
            #'wealth_sensitivity': [0.1, .03, .05]
            #'ability_to_carry_mortgage': [0.35, 0.28, 0.2]
            #'savings_rate': [0.5, 0.3, 0.1]
        },
        # {
        #     'density': [1, 100],
        #     'subsistence_wage': [10000, 30000]
        # },
        # {
        #     'density': [1, 100],
        #     'gamma': [0.001, 0.02, 0.7]
        # }
    ]
    
    # Define fixed parameters, replaced by variable_parameters if there is a conflict
    fixed_parameters = {
                'run_notes': 'Debugging model.',
                'subfolder': None,
                'width':     50, #30,
                'height':    50, #30,
    
                # FLAGS
                'demographics_on': True,  # Set flag to False for debugging to check firm behaviour without demographics or housing market
                'center_city':     False, # Flag for city center in center if True, or bottom corner if False
                # 'random_init_age': False,  # Flag for randomizing initial age. If False, all workers begin at age 0
                'random_init_age': True,  # Flag for randomizing initial age. If False, all workers begin at age 0
    
                # LABOUR MARKET AND FIRM PARAMETERS
                'subsistence_wage': 40000., # psi
                'init_city_extent': 10.,    # CUT OR CHANGE?
                'seed_population': 400,
                'init_wage_premium_ratio': 0.2, # 1.2, ###
    
                # PARAMETERS MOST LIKELY TO AFFECT SCALE
                'c': 300.0,                            ###
                'price_of_output': 10,                 ######
                'density':600,                         #####
                'A': 3000,                             ### 
                'alpha': 0.18,
                'beta':  0.75,
                'gamma': 0.12, ### reduced from .14
                'overhead': 1.5,
                'mult': 1.2,
                'adjN': 0.15,
                'adjk': 0.10,
                'adjn': 0.15,
                'adjF': 0.02,
                'adjw': 0.02, 
                'dist': 1, 
                'init_F': 100.0,
                'init_k': 500.0,
                'init_n': 100.0,
    
                # HOUSING AND MORTGAGE MARKET PARAMETERS
                'mortgage_period': 5.0,       # T, in years
                'working_periods': 40,        # in years
                'savings_rate': 0.3,
                'discount_rate': 0.07,        # 1/delta
                'r_prime': 0.05,
                'r_margin': 0.01,
                'property_tax_rate': 0.04,     # tau, annual rate, was c
                'housing_services_share': 0.3, # a
                'maintenance_share': 0.2,      # b
                'max_mortgage_share': 0.9,
                'ability_to_carry_mortgage': 0.28,
                'wealth_sensitivity': 0.1,
            }


ownership

MPL     n
N       F
extent  k   % I just removed N/f because by def'n N=Fn, I think 
ownership       

Experiments. thinking about this. I am interested in tests to fine-tune the model but let's instead go on to work with the determinants of ownership. 



navigate to the housing_app folder in Terminal. cd Devel/housing_app
source activate housing
python batch_run.py


July 8 DRR

The price a property might sell at would be $\omega - {dc} + \mathbb{A} + rural-op-cost + conversion cost$. For simplicity, we drop the last two terms. Conversion cost would be amortized over a very long period and would be small in any case. We have hidden the rural-op-cost in $a\psi$.


Equation~\ref{eqn-property-investment-return2} implies a `bang-bang' control---with all sales going to the richest participant unless there are limits on the size of capital flows. For our simulation, we implement such limits.  Removed from Ch modle 


---

THIS IS THE CODE THAT I HAVE WORKING
The figure simply shows that both variables behave as expected.
because there is no lag and no partial adjustments it all happens in the first 5 cycles.


Changes are indicated with a training ######

import matplotlib.pyplot as plt

# Initialize variables
A     = 10
gamma =1.2
transport_cost_per_dist = 1
population = 1000
transport_cost_per_dist = 4
density = 10
seed_population = 0
#P=10000
t = []
wage_premium_list = []
population_list   = []
P_list            = []   ######

for time_step in range (10):
    # Calculate wage premium and population #REVERSE ORDER
    wage_premium = gamma*A * population**(gamma-1) #actually this was wage
    population =  2*(wage_premium / transport_cost_per_dist)**2 * density + seed_population
    P = population/100  ######
    
    # wage_premium = gamma*A * population**(gamma-1)#actually this is wage
    # population = 2*(wage_premium / transport_cost_per_dist)**2 * density + seed_population

    # Record output
    t.append(time_step)
    wage_premium_list.append(wage_premium)
    population_list.append(population)
    P_list.append(P)    ############

# Plot wage premium and population against time
import matplotlib.pyplot as plt

# IS THIS THE SINGLE FIGURE   Sept 9 YES
plt.figure(figsize=(10, 6))
plt.plot(t, wage_premium_list, label='Wage Premium')
#plt.plot(t, population_list, label='Population')######
plt.plot(t, P_list, label='Population/100')      ######
plt.ylabel('Value')
plt.xlabel('Time')
plt.title('Population and Wage Premium Over Time')
plt.legend()
plt.show()



 

 ax.set_ylabel(model_out_key, fontsize=20)

 figure 6.1 model logic says we are up in the uppeer block. This is what we need to unpack more. There is a cycle that has to work:

(productivity + wage) -> adjust(wage, n) -> population -> productivity 
 
How can people go up when workers goes down?

why does it start removing agents at step 35


Pushed another change

#add horizontal and vertical reference lines
plt.axhline(y = 0, xmin = 0., color ='black')# zero 
plt.axvline(x = 10, ymin = 0., color ='red')
plt.text(10.5, 400000, r'intitial city boundary', fontsize=15)
plt.text(1, 7000, r'zero rent', fontsize=15)

type
git stash
then 
git pull origin master





"distance_from_center": lambda a: getattr(a, "distance_from_center", None) if isinstance(a, Land) else None,


---

        }
        agent_reporters      = {
            "time_step":         lambda a: a.model.time_step,
            "agent_class":       lambda a: type(a),
            "agent_type":        lambda a: type(a).__name__,
            "id":                lambda a: a.unique_id,
            "x":                 lambda a: a.pos[0],
            "y":                 lambda a: a.pos[1],
            "distance_from_center": lambda a: getattr(a, "distance_from_center", None) if isinstance(a, Land) else None,
            "wage":                 lambda a: getattr(a, "wage", None) if isinstance(a, Land) else None,
            "is_working":           lambda a: None if not isinstance(a, Person) else 1 if a.unique_id in a.workforce.workers else 0,
            # "is_working":        lambda a: getattr(a, "is_working", None),
            "working_period":    lambda a: getattr(a, "working_period", None),
            "property_tax_rate": lambda a: getattr(a, "property_tax_rate", None),
            "net_rent":          lambda a: getattr(a, "net_rent", None) if isinstance(a, Land) else None,
            "warranted_price":   lambda a: getattr(a, "warranted_price", None) if isinstance(a, Land) else None,
        }




KeyError                                  Traceback (most recent call last)
File ~/anaconda/envs/housing/lib/python3.9/site-packages/pandas/core/indexes/base.py:3629, in Index.get_loc(self, key, method, tolerance)
   3628 try:
-> 3629     return self._engine.get_loc(casted_key)
   3630 except KeyError as err:

File ~/anaconda/envs/housing/lib/python3.9/site-packages/pandas/_libs/index.pyx:136, in pandas._libs.index.IndexEngine.get_loc()

File ~/anaconda/envs/housing/lib/python3.9/site-packages/pandas/_libs/index.pyx:163, in pandas._libs.index.IndexEngine.get_loc()

File pandas/_libs/hashtable_class_helper.pxi:5198, in pandas._libs.hashtable.PyObjectHashTable.get_item()

File pandas/_libs/hashtable_class_helper.pxi:5206, in pandas._libs.hashtable.PyObjectHashTable.get_item()

KeyError: 'distance_from_center'

The above exception was the direct cause of the following exception:

KeyError                                  Traceback (most recent call last)
Cell In[120], line 49
     46 axs[0, i].grid(False)
     48 # Create scatter plot for distance from center vs warranted price
---> 49 scatter = axs[1, i].scatter(land_agents['distance_from_center'], land_agents['warranted_price'], c='blue', alpha=0.5)
     50 axs[1, i].set_title(f'Distance vs Warranted Price at Time Step {time_step}')
     51 axs[1, i].set_xlabel('Distance from Center')

File ~/anaconda/envs/housing/lib/python3.9/site-packages/pandas/core/frame.py:3505, in DataFrame.__getitem__(self, key)
   3503 if self.columns.nlevels > 1:
   3504     return self._getitem_multilevel(key)
-> 3505 indexer = self.columns.get_loc(key)
   3506 if is_integer(indexer):
   3507     indexer = [indexer]

File ~/anaconda/envs/housing/lib/python3.9/site-packages/pandas/core/indexes/base.py:3631, in Index.get_loc(self, key, method, tolerance)
   3629     return self._engine.get_loc(casted_key)
   3630 except KeyError as err:
-> 3631     raise KeyError(key) from err
   3632 except TypeError:
   3633     # If we have a listlike key, _check_indexing_error will raise
   3634     #  InvalidIndexError. Otherwise we fall through and re-raise
   3635     #  the TypeError.
   3636     self._check_indexing_error(key)

KeyError: 'distance_from_center'