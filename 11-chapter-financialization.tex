\chapter{Financialization} \label{chapter-financialization}
\epigraph{Financialization  occurs when housing is treated as a commodity—a vehicle for wealth and investment—rather than a social good.}{Special UN Rapporteur on the right to adequate housing, Leilani Farha}

%\epigraph{}{}
%https://www.ohchr.org/en/special-procedures/sr-housing/financialization-housing
%https://www.tni.org/en/publication/financialisation-a-primer#Q5

%*** ADD MORE ON THE LITERATURE ON FINANCIALIZATION TO SETUP OUR WORK, AS IN THE ABOVE CHAPTERS


FINANCIALIZATION VS SUPPLY


This thesis is a study of financialization of the housing market. Financialization is not a new concern. In 2013, Tomaskovic-Devey and Lin wrote that ``The U.S. is now a financialized economy, where the financial sector and its priorities have become increasingly dominant in all aspects of the economy.''\cite{tomaskovic-deveyFinancializationCausesInequality2013}. ***E ADD SOMETHING LIKE: THIS INCLUDES THE HOUSING MARKET. By 2022 Financialization had become a major theme in Canadian  discussions of what is increasingly seen as a national housing crisis. The Ontario Housing Affordability Task Force reported that ``This has home ownership beyond the reach of most first-time buyers across the province, ... Housing has become too expensive for rental units and ..  in rural communities and small towns. The system is not working as it should.''   Aled ab Iorwerth, Deputy Chief Economist at CMHC argued
\begin{quotation}
     “… Canada’s approach to housing supply needs to be rethought and done differently. There must be a drastic transformation of the housing sector, including government policies and processes, and an ‘all-hands-on-deck’ approach to increasing the supply of housing to meet demand.”\cite{CanadaHousingSupply2022}
\end{quotation}



Financialization is a term %***E WOULD CUT SOMETIMES 
sometimes used to describe the development of financial capitalism during the period from 1970 to present, in which debt-to-equity ratios increased and financial services accounted for an increasing share of national income relative to other sectors. % ***E WOULD FILL OUR WHAT IT IS - DEFINITION AND EFFECTS - MORE BEFORE GOING INTO WHEN IT STARTED. 

While there is no clear point in time when global financialization began, Thomson and Dutta,  \cite{thomsonFinancialisationPrimer2018}, suggest that  15 August 1971, when  President Richard Nixon announced that the United States would unpeg the dollar from gold, marks a break point. The accelerated growth in global liquidity and prompted a surge of financial liberalisation and deregulation and undermined the Bretton Woods System.  Synthetic derivatives were created soon after: The Chicago Mercantile Exchange launched futures contracts written on financial instruments the following year and the Chicago Board of Trade introduced the first interest rate future contracts three years later. Arbitrage, options trading, and various other activities grew exponentially. By 2011, the over-the-counter (OTC) and exchange-traded derivatives market amounted to almost \$800 trillion.  %***tHIS IS  VERY NICE, CLEAR SUMMARY, BUT IT WOULD HELP TO MAKE MORE CLER WHAT THESE THINGS MEAN. 

%***E UNLESS THE FOLLOWING PARRAGRAPH IS A REALY DISTINCT DEFINTION, i WOULD MOVE THIS UP ABOVE THE TIME LINE. 
Financialiation is also described as ``a process whereby financial markets, financial institutions, and financial elites gain greater influence over economic policy and economic outcomes. Financialization transforms the functioning of economic systems at both the macro and micro levels''.\cite{palleyFinancializationWhatIt2007}. 

Financialization in our analysis market is the increasing control of the stock of urban land and housing by investors in order to capture the scarcity rent generated by the people of the city.  

% ***E MAYBE MOVE THIS UP TO INTRODUCE THE DIFFERENT TERMS i THINK THIS CLARIFIES THE QUESTION i HAVE ABOUT HOW THE DIFFERENT MEANINGS FIT TOGETHER
The term financialization actually has different meanings at different levels of analysis.  In finance, it is a term for the process of developing the legal instruments that facilitate financial transaction.  At the microeconomic level it is a collective noun for the growth in the number of individual transactions that create or transfer financial assets. At a macro level it refers to the effect on the system itself of the new instruments and the increase in the type of transaction that they make possible. It is the macro effects that are our main interest.

We therefore begin in Section~\ref{section-financialize}  with a definition of the act of financializing transactions and markets,  and discuss some significant examples. To model the effects  financializing housing   an \gls{ABM} we are forced to carefully specify the microeconomics of  individual decisions. In Section~\ref{section-micro} we describe the microeconomic objective function that individuals and our bank agent employ in deciding to purchase a property. Finally, in  Section~\ref{section-system} we explain how financialization might affect the housing market as a system and some consequences for society in general. At that point we can introduce our specific hypotheses and how we intend to test them.

%We  focused on modelling the  effects of financialization on and through the urban housing market to \textbf{WHY?}. 

%This is where the general process of financialization At that point we can introduce our specific hypotheses and how we intend to test them.
% We therefore begin with a narrow and strict definition of the act of finacializing, followed by  an explanation and examples. We then go on to explain how the term is applied at the level of systems and while markets.
% At that point we can introduce our specific hypotheses and how we intend to test them.

\section{Financial instruments} \label{section-financialize}
The word ``financialization'' has several quite different meanings.  

%*YOU SAY ABOVE YOU ARE GOING TO NARROW TO A SPECIFC DEFINITION, CAN YOU MAKE VERY CLEAR HERE WHAT THAT IS. LIKE THESE ARE ALL POSSIBLE DEFINITIONS. THEN STATE VERY EXPLICITLY, THIS IS THE ONE WE ARE USING. 

To financialize anything is to create a  \gls{financial instrument} that represents it and can be bought and sold as an investment. Not all financial instruments are instruments of financialization ***E MAYBE ADD IN WHAT SENSE... %like in the sense we mean, in the macro sense, in the technical sense of the word.. 
For example, mortgages are  financial instruments, but another step is required to create a financialized  market in mortgages. ***E SAY WHAT THE STEP IS (EG. The additional step of trading these mortgages on a market is required to create a financialized market. 

%*E i'M A BIT CONFUSED HERE... i THINK YOU ARE GOING INTO EXPLAINING HOW THEY BECAME FINANCIALIZED, BUT IT'S SET UP LIKE AND EXAMPLE... CAN YOU SAY WHERE YOU ARE GOING BEFORE OPENING THE DETAILS ABOUT THE MORTGAGE. lIKE THEY BECAME FINACICLAIZED OVER TIME... AT A BASI LEVEL A MORTGAGE MAKES POSSIBLE ETC..

\subsection{Mortgages}
Mortgages, for example, are a financial instrument that allows lenders to  participate in housing purchases in the present in exchange for a future flow of payments.  The mortgage does not create housing, but it enables the prospective buyer to become the nominal owner of an asset that produces a stream of benefits. The stream of benefits from secure housing near a source of income generally exceeds a buyer's current assets. The mortgage enables the  transfer of ownership because it makes it possible to transfer the rights to a substantial fraction of the future income of the buyer to the mortgage holder who, in effect, is the owner until the terms of the mortgage are fulfilled.  If the mortgagee fails to make those payments the mortgage holder repossess the asset. The security provided by this instrument ``de-risks'' the transaction, making it safer and therefore easier to achieve the mutual benefits of the sale.

Originally mortgages were an arrangement between just the buyer and the seller, but in the 1870s in the USA, insurance companies stepped into these transactions, paying off the seller and then collecting the principle and interest payments on behalf of the bank's shareholders. This innovation was an example of financialization that made it possible for individual insurance companies to profit by providing money to buyers. When mortgage lending was regulated and insured  by the American government during and after the great depression it facilitated a massive increase in home ownership in the USA and contributed to the post-war construction boom and the suburbanization of American cities. 

These mortgages were still agreements between individual lenders and buyers. When financial institutions developed markets that let them buy and sell bundles of mortgages among themselves, we say the mortgage market itself was financialized. There was then a market for the  promises to pay in addition to the underlying market for housing and the market for mortgage-type loans loans

The transactions in this market do not directly affect the mortgage conditions or the home: they simply add a new product for investors or banks to buy and sell. This new market was thought to further reduce the risk for lenders, but  between 2007 and 2010 in the USA the sub-prime mortgage crisis destabilized the financial institutions that were playing this new market. \footnote{Arguably, that crisis resulted from overselling risky and predatory mortgages by lenders with more money to lend than the market could absorb. Eager investors seeking higher or more secure returns were willing  to buy bundles of  mortgages that were in theory de-risked. Pooling uncorrelated individual risk, as the insurance industry does, produces lower overall risk. (The instruments that enabled the speculative bubble were mortgage-backed securities (MBSes) and collateralized debt obligations (CDOs).) Pooling does not reduce systemic risk, however. As it turned out, the mortgage default rate rose, lenders' liquidity fell, mortgage rates were pushed up, defaults increased, and the market for the new instruments collapsed, taking down major financial institutions. The details don't matter for this thesis, but they provide a cautionary tale about the the potential costs of financialization.}.  

Sixty-six percent of Canadian homes are owner-occupied and about a third of the value of the homes is held as mortgages. Approximately two-thirds of the net land rents associated with housing, therefore, accrue to owner-occupiers \cite{nemtinFinancializationHousingSocial2021}. {\color {red}CHECK THESE NUMBERS AND IF THEY CITE A BETTER SOURCE WE SHOULD GET} 

The mortgage demonstrates the two aspects of financial instruments. It is both a financial instrument that enables  purchase and a financial asset that can be bought and sold. When we consider urban housing, it is the right to future income generated by capital, labour, and the city itself through the agglomeration effects that drive productivity. It is an instrument that indirectly captures a share of the urban rents. As productivity rises, wages rise, rents rise, property prices rise and mortgages rise. 

For urban theory and policy formation it is important to distinguish between financial instruments that enable production of real assets, and instruments like  mortgages that primarily facilitate the transfer of real assets or rights to real estate  income. Housing developers borrow to purchase land for development and builders borrow to finance construction. While important, the financial instruments involved are not driving the financialization of housing.  The size of the loans involved is affected by the amount of land purchased and the potential rents earned by that land, but the degree of non-occupant ownership is not affected. *** CLARIFY

%*E IF YOU SAY IT WAS THOUGHT TO LOWER RISK, CAN YOU FINISH THAT THOUGHT. %DID THE CRISIS SHOW IT DIDN'T? OR IS IT UNCLEAR WHETHER IT DOES? iS IT A DEBATED POINT? OR CHEANGE THE SENTENCE SO YOU DON'T SET UP IT WAS THOUGHT AND LEAVE THE CONCLUSION HANGING...

\subsection{Stocks and stock markets}
The \gls{joint stock company}, the basis of the modern stock market and an important mechanism for channeling investment  into productive activity,  is probably the most important financial instrument in the capitalist economy. Originally a tool to allow a group of investors to pool their money, take on large projects, and to share risks, the stocks themselves rapidly became objects of trade and speculation. Share prices on the stock market are not tightly tied to the productivity of the company they represent, which makes it clear that the financial instrument is something different from the real asset. 

Stock markets are  generally described as a primary means of efficiently mobilizing long term savings and investment for  fixed capital formation \cite{azfarMarketMobilizedCapital2003}. Most stock transactions however are speculative in nature and there is significant disagreement about the link between stock investment and investment in productive real assets \footnote{Mork et. al \cite{morckStockMarketInvestment1990} identify four theories that attempt to explain the correlation between stock returns and subsequent investment \begin{quotation}The first says that the stock market is a passive predictor of future activity that managers do not rely on to make investment decisions. The second theory says that, in making investment decisions, managers rely on the stock market as a source of information, which may or may not be correct about future fundamentals. The third theory, which is perhaps the most common view of the stock markets influence, says that the stock market affects investment through its influence on the cost of funds and external financing. Finally, the fourth theory says that the stock market exerts pressure on investment quite aside from its informational and financing role, because managers have to cater to investors' opinions in order to protect their livelihood. For example, a low stock price may increase the probability of a takeover or a forced removal of top management. If the market is pessimistic about the firm's profitability, top management may be deterred from investing heavily by the prospect of further erosion in the stock price.\end{quotation} None of the point to a direct connection between stock market investment and real investment.}.  Mutual funds, which pool risky stocks, are a %financial instrument built on top of the stock market.


\subsection{Investment trusts}
An example of a financial instrument designed specifically to support rent extraction and which  increases the degree of financialization of the housing supply is the  Real Estate Investment Trust (REIT).  A REIT is a company that owns, operates, or finances income-generating real estate and distribute the income to shareholders. While the company itself is an management system, the shares are are simply financial assets that are that are sold to individuals and organizations that want to share in real estate revenues and capital gains. There are other large owners of residential real estate such as life insurance companies and pension plans that behave similarly, but REITs are a relatively new financial instrument which is  expanding rapidly and attracting political attention for their effect on housing markets.  % REITs that develop new properties generally don't sell the properties they construct.

REITs have become increasingly popular in recent years.  An S\&P-Dow Jones research bulletin reported that over the  25 years to 2017, REITs outperformed stocks, bonds, and commodities \cite{GET-Dow-Jones-research-bulletin}. %***E COULD YOU SAY THE RELATIONSHIP IN A WAY THE CLARIFIES THE RELEVANCE OF THIS NEXT SENTENCE? % ***E I MEAN LIKE IS THIS A SIDE EFFECT. IT FEEL LIKES THIS IS A BIT OF A THROWAWAY STATEMENT THAT MIGHT HAVE MORE TO IT? 
Because they have outperformed competing investments they have attracted  capital from other uses, particularly during the recent period of low interest rates.

Developed in the USA  in 1960 (as an amendment to the Cigar Excise Tax Extension!) and in Canada in 1993 \cite{GET_REITsDevelopedDates}, REITs are similar to mutual funds in making it possible for an individual, often small investors to earn dividends from real estate investments without having to buy, manage, or finance any properties themselves. 
***E YOU MIGHT MAKE A SPECIFIC SECTION WITH THE PARAGRAPHS BELOW ABOUT CRISTICISM/OR QUESTIONS OR DEBATE ABOUT REITS % i FEEL LIKE THIS IS SEPARATE AND MAKE COME ACROSS AS CRITICIZIG, BUT IF YOU MAKE A SECTION THEN IT'S JUST SUMMARIZAING THE CONVERSATION HAPPENING. SEEMS MORE NEUTRAL
\footnote{There are questions about preferential tax treatment, and whether some REITs are really just inappropriately sheltered real-estate corporations. } For the purpose of this thesis, REITs are simply one of the mechanisms for the financialization of housing.

They are not simply a neutral tool for saving however. According to a paper \cite{wangAnalyzeImpactREITs2021} on REITS in the Irish housing market, ``REIT successfully reconnected the international financial market and the Irish real estate market.'' In other words, in Ireland, REITS have made it easier for international capital to buy Irish land. The entry of outside and footloose capital has had an effect on the resident population:  ``the large-scale acquisition of Irish real estate by REITs and other real estate buyers has also caused some new problems. First, the active management of assets by REITs and other investors has led to a rapid increase in rents''.\footnote{In  IMF working paper ``Capital Account Liberalization and Inequality'' \cite{furceriCapitalAccountLiberalization2015}  Furceri and Loungani reported that that for 149 countries from 1970 to 2010, ``after countries take steps to open their capital account, an increase in inequality in incomes within countries follows'' . The observation is consistent with our argument  that domestic rent-seeking in housing markets will increase inequality.}   

There is evidence that REITs affect real estate markets in other ways. Bat et all  \cite{batRolePublicREITs2022} reprot that  ``they are actually financial actors that aggressively buy up property assets and manage them to extract wealth at taxpayers’ expense. '' and ``they have expanded the pool of capital available for transactions that monetize real property and turn it into tradable assets – financial widgets with little or no connection to the real purpose of the productive enterprises that occupy the properties they own.''

\subsection{A comment on financial instruments and financialization}
We have called attention in this section to the development  several important financial instruments -- mortgages, stocks, and REITs --  with the intention of  disentangling the concept of financial instruments from the term ``financialization''.  We say that financialization is occurring if there is an increase the stock of financial assets associated with a stock of real assets. The development of new instruments facilitates this process. 

\section{The microeconomics of  financialization}\label{section-micro}
At  the microeconomic level, where individual decisions are made, financialization happens,  real assets are bought by individuals and institutions. Individual home-buyers are buying a stream of housing services and possible capital gains. Institutional buyers are buying a stream of net rents plus speculative gains. Each agent has their own interest rates, discount rates, mortgage share, information, and expectations, so individual bids can differ.

%In the process, the real asset takes on an additional and separate aspect as financial object that can be bought and sold. 

The feature that matters most to investors is the  \gls{rate of return} that  an asset offers. For any level of risk and liquidity, an investor will choose to purchase the asset with the highest rate of return. When a home is bought to live in, the potential buyer makes a similar calculation, perhaps with a greater emphasis on the value of the stream of housing services. When home prices are rising rapidly speculative gains may become the dominant concern of home buyers. 

To model the investment decision  we therefore need to calculate the \gls{rate of return} on each property taking into account the rental revenue, the potential capital gain, the investors' costs of capital, incomes, and assets. The analysis will show the incentive structure driving the  financialization of the housing market. We do this in detail in Appendix~\ref{appendix-bid-price} 

We first calculate the net present value of the purchase, then divide by the amount of capital employed, which we assume is simply the size of the down payment made at the beginning of the period. This gives us a rate of return.\footnote{A common approach would be to calculate an internal rate of return (IRR), but  the IRR is in general the solution to a polynomial and does not guarantee a single-valued result.\cite{robinsonOPTIMALTERMINATIONIRR1996} Multiple real-valued  IRRs may arise;  complex-valued IRRs may arise;  the IRR is, in general, incompatible with the net present value (NPV) in accept/reject decisions; the IRR ranking is, in general, different from the NPV ranking; the IRR criterion is not applicable with variable costs of capital. Ways to salvage the IRR as a usable criterion have been proposed that are consistent with our approach \cite{magniAverageInternalRate2010}, which is to calculate an NPV then convert it to a rate,} 

\begin{eqnarray}
Rate\ of\ return\ on\ capital\ invested = \frac{\delta \left((1+ \dot P_M^e - (1+r)m\right)}{1-m} + \frac{\mathcal{R}_N}{(1-m)P_B}\label{equation-RoR}
\end{eqnarray}
where 

\begin{tabular}{lll}
 $\delta$       &=& individual discount rate \\
$\dot P_M^e $   &=& expected rate of price increase \\
$ r$            &=& mortgage interest rate \\
$m$             &=&  fraction of the price that can be mortgaged \\
$\mathcal{R}_N$ &=&  net  market rent
\end{tabular}
Equation~\ref{equation-RoR} makes it clear that the estimated rate of return depends on subjective magnitudes,$\delta$ and $\dot P_M^e$, attributes of the property, $ \mathcal{R}_N$, and  individual financial position, $r$, $m$.

The individual's Rate of return on capital  invested  for each project  is compared to the investor's target rate, $r_{target}$, providing  an investment criterion:
\begin{eqnarray}
r_{target} \le \frac{\delta \left((1+ \dot P_M^e - (1+r)m\right)}{1-m} + \frac{\mathcal{R}_N}{(1-m)P_B}
\end{eqnarray}
In Appendix~\ref{appendix-bid-price} this expression is solved for $P_B$, the  maximum price that allows the investor can pay and still achieve at least her  target rate of return, $r_{target}$.  

\begin{eqnarray}
P_B & \le    \frac{\mathcal{R}_N}{(1-m)r^{target}-\delta \left(1 + \dot P_M^e - (1+r)m\right)} \label{equation-Bidprice}\end{eqnarray}
% P_B & \le    \frac{\mathcal{R}_N}{(1-m)r^{target}-\left[ \delta(1+L(P)- (1+r)m\right]}
We call this  $i's$ maximum bid and compute it for all potential buyers. In each sale the highest $P_B$ will make the purchase. The denominator can be seen as an adjusted rate of return for capitalizing net rents, analogous to the value of $r$ in  the standard capitalization formula.

The market mechanism then simply has to compare the bid price  $P_B$ with the seller's reservation price and apply a bargaining rule to determine how any surplus is allocated. .% unless there are limits on the size of capital flows. For our simulation, we implement such limits. 

Since  agents do not have perfect information, the calculation is done with their best approximation of values. % The agent does not know the future. 
 The rate of price growth $\dot P$, is an approximation based on rents and past market behaviour.\footnote{Case and Shiller ``..we see a market largely driven by expectations. People seem to form their expectations from past price movements rather than having any knowledge of fundamentals. This means that housing price booms will persist as home buyers become destabilizing speculators.''Case and Schiller, \cite{caseThereBubbleHousing2003}} Details of the derivations, and implementation, are discussed in the Appendix. % We calculate the bid price in Appendix \ref{appendix-bid-price}.

%Thus, the 1-year expectations are fairly well described as attenuated versions of lagged actual 1-year price changes, Case and Schiller, \cite{caseThereBubbleHousing2003} p282

%  GET?? Case, Karl E., and Robert J. Shiller. 1988. “The Behavior of Home Buyers in Boom and Postboom Markets.” New England Economic Review (November– December), pp. 29–46.

\subsection{Implications of the bidding rule}
\begin{enumerate}
\item A large $m$ magnifies the return. (The downpayment is smaller as a fraction of the price, increasing the investor's leverage). 
Given the  common rule that mortgage payments cannot exceed some fraction of disposable income, the wealthy will be able to borrow larger amounts and at lower interest rates than the less wealthy. At any distance from the centre they will be able to make a higher bid.

\item A lower mortgage interest rate increases the return by lowering interest payments. The cost of capital is known to differ for rich and poor.  The wealthy can generally borrow  at lower interest rates than the less wealthy. 

\item A lower discount rate $\delta$ reduces the subjective rate of return.  Poverty in assets and cash liquidity constraints are correlated with higher rates of time preference  \cite{carvalhoPovertyTimePreference2010}\cite{holdenPovertyMarketImperfections1998}. If agents discount at their borrowing rate, wealthier agents may have a lower subjective rate of time preference and therefore value properties more highly. 

\item Higher expected price appreciation increases the attractiveness of an investment. Financial corporations and the wealthy are likely to have better price forecasts than  the occasional home buyer.

\item Higher rents make the unit more profitable. Higher expected  rents may result from expecting greater price appreciation  leading to raising rents for tenants. Lower discount rates may give future rent increases greater present value.

\item Lower maintenance costs increase profits. There may be scale economies in the maintenance  of rented housing. 

\item Lower tax rates decrease holding costs and increase the value of the investment. There may be opportunities to shelter income with land held for investment (speculative) purposes. Tax treatment of income and capital gains as well as interest deductibility may also provide advantyages for institutional buyers and investors.%\footnote{Case and Schiller \cite{LOST_CaseandSchiller} observe that (source?) `` ... increases in real per capita income all are positively related to excess returns or price changes over the subsequent year.''} 
\end{enumerate}

Some  of these conditions (1-3) hold generally for wealthier actors. Others (4-7) may be available only to institutional investors.  Financial corporations in particular may have advantages relative to individual investors, making it  reasonable to expect that financial corporations increasingly dominate urban land 
markets.\footnote{Fr\'ed\'erick Demers \cite{demersModellingForecastingHousing2005} found that the response of housing investment to interest rates has become more pronounced over time. This suggests a rising share of financial investors relative to buyers focused on housing services. Case and Schiller \cite{caseThereBubbleHousing2003} observe that `` ... increases in real per capita income all are positively related to excess returns or price changes over the subsequent year.''}  

Since interest rates are lower for those with higher wealth, the analysis implies, consistent with the empirical evidence, that net returns for investment are increasing with wealth. Large wealth holders will get higher expected and actual rates of return on land than those with lower wealth holdings. Managers of large pools of capital will have an even greater   advantage. Overall, Equation~\ref{equation-Bidprice} implies  sales generally go to the richest participant.
 
%  \footnote{Case and Schiller \cite{LOST_CaseandSchiller} observe that (source?) 
%  `` ... increases in real per capita income all are positively related to excess returns or price changes over the subsequent year.''} 

% The conclusion that we draw from the analysis above is that  financialization of urban housing benefits a rentier class of urban landholders. There is evidence that it benefits a globally distributed class of rentiers.  

\section{Financialization as system change}\label{section-system}
We established that, at least in theory,  financial institutions and the wealthy are likely to own increasing shares of the housing stock. the theoretical conclusions is consistent with what has already happened in the Canadian Housing market. Recent data from Statistics Canada \cite{fontaineResidentialRealEstate2023} suggests people who own more than one property in Ontario make up more than 25\% of buyers in the province. (The proportion of investors among owners varied from 20.2\% in Ontario to 31.5\% in Nova Scotia.)
Just under one in five properties overall was used as an investment.
In Ontario 41.9\% of condominium apartments are investment properties.\cite{statisticscanadaBuyRentHousing2022}

The immediate social implications are fairly obvious. As Statistics Canada points out, these trends might limit the number of properties available to buyers who intend to use it as a primary place of residence. \cite{fontaineResidentialRealEstate2023}. Statistics Canada reports that latest census release, two-thirds of Canadians owned a home in 2021, down from a peak of 69 per cent a decade earlier. The decline is was higher for younger members of the population. 

When the homeownership rate goes down, the rental rate goes up. The 2021 Canadian Housing Survey reported that the number of renter households increased  at over twice the pace of owner households, pushing down the homeownership rate in Canada. If the trends continue, Urban Canada will gradually change from a society dominated by homeowners to a predominantly tenant society. Since wealthier buyers are advantaged in the market, the younger and poorer parts of the community will be increasingly excluded from ownership. Financialization will increase income and asset inequality in cities.

Combined with rising housing prices the effect will be to squeeze lower-wage households closer to what we have termed to subsistence level and make it harder for low-wage workers to live in the city. The city requires low-wage workers for many of the services, so labour shortages are a possibility. Labour shortages will squeeze some activities out of the city and are likely to reduce productivity. Labour shortages may push up wages, but in rental markets, landlords can capture much of any increase in wages. 

The incentive structure in our model was derived purely from the point of view of an individual investor. Examination shows that investment incentives favour the wealthy and institutional buyers, but that does not necessarily imply that the process of financialization will drive social transformations. Individual choices are at most  a link in the chain. Modelling  allows us to identify which parameters are most influential. 

A question that is especially important is whether the process of financialization and tenentization our micro model suggests is reversible:  high levels of home ownership we have seen throughout the 20$^{th}$ century, as Purdy \cite{purdyPropertyOwningDemocracyHome1993} suggests, may be ``a transitory phenomenon of the 20th century.''

% .#https://www.jstor.org/stable/j.ctt80wdt Housing the North American City
% MICHAEL DOUCET
% JOHN WEAVER
% Copyright Date: 1991
% Published by: McGill-Queen's University Press
% Pages: 608
% https://www.jstor.org/stable/j.ctt80wdt

 %how financialization might affect  the housing market as a system and some consequences for society in general. At that point we can introduce our specific hypotheses and how we intend to test them.
%***E IF YOU WANT TO INTRODUCE ANEW USE OF THE TERM, CONTEXTUALIZE IN TERMS OF HOW YOU ARE USING THE WORD. IS THIS AN EXTENSION OF HOW YOU USE IT? THIS SEEMS RELATED TO THE MACRO VERSION YOU MENTION ABOVE. MAKE THE RELATIONSHIP CLEAR. I ALSO WONDER IF THIS WOULD BE USEFUL TO MOVE UP. IT FEELS LIKE IT MAKE BELONG WITH THE BROADER CONTEXT AT TEH BEGINNING OF THE CHAPTER? GOOD IDEA I will try it

*e tHIS IS CLEAR BUT COULD BE HELPED BY RESTATING HOW THEY MOVE IT ONTO FINANCIAL MARKETS. JUST A QUICK SUMMARY STATEMENT OF HOW THEY ARE FINACIALIZATION. %EVEN JUST something like... They put the ownership of housing onto financial markets. just to keep us oriented in what we are talkign about.

*E I FEEL LIKE THERE IS SOMETHING MISSING BETWEEN THESE TWO PARGRAPHS. %PERHAPS JUST FLESHING OUT WHAT FINANCIALIZATION LOOKS LIKE TECHNICALLY. MAYBE ALSO INSTRODUCE POSIBLE EFFECTS... LIKE EVEN JUST INTRODUCE THEM AS QUESTIONS? IT'S BEEN SUGGESTED OR SHOWN THAT ITS CONTRIBUTING TO THE HOUSING CRISIS. tHIS WOULD ALSO BE A GOOD PLACE TO EXAPLIN WHAT YOU MEAN BY "aS SYSTEMS CHANGE... BECAUSE i THINK THAT IS A BIG PART OF WHY YOU SAY NEXT THAT IT NEEDS TO BE UNDERSTOOD... BECAUSE IT HAS SUC BROAD EFFECTS

We need to understand the economics of financialization.
% \section{Literature on theory and evidence} % PROVIDE EVIDENCE 	mention theories?
There is substantial evidence that the financialization of urban housing is underway in Canadian cities..

Two questions arise when we observe the growing participation of global capital in the urban housing system: 
\begin{enumerate}
\item How far will the financialization of urban land go? 
\item That are the implications for the urban economy and the welfare of the urban population? 
\end{enumerate}

We can demonstrate that in the absence of policy interventions, differential access to finance capital ensures that capital owners acquire an increasing share of urban land % over time
and therefore capture the growing land rents from urban productivity growth. 

With this insight, growing wealth inequality emerges within a simple, widely accepted model of the urban land market. In the limit, urban residents are tenants, and new residents without capital no longer receive any of the increases in rents arising from the growing productivity of the city. 

%The first question, therefore, is reduced to which capital holders will increase their share of urban land and whether there is any reason to expect the process of financialization process to stop or reverse itself.

% \section{The incentives for financialization}
%Instead, drawing on the ideas of Jane Jacobs, Lucas proposes the city as the unit of analysis. Lucas, Robert (1990), "Why Doesn't Capital Flow from Rich to Poor Countries?," American Economic Review Papers and Proceedings v. 80, no. 2 (May) pp. 92-96.  
%Jacobs, Jane  (1969), The Economy of Cities (New York: Random House).  
% The Death and Life of Great American Cities \cite{jacobsDeathLifeGreat1961}

MOVE The mortgage share and interest rate are functions of the agents wealth %Both the  share of the price  that can be mortgaged, $m$, and the interest rate and the interest rate paid, $r$, are functions of the agent's wealth. 
The discounting factor may be correlated with wealth as well. 

%%%%%%%%  VVVVVVVVVVVVVVVVVVVVVVV   This section May 18 to cut?  V
%%%%%%%%  ^^^^^^^^^^^^^^^^^^^^^^^   This section May 18 to cut?  V
% TODO - add interest rate discussion - (borrowing rates drive land prices up, even if there is no development or improvements, simply because it makes it worth a larger--the effect of low rates, especially for institutional actors have driven a large effect)
%\begin{enumerate}
%
%\item  the buyer and seller calculate the value of the property  differently. 
%
%\item  the  buyer and seller may have different expectations of the path of prices and therefore the stream of rents.
%%There are two standard ways that expectations are modeled
%%	\begin{enumerate}
%%	\item \textbf{Adaptive expectations.} Expectations are largely based on what has happened in the past. 
%%	Under normal conditions most people  have relatively weak incentive to get forecasts about inflation correct and lack the resources and time to purchase expert advice. 
%%	Recent price trends are easily available and likely to be the main source of  information.
%%	\item     \textbf{Rational expectations.} Expectations are based on a model of the future economy. 
%%International investors and banks employ economists and other experts to  forecast prices, exchange rates, and trends in the economy.
%%	\end{enumerate} 
%\end{enumerate}
% Why would  discount rates differ between identical workers? Buyers and sellers are not identical in wealth, . 
%%We could implement the first  explanation either by generating expectational errors based on functional class or wealth. 

\subsection{Financialization and productivity}





When  a productive asset is acquired as a financial asset it remains productive.  The financial instrument is separate form the real asset, at least in principle, and is traded in different markets. Why then is financialization an issue?  Theory suggests that financialization is positive.  The major argument is that finacialization enables real investment. In theory then, financialization of the housing market therefore  to more housing production. It is striking that after 40 years of growing financialization across we face a housing crisis , increasing homelessness and falling rates of home ownership.  Palley \cite{palleyFinancializationWhatIt2007} reproits that 

\begin{quotation}At the macroeconomic level the era of financialization has been associated with generally tepid economic growth.\dots  In all countries except the U.K., average annual growth fell during the era of financialization that set in after 1979. Additionally, growth also appears to show a slowing trend so that growth in the 1980s was higher than in the 1990s, which in turn was higher than in the 2000s. \end{quotation}

% African land or land in Northern Ontario 
% Land acquired by holding companies may even be made more productive. The theory is that 
% the goal of such investments, however, is generally to achieve a capital gain over time. Financial analysis is essentially about rates of return on financial capital invested. The opportunity for capital gains  attracts financial capital to the housing market.%Financial managers have no interest is n in assets that are not expected to increase in value. 

\section{Financialization in a modern urban theory of rent}

To summarize, there are these three meanings 
% - 3 meanings of financialization
% - here's what we need to know about this for this theory which is a 
% - how it connects to what's next - it actually requires us to build a modern theory of rent for xyz reasons.

Financialization of the housing market is about capturing the surplus in an urban economy so we next go to the theory of rent which is an theory accounting for distribution of surplus. 

A large part of the surplus appears as locational rents, so we then go to developing the spatial and urban model in which rent operates.  

Since we are looking at how financialization works to claim the urban productivity premium or value created in cities, we also need to account for how value is produced in cities, so we then introduce theories of how productivity scales in the urban context.

%because a key feature of this model is that it formally captures how finalization works to capture the  
%financialization is aimed at capturing the 
%rents generated by growth and specifically urban growth..

%WHY GROWTH HERE.
%In Chapter \ref{chapter-growth}, we integrate modern growth theory into our urban theory of rent.
%Together these pieces can be formalized as a theory of urban rent, that is a theory that captures the dominant dynamics of financialization described above.  We develop in the second part of this thesis, Part \ref{part-model}, on the model. 


This chapter has discussed the theory of financialization. In the following chapters we will discuss the three pieces of theory needed to build a model of financialization, rent, urban spatial models, and growth theories. From there, we will bring these things together into a formal model in Part \ref{part-model}.