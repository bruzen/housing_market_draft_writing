%==================================================================
\chapter{Methodology} \label{chapter-methodology}
%==================================================================
\epigraph{The choice, then, is not whether to build models; it's whether to build explicit ones.}{Joshua M. Epstein \cite{epsteinWhyModel2008}}
%\epigraph{One of the very important components in the urban and agricultural land use model is the so-called \gls{bid-rent curve}. Regional and urban economists, city planners, and economic geographers have used this curve extensively as an analytical device.}{Yeung-Nan Shieh \cite{shiehWilhelmLaunhardtBidRent2004}}

% CITE DEREKS PIECE ON MODELS AS BUILDING BLOCKS---MODELS AS eAGOOD PARTS---CITE TERRY'S THESIS---SHOULD DO WHAT IT DOES HOW IT DOES, BUT ABILITY TO COARSE GRAIN- ACTUALLY SWITCH BETWEEN THESE TWO PIECES-  WITH AN EQUILIBRIUM ASSUMPTION AS THE BRIDGE.

% \epigraph{One of the very important components in the urban and agricultural land use model is the so-called \gls{bid-rent curve}. Regional and urban economists, city planners, and economic geographers have used this curve extensively as an analytical device.}{Yeung-Nan Shieh \cite{shiehWilhelmLaunhardtBidRent2004}}
 % \epigraph{It may be tempting to specify an aggregate production function that directly relates primary factors to final output, as is customary in much economic analysis. This standard simplification is often inadequate, however, because cities are characterized by increasing returns to scale, and how psuch increasing returns are generated has potentially important policy implications. In particular, detailed assumptions are needed about rlabor, the nature of products, the production function of individual firms, the input-output structure that links firms, and how firms compete.}{Spence et al. \cite{spenceUrbanizationGrowth2009}}
 % TODO ADD THE PURPOSES OF MODELLING NOTE TO eDISCUSSION OF HOW  WE ARE USING ABMS

% SUMMARIZE: WE ARE MODELLING THE RELATIONSHIP BETWEEN FINANCIALIZATION AS sentsGRAWTH VALUE CAPTURES AND PRODUCTION IN CITIES TO DO THIS WE  BUILD ABM

% ROOTED IN NEOCLASSICAL AND CLASICAL (MOVE THAT SECTION UP??) 

% ABM ADDS WHAT?? GO INTO HISTORY OF ABM AND THEN HOW YOU USE IT TO MODEL THAT ASPECTS YOU HAVE INTRODUCED IN EARLIAR SECTIONS


So far, we have introduced the theoretical frameworks for understanding locational value, the structure of the city and how productivity scales with wealth. In the following chapters, we introduce a model that brings these together to explore the effects of financialization on the urban system. In this chapter, we consider some of the broader methodological issues involved. Section~\ref{sec-purpose} considers the general purpose and strategy of modelling. %: what is the purpose of modelling? what claims does it allow about the outside world? and what is the modelling strategy?
Section~\ref{sec-types} explains what type of model we use and why,  describing how we integrate agent-based modelling and equilibrium approaches in the model.   % in particular what is the underlying paradigm for modelling each component, what are the agents, what simplifications are justified, what uses are made of equilibrium reasoning? 
% Section~\ref{sec-other} describes our approach to several methodological matters specific to our project and the final section does what?

\section{Purpose and strategy}\label{sec-purpose}
% The goal of modelling in general is to have an accessible and useful description of reality.  

The purpose of the model is its goal, and the strategy is the means by which it seeks to achieve its goal. To define our purpose and strategy, we are guided by the guidelines laid out by Mader et al. for the properties of a good model \cite{maderConstructionVerificationModels2007}. % They emphasize that it should have a clearly specified object of modelling and a clearly specified purpose. They go on to 
% They list the following epistemological criteria, 
They argue that a model should be: 
\begin{enumerate}   
\item  Truthful. The model has to represent the relevant behaviour of the system. % we are describing.
\item  Complete. It must include necessary features of the system.
\item Simple. It must be possible to verify and debug the model.
\item Understandable. Elements should be clearly described or derived and documented.
\item  Traceable.  It should be obvious what elements of the artifact went into which design decision and are reflected at which point in the model.
\item  And finally, efficiently constructable and maintainable.
\end{enumerate}

Frits Vaandrager %\cite{} %(2010 \url{http://www.cs.ru.nl/~fvaan/PV/what_is_a_good_model.html})   
augments their list to produce the following.\footnote{Vaandrager's useful list appears on his faculty webpage at the Institute for Computing and Information Sciences, Raboud University \url{http://www.cs.ru.nl/~fvaan/PV/what_is_a_good_model.html}. % I have not found it published elsewhere, but  
} A good model: \begin{enumerate}
     \item has a clearly specified object of modelling;
     \item has a clearly specified purpose;
     \item is traceable: each structural element of a model either (1) corresponds to an aspect of the object of modelling, or (2) encodes some implicit domain knowledge, or (3) encodes some additional assumption;
     \item is truthful: relevant properties of the model should also hold for the object of modelling;
     \item is simple (but not too simple);
     \item is extensible and reusable;
     \item has been designed and encoded for interoperability and sharing of semantics. 
 \end{enumerate}
These principles inform the purpose and strategy of our model.

\subsection{Model purpose}

Edmonds et al. lay out seven purposes of modelling \cite{edmondsDifferentModellingPurposes2019}. These range from prediction of new data and explanation of patterns in the world, to analogy that merely illustrates some phenomena informally. % Following Epstein who proposed sixteen reasons in addition to prediction for modelling  \cite{epsteinWhyModel2008},  
One of the purposes they identify is \gls{theoretical exposition}, which means the purpose is to explore the relationships between variables in the model. This is the purpose of our model. % The mechanisms in the model display. 
While it would require further empirical work to make specific claims about the outside world, 
% whether explanatory or predictive, % Thia does not make it possible to make claims about 
the  model can illustrate relationships and give insight into the implications of theory. 

\subsection{Model strategy}
%Given the purpose, we must select a strategy as a means to achieve that purpose.  
%  Model strategy is the means by which the model purpose is achieved. %In addition to modelling purpose, there is modelling strategy, 
A common approach to strategy is to follow principles or rules of thumb suited to the purpose of the model. % that guide the modelling. % work. % practice. 
%is With a model built for theoretical exposition, the challenge is to systematically explore the relationship between parameters, % show the results, and % qualify any results, making and make clear the limitations of the model. 
A model designed for theoretical exposition  
% A model built for theoretical exposition 
is built to illustrate the implications of theory, not to establish that theory applies in any given situation beyond the model. Our strategy thus has two aspects. % focuses on two things. % seeks to achieve two things. % has two features.
% With theoretical exposition we 
% To achieve aims specific to the purpose of theoretical exposition, we focus on two things. 
% The purpose of our model is theoretical exposition, so our strategy is designed to achieve aims specific to that purpose. 
Firstly, to understand the implications of theory, we remove elements that are not central to understanding the central theoretical questions. Secondly, since we are not attempting to make predictions, 
we focus on patterns and qualitative relationships rather than quantitative values. Our strategy is thus to keep the model simple and descriptive. Edmonds et al. use the acronyms KISS, `keep it simple stupid,' and KIDS, `keep it descriptive stupid' \cite{edmondsDifferentModellingPurposes2019}.

% \subsection{Applying the principles}
% Since simplicity is key to achieving these goals, % That leads us to a long list features of real urban systems that 
We intentionally leave features out of our model to focus attention on the features that matter most to understanding how financialization captures the value produced in cities. We have two basic propositions:
\begin{enumerate}
    \item Financialization of the urban housing stock extracts wealth produced by urban agglomeration effects. 
    % \item Financialization of the urban housing stock changes the class structure of urban society. 
    \item Financialization of the urban housing stock can limit urban productivity growth. 
\end{enumerate}
% Each  of these represents a high-level and general proposition. 
In considering whether to add a feature to the model, we ask: % varying housing density for example, we ask: 
\begin{enumerate}
    \item Is it necessary to demonstrate the principle? 
    \item Would incorporating it result in falsifying our result?
    \item Would incorporating it result in a qualitative difference in our result?
    \item Would incorporating it result in clarifying our result?
   % \item can it be added at a later point to get more neuanced results? 
\end{enumerate}

For example, we need to model how firms set the urban wage and how land markets distribute surplus value to understand the relationship between financialization and the housing market. % production in the urban system. 
 % It should be clear that, 
However, while varying housing density across our model urban system is technically straightforward, it is not necessary to explore or demonstrate any of our two propositions. Including density variation would not undermine our results nor clarify them. % It is simply not useful to incorporate this feature in our model. 
We can say the same about other extensions discussed in Chapter~\ref{appendix-future-work} on future work. %For instance: 
%\begin{enumerate}
  %  \item Adding a detailed production sector with multiple firms. 
   % \item Adding a detailed labour market.
    %\item Adding households of different sizes.
    %\item Adding an income distribution. 
    %\item Adding a range of distinct occupations.
    %\item Incorporating more complex lifestyle choices.
    %\item Articulating the transmission mechanism from productivity to wages and from wages to population. 
    %\item Incorporating building costs. 
    %\item Adding developers.
    %\item And including zoning regulations.
%\end{enumerate}
Those extensions are interesting and would add fine-grained detail to our understanding of how the effects of the financialization of the housing market are distributed, but none of them are essential to exploring the basic relationship. % is likely to affect the qualitative results or make our argument easier to grasp. 
% While making it 
% In the thesis, we have described a stylized model to  establish how the urban system generates rents. Even with its simplifications, the model can describe key features of urban structure and urban history, and addres the core hypotheses.   %In this section, we illustrate some of the insights supported by the model. 
% Extensions can incorporate variations in wages, density, transportation costs, preference, and even building technology and codes. The limitations of the simple, continuous, equilibrium-based versions described above can be overcome using agent-based models to model the evolution of complex and much more realistic urban systems. 

% Our core question in this thesis is about the effects of financialization capturing the value produced as cities grow. We have combined these  approaches to bring together the two pieces you need to understand the financialized capture of the value of the urban system: a model of the urban system and a model of the mechanism by which they capture it. 

% None of these represent methodological transgressions although the combinations may be unusual. 
% The methodological questions that this combination raise are interesting and we get additional insight into the financialization of housing markets by examining them further.

% We begin by reviewing the modeling paradigms, then discuss in some detail the relationship between analytic neoclasical economic analysis and agent-based modelling, how we've integrated the two and why we believe this is a powerful technique for the tendencies inherent in agent-based models, and in particular in understanding the long term distributional questions. 


\section{Types of models}\label{sec-types}
We combine two distinct approaches to modelling social systems: agent-based techniques and equilibrium reasoning. 
Where agent-based models describe behaviours and observe the outcome of the behaviours computationally, most economic modelling is built around equilibrium conditions that are identified \textit{a priori}. Agent-based modelling and equilibrium analysis are complementary, however, and we employ both. We combine a spatially explicit agent-based land market model with equilibrium assumptions about rent levels and an equation-based model of urban production. We employ equilibrium arguments to get clear inputs for the ABM from the peripheral models, where there is a great deal of existing theory about how that part of the system should behave.\footnote{Combining agent-based models with components represented as systems of equations is an established approach to modelling:``\dots in an agent model, physical subsystems can be represented naturally using systems of differential equations (e.g. climate, energy, etc.)  (Chappin, Dijkema and Vries, 2010; Chappin and Dijkema, 2009; Davis et al., 2009; Nikolic, 2009)''  (\cite{chappin_simulating_2011} p 61).} 

\subsection{Agent-based modelling}
Agent-based models employ populations of simple submodels (automata, or agents) \cite{shalizi_methods_2006}. They are part of a wide class of models that incorporate the behaviour of distinct entities and explore the interactions among them. The approach was developed in the 1940's when von Neumann and Ulam introduced cellular automata, a simple form of agent model where agents are typically squares on a grid following rules based only on the state of their neighbours \cite{banksStatisticalChallengesAgentBased2021}.

ABMs have proven effective across a range of domains. Examples of ABMs range from models of Bali's traditional economic and social structure, delays in traffic flow, economic bubbles, 
traditional foraging patterns, % (in the Sugarscape models), 
social segregation, and ecological succession networks, to patterns of poverty and crime \cite{open_agent_based_modeling_consortium_comses_????}. %, _netlogo_????}. 
Recent results from disciplines including physics, biology, anthropology, and economics have built a record of useful empirical results \cite{parkerMultiAgentSystemsSimulation2003, parker_multi-agent_2003, helbing_social_2011-1}. 
These have shown ABMs reliably produce phenomena through simulation that were difficult if not impossible to derive from simpler analytical/closed-form expressions. 
% Related modelling approaches include individual-based models, cellular automata, Ising models of atomic spin, the use of swarm phenomena for modelling fabric or fluid motion, and even finite element analysis and parts of graph theoretic modelling.  
% While the details of implementation differ, common principles and patterns of behaviour appear that in ABMs apply across domains \cite{shalizi_methods_2006}.  

In an ABM, %\gls{ABM},
agents are defined as having adjustment rules or behaviours that respond to environmental variables.\footnote{Modelling agents with agents are defined as having adjustment rules and behaviours has a long history in economic, going back to the 1838 Cournot duopoly model \cite{cournotRecherchesPrincipesMathematiques1838}, for example, which is analyzed using `reaction functions' which simply describe a firm's optimal response to a second firm's output choice. In Cournot's simple case an equilibrium can be directly computed.} 
Unlike equilibrium approaches common in economic analysis, agent-based modelling commits to using computational methods to mimic the distributed decision-making of complex systems like cities. A program is written that considers each agent sequentially and updates agent and system states as it goes. The program is allowed to iterate, and the values of any state variables of interest are recorded at each step. The model may or may not settle into a steady state. As with other modelling approaches, ABMs can be used explore the behaviour of the system by varying individual parameters, and to explore the parameter space using Monte Carlo methods.

%agent-based models replace single equation models of sub-systems with populations of simple submodels (automata, or agents) \cite{shalizi_methods_2006}. 
% Just as a system of equations can produce emergent phenomena, so can subsystems of agents. 
% Emergent phenomena may occur because 
One advantage of beginning with an agent-based model is that we are not imposing linearity, or \textit{a-priori} distributions on the behaviour of agents. In an ABM, agents do not make exactly the same decision at exactly the same time. Decisions and timing depend on the speed of information flows, proximities, and other factors. As a result, suites of agents behave differently than the ``representative agents" typical in economic models \cite{darley_towards_1999, tesfatsion_agent-based_2002}. 
%We employ agent-based methods in modelling the labour and housing market. We allow agents to decide to work if the wage justifies commuting to work at the center.\footnote{In a computationally faster version we compute the distance to the boundary once, rather than having each resident or potential resident decide individually if it is worth entering the urban labour market.} We allow individual worker agents to decide whether to purchase a home and to set their own reservation prices when it comes to selling when they retire. Although we limit the variety of agents in this study, the agent-based modelling approach allows us to introduce continual of agents or whole new classes of agents easily.
 % Imposed assumptions might make the model more tractable, but they have little empirical or theoretical justification. 
By including individuals explicitly, agent-based models offer nuance in exploring the effects of individuals on a system, and the make it possible to study distributional effects, space, and individual differences in a much richer way. The structures that emerge result from assumptions about the decisions of those in the system rather than from assumptions about the behaviour of the system. Agent-based models thus may reveal non-obvious implications of structural changes in particular locations, for particular individuals \cite{darley_towards_1999, happe_agricultural_2004}. 
%cite some of the people who show it can be used for structural change)
% Studying large messy models like these has many of the features of studying real social systems. 
%Where it is helpful, other types of models can be linked with agent-based models or used as subsystems. % 

 %There is a large literature on agent-based models. 

%There are good text books as well. Agent models make it possible to represent phenomenal that are hard to represent otherwise including traffic jam, bubbles and crashes, power law and other fat tail phenomenal, critical transitions

%%%%%%%% (Economics has looked at those areas where the behaviour of many individuals can be reduced. Sociology has looked at those areas where individuals matter. Because of constraints from the (1800s) economics has used math and sociology has not.) ABMs are starting to bring math to social theory. Network theorists and big data people are getting hired both in sociology and and economics for this reason. It is because the math is a different kind of thing.

A challenge with ABMs is that they can be hard to extract meaningful information from. ABMs can be computationally intensive, raising significant problems for analysis since there is so much happening in them. This can make them difficult to use in a practical context \cite{banksStatisticalChallengesAgentBased2021}. % Furthermore, since much of the interest in ABM's lies in system behaviour in response to changes in policy \cite{helbing_social_2011-1}, %integrative design paper 
% questions of observability become central. 
% Raising significant challenges for analysis 
%They thus make it possible to look at the effect of the structural impact of policy makers on real individuals. 
% 
% Of course 
Their complexity is also their strength, however. An advantage is that agent-based models represent the structure of individuals interacting in social systems. % They represent specifically the structures that policy makers engage with: how do actions affect individuals, communities, and societies it affects how people act. It provides the capacity to look at those different levels. % They are process models.
% They provide a natural form for modelling transient and long-term dynamics as individuals interact over time.  %They include interacting individuals so 
They thus make it possible to look at the structural impact of interventions on particular  individuals and in particular contexts. This makes agent-based modelling particularly relevant for policymakers because the can provide a more accurate picture of the whole systems policymakers would be working in.

% Since the system level behaviour comes from the interaction of individuals the model does not need to impose distributions on the outcomes over sets of individuals. The distributions can emerge from the system. 



%\textbf{Agent-based models have a number of advantages:} 
%Maybe include some of these.
%
%* We want to model dynamics
%What are the transient and long term effects? 
%How do the details of what individuals do affect the system. How do the changes in the system affect particular people. Not just what is the mean, but what is the distribution? How does it affect an individual?
%
%* We want to be able to model structural change.
%"Although there are proposed examples of SDs with changing structures (Duggan, 2008), they have not yet matured: in SD the structure of the system is fixed (Yücel, 2010)."
%The other models fix structure.
%
%* There are advantages to process models. 
%It does what we want in the way the system does it.
%
%The cost is that disabragating makes for larger models. They are computationally more intensive and they are more difficult to draw insight from. 
%
%If a simpler model is adequate, a more complex one should not be used. 
%
%This suggest a process model where phenomena that emerge from interacting individuals are modelled as 
%And phenomena represented well by continuos processes are represented in that way (NEwton is many interacting bodies.)
%And where events are modelled using discrete events. 

% This section reviews approaches to models and makes the case that computer models and specifically agent-based computer models are particularly well suited to the problem of understanding how interventions shape social systems.

% "Fundamentally, I'm not sure that agent-based modeling amounts to anything other than object-oriented programming for disaggregated simulations `` \cite{http://vserver1.cscs.lsa.umich.edu/~crshalizi/notebooks/agent-based-modeling.html}

% "Complex  systems models can also serve as stochastic models {\bf Ergodic deterministic systems might as well be stochastic}  Some of them are related to standard, modern stochastic models e.g., {\bf agent-based models are “interacting hidden Markov models” or “dynamic Bayes nets with latent variables}"\cite{Cosma talk on stats complex}


%%%%%%%%%%% FROM PAMPAS DOCUMENTATION 
% "We adopt agent-based modeling as a suitable approach to quantitatively model agricultural systems, their \textbf{structural change}, and endogenous adjustment to policy interventions (Happe et al., 2004). Agent- based modeling is a powerful technique for simulating the actions and interactions of autonomous individuals to \textbf{assess emerging system level patterns} (Gilbert, 2008; North and Macal, 2007). An ABM consists of a collection of autonomous and heterogeneous decision-making entities (agents) interacting with one another and an environment. Agents have \textbf{information} about attributes or state of other agents and the environment, and have access to past and current values of their own state variables (e.g., economic outcomes). Agents make \textbf{decisions} using both prescribed rules nd analytical functions; decisions are based on the information agents have available (Gilbert, 2008). An ABM also includes \textbf{rules} that define the relationship between agents and their environment, and rules that determine \textbf{scheduling} of actions in the model (Parker et al., 2003)."

\begin{figure}
    \centering
  \begin{tikzpicture}[scale=0.6]
\draw[thick,<->] (0,10) node[above]{$Price$}--(0,0)--(10,0) node[right]{$Quantity$};
\node [below left] at (0,0) {$0$};
\node [below] at (5,0) {$Q^*$};
\node [left] at (0,5) {$P^*$};
\draw(1,1)--(9,9) node[right]{$Supply$};
\draw(1,9)--(9,1) node[right]{$Demand$};
\draw[dashed](0,5)--(5,5)--(5,0);
\end{tikzpicture}  
    \caption[Supply and demand]{Supply and demand.}
    \label{fig:SandD}
\end{figure}

\subsection{Equilibrium reasoning }\label{subsec_Equilibrium_reasoning}
Economists rely heavily on equilibrium analysis in their study of systems. 
%ADD BACK? stylized fact---appearing data over again and again... toy in real world. distributional effects of urban wealth growing. The ADVANTAGE OF THIS is that iT ENABLE THIS BROAD UNDERSTAND OF HOW PATTERNS PLAY OUTj
%In Economic modelling the most familiar approach is the analysis of systems in equilibrium. 
Under the equilibrium approach, a set of necessary conditions describing the steady state of interest are imposed. This approach produces tractable models that can often be solved explicitly. It % is a productive methodology partly because it 
bypasses the often intractable and complex process of adjustment, focusing on the conditions that must be true if a particular situation is to persist. This tractability was particularly important before computers became widespread.
% ANALAGOUS TO RESILIENCE.

The most familiar example of an equilibrium model is the ubiquitous supply and demand model. A supply and demand model consists of two curves, each describing a functional relationship between two variables, price and quantity. Each curve represents the preferences of a class of agents, and the model explores the combinations that satisfy the behavioural intentions of both classes i.e. are on both curves.  The demand curve, illustrated in Figure~\ref{fig:SandD} visually represents the quantity of a product demanded by potential buyers at each price.  The supply curve is the quantity that suppliers would be willing to sell at each price.  Equilibrium, in this example at the combination  $(Q^*,P^*)$, refers to a state in which the preferences of both agent classes are satisfied. A situation is unlikely to persist if either class of agents is unsatisfied with the combination of price and quantity. The question at the heart of the equilibrium approach is, ``What conditions are necessary if the situation is to be stable?"  In the case of supply and demand, The quantity supplied must equal the quantity demanded, and the price must be the same for both classes.\footnote{Taxes introduce an additional condition that results in different prices for the two sides.} If the two curves can be described mathematically, equilibrium prices and quantities can be derived by solving the two-equation system.


% It is helpful to remember that 
% The approach makes it possible to simplify analysis and understand long run patterns.  The technique evolved before computers made simulations relatively easy. To achieve tractability, it was necessary to limit the number of variables, and independent decision-makers by employing a `representative agent,' i.e. one agent with consistently defined qualities that represents the class of agents.  

% To extend the analysis to dynamic systems, `laws of motion' (adjustment rules) can be added as either difference or differential equations.  Models quickly become challenging with more agents or dynamic processes, and economists, like other modellers, now very often resort to simulation and numerical methods. 


\subsection{Where we use agent-based models and equilibrium reasoning}
We use an agent-based model for the core model of the process of financialization, where we need to understand individual behaviour and distributional effects. % Because our focus is 
%To understand the relationship between financialization in the housing market and the production of value, central housing and financial sub-model using an agent-based model % we use an agent-based model in the central housing and financial sub-model because that is where the processes most interesting for our work happen. 
The agent-based model allows us to explore how patterns of distribution emerge within a spatial system with many agents.

%We draw heavily on existing economic analysis of cities to identify relevant behaviours and parameters. We also employ three key economic equilibrium conditions drawn from the economic literature: locational, labour market, and  housing market equilibrium conditions 

% In a market with any degree of churn, market prices for land should converge on use values, and these will vary systematically with transportation costs. This is an implication of consumer choice theory. The market-based argument allows us to compute land rents from the transportation cost and urban wage premium that drive urban locational decisions. For simplicity, we present the model for the case where individuals have the same preferences, employment opportunities and transportation costs. We can, if and when we apply the model to questions of urban design and inter-individual distribution, easily introduce site-specific and person-specific features.  

%who make decisions at the margin within an agent-based model (ABM) in which agents make periodic choices among a small set of alternatives. 

We use equilibrium approaches to model urban production. There are two reasons for this choice. First, firms can generally adjust much more quickly than the housing market, so it is reasonable to assume that the quickly adjusting variables like price and employment are close to equilibrium values.  Second, we want to drive the central process with a sub-model that is well understood to ensure that results can be explained easily. Nothing prevents us from replacing the sub-model with one with long lags or a different theory of firm behaviour. 

% We use the standard urban firm model which is based on equilibrium analysis to reproduce key features of the urban production system. 

% In the production sector, 
 % equilibrium levels as inputs to the other parts of the model

% HOW

% Equilibrium wage levels are passed to the spatial model where another equilibrium-based calculation is used to determine target population and locational rents. Population feeds back into the agglomeration function with a lag. Agents in the housing investor agents correctly observe the levels of wage and rent that apply for them in each period. 

% Other variables such as interest rates and mortgage requirements are parameters in agent decision-making. Agents use the observed values when they enter the housing market and make bids in response to the wage. They work, save and eventually retire, putting their homes on the market. Investor agents estimate the revenues and capital gains they would earn if they buy available properties, and make bids as well. Sellers accept the highest bid they receive. 

% WHAT ABOUT THE STUFF BELLOW THIS?

% An example of \gls{equilibrium reasoning} is the \Gls{Alonzo model} approach to imposing a locational equilibrium condition, \[U_i(d_k,\dots)=U_j(d_l, \dots),\] where $U_i(d_k,\dots)$ represents agent $i$'s utility at distance $d_k$. The condition says that identical individuals must get the same utility no matter how far they live from the city centre. If that were not the case, individuals would move to a location where they get higher utility. For utility to remain constant as transportation costs rise, some other variable must compensate. In this class of models, the rent charged for the use of land must fall as distance increases. This is an equilibrium condition. 

% Equilibrium locational choice by commuters therefore determines the extent and ultimately population of the city.\footnote{More complex models allow home sizes and lot sizes in the suburbs to increase as well.} Since land value is \gls{capitalize}d rent, land values also decline toward the edge of the city until they are equal to the rural value of the land. An adjustment process that requires people to move to a different location would take a long time to work through the system. Home prices and rents, however, can usually adjust more quickly. % different location would take a long time to work through the system. Prices, however, can adjust much more quickly. 

Our model of firm behaviour uses a single representative firm to determine the wage. This approach does not explicitly model multiple interacting firms but using  equilibrium reasoning here has the advantage of simplicity and being among the most common approaches to modelling production. 
We sacrifice some complexity in this part but the model of production is less central to our model and we get the advantage of connecting the work with standard work on theory of the firm using simple, well-understood models. 
Using a representative firm allows us to focus on the relationship between the scaling of urban productivity and the financialized ownership that can extract that value, which is the core conceptual contribution of this work. 

There is no loss of generality in using a single firm: we are not interested in the structure of the production sector. Because our focus is financialization in the land market, we only require that wages and employment behave as economic theory tells us they should. Our representative firm is a neoclassical profit-maximizer, adjusting employment and capital stock in every period toward the optimal quantities according to the marginalist rules for maximizing profit.  The model is fully extensible and future implementations could replace the equilibrium model of representative firms with a larger agent-based model of interacting firms. Our overall model is sufficiently modular to plug in a different production sector, perhaps to introduce agents or to link to a model developed elsewhere. 


By integrating the agent-based land market model, the equilibrium assumption about rent and the equation-based model of urban production, three pieces which are traditionally studied separately, this work combines elements of classical rent theory and neoclassical distribution theory with agent-based modelling.




\section{Summary}\label{sec-summary}
In this chapter, we have made our solutions to certain high-level modelling questions explicit. We have explained our purpose and strategy in building the model. We then described the type of modelling used for each piece, the use of agents and the role of equilibrium analysis. In our case, the purpose of the model is \gls{theoretical exposition} and we describe our strategy as an attempt to make a model that is simple, extensible, and reusable, while meeting rigorous criteria for good modelling practice. % 

The model type is a blend of agent-based and equilibrium modelling.
In integrating models of the firm, the urban spatial system, the housing market and the financial system,  % which are traditionally studied separately, 
this work combines elements of classical rent theory and neoclassical distribution theory with agent-based modelling.  We have a spatially explicit agent-based land market model, an equilibrium urban rent model, a neoclassical equation-based model of urban production, and a rule-based mortgage provision system.

The work thus draws on a variety of methodologies. % and raises subtle modelling issues. 
We have at each point grounded modelling decisions in established theory and eliminated elements that would add unnecessary complexity, while focusing on building an extensible model that illustrates the relationship between urban production and the capture of rent through land markets.

 % Finally, given our purpose and methodological decisions, we have focused on developing a model that is both simple and descriptive. 



 %ADD BACK WE USE IN TWO PLACE WITH DIFFERENT PURPOSES ...  So in the model we have MODEL OF PRODUCTION LAND MARKET LINK BETWEENTHE AND RENT WHICH IS NOVEL application serves as link between them.



% We identify the classical period loosely as 
% Neoclassical and classical theories are often seen as opposed ideologically, since the classicals 
 % The  opposition in our view is exaggerated. 
% Many  who employ ABMs to analyze urban systems are deeply skeptical of neoclassical assumptions while ignoring the rent;-based distributional analyses of the classicals.

% In ABMs it is often makes sense to simplify models, less for computational reasons then to % The problem of model complexity remains a challenge with ABMs, however, and modellers introduce simplifying assumptions. These are not as a rule needed for the computational model but are helpful in maintaining a focus on the key theme  of the model.  %(ADD distinction between detailed models and detailed explicit models)


% These ideological overlays are not helpful in making choices about how to model. We would argue that ABM modelers who assume mainstream economists actually believe that the neoclassical technique of reasoning based on informed rational agents is based on faith or ideology are mistaken. 
%The neoclassical approach is a modeling technique and a methodological convenience. that is widely used across economics. %, that is valuable when it is used correctly.  
%To the extent that it is limited, its limitations may be explored by making it interoperable with a model that makes it systematically possible to relax each constraint, and draw on assumption to study.
% The limatations then become testable, and it's use

% We consider the rent theory of the classicals an essential tool for analyzing urban land markets, and we agree with neoclassical economists that wages and prices are set in factor markets. At the same time, we agree with agent-based modelers that complex systems, including markets, are usefully modeled by focusing on the individual and local decisions of % in this context how %easily and productively 
 %, linking traditions that have been separate. 


% In general, there is relatively little work rigorously linking analytic and agent-based models, so the results can be understood formally, in relation. % incorporated into prior traditions and

% Classical economics theory actually maters centers theory- both theory as embedded in largely theory and the 100 years of sophisticated thinking. a new technology does not make obsoletes inline.. ltos has been worked out. eg. refactor..

% 1. how is it the same, how can it vary. switch from complexity to simplicity as easy. the coarse grainning piece.  small models can be best, can be beneficial to 


% 2. This approach makes it possible, in particular, to study the distributional and resilience and hysteresis implications of mainstream economic models.---regimes and patterns, protection. (engineering approach in complex system- margin of error with the general distributional properties of the system) (cycling and regimes-)

% can't with ODEs model a set of things. a generation of thinking...

% 3. the idea of a frontier, an attractor

% don't believe people are optimizing

% do believe that gives us an approximation o


% it is an idea
% some heterodox economissts chose to think is a big thing, the thing we're doing with rationality at the agent level
% we're getting a plausible way to model short way behaverio will lead
% to what they do if they just stumble along.. specific leads to generalequilibrium

% living in it, they don't try to move.
% the equilibrium cocnept is a plausible place ot end up.. for using a kind of plausible hack for hte short term.
% they act differently. much more like evolutionary game method- agents have a certain rule and we see what the agent does. 

% Most analagous to the evolutionary game models, which are most alike this..
% in the game models, you can put a local behaviour liek behaver which is a drive from calculus- like what's the best I can do.

% question which is interesting---whether or not if its a rule of thumb,.. stochastic rule willgive you the same thing, now a testable modelling question.

% NASH rule is optimal given what others have done.. .

% true like the horizon is true.---a direction parts of the system are always pushing to by their own logic. look at the implications of that
%---long run in a sense- it is a tendency-disregarded-
%---deep kind of truth to this tendency, like a current underwater, sometimes obscured- but to understand what are something analogous to forces. 
% the concept of rent is in line with this. What would be the most self interested thing, what is the most that could be taken out.
% It is a distinct class of question philosophically in line with the class of equilibrium questions.

% not the exact amount charged
% it is what it's worth to be there.
% when their pushing---they're competing to claim the edge case of rents, specuating on claiming high trent, wh.. who can fit in a community. 


% force, always a pull towars this much
%  a constriang as far as you can go, and as much you can do
%  economics obscures any infra-marginal gains
%  -hill climbing algorithm.

% no equilibrium and no optomization algorithm
% what is rent. 
% we don't see any opotmization in our financial agents. 
% they are trying to get the largest investment return they can.. they are locally optomizing..---they don't optomize. it is a good rule.

% we do use optomization theory.
% use wehn we are assuming an optomistation and will move iftheyre any differented..

% thinking about this things it moves towards but doesn't reach
% in Riccardo farmers go toward the margin- everyboddy short is collecting rents..
% Prepares them to be used in corase grainin gand compared and varried

% ***. CITY We are focused on the evolution of a city. In a city, people and organizations make individual decisions independently, constantly adjusting and shaping the city they inhabit. As a result, cities evolve continuously and and don't reach any final equilibrium. 

%In any city, numerous people and organizations in the city are making individual decisions independently, constantly adjusting in sensible ways to the changing parameters of the city they inhabit. As a result, cities evolve continuously and may never be in equilibrium.  To model this complex, dynamical, multi-agent economic system we  rely on techniques and insights from several fields and we  employ two distinct modeling approaches. We develop and explain the theoretical foundation of our work using a range of results from the economic  literature. We go on to implement the theoretical model in the agent-based modeling (ABM) framework. 
% In doing so we are attentive to the strengths and weaknesses of the approaches that we are adopting. 

% COMPLEX SYSTEMS
% COMPLEX ADAPTIVE SYSTEMS
% DISTRIBUTION
% ABMS---as like attactors in the model

% in between dependence
% smooth but the parts matter
% a grove, an atractor

% %----------------------------------------------------------------------
% \subsection{Equilibrium versus agent-based modeling}
% %----------------------------------------------------------------------

% distribution and long-run dynamics---market-level things really well but can't disaggregate. conceptual clarity with which you see the long-run market pathway. what state of affairs can persist. agent-based models aren't constrained by it's end result. classical economics uses equilibrium concepts at the market level, generally bypassing the adjustment process while agent-based models focus on adjustment processes without imposing equilibrium conditions.  In this model, we want to deal with adjustment processes in the financial and housing markets, but want to retail the market-level discipline for the rest of the system. , is what makes it possible to explore the distributional implications of financialization in a way that is consistent with standard \gls{classical} and \gls{neoclassical} economic analysis.   limit or \gls{frontier}  The it functions as an . this work uses that kind of frontier, drawing on \gls{equilibrium}, \gls{dynamical system}, and \gls{agent-based} analysis, is treated in the discussion of methodology in   which, based on the literature can reasonably be seen as an equilibrium or attractor the market pushes agents toward.


%Some adjustments are slow and some are fast. Rates of adjustment can matter. 


%Classical economists paid great attention to the extraction of rent or surplus as the foundation of the class structure of the day. In contrast, the neoclassical economists emphasized the market allocation of income based on one's contribution to production at the margin.



%ALTERNATE PHRASING? Transportation costs depend on distance from the center, so land close to the center is more attractive than land farther away.  The equilibrium concept is that a market with identical individuals with identical incomes and transportation costs will result in identical utilities. The result is that land rent must decline with distance from the central place to offset rising transportation cost.


