
% \chapter{Ownership} \label{chapter-ownership}
% 'cg_tax_invest': [1.0, 0.5, 0], # done
% 'cg_tax_per': [1.0, 0.5, 0], # dr done
% 'r_investor': [0.2, 0.1, .05], # has set up for morning
% 'c': [500, 300], # k ran
% 'density': [100, 150], # k ran
% 'wealth_sensitivity': [0.15, 0.1, 0.5], # k ran
% 'property_tax_rate': [0.1, 0.05, .01], # k ran

% FIGURES IN _plots folder with chaty in the  _code folder

% interest rates - check impact in code
% FIX. THIS DOESN'T REALLY LAY OUT WHAT WE'LL DO WITH THE REST OF THE CHAPTER
In this chapter we look at ownership. The pattern of ownership established as a result plays a crucial role in determining how locational rents within the city are distributed. The consequences of any shift from owner-occupancy to tenancy, shown in red, are dramatic. 
%\includegraphics[scale=.5, angle=-90]{fig/IMG_2691.jpg}
We are observing these processes underway in the real world. We need to know if they are explained by a theoretically-consistent model of the urban economy that incorporates the financialization process.

\section{Ownership of housing}

Recall that our first hypothesis is that in an \Gls{Alonzo-Jacobs model}, the financial sector will tend to take over a growing share of property ownership within a city. 
With  plausible parameter settings our baseline model consistently produces an evolutionary trajectory of class ownership of housing similar to that illustrated in Figure ~\ref{fig:Baseline_ownership_trajectory}. The distribution of ownership then determines the distribution of the locational land rents, which are the product of agglomeration effects. This general result arises from the specification of the financial sector and the independent behaviour of the agents. It is driven by the operating rules of the financial system. 

\begin{figure}
\centering
\begin{tikzpicture}
  \node at (0,0) (1) {URBAN/PRODUCTION PLATFORM};
   \node[above=1mm of 1] (A1) {$\uparrow$};
   \node[above=1mm of A1] (Pop) {POPULATION};
   \node[above=1mm of Pop] (APop) {$\uparrow$};
   \node[above=1mm of APop]  (3) {BIDS};
   \node[above=1mm of 3] (4) {$\uparrow$};
   \node[above=1mm of 4] (5) {OWNERSHIP};
   \node[above=1mm of 5] (6) {$\uparrow$};
   \node[above=1mm of 6] (7) {RENT DISTRIBUTION};
   \node[above=1mm of 7] (ARent) {$\uparrow$};
    \node [above=1mm of ARent, text width= 6cm, text centered, red] {Tenant sojourner cities\\ Eviction of families from cities\\Growing inequality\\ End of urban home-owning class};
   % \node[above=1mm of 1] (2) {$\uparrow$};
   % \node[above=1mm of 2] (3) {Growing inequality};
   % \node[above=1mm of 3] (4) {$\uparrow$};
   % \node[above=1mm of 4] (5) {Eviction of families from cities};
   % \node[above=1mm of 5] (6) {$\uparrow$};
   % \node[above=1mm of 6] (7) {Tenant sojourner cities};
\end{tikzpicture}
\caption{TODO}
\label{fig:enter-label}
\end{figure}

 Figure ~\ref{fig:Baseline_ownership_trajectory} shows an initial situation with close to 100 owner-occupiers and a small number of investor-owners. Although the number of owner-occupiers grows in the initial period in Figure ~\ref{fig:Baseline_ownership_trajectory}, over time financial capital acquires an increasing share of the housing stock. By the time the city reaches its maximum size, the city has been transformed from a city of homeowners to a city of tenants.

%DISCUSS THE PARAMETERS THAT AFFECT OWNERSHIP? We have conducted parameter sweeps to identify policy parameters that can affect the distribution of ownership.


\begin{figure}
    \centering
    \hspace{4cm} % Adjust the amount of space as needed
    \includegraphics[scale=0.8, trim={0 1cm 0 1.8cm}, clip]{fig/Analysis/Ownership_Data_1.pdf}
    \caption{The transformation from a city of homeowners to a city of tenants in the baseline model}
    \label{fig:Baseline_ownership_trajectory}
\end{figure}

%\includegraphics[scale=.5, angle=-90]{fig/IMG_2688.jpg}


%\includegraphics[scale=.5, angle=-90]{fig/IMG_2689.jpg} %Density determination

%\includegraphics[scale=.5, angle=-90]{fig/IMG_2690.jpg}% 2 FIGURES NEEDED  
%$\includegraphics[scale=.5, angle=-90]{fig/IMG_2692.jpg}

%To illustrate a decline in investment we can reduce Kadj.

%\includegraphics[scale=.5]{fig/IMG_2693.jpg}


Initially, 100\% of locational land rent accrues to residents. In the end, 100\% accrues to the owners of financial capital. Total rents at the end of the period of growth are eight times their initial size.\footnote{Recall that the radius of a circular city is proportional to the wage premium. Rent can be visualized as a cone of volume $\pi r^2 h$ where $h$ is the wage premium. If we double $h$, the volume is increased by a factor of 8.}

\section{Capital gains taxation}
The tendency for investors to own an increasing share of the housing stock is robust in our model. Can it be affected by public policy? Consider a capital gains tax on investors. If capital gains are part of the objective of investors, a capital gains tax will, in theory at least, reduce investor participation. Figure ~\ref{fig:CGinvest_ownership_trajectory} shows this is in fact the case. A capital gains tax on housing investment will have a powerful effect.
\begin{figure}[htb]
    \centering
    \includegraphics[scale=.8, trim={0 1.4cm 0 0},clip]{fig/cg_tax_invest-Main-_011821.pdf}
    \caption{The effect of a capital gains tax on investors}
    \label{fig:CGinvest_ownership_trajectory}
\end{figure}

The previous result was achieved in a situation where homeowners do not pay a capital gains tax when they sell. This is a policy that has been questioned on ethical grounds because it favours owners over renters.  Figure ~\ref{fig:CGpers_ownership_trajectory} shows that raising the capital gains tax on owner-occupiers will also have a powerful effect. 

\begin{figure}
    \centering
    \includegraphics[scale=.8, trim={0 1.4cm 0 0},clip]{fig/cg_tax_per-Main-021508.pdf}
    \caption{The effect of a capital gains tax on homeowners}
    \label{fig:CGpers_ownership_trajectory}
\end{figure}
In this case, the capital gains tax on investors is set at 15\%. The dotted line represents the case with no capital gains tax for homeowners. When the tax is higher than the rate of investors, shown by the solid and the dashed lines,  home ownership falls more than in the base case. In our model, the rents will not fall for tenants, however. Landlords by assumption charge tenants the full locational value, and that does not depend on whether there are capital gains for property owners.

%%%%%%%%%%%%%
\newpage

\subsection{The cost of capital for investors}
% 'r_investor': [0.2, 0.1, .05] has set up for morning
Figure ~\ref{fig:capital_ownership_trajectory} shows the effect of increasing capital costs for investors. Higher capital cost might be expected to slow the rate of housing acquisition. Surprisingly it appears to increase the speed at which investors purchase but not the final level of investor ownership.  In addition, it appears to decrease population, suggesting that the cost of acquiring housing rises. The cost of capital for investors is not an obvious policy tool, so we don't think this result is of great interest politically, but it warrants further study.
\begin{figure}[h!t]
    \centering
    \includegraphics[scale=.8, trim={0 1.4cm 0 0},clip]{fig/r_investor-Main-024857.pdf}
    \caption{The effect of raising capital costs}
    \label{fig:capital_ownership_trajectory}
\end{figure}

\newpage
\section{Transportation costs}
Roads and road congestion as well as the transit system are public responsibilities. They affect the cost of transportation. Can reducing the cost of transportation affect the ownership ratio. In Figure~\ref{fig:c_ownership_trajectory}, we reduce transportation costs for everyone by 40\%. 

The effect, which is not obvious in advance, is to decrease the home-ownership ratio but to greatly increase the size of the city. The reduction in transportation costs leads to an increase in land rents, and the increase in land rents increases the rate of increase of property prices, leading to higher expected capital gains.

The city grows faster and is larger with lower transportation costs. This is what we expect in the Alonzo model in this model.
% 'c': [500, 300],


\begin{figure}[h!t]
    \centering
    \includegraphics[scale=.8, trim={0 1.4cm 0 0},clip]{fig/c-Main-121459.pdf}
    \caption{The effect of decreasing transportation costs}
    \label{fig:c_ownership_trajectory}
\end{figure}


\newpage
\section{Density}
% 'density': [100, 150]
Density is a policy variable that is much discussed. Increasing density is proposed as a way to increase housing availability and the ability of people to buy homes. Our results indicate that The effect on city population is significant but it has no effect on city extent and no effect on the ownership ratio. The first two results are expected, the third less so but easily explained. 

The share of homeowners is not affected, but the number of homes is. Since rent per unit is unchanged, density alone does not affect the relative advantage of new home buyers and investors, leaving the ratio constant. There are more homes, however, and more home-owners. 

\begin{figure}[h!bt]
    \centering
    \includegraphics[scale=.8, trim={0 1.4cm 0 0},clip]{fig/density-Main-123417.pdf}
    \caption{The effect of increasing density}
    \label{fig:density_ownership_trajectory}
\end{figure}


\newpage

\section{Wealth sensitivity}
Wealth sensitivity in our model is a parameter of the banking system. It determines how easy it is for people with limited assets or income to get a mortgage. A higher value is more restrictive:  makes it harder for an asset-poor person to get a mortgage.

A restrictive mortgage regime slightly reduces labour supply and city population. It has no noticeable effect on the ownership ratio. Since the purpose of a restrictive policy is to reduce defaults and bankruptcies, the small impact on other variables suggests that as a policy it can be adopted without needing to consider the  economic impacts on the city. 
% 'wealth_sensitivity': [0.15, 0.1, 0.5]

\begin{figure}[h!bt]
    \centering
    \includegraphics[scale=.8, trim={0 1.4cm 0 0},clip]{fig/wealth_sensitivity-124913.pdf}
    \caption{The effect of wealth sensitivity of mortgage access}
    \label{fig:wealth_sensitivity_ownership_trajectory}
\end{figure}

\newpage

\section{The property tax rate}

This variable is a matter for policy debate because there are good reasons to think, first, that property taxes are a barrier to ownership, and second, that the property taxes are not paying all the costs of public infrastructure and services to homeowners, leading to significant inequities.

% 'property_tax_rate': [0.1, 0.05, .01] k doing}
\begin{figure}[h!tb]
    \centering
    \includegraphics[scale=.8, trim={0 1.4cm 0 0},clip]{fig/property_tax_rate-Main-130755.pdf}
    \caption{The effect of property tax rates}
    \label{fig:property_tax_ownership_trajectory}
\end{figure}