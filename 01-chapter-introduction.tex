\chapter{Introduction}

% \vspace*{\fill}
\epigraph{``cities..% the 'force' we need to postulate account for the central role of cities in economic life is of exactly the same character as the 'external human capital'}{Robert E. Lucas Jr., \textit{ON THE MECHANICS OF ECONOMIC DEVELOPMENT (1988)}
}


%Cities are the crucible of civilization, the hubs of innovation, the engines of wealth creation and centers of power, the magnets that attract creative individuals, and the stimulant for ideas, growth, and innovation." also dark side. "prime loci of crime, pollution, poverty, disease, and the consumption of energy and resources 214 [[West Scale]]
 
%Missing- transfer of money vs put in a spatial framework.
%Most of the economic theory talks about where people go, and it doesn't talk about the value they create in the city and where that goes. That's finacialization, capturing those benefits is what capitalists are doing now.
% rent is being in the house/what they pay, the transfer of money, vs what cities are, and how that produces value..

In Canada faces a housing crisis. %In the last few years, the need for affordable housing has come into focus as one of the most pressing issues facing Canadians. % #e (ADD STRIKING STATISTIC - A NUMBER OR QUOTATION HERE.) 
% As more and more Canadians are finding 
As more Canadians find housing unaffordable, the effects now reach far into the middle class, causing everything from declining home ownership rates to increasing poverty and displacement. % number of Canadians able to afford housing at all, leading to vacancies, poverty, and displaced workers. % (#e FILL THIS OUT A BIT - ADD SOME DETAILS HERE ABOUT THE EFFECTS OF THE CRISIS - THAT SET UP THE RESULTS OF THE THESIS)
% RATE OF INCREASE -VITAL SIGNS REPORT, CMHC

There has been extensive debate about the drivers of the crisis. Proposed explanations include supply shortages, stagnating incomes, and the financialization of housing ownership.
% (Centering on two dominant stories, a story of supply and demand and one of rights.) %FIX and cite
There has been less work on the implications for productivity. 

The productivity implications of the housing crisis are the focus of this thesis.
%(#e DEFINE TO SET UP THE DISTRIBUTIVE FEATURES OF ECONOMY). 
 % #e ADD: the effects of housing affordability are pervasive / complex /run through the whole system / go far behind the obvious /direct effects / extend in non-obvious ways through the economy/whole system. What is at stake at a broader level is. 
% Yet, the economics is clear that what's at stake is the productivity of cities, the distributive features of the economy and the impact of the middle class % THIS IS A RESULT NOT AN INPUT. WHAT GOES HERE? #E MAYBE ADD A CLARFIYING CONCLUDING SENTENCE HERE.... TO SAY SOMETHING LIKE THE EFFECTS OF HOUSING PRICES ARE NOT LIMITED TO 
The greatest price increases have been in cities, where where people live and work, where %a great deal of 
production is concentrated, %in addition to being where wealth is created and accumulated, cities are also 
and where income is distributed. With humans becoming an increasingly an urban species, cities are a primary driver of technological development and increasing wealth. 
%Behind these observations is a

A fundamental feature in recent empirical work on scaling laws is % demonstrated in the recent literature on scaling laws: 
the persistent relationship between density and productivity. The productivity of cities increases super-linearly with population. Cities are the locus of a positive feedback loop. Rising populations raises productivity, rising productivity attracts more people and resource.

% (TIE BACK TO HOUSING CRISIS - EG. The affordability 
% \textbf{HOW WE ADD IT BACK IN}
This thesis presents a spatial model of the city that incorporates distributional issues and financialization and allows us to examine the productivity implications of the housing crisis. The model that incorporates the scaling of productivity in cities within a standard urban model. 

Our approach is constructed by drawing together insights from %a number of sub-literatures from 
economics and the study of cities, including rent theory, production functions, the standard urban model, growth theory, urban growth theories, financialization, and the theory of distribution.
We relate this to the scaling models from the study of complexity. This gives insight into %a deeper look at 
distribution in cities, the effect of financialization, and effect of distribution and financialization on the growth and development of cities. 

The urban model we've built is based on work from %those developed in 
geography, planning and urban economics. The organizing principle in  the spatial models of all three disciplines is an economic variable, land rent, which is the link to distribution, financialization and continuing productivity. %*** (another sentence on why this is great) --The space-less quality of the study of finance leaves out xyz GET PHRASING- CAN'T SEE- INTEGRATION OF SPACE NEGLECT GROWTH FACTOR. 

%\textbf{WHY IT'S BEEN MISSING} EXPLAIN BETTER HERE WHY SPACE HAS BEEN LEFT OUT, AND WHY THAT LET'S US NEGLECTS SPATIAL RENTS AND MISS THE RELATION BETWEEN SPACE AND PRODUCTIVITY.
% We see urbanization and continuing and financialization accelerating. Financialization is driven by capital seeking profits, but what is the source of the rents they capture? The answer is in conventional urban theory, which allows us to identify the spatial distribution of those rents and traditional rent theory, which allows us to understand the social relations of those, those rents, the classical economists spend a great deal of time on that question. And we're very clear about it.
%We're talking about what is the puzzle? This is the teaser for this thesis and this thesis offers an answer to and I've just started to suggest that the teaser is given that 

% fig
% #E I THINK THE PARAGRAPH ABOVE IS SAYING:  
Financialization is usually understood as capital seeking profit, but the source of the rents captured and the broader social and economic effects have not been fully captured. The current models for understanding financialization and it's effects don't predict the actual trends we are see. We argue this is because they miss the importance of space FIX. 
By integrating classical and neo-classical economic approaches with traditional urban theory, which allows us to identify the spatial distribution of those rents we can build a more comprehensive model of financialization and its effects. 

In short, the effects of financialization on cities and economies has not been fully accounted for because the tools of the different, relevant disciplines have not been adequately integrated. The current approaches to describing the financialization of housing and its effects predict / explain /account for the housing crisis and effects on home ownership and access to housing, but our work shows that there are broader effects that have not been accounted for / predicted. 

The housing crisis raises the question of whether Canadian cities can continue to attract people and accumulate wealth for its residents and industries, and whether they can even sustain their growth.

This fuller picture is made possible by bringing together classical rent theory, neo-classical XX and urban theory to create/and using/along with a agent-based model that allows us to .... (What the modelling technique enables) 

Our focus is land rents, but in the context of an urban economy. 

% \textbf{WHICH GIVES US THESE CONTRIBUTIONS}
% \section{Contributions}
%PROBABLY INTEGRATE WITH THE ABOVE, MAYBE MAP/HIGHLIGHT CONTRIBUTIONS IN A FIGURE/TABLE

%This work is important for understanding the current policy context. 
The analysis makes clear that in addition to the recognized distributional consequences, the housing crisis has productivity impacts that should be considered in developing urban and housing policy. Particularly it centers concern with implication for urban development of growing rent extraction by the financial sector. 

In addition to the core contribution linking housing and productivity, %(rent is key to financialization, however the main urban models don't observe the distribution of rents)
there are three methodological lines of contribution, and there are policy implications SUMMARIZE METHODOLOGICAL CONTRIBUTIONS

\section{Document overview}
% #E ADD: This thesis develops a conceptual framework for a model of the housing market and then describes the model and the insights it produces. 
There are three parts, Part \ref{part-background} gives the background and develops the % #E theoretical framework for our analysis??
theory for our analysis, Part \ref{part-model} develops the model and analyzes the results, and Part \ref{part-system}, puts the model and theory in the context of a larger system, using methods of systems analysis and design, then discusses the potential for interventions and policy.

Chapter \ref{chapter-background} sketches how this thesis relates to four major fields ..., and the role of space as a unifying factor across three of the fields.

In Chapter \ref{chapter-rent}, we link classical rent theory, neoclassical production theory, neoclassical growth theory, the scaling literature, and urban spatial models.

Chapter \ref{chapter-growth} focuses on growth. To show how our model is directly connected with this broad collection of linked theories, we use the Cobb-Douglas function, which is used across this entire range of literature 

% After we develop the mathematical description of the relationship among these will discuss  in more detail, rent theory and our contribution, scaling laws, ......  and other issues in the literature that draw on parts of this model and 

 % ???  apply to the specific situation we're in why rent theory is related to discussions of exploitation why it might lead the inefficiencies, whether or not this links with other important models in the literature.

Chapter \ref{chapter-space} is on space and the Alonzo model.

In Chapter \ref{chapter-financialization} we  provide a description of finacialization and show it is a a form of rent-seeking in the housing market and ?? the potential consequences of fiancialization in the housing market. 

In Chapter \ref{chapter-model} we  describe an illustrative agent-based model of the urban system. Most of the analysis of urban systems has employed analytical models with roots that go back to von Thunen () and more recently Alonzo. These models are extremely useful, but necessarily abstract from the concrete  and variable individual behaviour and  the details  of dynamics that make real cities path-dependent. XXX (Dawn) have shown that agent-based models can reproduce the features of the analytical models, at least in simple cases. 


In the context of this we position the results in the larger system, analyze the potential for a range of interventions, and suggest policy implications.

% After we develop the mathematical description of the relationship among these will discuss in more detail, various relevant applications, and issues in the literature that draw on parts of this model and apply to the specific situation we're in why rent theory is related to discussions of exploitation why it might lead the inefficiencies, whether or not this links with other important models in the literature.

% Because we draw on a wide range of methods and literatures, we discuss the relevant literature and  methodologies in the chapters where they apply 