\chapter{Introduction}

Cities are a central feature of human society - Human beings are increasingly an urban species. Cities are one of the primary sources of technological development and increasing wealth. Behind these observations is a fundamental feature demonstrated in the recent literature on scaling laws: the productivity of cities increases super-linearly in population. CXities are the locus of a positive feedback l;opop

Cities are where people live and work, where a great deal of production is concentrated, where wealth is created and accumulated, and where income is actually distributed. 

In modern Canadian cities today, a housing crisis raises the question of whether Canadian cities can continue to attract people and accumulate wealth for its residents and industries, whether in fact it can even sustain their growth.

Several explanations of the current situation are commonly proposed. The first is simply that the problem of housing is a supply and demand problem where supply is blocked by some features of urban regulation. The second explanation is that the distribution of income has changed in some way that mean a significant fraction of the population are unable to afford satisfactory housing, and therefore this is the problem that must be solved.  The third common explanation currently is that financialization of the housing market  is changing the way the city economy is working, redistributing income and potentially threatening the long term growth and wealth creating capacity of the city.

Ideally, an analysis of the current state of cities will incorporate supply, 

The phenomenon of growing productivity was initially identified and estimated in the economics literature production at the national level. The estimated functions linked capital and labour inputs to output.  Soon after the  the earliest econometric models of output  were estimated, it was found that equations were not stable over time. Productivity grew over time
(We can do the arithmetic with the cobb douglas to illustrate) 

Faced with this puzzle, Robert Solow introduced a term that was time dependent, and an entire literature developed to explain this term. One productive stream explained growing productivity in terms of agglomeration effects- more people, more workers more firm or more diversity of firms appeared to be associated with growing productivity. Two major schools emerge - roughly speaking,  the Marshallian explanation, which emphasized firm-level processes and the Jacobs model which focusses on the creative effect of agglomerations of people in cities. Both have receives empirical support.

(We can do the arithmetic with the cobb douglas to illustrate)

Louis M. A Bettancourt and others applied similar models at the level of cities, but rather than a time-dependent term, they introduced a population-dependent term and found evidence from cities around the world that productivity rose as population rose: The scale of the city has a positive effect. The result  was one of a wide range of scaling results identifies in a great variety of systems examined in the complexity literature 



