\chapter{Introduction}

Cities are a central feature of human society - Human beings are increasingly an urban species. Cities are one of the primary sources of technological development and increasing wealth. Behind these observations is a fundamental feature demonstrated in the recent literature on scaling laws: the productivity of cities increases super-linearly in population. Cities are the locus of a positive feedback loop: rising populations raises productivity, rising productivity attracts more people and resource,

Cities are where people live and work, where a great deal of production is concentrated, in addition to being where wealth is created and accumulated, cities are also where income is actually distributed. 

In Canada, there is a housing crisis. In the last few years, the need for affordable housing has come into focus as one of the most pressing issues facing Canadians. As more and more Canadians are finding housing unaffordable, the effects are being seen in everything from declining home ownership rates to an increasing number of Canadians unable to afford housing at all.

There has been extensive work on the drivers of the crisis, including supply shortages, stagnating incomes, and the finacialization of housing ownership.

There's been less work on the implications for productivity. The housing crisis raises the question of whether Canadian cities can continue to attract people and accumulate wealth for its residents and industries, whether in fact it can even sustain their growth.

This thesis presents a spatial model of the city that incorporates distributional issues and financialization and allows us to examine the productivity implications of the housing crisis. The model that incorporates the scaling of productivity in cities within a standard urban model. 
The urban model is based on those developed in geography, planning and urban economics. The organizing principle in  the spatial models of all three disciplines is an economic variable, land rent, which is the link to distribution, financialization and continuing productivity. (another sentence on why this is great)

The analysis makes clear that in addition to the recognized distributional consequences, the housing crisis has  productivity impacts that should be considered in developing urban and housing policy. 


OVERVIEW OF DOCUMENT

1. the core model and analysis - do a model of the endogenous dynamics of the model.
2. the resilience analysis -
but this is coupled with a larger system. we're interested in how it is coupled..

low interest rates have been key to financialization 
now they're going up?

we drive the system with signals to see how. and look at the external driving variables.

but what happens with changing interest rates? to explore we drive the system with external signal to explore how it is coupled with the larger economy and get an interesting resilience result, that it is actually a kind of ratchet pumping wealth out of communities on the upswing and on the downswing.
3.  policy analysis - finally we take a second step out to position the model within a larger dynamical system and do a systems analysis of the model and suggest policy implications. 

----------

% ?Ideally, an analysis of the current state of cities will incorporate supply, 

\section{Agglomeration discussion}

The phenomenon of growing productivity was initially identified and estimated in the economics literature production at the national level. The estimated functions linked capital and labour inputs to output.  Soon after the  earliest econometric models of output  were estimated, it was found that equations were not stable over time. Productivity grew over time
(We can do the arithmetic with the cobb douglas to illustrate) 

Faced with this puzzle, Robert Solow introduced a term that was time dependent, and an entire literature developed to explain this term. One productive stream explained growing productivity in terms of agglomeration effects- more people, more workers more firm or more diversity of firms appeared to be associated with growing productivity. Two major schools emerge - roughly speaking,  the Marshallian explanation, which emphasized firm-level processes and the Jacobs model which focuses on the creative effect of agglomerations of people in cities. Both have receives empirical support.

(We can do the arithmetic with the Cobb Douglas to illustrate)

Louis M. A Bettancourt and others applied similar models at the level of cities, but rather than a time-dependent term, they introduced a population-dependent term and found evidence from cities around the world that productivity rose as population rose: The scale of the city has a positive effect. The result  was one of a wide range of scaling results identifies in a great variety of systems examined in the complexity literature 


\section{Background}

\subsection{Drivers of the housing crisis}
supply and demand, stagnant income, and finacialization of housing

Several explanations of the current situation are commonly proposed. The first is simply that the problem of housing is a supply and demand problem where supply is blocked by some features of urban regulation. The second explanation is that the distribution of income has changed in some way that mean a significant fraction of the population are unable to afford satisfactory housing, and therefore this is the problem that must be solved.  The third common explanation currently is that financialization of the housing market  is changing the way the city economy is working, redistributing income and potentially threatening the long term growth and wealth creating capacity of the city.


\subsection{How we do the resilience analysis}

- what will we do?