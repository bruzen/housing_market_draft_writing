%https://www.thriftbooks.com/w/an-essay-on-urban-economic-theory_yorgos-y-papageorgiou_david-pines/10122668/#edition=9531987&idiq=14152085
BOOK BLURB
Over the past thirty years, urban economic theory has been one of the most active areas of urban and regional economic research. Just as static general equilibrium theory is at the core of modern microeconomics, so is the topic of this book - the static allocation of resources within a city and between cities - at the core of urban economic theory. An Essay on Urban Economic Theory well reflects the state of the field. Part I provides an elegant

PAGE 14

But the fundamental concept of a central place system was first elaborated in a comprehensive manner by Walter Christaller (1933), whose explicit aim was to understand the laws that determine the number, size, function and spacing of settlements over an homogeneous area. It has often been said that Christaller's deductive structure is a theory about the location of tertiary activities, which stands alongside the work of von Th\"Unen (1826) on the location of primary activities and that of Weber (1909) on the location of secondary activities. More importantly, it can be seen as the original integrative framework which introduced the basic concepts necessary for L\'osch's theory of economic regions. Stepping on Christaller's shoulders, L\'osch (1940) gave us in first approximation reasons why economic activities tend to agglomerate over an otherwise featureless plain. 

PAGE 15  Paul Krugman (1991) who applied a differentiated product approach within a monopolistic competition framework, developed by Dixit and Stiglitz (1977), in order to study how agglomeration shapes urban struc- ture. In all the papers that follow his approach there is a unique, fundamental agglomeration advantage manifested in a self--enforcing, dynamic manner ('cir- cular causation'). Namely, an increase of the population in a particular city implies an increased demand for brands which attracts new firms, each produc- ing a new brand ('backward linkages').
In Krugman (1991), where the dispersion force is provided by an immobile agricultural population, low transportation cost leads to agglomeration and high transportation cost leads to dispersion. By contrast in Helpman (1998), where the dispersion force is pro- vided by residential crowding, low transportation cost leads to dispersion and high transportation cost leads to agglomeration.

\href{https://medium.com/@OECD/the-productivity-and-equality-nexus-is-there-a-benefit-in-addressing-them-together-60a46ab4fd09}{
OECD research has highlighted how the rise in inequality over the last three decades has slowed long-term growth through its negative impact on human capital accumulation by low income families.}


Riucardo  defined rent as "the difference between the produce obtained by the employment of two equal quantities of capital and labour."

``The rent of land, therefore, considered as the price paid for the use of the land, is naturally a monopoly price. It is not at all proportioned to what the landlord may have laid out upon the improvement of the land, or to what he can afford to take; but to what the farmer can afford to give.'' — Adam Smith, An Inquiry into the Nature and Causes of the Wealth of Nations, Book I, Chapter XI "Of the Rent of Land" A Theory of the Urban Land Market
\vspace{1cm}
\textbf{ALONZO PAPER:}Chapter
A Theory of the Urban Land Market
By William Alonso
Book
Readings in Urban Analysis
Edition 1st Edition
First Published 2014
Imprint Routledge
Pages 10
eBook ISBN 9781315128061
Share
Share
ABSTRACT

 This chapter presents a non-mathematical overview, without trying to give it full precision, of the long and rather complex mathematical analysis, which constitutes a formal theory of the urban land market. It is a static model in which change is introduced by comparative statics. And it is an economic model: it speaks of economic men, and it goes without saying that real men and social groups have needs, emotions, and desires, which are not considered. \textbf{The chapter considers the urban businessman, who, we shall assume, makes his decisions so as to maximize profits. A bid rent curve for the businessman, then, will be one along which profits are everywhere the same: the decision maker will be indifferent as to his location along such a curve.} The household differs from the farmer and the urban firm in that satisfaction rather than profits is the relevant criterion of optional location.

\vspace{1cm}
\textbf{TRADE AND THE DIFFUSION OF THE INDUSTRIAL REVOLUTION Robert E. Lucas, Jr.
Working Paper 13286 http://www.nber.org/papers/w13286
NBER August 2007.}
We need to add a second feature, focusing on the role of cities as centers of intel- lectual interchange, as the recipients of technological inflows. Scale or agglomeration effects are central to this role.7 I will treat these as external effects, and modify (2) to the form FIX HAS ERROR 
%FIX HAS ERROR \[dh =μ[1-x(h)]^\xi  h^{1-\theta}H^\theta \hspace{1cm} (6) \] {\color{red}Think of the new term [1 − x(h)]ς as a kind of agglomeration effect, according to which the rate of technology inflow to any individual is an increasing function of the city population.} 

 \vspace{1cm}
\textbf{Urbanization and Growth Edited by Michael Spence, Patricia Clarke Annez, and Robert M. Buckley}

 In all known cases of high and sustained growth, urban manufacturing and services led the process, while increases in agricultural productivity freed up the labor force that moved to the cities and manned the factories. 

 robust empirical evidence shows that productivity increases with the size of cities. 