% \documentclass[11pt]{amsart}
% \usepackage{geometry}                % See geometry.pdf to learn the layout options. There are lots.
% \geometry{letterpaper}                   % ... or a4paper or a5paper or ... 
% %\geometry{landscape}                % Activate for for rotated page geometry
% %\usepackage[parfill]{parskip}    % Activate to begin paragraphs with an empty line rather than an indent
% \usepackage{graphicx}
% \usepackage{amssymb, amsmath}
% \usepackage{epstopdf}
% \DeclareGraphicsRule{.tif}{png}{.png}{`convert #1 `dirname #1`/`basename #1 .tif`.png}

% \title{Individual choice and Cobb Douglas optimization}
% \author{David Robinson, Kirsten Wright}
% %\date{}                                           % Activate to display a given date or no date

% \begin{document}
% \maketitle
% Source :speculation-model.tex
\section{variables}

USEFUL FOR UNDERSTANDING HOW MUCH PEOPLE SPEND AFTER THEY RETIRE. SET UP THE STORY. WE'D NEED TO THINK ABOUT HOW LONG THEY'D LIVE TO ASSESS IT

\begin{table}[htp]
\caption{default}
\begin{center}
\begin{tabular}{cp{8cm}}
r& global return on capital\\
U(h,x)  & individual utility for deriving consistent demand\\
 h	 &quantity of housing services for consumption \\
x     	&commodity consumption\\
$p$ 	& rental cost of a unit of housing for one period. \\%We will treat this as the $\frac{1}{T}$ of the capitalized expected stream of rent payments\\
 1			&price of the composite good, $x$\\ 
 r			& Return on capital, bank rate of interest available to a household\\
$P_{0}$ 		& Price to purchase a unit of housing at time $0$ \\
 $\kappa$ & the costs of housing transactions.\\
 $\delta_t$& discount factor for period t\\
T 			& remaining life\\
 $P_T-P_0$	&Price appreciation on the interval from 0 to $T$.\\
$\phi^s<1$	&Sellers share of sale price after transaction costs\\
$\phi^b>1$	&buyers cost with transaction costs\\
 $M\leq mP_0h$ 	&the size of mortgage, where $m\leq 0.8$\\
%	&the present value of mortgage payments to $T$.  Maybe $0.8M(1-\frac{1}{(1+r)^{T+1}})$\\
$C^C $&carrying cost of a mortgage. \\
$C^T$&Transaction cost cost of of ownership \\
$r^h$&rate of  return on the housing investment\\	
 % $P$& the path of housing prices\\
  Y & the path of annual income\\
$W_0$& Wealth at the beginning of the period (inheritance)\\
W&asset wealth, consisting of inheritance, $W_0$, minus mortgage, rate $M$,  plus sale price of home,  $P_Th$  ??? \\
%  E& the expected path of prices\\
%  &asset value of \\
%  & \\
\end{tabular}
\end{center}
\label{default}
\end{table}

We are considering market in which one good is desired  both for its use-value, $h$, and as a financial asset\footnote{ In many advanced economies, the average annual real return on housing assets between 1950 and 2015 lies between 5 percent and 8 percent, comparable in magnitude to that of equity investment but with a lower standard deviation (Jord\'a and others 2017) From IMF’s Global Financial Stability Report – April 2018}.   We need to work out the return on housing as an asset for a homeowner. We can treat the rental price, p, as the cost of the services from a unit it of housing. The market price of a unit of housing  should then be the present value of the housing services plus the present value of the expected price appreciation. 
\section{Individual choice}

The rate of return on housing is an element of household choice. A household chooses how much housing on any sort on offers is optimal. If the housing is purchased the return enters the lifetime budget constraint. The rental-ownership choice is made by selecting the alternative unit with the highest lifetime utility,. 

The life cycle hypothesis provides a simple way to integrate assets into housing market behaviour, so we assume the life cycle hypothesis holds for this person.\footnote{Modigliani, Franco, and Richard H. Brumberg, 1954, “Utility analysis and the consumption function: an interpretation of cross-section data,” in Kenneth K. Kurihara, ed., Post- Keynesian Economics, New Brunswick, NJ. Rutgers University Press. Pp 388–436.} In the simple case without discounting, she divides lifetime earned income, $\sum_t{Y_t}{T}$, and non-income wealth, $W$, equally over the remaining $T$ years of her life.  
    \[C = \frac{W+\sum_tY_t}{T}.\]

A low marginal propensity to consume out of financial assets is consistent with Poterba and Samwick\footnote{Poterba, J. and A. Samwick, Stock Ownership Patterns, Stock Market Fluctuations and Consumption, Brookings Papers on Economic Activity, 1995, 2, 295–357.}. They examine luxury good demand when stock prices rise, and their results find sensitivity only in the automobile market, leading Shleifer\footnote{Shleifer, A., Comment on ‘‘Stock Ownership Patterns, Stock Market Fluctuations and Consumption,’’ Brookings Papers on Economic Activity, 1995, 2, 358–59.} to suggest that the marginal propensity to consume from stock market wealth is negligible. Engelhardt\footnote{Engelhardt, G. V., House Prices and Home Owner Saving Behavior, Regional Science and Urban Economics, 1996, 26, 313–36} finds that housing wealth can have a large though asymmetric impact on consumption. Households reduce consumption by as much as 30 cents for a one dollar decline in house equity. However, when house equity increases, the impact on consumption is negligible. Case, Quigley and Shiller\footnote{Case, K. E., J. M. Quigley and R. J. Shiller, Comparing Wealth Effects: The Stock Market Versus the Housing Market, Cambridge, MA: National Bureau of Economic Research, Working Paper 8606, 2001.} found that for housing wealth of \$1 increase in house prices (with mortgage debt constant) results in a consumption increase of 5 cents to 9 cents in the current year.

We follow the classic  Ando and Modigliani paper and let the consumption in each period be a different but fixed fraction of lifetime wealth, $\Omega_t$.\footnote{Ando, Albert, and Franco Modigliani, 1963, “The ‘life-cycle’ hypothesis of saving: aggregate implications and tests, “ American Economic Review, 53(1), 55–84. }  Then $C_t=\Omega_t(\sum_{\tau}Y_{\tau}+W)$, with $\Omega_t$ plausibly being in the range 0.05 to 0.09 after Case, Quigley and Shiller. This amount is allocated to the purchase of housing services and the composite good x in each period. 

Individuals maximize utility  subject to spending  no more than C:
\[\displaystyle\max_{h,x}h^{\alpha}x^{\beta}\quad\quad \textrm{s.t.} \quad P_h+x \leq \Omega(\sum_{\tau}Y_{\tau}+W)\]

To keep the analysis simple we will assume that  housing is bought just once at the beginning of the analysis period. We will also assume that income is not affected by changing housing value.  .  
\[W= W_0 -P_0h+\delta_TP^e_Th-\kappa \]

$\delta_TP_Th-P_0$ is the expected gross increase in house value and $\kappa$ is the associated costs of housing transactions. 

The current price, $P_0$,  is established in the market at the time of purchase by bargaining between buyer and seller over the expected net gain, $\delta_TP^e_Th -P_0h-\kappa$. Policies or events that affect $\kappa$ or  $P^e_T$ will affect the path of prices.  

\section{Household behaviour}

USE A COBB DOUGLASS TO ASSESS IT.

We can use a Cobb-Douglas instantaneous utility function $ u(h_t,x_t\ )=h_t^{\alpha}  x_t^{\beta}$. This function yields a constant share lifecycle wealth going to housing. 

Lifetime utility is	\[\Sigma_{\tau=0}^T \delta_u(h_{tau},x_{tau}) \]


The first order condition for this problem is 
\[ \nabla u= \lambda \nabla \big[ ph+x - C \big] \]
The effect of expected housing appreciation then enters only through its contribution to the wealth term in C.
\[ \nabla u= \lambda \nabla \big[ ph+x -\Omega_t\big(\sum_{\tau}Y_t+W_0 -m P_0h-\delta_TP_Th + \kappa\big) \big] \]
so 
\begin{eqnarray*}
\frac{\alpha}{h} u&=& \lambda \big[p-\Omega(-mP_0-\delta_TP_T)\big]\\
 \frac{\beta}{x} u&=&\lambda\\
\end{eqnarray*}
Dividing,
\[\frac{\alpha x}{\beta h} =P_0+\Omega(0.8P_0-P_T)\]
so
\[h=\frac{\alpha x}{\beta (p+\Omega(0.8P_0-P_T)} \]
\begin{eqnarray*}
h^d &=&\frac{\alpha}{\alpha+\beta} u\\
x^d &=&\frac{\beta}{\alpha+\beta} U\\
\end{eqnarray*}



%\section{the investment dercision}

\section{I THINK THIS LARGELY REPEATS THE ABOVE - The Speculative motive in Housing Demand}

We are considering market in which one good is desired  both for its use-value, $h$, and as a financial asset\footnote{ In many advanced economies, the average annual real return on housing assets between 1950 and 2015 lies between 5 percent and 8 percent, comparable in magnitude to that of equity investment but with a lower standard deviation (Jord\'a and others 2017) From IMF’s Global Financial Stability Report – April 2018}.   We need to work out the return on housing as an asset for a homeowner. We can treat the rental price, p, as the cost of the services from a unit it of housing. The market price of a unit of housing  should then be the present value of the housing services plus the present value of the expected price appreciation. 

If a market was in equilibrium, the rate of return on a unit of housing should equal the rate of return on other assets. There would be no financial advantage to home ownership. There are some apparent advantages, including  tax exemption for capital gains on housing.\footnote{These exemptions hhave a significant cost for the government, and affect consumer behaviour. For a GE study of  the US system see Bank of Canada Working Paper 2013-33, September 2013, ``Housing and Tax Policy,'' by Sami Alpanda and Sarah Zubairy,}\footnote{Gervais (2002) considers a general-equilibrium life-cycle US economy with heterogeneous individuals, where agents can either own or rent a house. He finds that individuals at all income levels would rather live in a world where imputed rents are taxed, or one where mortgage interest payments are not deductible, and both policies have very small distributional effects in the long run}

%Start by ignoring discounting.

We need a function to calculate the return on housing.  Price appreciation on a unit of housing is $P_T-P_0$. Transaction costs including real estate fees, take a fraction from the value of the final sale, leaving $\phi^sP_Th$ where $\phi^s<1$ is the share remaining for the seller. Moving costs, legal fees and land registration costs add to the price of purchase, so that the full cost of purchase is $\phi^bP_0h$, where $\phi^b>1$

The net gain from appreciation with transaction costs is something like \[G^N=\phi^s P_T-\delta^T\phi^b P_0=P_0(\phi^s \frac{P_T}{P_0}-\delta^T\phi^b)\]%=P_0\psi(T)\]
where $\delta^T$ is the discount factor for $T$ periods. The expression $(\phi^s \frac{P_T}{P_0}-\delta^T\phi^b)=C^T$ summarizes transaction costs. $\phi^s$ and $\phi^b$ represent  fixed costs that  encourage longer ownership terms, decrease mobility and make the use of the housing stock less efficient. 

If  owners take on the maximum mortgage, $mP_0h$ and pay interest at rate $r$, %we can set the  undiscounted carrying cost of ownership is simply\footnote{
and do not pay down the mortgage the value of the infinite sum of payments $ rP_0$  minus the long tail of payments after T discounted is 
\[C^C= rM\left(\frac{1}{r} - \frac{1}{r(1+r)^T}\right)= mP_0\left(1- \frac{1}{(1+r)^T}\right) \]%=\Gamma(r,T)\]
%\[C^O= rTP_0\]
The return on the investment net of carrying costs is 
\begin{eqnarray}
R&=&G^N-C^C \\
%&=&P_0h\left(C^T-\right)\\
&=&P_0\left[ \left(\phi^s \frac{P_T}{P_0}-\delta^T\phi^b\right)      -m\left(1- \frac{1}{(1+r)^T}\right) \right]
\end{eqnarray}


The \textbf{effective rate of  return on the housing investment}, $r^h$ satisfies 
\[\left[ \left(\phi^s \frac{P_T}{P_0}-\delta^T\phi^b\right)      -m\left(1- \frac{1}{(1+r)^T}\right) \right]=(1+r^h)^T\]
%or
\[r^h=\left[ \left(\phi^s \frac{P_T}{P_0}-\delta^T\phi^b\right)      -m\left(1- \frac{1}{(1+r)^T}\right) \right]^{1/T}-1\]
and housing is a good financial investment if $r<r^h$,, i.e., id
\[r< \left[ \left(\phi^s \frac{P_T}{P_0}-\delta^T\phi^b\right)      -m\left(1- \frac{1}{(1+r)^T}\right) \right]^{1/T}-1\]

%When will it be the case that 
%\[r<\left[\phi^s \frac{P_T}{P_0}-\phi^b - \frac{1}{r}\left( 1- \frac{1}{r(1+r)^T}\right)\right]^{1/T}?\]
Writing $\frac{P_T}{P_0}= 1+r^p$, where $r^p$ ifs the compounding rate of growth of housing prices over the period $0-T$,  

\[r^p>\frac{r^T+ \frac{1}{r(1+r)^T}+\phi^b}{\phi^s}-1\]
