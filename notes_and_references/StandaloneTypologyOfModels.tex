

\documentclass[preview, 12pt]{standalone}
\usepackage{geometry}                % See geometry.pdf to learn the layout options. There are lots.
\geometry{letterpaper}                   % ... or a4paper or a5paper or ... 
%\geometry{landscape}                % Activate for for rotated page geometry
%\usepackage[parfill]{parskip}    % Activate to begin paragraphs with an empty line rather than an indent
\usepackage{tikz, pgf}
\usetikzlibrary{shadings, shadows, shapes, arrows, calc, positioning, shapes.geometric, decorations.markings,math,arrows.meta}
\usepackage{pgfplots}
	\pgfplotsset{width=7cm,compat=1.8}
	 \usepackage{calc}
\usepackage{mathtools,amssymb}
\usepackage{graphicx}
\usepackage{xcolor, colortbl}  
\usepackage{amssymb, amsmath}
\usepackage{epstopdf}
\DeclareGraphicsRule{.tif}{png}{.png}{`convert #1 `dirname #1`/`basename #1 .tif`.png}

\begin{document}
%%%%%%%%%%%%%%%%%%%%%%%%%%%%%%%
\section{Utility Function and budget}

\[ U=(G^\delta H^\beta  A_1^{\alpha_1}  A_2^{\alpha_2}A_3^{\alpha_31}\]
\[ U=(\frac{D}{p})^\delta H^\beta  A_1^{\alpha_1}  A_2^{\alpha_2}A_3^{\alpha_31}\]

\[pG=D=Y-M-\sum_jT_j\]

\begin{description}
\item[G]Hicksian composite good  (food) (for necessities $g$, use $G-g$)
\item[p]price index
\item[H]Housing  (for necessities $h$, use $H-h$. For houses sharing $H-s$ (for suite))

Housing can be a composite of size and other attributes: $H=H_s^(\eta_s) H_q^(\eta_q)\dots H_k^(\eta_k)$
\item[M] mortgage and housing costs 
\item[$A_i$] Amentity $i$
\item[$T_j$]Transportation cost $j$


\end{description}

\section{A Possible Typology of Models and Experiments}
While there  are very many variations on the basic urban model and many potential experiments with each model there are only a few of immediate interest if the goal is to text the ``resilience'' of equilibria.

These models may exhibit irreversibilities in variables such as distribution, homelessness, city form, and class structure. 


The  basic strategy for examining the system resilience is to shock a model (experiment) and then see if diagnostic variables recover. (This needs more precise expression.)


The first task is to select a subset of models an experiments that are of particular interests with respect to .

The second is to construct  a model that allows those case to be examined. Ideally the model would be easily adapted to other experiments.

The following is a an attempt to develop a typology with a clear  progresssive structure.

Feedback - wealth  allows  upgrading. This advantages the rich. Maybe this 


\subsection{Models}
\begin{enumerate}
\item \textbf{A: The Basic Model}

The workhorse of urban economics is the circular city model. Some feature of the central place generates rents. It may be that it is the only employment centre. It may be economies of scale to a single activity or synergies arising from various externalities\footnote{We are interested in agglomeration economies. The wage  structure would then be related to the population or industry  structure. Externalities driving agglomeration may be classified  into two types, the  or so-called ``Marshalian''  and ``Jacobs'' externalities.}. 


In the simplest model, the central place pays a uniform wage, $w$ to all employees, who have identical preferences and transportation costs. $w$ is an attribute of individual residents. Residents  purchase or rent equal quantities of land at differing locations $l$ for identical housing.  

There are transportation costs $T$ that depend on distance from the  central place, so land close to the central place is more attractive than land farther from the central place.  

The equilibrium concept is that a market with identical individuals with identical incomes and transportation costs will result in identical utilities. The result is that land rent must decline with distance from the central place to offset rising transportation cost. 

The size of the city is determined by population and lot size. Income and transportation costs will interact with lot size. The basic model can be initialized by matching the number of properties to the size of the population. 

If population exceeds the number of properties there are three margins to consider
	\begin{enumerate}
		\item The land supply can increase. There may be a conversion cost
		\item The land per-capita may decrease. This is not simple in a city with land-use regulations, zoning, and fixed capital in homes. A conversion process has to be defined
		\item A homeless population can emerge. 
	\end{enumerate}
	


It is convenient in this model to use a Cobb-Douglas utility function that has the property that a fixed fraction of income is spent on housing.  We can start with the assumption that earnings are fixed for the lifetime at the one-period wage, $w$. Then total spending on housing is $\beta Y, \beta <1$ and $ Y=w$. Let the transportation cost for a specific location $l$ be $T(l)$. The  equilibrium price at that location will be $P(l)= \beta Y-T(l)$.


It is convenient but not necessary to assume that land outside of the residential limit is costless. It is common to assume a fixed price for agricultural land. 

There is no fixed boundary and the size of the city is determined by the utility that can be achieved in competing regions of competing


\item \textbf{Y: The Basic Model with Income Differences}
This will result in segregation by neighbourhood depending on income. 

Income can be purely earning, which requires a distribution of $w$ across agents. Income  might include investment income, which  a private rate of return and a distribution of assets across agents. \footnote{A more subtle model could allow individual wages to be linked to the agglomeration of other workers - say engineers. we can imagine a city that has centres of agglomeration by profession or by complementarity. Depending on the production function, this should emerge endogenously.}
\footnote{Sufficient investment income could lead individuals to locate in cheap properties at the edge of the city.  Income might also be invested in property affecting the quality of a unit. This would require incorporating unit quality in the attribute list for each property, and introducing a quality preference  in the attribute s of residents.}


\item \textbf{L: The Basic Model with Locational Preferences}
This will result in segregation by neighbourhood depending on preferences.

One version would be include distance to the edge of the city as an amenity in the utility function. Another would be to locate amenities within the city. These would lead to higher prices near amenities.

A natural variant would be to have earning depend on location. If there were several locations  a polycentric city would emerge.

\item \textbf{T: The Basic Model with Varied Transportation Cost }
This will result in segregation by neighbourhood depending on income and Transportation costs. Experiments include cars for the rich and  transit. 

Diagnostics include change in total transportation cost and differential welfare effects.


\item \textbf{R: The Basic Model with a Rent-own choice}
This may result in the emergence of classes. Agents must have the capacity to borrow to purchase. Attributes of the agents and must now include  net assets,  an available interest rate, and a permissible mortgage.

We imagine a banker setting the mortgage rates and size. This can be done at the beginning of each period for each agent. 

With no income differentia we expect equal utiliites

\item \textbf{YR: The Basic Model with Earnings (Y) Differences and a rent-own choice}
This model is very likely to generate diverging classes as income differentials permit some to capture land rents from others. This is highly likely if borrowing costs decline with income and asset ownership.

\item \textbf{L: The Basic Model with Variable lot size}
This is achieved by making lot size a choice variable for households, in which case we will get a tradeoff between transportation cost and lot size and distance. Results for this model are known. Density  falls with distance from the centre. 

\item \textbf{YL: The Basic Model with Earnings (Y) Differences and Variable lot size}
The wealthy choose larger homes and lots farther form the centre

\item \textbf{S: The Basic Model with constant lot size and variable density}
This is achieved by allowing stacking of housing units. Results for this model are not known. This introduces a step change in housing form, and emphasizes unit size.

This model should produce some interesting spatial patterns, especially if couples with the possibillity of secondary central places.

\item \textbf{YS: The Basic Model with Earnings (Y) Differences, constant lot size and variable density}

This model should produce some interesting spatial patterns, especially if couples with the possibillity of secondary central places.


\item \textbf{IR: The Basic Model with outside investors and rent-own}

\item \textbf{IYR: The Basic Model with outside investors, Earnings differentials and Rent-own choice} This model is of interest if borrowing costs decline with income and asset ownership.



\end{enumerate}
\subsection{Experiments}
There are various experiments of interest. You will have to pick key ones. It is not necessary to do all of them in every model. 

	\begin{enumerate}
		\item increase population
		\item increase wage\
		\item add hard boundary (limit land)
		\item Introduce differential incomes
		\item Introduce differential access to capital
	\end{enumerate}

\newcommand{\cred}{\cellcolor{red!30}}
\begin{table}[htp]
\caption{Potential experiments: \textbf{Pick some}}
\begin{center}
\begin{tabular}{|c|c|c|c|c|c|}\hline

  &\multicolumn{5}{c|} {experiments}\\ \cline{2-6}
Model  &1 &2  & 4 &4  & \\ \hline
 A& \cred& \cred  &  \cred & \cred  & \cred  \\
 Y& \cred   & \cred   & \cred   &\cred    &\cred   \\
 T & \cred   & \cred   & \cred   &\cred    &\cred   \\
 R & etc &  &  &  & \\
 L &  &  &  &  & \\
 S&  &  &  &  & \\
 I &  &  &  &  & \\
 YR &  &  &  &  & \\
 IR &  &  &  &  & \\
  IYR&  &  &  &  & \\\hline
\end{tabular}
\end{center}
\label{default}
\end{table}%

\end{document}


