
% \documentclass{standalone}
% \usepackage[dvipsnames]{xcolor}       \usepackage{calc}     
% \usepackage{tikz}
% \usetikzlibrary{shadings, shadows, shapes, arrows, calc, positioning, shapes.geometric}
% \usepackage{pgfplots}
% \pgfplotsset{compat=1.16}
% \usepackage{mathtools,amssymb}

% \begin{document} 
% \section{Notation for Urban and Production Sectors}

% $t$     & Time \\ 

t is time
TODO - uncomment the commented out ones with fraction- unidentified control sequence

delta is density ** - - infitesmal density increase as the city moves out. [[adjustment speed for wage N]] imagine a density function over the city. 
using the geometry for the derivative of that with respect - how does it change if you increase the amount under the integral, wehre does it go..  - in cont terms.  

N is population, n
d is diameter %(https://www.overleaf.com/project/606a6b286ae1c9f203fadab5 ). \\
omega is wage premium - which gives  the parented population
tau is linear transport cost per unit distance


\newpage
\begin{longtable}{lp{10cm}}
\caption{Notation}                       \\

\hline           &  \textbf{Productivity} \\ \hline
$K$              &  Capital               \\ 
$L$              &  Labour                \\
$N$              &  Population, which equals labour, $L$                  \\ 
$\alpha$         &  Elasticity of output with respect to capital          \\
$\beta$          &  Elasticity of output with respect to effective labour \\
$\gamma$         &  Elasticity of the urban agglomeration effect          \\ % , $\Lambda(n)$, for illustration \\
$Y=N^\gamma K^{\alpha }N^{\beta }$  &  Urban output                       \\
% $L$              &  Labour supply \\ %the number of workers, which, in the standard circular city model, equals the number of lots of size $s$  when workers live on identical individual lots. % Unless $d^{max}>d^*$ v  \frac{\pi}{s}(\frac{w}{\tau})^2 =
% $n_i$  &  Number of workers employed by firm $i$ \\
% n=\sum_i n_i$  &  Number of workers, the urban population in the model \\
% $\#f=\frac{n}{n_i}$&number of identical firms \\ %not used
% $f$  &  Number of firms =1 \\
% $n =f n_i$  &  Aggregate labour \\
% $\Lambda(n)$    &  Labour-augmenting agglomeration effect \\
% $n^\gamma$ & The labour-augmenting agglomeration effect,  modelled as an exponential function of the number of people \\
% $\Lambda(n)n_i$ &  Effective labour for firm $i$ \\
% $\Lambda'=\die{\Lambda(n)}{n} $ & Derivative of the labour-augmenting agglomeration effect\\

%%$Y_i=K_i^{\alpha }(\Lambda(n)n_i)^{\beta }$  &  Urban firm $i$'s output \\

%%$Y=\frac{n}{n_i}K_i^{\alpha }(\Lambda(\sum_i n_i)n_i)^{\beta }$  &  Aggregate output of all firms in the city \\
% $\die{Y}{n}=\beta\frac{1}{n} Y  \left( 1+ \frac{n\Lambda'}{\Lambda} \right)$  &  Social marginal product of labour \\
% $Y_i=K_i^{\alpha }(\Lambda(n)n_i)^{\beta }$    &  Urban firm $i$'s output \\
% $\die{Y_i}{K_i}	=\alpha \frac{1}{K_i} Y_i $  & Marginal product of capital for firm $i$ \\
% $\die{Y_i}{n_i}	=  \beta\frac{1}{n_i} Y_i $  &  Marginal product of labour for firm $i$ \\
%%$\eta=\frac{n_i\Lambda'}{\Lambda}$  &   Marginal agglomeration effect on a firm's output of increasing it's own labour stock \\
% \hline
	% &\textbf{Amenity}\\ \hline
% $A(d, n)$   &  Agglomeration amenity          \\

\hline           &  \textbf{Labour market}      \\ \hline %and urban stucture??
$\psi$           &  Rural wage                  \\ % 
$\omega$         &  Urban wage premium          \\
$\tau$           &  Transportation cost per unit distance \\
$d$              &  Distance of a residence from the centre of the city \\
$d^* = w/\tau$   &  City extent \\ % Maximum distance commuters will travel \\ % to get the wage premium \\
$\mathcal{R} = w-\tau d$ & Rent at distance $d$ \\ 
% $\zeta$          &  Population density at distance $d$     \\
% $s$              &  Lot size      \\
% $\psi$  &  ?Per-period cost of a unit of productive capital \\
% $\omega + \psi$  &  Urban wage including rural wage \\ %***
% $\textit{t}$ & {\color{red}transportation cost per km} \\%use   c?
% $w^n=w-\tau d$ & Wage  premium net of transportation costs \\
%% $\Omega=\frac{w+\psi}{\psi}$  &  Ratio of the urban wage to the  cost of capital \\
%% $\Pi$	   &  Profit \\
%% $ER$	   &  Excess return to capital \\ 
% \hline &\textbf{Spatial structure in the circular city} \\ \hline		
%% $d^{max} = w/\tau$  &  Maximum distance commuters at which residents enjoy the urban amenity \\
%% $d^{**} = max(d^*, d^{max})$  &  radius of the city \\
%% $U$                     &  Worker utility **\\ %, a function of location and prices \\
%% $U^{urban}=U^{rural} $  &  Migration equilibrium assumption ** \\
% \hline & \textbf{Labour market} \\ 

\hline           & \textbf{Financial market}    \\ \hline
$P$              &  Property price              \\
$\dot p$         &  Rate of price growth        \\
$\bar r$         &  Prime interest rate         \\
$r_{target}$     &  Investor or banks target interest rate, $\bar r + margin$ \\
$r =  r_i$       &  $i$'s personal borrowing rate  \\
$r_i^{disc}$     &  $i$'s subjective discount rate (possibly $r_i$)           \\
$\delta = \delta_i^T$ &  $i$'s discount factor for a period $T$               \\
$W$  & Wealth \\
$m = m_i(W_i)$   &  Wealth-based share of home price worker $i$ can mortgage           \\
$IS_i = IS_i(\omega+\psi$)  &  Income-based share of home price worker $i$ can mortgage \\
$\rho$           &  Rent ratio             \\
$\kappa$         &  Operations ratio       \\
$\sigma$         &  Property tax share     \\ % (also considered $\chi \Gamma$  \rotatebox[origin=c]{180}{$t$} \reflectbox{$t$})
$t$              &  Time                   \\
\hline
\color{black}
\end{longtable}  

\newpage
\begin{longtable}{lp{10cm}}
\caption{Bidding Mechanism Notation}                       \\
\hline
$P_0$            &  Purchase price              \\
$P^e_T$          &  Expected price at the end of period T \\
\hline
\color{black}
\end{longtable}  


%%\item[$$]
%%\item[$$]
%%\item[$$]
%Rural producers pay a wage $\psi$. this covers a standard house, lot, entertainment, diet and consumption pattern. We  choose units so that per-period cost of a unit of productive capital is also $\psi$


\section{Additional Assumptions}
\begin{enumerate}
\item There is full employment, no frictional unemployment, and no labour adjustment costs.
\item Firms set output and factor inputs to maximize profits, so factors are paid the value of their marginal private product
\item Demand for output is perfectly elastic (constant price = 1)

\end{enumerate}

