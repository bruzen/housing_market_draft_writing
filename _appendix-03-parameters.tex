\chapter[Model Implementation]{Model Implementation}
\textbf{May 29 -- added: T, a, b, d}
\section{Setting up the model:\newline parameters and initial values}
% It is convenient to set up the model using initial values that are close to equilibrium values 
% For some variables, we choose arbitrarily starting values that seem resonable, because that makes it possible to check that variables that follow from computation also seem resonable, as a kind of informal sanity check.
It is helpful if initial values are somewhat `realistic' to assess whether model behaviour is plausible. 
% We have to do some initial calculations because we want to start near the equilibrium value. 
There are several parameter values set. There are also endogenous values computed to initialize the model. These values are listed in sections \ref{sec-param-values}, and \ref{sec-init-value-list} respectively. The following sections discuss where those numbers come from, beginning with a discussion of constraints on parameter values, followed by a discussion of parameters and initial values for variables. % These are summarized in TABLE.

\section{Relationships that constrain parameters }

Explaining relationships that constrain parameters if the model is near an equilibrium path: 

CONSTRAINTS DIRECTLY FROM RESEARCH ON THE URBAN WAGE PREMIUM
\begin{enumerate}
     \item  Empirically the urban wage premium  $\omega$ is some fraction $p$ of the non-urban wage $\psi$.   This implies that if we set the subsistence wage, The urban wage is $(1+p)\psi$ and $\omega= p*\Psi$.
\end{enumerate}

CONSTRAINTS DIRECTLY FROM THE CORE URBAN MODEL
\begin{enumerate}
    \item From the urban model, the extent of the city is determined by transportation costs:   $\omega =c*width$.  Only two of these parameters can be set independently.

    \item From the urban model population is determined by area and density   $N=density* f^2$. we assume the city is symmetric  and we assume plots  of land are identical, so  - $area=f^2$. If there is a core (or seed) population it must be added.  
   
    \item From the urban model the urban wage is the subsistence wage plus the urban wage premium $\omega+\psi$ 
\end{enumerate}

CONSTRAINTS DIRECTLY FROM THE AGGLOMERATION/SCALE MODEL
\begin{enumerate}
    \item From the scale literature the value of aggregate output is $Y=AN^\beta$ for the city. This establishes a relationship between urban GDP and population. We initially make population equal to the workforce. 

    \item Empirically, $\beta \approx 1.13$
\end{enumerate}

CONSTRAINTS DIRECTLY FROM THE NEOCLASSICAL THEORY OF THE FIRM
\begin{enumerate}
    \item From neoclassical theory of the firm the competitive equilibrium wage is the  marginal product of labour in the firm.      This should hold both in the urban and rural economies. This implies       $ MPL^R=\psi$  and $ MPL^U=\omega+\psi $.

    Using a Cobb-Douglas production function, we can get explicit expressions for the marginal productivities of labour and capital. 
\end{enumerate}

CONSTRAINTS DIRECTLY FROM THE COBB-DOUGLAS PRODUCTION FUNCTION
\begin{enumerate}
\item The firm production functions are 
\begin{align}
Y_R  &= \quad  A_FK^{\alpha_F}L^{\beta_F}\\
Y_{R, )} &= N^\gamma A_FK^{\alpha_F}L^{\beta_F}
\end{align}
We start with firm labour forces  being equal at $n$.

\item $\sum_X \alpha_X \le 1$ is required for diminishing returns to scale. 

% \item From the theory of the firm with a general production function, an increase in the price of one-factor  results in increasing factor proportions for other factors.  
\end{enumerate}


\section{Parameter values}\label{sec-param-values}

\subsubsection{Time}
 The computational cycle is a year, so all time-dependent variables, such as wage, transportation cost, and interest rates, are specified for the yearly interval. The mortgage period is set arbitrarily, value 5 years. 

 
\subsubsection{City}
\begin{description}
\item [density] workers per lot, maybe 100 initially, Combined with the city extent, $f$, this gives suburban population. % A high value reduces the computation time by reducing the number of cells, but it means we have a representative agent in  each cell. 
Notice we can set density individually for each cell in the grid, so density is really a vector variable.  % Also explore a density of 1 with a higher seed population.
\item [seed population] $P_0=0$. % An initial population at the center that anchors the population. % We need to seed it with an initial core urban Cities aren't unseeded. Zero initially, adjustable to get reasonable behaviour or to explore additional factors such as a service population or a retired population. 
\item [baseline population]  $N_0=\mathrm{density} * f_0 * f_0 + P_0$ 
\begin{lstlisting}
 self.baseline_population = density*width*height + self.seed_population
\end{lstlisting}

\item [workforce of a standard rural firm] $n=100$. This is also the initial value for urban firms.


\item [subsistence wage] $\psi$ = \$40,000. %  that is within range of reasonable subsistence wage in Canada for after-tax income. Try also lower values.


\item [initial wage premium ratio] $p_0\in\{0.04,0.2,.30\}$ (empirical range from various sources). The percentage by which urban wages exceed rural wages 
\item [worker share of urban agglomeration surplus] 0.72 or 0.8 $\lambda=\beta_F$ (Workers are getting the same share of the surplus and they do of output with no agglomeration effects.)
\item [urban agglomeration coefficient $\beta$ = 1.12] for the city: elasticity of urban output with respect to population (empirical)

\item [scaling adjustment factor] $z\in[0,1]$ Begin with z=0.5. Scaling adjustment for calculating the number of new entrants. BAD NAME. ****

\item [equilibrium cost of transportation] $c = \omega/f$. Constant.
\item [property tax rate annually. $\tau=0.04$] This is applied to the assessed value.

\item[a: housing services share] Share of subsistence wage going to housing 
% services, a.

\item[b: maintenance share  of housing services ] Share of housing services going to maintenance.

\item [Annual maintenance]
\[\mathcal{O}=   ba\psi \]

{\color{red}
\item[Assessment ratio] 
Needed because the property tax is based on assessed value, which is less than market values. It is generally a fraction of a lagged value, say $\kappa$,  of the warranted price
\[P^{assesssed}=  \kappa P_{t-1}^{market}\]
$\kappa$ is associated with the tax rate multiplicatively. Annual property tax is $\tau\kappa P_{t-1}^{market}$

\item [Property tax] is based on the assessed price, $\mathcal{P}_{A}$, which is a lagged market price.

\[\mathcal{T} = \tau\kappa  \mathcal{P}_{A} =  \tau\kappa \mathcal{P}_{M, t-1} \]
}
\end{description}


item [Amenity]
$\mathbb{A}=0$
The local site-specific and urban amenity \textbf{function}. %Also considered symbols $\bar\forall$ and $\mathbb{H}$ for Hedonic value. 
 Assume it is zero initially.

\subsubsection{Land}
\begin{description}
\item[distance from center  $d$] 
\end{description}

\subsubsection{Person}
\begin{description}
{\color{red}
\item[subjective discount rate] set $\mathrm{delta} = 0.10$.  For a formula based on Kureishi et al, consider $\delta- 20-0.22*age$.}
\footnote{Note that this value is a subject of serious debate on several accounts. It is generally thought to be higher than the bank rate. In addition, it has been shown to be age-dependent. Using Japanese data Kureishi et al \cite{kureishiTimePreferencesLife2021} found the discount rate to be linearly and negatively dependent on age, as well as health, wealth, and schooling with a mean of 14.14\%  declining %(at about 22.% per year) 
to about 10\% (see Fig. 4) neat 65 years of age and more rapidly if risk aversion is accounted accounted for. 

 \underline{Kureishi et al Figure 3} \includegraphics[scale=.35, angle=-90, trim= 5cm  0cm 5cm 0cm,clip]{fig/age_dependent_discount_rates.pdf}
}



\item[working period] age or working period out of the total number of working periods. People are initialized with a random working period between 0 and the number of working period (40).
\end{description}

\subsubsection{Firm}
\begin{description}
\item [initial workforce of a standard urban firm] $n_0=n$.  The urban firm workforce will evolve over time: $n_{Ut}$. 

\item [price of output] = 1. (If you wanted to be able to vary this, you would put a price parameter in front the % of marginal products. 
\gls{marginal product of labour}. %pMPL)

\item [firm cost of capital] $r = 0.05$ or 5\%.

\item [initial wage premium]  \[\omega_0 = p \psi\] where $p$ is the wage premium ratio.

\item  [$\mathbf{\alpha_F}$] = 0.18 (typically 0.2 in macro models, but empirically values are lower, and lower values leave room for agglomeration effects),  for a single firm with a production function $Y_{Fi}=A_F K_i^{\alpha_F }L^{\beta_F}_i$ (empirical range =)\footnote{``In the U.S. from 1948-1995 the capital elasticity bounds were 0.18-0.33, and rose to 0.21-0.39 in 1995-2018.'' Dietrich Vollrath, (2021) The Elasticity of Aggregate Output with Respect to Capital and Labor. They also found that the capital elasticity in the private business sector is 0.09-0.28 from 1948-1995, and 0.16-0.33 from 1995-2018, and it is lower if intellectual capital is excluded.}

\item  [$\mathbf{\beta_F}$] = 0.72 (typically 0.8 in macro models, but empirically values are lower, and lower values leave room for agglomeration effects). Also called the elasticity of output with respect to labour for a single firm. (empirical)\footnote{Douglas  obtained a result for the exponent of labour of 0.75—which was subsequently confirmed by the National Bureau of Economic Research to be 0.741. Lower values  are associated with stronger decreasing returns to scale for the firm in our model.} 

\item [wage adjustment coefficient for new workers ] $adj^{new}_\omega=.5$

\item [wage adjustment coefficient for existing workers] $adj^{existing}_\omega=.5$
\end{description}
    

\subsubsection{Bank} % and mortgage parameters
\begin{description}
\item [mortgage period]  mortgage\_period (T)
\item [cost of capital for the bank] r\_prime. The bank's interest rate, $r$, is just the bank rate (r\_prime) set by the Bank of Canada.  
% The bank's interest rate, $r$, is just the bank rate (prime rate? set by the Bank of Canada. Exogenous. Just assign  a value like 4\%.

\item [r\_margin = 2\%] The bank's required markup on funds when it lends.  

\item [r\_target] reference rate that the bank demands on loans on 
$ r^{target}= r^{prime} +r^{margin}$ ? 

\item[lending rate/borrowing rate $r_i$]  
\[r_i=r^{prime} + r^{target} + \mathrm{individual\_wealth\_adjustment}\]

\item [{ability to carry a mortgage}] the share of income that can be used to cover mortgage interest. This represents the rule of thumb for households that they should not spend more than 25\% or 30\% of their income on on housing. We use 0.28.

\item [\gls{maximum mortgage share} of price] parameter in calculation of $m_i$. Specific to the functional form used. Set at 0.9. Based on stylized facts from the literature.  % No empirical estimates are available.
\item[wealth sensitivity] of maximum mortgage share of price. This is a parameter in the calculation of $m_i$. It is specific to the functional form used. Set at 0.1. The choice of a function is based on stylized facts from the literature.  %No empirical estimates are available. 
Used in Equation~\ref{eqn-wealth-based-mortgage}.
\end{description}

\subsubsection{Realtor}

\section{Transportation costs}
Transport costs have two parts:
1) fuel and vehicle costs per km
2) time costs per km

\subsection{Vehicle related costs}
Use one year as the wage period, converting transportation costs per km to annual cost for consideration in the household budget. Starting with the cost per km, calculate the cost per year:

\textbf{cost per km =$\textit{t}$}:. \$0.59   (from  Ontario data, 2021). sensitive to congestion, use of subways (\$5 /day?), 

 \textbf{work trips per year} 2 way * 5 days/week * 50 weeks work days = 500. [range: 450-550]

\textbf{cost per km-year} = work trips per year*cost per km

=\$0.59/km*500 trips/year  =  \$295/km year 

\subsection{Time costs}
\textbf{time per km}. range: 20km/hr -> 3min/km, 40km/hr -> (1.5min/km - 3min/ km)per trip 

(New York rush hour is much slower:  4-9km/hr ->6-15 min/km)

\textbf{time  per km-year} = work trips per year*time t per trip = 500* 3min  = 1500 min/km year = 25 hours= 3-3.5 days/km
 
\textbf{time cost per km-year} =  (days per km-year /work days/year)*wage premium per year  = 3/250 = 0.012 years/km year. ?

\textbf{money cost of time per km year} 

= time cost per km-year* wage(including subsistence) 

= 0.012 year* wage per year

\subsection{Total cost per km year of commuting for one agent}
\textbf{money cost of time per km year + \$295/km year * distance} \\
= (0.012 w+ \$295)/km year 
    \begin{quotation}
    \textbf{Example}
    To get a sense of the required wage if we have this annual cost structure, assume city\_extent $d^*$ is 30 km. At this point the transport cost is equal to the wage

\[(0.012 w+ \$295)/km year)*30 =  w\] 
\[.36w+ 8850=w\]
\[w=13828.12\]
\begin{quotation}
\textbf{PLAUSIBILITY CHECK}
This is plausible land rent, but does not include building rent. 
Capitalized at 5\% this house is worth \$ 276,562, a fairly cheap house 30 miles from city centre
\end{quotation}
\end{quotation}



\subsection{Value of transportation price to use in model}
\[ {c}=(0.012 w+ \$295)/km year \]




\subsubsection{Locational rent}  $\omega-cd$ 

\subsubsection{Warranted market rent} $\mathcal{R}_W = \omega - cd+ + a\psi + \mathbb{A},$
where $A$ is the local site-specific and urban amenity function. The Warranted \gls{market rent} is the amount needed to fully pay housing costs and locational rent. This will include taxes and maintenance. Use  $\mathcal{R}_W$ as an initial rent cost for tenants.\footnote{The market rent can in principle differ from the economically warranted rent as individual investors make decisions about rents and setting rents based on their own expectations, which include a wide range of factors. To simplify the model, we assume that the market rent equals the warranted rent. We are in effect imposing a long-term equilibrium value rather than modeling the adjustment process.} 

\subsection{Initial market rent} 
Initial rent cost for tenants  $\mathcal{R}_{M, 0}= \mathcal{R}_W$.

\subsubsection{Warranted market price}  
This is  an initial value for market price 
 $P_W=\frac{\mathcal{R}_W }{r}$,  

This is the present discounted value of the flow of housing services.  Use this as initial market price.  It can reasonably be seen as an equilibrium or attractor towards which prices should  evolve.  Notice that it includes $\mathbb{A}$

{\color{red}
Net benefit may be economically interesting later . You do not need to include it. 

If you own a home, after maintenance and taxes your \textbf{net benefit B} in each period is 
\[B= \omega-cd +(a-ab)\phi +\mathbb{A} - \tau P^{assesssed}\]
Capitalizing this we get a measure of the value of net benefits, 

If the assessed price is a fraction, say $\kappa$,  of the warranted price
\[B^{capitalized}=\frac{\omega-cd +(a-ab)\phi + \mathbb{A}}{(1- \tau\kappa)r}\]
}
\subsubsection{Annual property taxation}

\subsubsection{Net rent for investors} {\color{green}}
\[\mathcal{R}_N = \mathcal{R}_W - \mathcal{O} - \mathcal{T}= (\omega - {dc} + a\psi )- a b \psi - \tau  \mathcal{P}_{M, t-1}\]

The \gls{net rent} is the warranted rent net of costs. Homes must be maintained and taxes paid, so an investor would receive, not the total warranted rent, but the net rent after expenses,

\subsection{Initial values for endogenous variables} \label{sec-init-value-list}

These are \textbf{initial values} that we calculate based on the parameters above. These initial values provide  a starting point consistent with the production theory that we apply. They will be adjusted as the model iterates.  Some, like "Output for a typical rural firm", are simply intermediate variables used in further calculations.% that have to be chosen because they are used in calculating other values are:



\subsubsection{City}
\begin{description}
\item [initial market rent] 
Initial rent cost for tenants  $\mathcal{R}_{M, 0}= \mathcal{R}_W$.

\item [Initial market price] 
$P_0= P_W=\frac{\mathcal{R}_W }{r}$,  

\item[number of urban firms]  $F_0=\frac{N_0}{n_0}$. % **** TODO FIX - this says the baseline population over the number of workers per firm. - the initial value is this. The ongoing calculation is the number of workers $n_0$ is the initialu urban workforce. just count the workforce

\item[elasticity of urban output with respect to population.] $\gamma$. Combining the two marginal productivity conditions, 
\begin{align}
\frac{n\psi}{\beta_F}  &= \frac{n(\psi+\omega)}{N^\gamma \beta_F}  \\
N^\gamma &= (1+\omega)/\psi)\\
\gamma &= \frac{log(1+p)}{log(N)}
\end{align}

\item [agglomeration surplus $\mathcal{S}$] We define the agglomeration surplus as the difference between the output produced by  the city with population N and a set of $N/n$ standardized rural firms each with a labour force $n$. 
\[\mathcal{S_0}=A^U N^\beta-F_0Y_R \] 
This is illustrated in Figure~\ref{fig:Agglomeration-surplus}

The surplus can also be expressed in terms of the wage premium and the workers' share of the surplus:
\[\mathcal{S_0}=N\omega_0/\lambda=\frac{N\omega_0}{\beta_F}\] 
The logic is that the wage premium is the worker's share of the urban surplus. We have assumed that $\lambda=\beta$ as a convenience but we could 
The logic is that the wage premium is the worker's share of the urban surplus. 

\item[urban scale coefficient $A_C$] Combining the two expressions for $\mathcal{S}$
\begin{align*}
 AN^\beta-\frac{N}{n}Y_F    &=\frac{N\omega}{\beta_F}\\ 
 AN^\beta   &=\frac{N\omega}{\beta_F} + \frac{N}{n}Y_F \\
 AN^\beta   &=\frac{N\omega}{\beta_F} + \frac{N}{n}\frac{n*\psi}{\beta_F}\\
AN^{\beta-1}   &=\frac{\omega+\psi}{\beta_F}\\ 
  A_C&=\frac{\omega+\psi}{\beta_FN^{\beta-1}}
\end{align*}
\textbf{NOTE: A is scale independent and all of the population agglomeration effects, which are confined to $N^\gamma$.}\footnote{We have assumed that $\lambda=\beta_F$ as a convenience but we could say $\lambda=zork*\beta_F$, making $A=\frac{\omega/zork +\psi}{\beta_FN^{\beta-1}}$ }
\end{description}


\subsubsection{Land}
\begin{description}
{\color{red}
\item[property taxes]
\begin{align*}
\mathcal{T} &= \text{mill rate} \times  \\
&= \tau \times \frac{\omega_t- {dc} + a\psi.}{r}
\end{align*}
}
]
\end{description}


\subsubsection{Person}
\begin{description}
\item [savings] $age_i*savings\_rate*\psi$
\item[personal wealth] $W_{i0}= savings + warranted\_price $
\item[average wealth] $\bar W_{0}= \sum_i W_i$
\item[Subjective discounting] The discount factor gives the present value of one dollar received at a particular point in the future, given the date of receipt and the discount rate.

{discount rate vs discount factor}

We have called the  discount rate is $\delta$ (it might be better to call it $rho$.) We assumed it equal to $r_i$. This is a pretty strong assumption.

The discount \textbf{factor} $\delta(t)$ is a function of $\delta_i$ and the time until the event, $t$
% delta(0)=1  delta (1)= 1/(1+r) 
\[\delta(t)=\left(\frac{1}{1+\delta}\right)^t\]
\[delta(T)=  (1/(1+\delta))^T\]   
    

eg r=.05  t=5  $\delta(5)$ =  $(1/(1+.05))^5 = 0.7835262$


% growth rate= rt
% growth factor =($1+r)^t$
% discount rate= r
% discount factor = $1/(1+r)^t$   
\begin{lstlisting}
def get_discount_factor(self):
    """
    The discount factor gives the present value of one dollar received at particular point in the future, given the date of receipt and the discount rate.
    Delta is the subjective individual discount rate for agent
    after one year. This will be close to the prime interest rate, r_i.
    """    
    delta = self.r_prime
    delta_period_1 = 1 / (1 + delta) 
    delta_mortgage_period = delta_period_1**self.mortgage_period
    sum_delta = (1 - delta_mortgage_period)/delta
    return sum_delta
\end{lstlisting}
% #   sum_delta = delta_mortgage_period * (1 - delta_mortgage_period) # Old
Delta could also depend on wealth. For example,  use the bank rate, which is the rational rate but people who are poor typically have higher rates.  It would not change as the central bank changes r-pirme
% delta could be wealth based typically higher for poor.

% sum\_delta is sum of the infinite series minus discounted infinite series after mortgage\_period years
% Here, it is the present value of annual payments from one to mortgage\_period years e.g. of mortgage payments or rent received. 
% delta\_mortgage\_period was previously called   delta\_period\_T 
% # Note delta_mortgage_period is subtracted to subtract the long tail. 1/delta gives the PV of an infineite series of payments

\begin{lstlisting}
# A version with delta depending on wealth
wealth = self.wealth
delta =
\end{lstlisting}
% savings =(sum(0-age)((1+r)**age)*savings_rate*subsistence)


\end{description}




\subsubsection{Bank}
\begin{description}
\item[mean initial wealth $\bar W$] $= \frac {a\psi}{r_prime}+savings_0$
\end{description}








\section{Model execution}
Variables that are recalculated in each step.
% \footnote{\begin{itemize}
%     \item Variables have to have a time subscript. 
%     \item We also distinguish the urban firm from the basic rural firm.  Only the urban firm changes its behaviour.  I use a subscript $_U$. 
%     \item I omit the subscript for $n_t$ because only the urban firm changes its workforce size.
% \end{itemize} }


\subsection{City}
\subsubsection{Wage premium ratio}

$p_t= \frac{\omega_t}{\psi_t}$ 


\subsection{Land}
\subsubsection{Warranted price} The value of housing services including locational services. (Copied from the initial value calculation.)

\subsubsection{Maintenance costs for T} The discounted value (sum\_delta) of maintenance cost (b) on  physical housing ($a\psi$) over the mortgage period (T). This may be used in computing bids.

\subsubsection{Value of taxes for T}
Property taxes are imposed on the  lagged property price. They  are discounted and summed (sum\_delta)  over the mortgage period (T). This may be used in computing bids. A buyer assumes they are constant and paid at the end of each year of ownership during the mortgage period.

Each mill is expressed as  1/1,000 of the value as determine by assessment \footnote{By capitalizing the mill rate at 5\%  we see that each `mill' is worth about 2\% of the warranted rents. Assessments usually understate the market value considerably. Mill rates are commonly about 1.5 and differ between municipalities. ***MAYBE MOVE PARAM VALUES TO PARAM DISCUSSION}

\section{Older parameter value discussion}

\subsection{Price growth}
% """ Note agents forecasts are linear, but the population growth 
% is power law distributed. Thus the model is conservative and agents
% underestimates the the value of the urban center. Actuall effects  
% could be stronger. onsider alternative appraoches to forcasting value, risk, etc.#     What time frame do agents consider? Is their model linear?


\subsection{Wage premium} \label{section-wage-premium}

Hirsch Wage \cite{hirschUrbanWagePremium2019} observe that, ``Following Glaeser and Maré \cite{glaeserCitiesSkills2001},  a  large  empirical  literature  has  investigated differences in wages across labour markets of different sizes. The general finding of this literature is that a significant urban wage premium exists. and that this premium consists both of a level effect and a growth effect that arises as workers gain urban work experience.'' 

Almeida et al \cite{almeidaUrbanWagePremium2022} found for Brazil, that the female urban wage premium is on average 11.3\%, almost double the average male premium of 5.7\% and that higher in formal and informal jobs and across various agglomeration levels. The premium is larger in denser areas.

\subsection{Spillover effects} \label{section-spillover}

Spillover effects can be large. Irwin and Klenow  studied learning in chip production focusing  on the key issue of spillovers. They found learning rates of 10 to 27 per cent, averaging 20 per cent. They indicated that a good part of learning is internal, and that national spillovers were no greater than international spillovers. " \dots a firm learns three times as much from an additional unit of its own cumulative output as from another firm's cumulative output, regardless of the other firm's country of location. However, rest-of-world cumulative production is typically more than three times any given firm's cumulative production. This means that the absolute contribution of world cumulative production to each firm's experience outweighs the absolute contribution of its own cumulative production. In this sense, spillovers are substantial." (pp. 1217-1218).



% \section{TODO FIX/CUT Table of parameter values}
% \renewcommand{\arraystretch}{2}
% \begin{longtable}{lp{5cm}cp{2.5cm}}
% \caption{Table of parameter values}\label{tab:parameters}\\
% \hline 
% Symbol         & Name                                 & Value      & Formula  \\ \hline
% $a$ replace    & Share of $\psi$ for land and building &   0.3         & \\
% $b$ replace    & Share of $\psi$ for maintenance       &   0.2         & \\
% $tau$ replace  & Property tax rate &  e.g 1.6\% = 16 mills             & \\
% $c$       & Transportation cost & \\
% $T$       & Period & 5 years      \\
% $r$       & Individual interest rate & 0.05 \\
% $\omega$  & Locational rent & 0.012  \\
% $\psi$    & Subsistence wage & 10000 \\
% $a$       & Share of subsistence wage for land and building & 1.0 \\
% $\tau$       & Tax share & \\

% ---       &  & \\
% $m_i$          & Individual borrowing-ratio           & 0.75-0.85  & $M/P^{ask?}$ \\
% $M^{max}_Yi$.  & Maximum mortgage based on income     &            & $\frac{0.28(\omega+w)}{r_i}$ \\
%  $M^{max}_P$   & Maximum mortgage based on the price  &            & $0.8*P_0$ \\
% $IS$           & Income share for housing debt        & 0.25-0.35  & higher for the poor \\
% $\rho$         & Rent ratio                           &            & $\frac{\omega-tau*d_i}{P_0}$ \\
% $\kappa $      & Operations ratio                     & 0.1-0.3    & e.g. $ 0.2\frac{\omega-tau*d_i}{P_0}$ \\

% $\sigma$       & Tax ratio                            & 0.25-0.35  & e.g. $ 0.3\frac{\omega-tau*d_i}{P_0}$ \\
% $\dot P $      & Price growth                         & []         & $\frac{P_t-P_{t-1}}{P_{t-1}}$\\
% % $P^T_e$        & Expected price in T years            &            & $P_0(1+\dot P)^T$ \\ % *** WAS $P^e_T$
% $\bar r$       & Prime interest rate                  &            & \\
% $r^{target}$   & Target interest rate                 &            & $\bar r + margin$ \\
% $r_i$          & Individual borrowing-rate            & \multicolumn{2}{c}{ $r^{target}+ K/(W-W_{min}) -K/(\bar W - W_{min})$}\\
% $\delta_i$   & Individual discount rate        &  USE $\rho$?          & To assign \\
% $\delta_i$     & Discount factor for T                &            & $\left(\frac{1}{1+r_i^\delta}\right)^T$ \\
% \hline
% \end{longtable}
% \renewcommand{\arraystretch}{1.0}

\section{Amenity}

\chapter{Three Questions}

The three questions I promised to deal with were
\begin{enumerate}
    \item should we add interest to the net rent calculation? (yes)
    \item should I subtract the purchase price from the calculation of the value of a purchase?
    \item in the bid price, how is the present-value deflator computed?
\end{enumerate}

\subsection{The value of an investment}
The value of the investment $V$ is  the capital gain, $P_{T}-P_{0}$, minus financing costs, plus the discounted sum ((of rents and interest on rents net of operating costs and taxes)), which I will call $\delta sum\mathcal{R}_N$.  

\subsubsection{Defining Net Rent carefully}

The annual \gls{net rent} is the property rental price  net of costs. Homes must be maintained and taxes paid, so an investor would receive, not the total warranted rent, but the net rent after expenses,
\textbf{Market rent is used in computing net rent. The initial value for the market rent is \gls{warranted rent}}.

market or warranted..
Warranted market rent is defined as value of housing services:
\[\mathcal{R}_M = \omega - cd+ + a\psi + \mathbb{A}\]

The market rent may include  a fraction of financing costs $F(v, rM)$  that depends on how tight the market would support additional useful analyses:
 \[\mathcal{R}_M=\mathcal{R}_W + F(vacancy\ rate) rM\]

Net annual rent for investors is then  defined this way 

\begin{align}\mathcal{R}_N &= \mathcal{R}_M - \mathcal{O} - \mathcal{T} + F(v, rM)\nonumber\\
&= (\omega - {dc} + a\psi )+ \mathbb{A} - a b \psi - \tau  \mathcal{P}_{M, t-1}  + F(v, rM)
\end{align}

This term enters into the computation of housing prices. We assume it is deposited each year and earns interest $r$

The discounted value with accumulated interest in T years 
\[\delta-sum\mathcal{R}_N=\delta^T\mathcal{R}_N (1+r)\frac{(1+r)^{T-1}-1}{r} \]

% This net rent term has the form
%  \[\delta sum\mathcal{R}_N = \delta(T)H(\mathcal{R}_{N,1} +\dots+ \mathcal{R}_{,T})\]

\subsection{Answer to question 1} about whether to add interest to the rent term: YES

\section{The value of an investment}
 
 This may be written in terms of the purchase price, and several individual parameters: interest  rate, share of the price that can be mortgaged and  discount rate: 

 Changes to $V = \delta(T) \left(P_T -P_0- (1+r)M + \right)  \delta-sum\mathcal{R}^w_N $
 \begin{enumerate}
     \item we replace $\delta$ with $\delta(T) = \delta^T$ in  to reflect that discounting compounds. 
     \item we replace $\dot P$ with $\dot P^T$ to reflect that price growth compounds. 
     \item we replace $(1+r)$ with $(1 + sumT(1+r))$ to reflect the fact that interest payments accumulate.  
     \item we replace $\mathcal{R}^w_N$ with $\delta sum\mathcal{R}^w_N$
     \item we replace $(P_T - P_0)$ with $\dot P^T P_0$ in Equation~\ref{first_sub} to correct an arithmetic error
\end{enumerate} 
(This is terrible notation)
\subsection{Answer to question 2} 
Do we subtract $P_0$: YES 

Four change just reflect compounding details that were glossed over previously. The fourth corrects an arithmetic error.

\begin{align}
V &= \delta^T \left(\dot P^T P_0- (1 + sumT(1+r))M +  \right) +\delta sum\mathcal{R}^w_N   \\
&= \delta^T \left(\dot P^T P_0 - (1 + sumT(1+r))mP_0\right)  +      \delta sum\mathcal{R}^w_N \label{first_sub}\\
  &= \delta^T \left(\dot P^T    - (1 + sumT(1+r))m    \right) P_0 + \delta sum\mathcal{R}^w_N 
\end{align}
This is the net present value of buying, and selling after one planning period. All rates are scaled to the length of the period to avoid the need for compounding calculations. The function has 4 individualized  parameters, $\delta$, $\dot p$, $r$, $m$, as well as any factors that affect the rent term.


\subsection{Rate of return on investment}
We divide V by the size of the down payment, $D$ to get the required rate of return  

\begin{align}
r^{return} 
  &= \frac{V}{D}  \nonumber \\
  &= \left(\delta^T \left(\dot P^T - (1 + sumT(1+r))m\right) \ \right) \frac{P_0}{D}  + \frac{\delta sum\mathcal{R}^w_N}{D}      \nonumber \\
  &= \left(\delta^T \left(\dot P^T - (1 + sumT(1+r))m \right) \ \right)\frac{P_0}{P_0-mP_0} +  \frac{\delta sum\mathcal{R}^w_N }{P_0-mP_0}  \\ 
  &= \frac{\delta^T \left(\dot P^T - (1 + sumT(1+r))m\ \right) }{1-m} +\frac{\delta sum\mathcal{R}^w_N }{(1-m)P_0}\label{revised}
\end{align}
Although this looks more complicated, there is only one change. It looks different from the original because the compounding effects has finally been made explicit.



\section{in the bid price, how is the present-value deflator computed}

Using the simplified version of equation~\ref{revised} we calculated  as Equation~\ref{eqn-bid-price} that the maximum bid that achieves the target return on the down payment is
\begin{align*}
P_B^{max} & \le    \frac{\mathcal{R}_N}{(1-m)r^{target}-\delta \left(1 + \dot P_M^e - (1+r)m\right)}  \end{align*}

Using the extended and corrected version this becomes
\begin{align}
P_B^{max} & \le    \frac{\delta sum\mathcal{R}^w_N } {(1 + sumT(1+r))r^{target}-\delta^T \left(\dot P^T - (1 + sumT(1+r))m\ \right)} \label{eqn-bid-revised} \end{align}

$\dot P^T$  is the expected rate of change in the market price over the term of the mortgage.