%https://www.thriftbooks.com/w/an-essay-on-urban-economic-theory_yorgos-y-papageorgiou_david-pines/10122668/#edition=9531987&idiq=14152085
BOOK BLURB
Over the past thirty years, urban economic theory has been one of the most active areas of urban and regional economic research. Just as static general equilibrium theory is at the core of modern microeconomics, so is the topic of this book - the static allocation of resources within a city and between cities - at the core of urban economic theory. An Essay on Urban Economic Theory well reflects the state of the field. Part I provides an elegant

PAGE 14

But the fundamental concept of a central place system was first elaborated in a comprehensive manner by Walter Christaller (1933), whose explicit aim was to understand the laws that determine the number, size, function and spacing of settlements over an homogeneous area. It has often been said that Christaller's deductive structure is a theory about the location of tertiary activities, which stands alongside the work of von Thiinen (1826) on the location of primary activities and that of Weber (1909) on the location of secondary activities. More importantly, it can be seen as the original integrative framework which introduced the basic concepts necessary for L\'osch's theory of economic regions. Stepping on Christaller's shoulders, L\'osch (1940) gave us in first approximation reasons why economic activities tend to agglomerate over an otherwise featureless plain. 

PAGE 15  Paul Krugman (1991) who applied a differentiated prod- uct approach within a monopolistic competition framework, developed by Dixit and Stiglitz (1977), in order to study how agglomeration shapes urban struc- ture. In all the papers that follow his approach there is a unique, fundamental agglomeration advantage manifested in a self--enforcing, dynamic manner ('cir- cular causation'). Namely, an increase of the population in a particular city implies an increased demand for brands which attracts new firms, each produc- ing a new brand ('backward linkages').