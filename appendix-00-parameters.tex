\chapter{Initial values} \label{appendix-parameters}

% Here are the parameters as they appear in the code. 
These are the values passed to the program on initialization.  Some are initial values for endogenous variables, %These are {initial values} that we calculate based on the parameters above. These initial values 
% providing a starting point consistent with the production theory. 
for instance `\texttt{init\_F},' `\texttt{init\_n},' and `\texttt{init\_k};' %are initial values, adjusted as the model iterates. 
some, like ``output for a typical rural firm,'' are intermediate variables used in further calculations;  % that have to be chosen because they are used in calculating other values are:
and some are parameters.  While parameters typically remain constant through the run, experiments can perturb them. For instance, introducing a forcing function driving the `\texttt{price\_of\_output}' is a natural way to model the effect of business cycles on firm production. 
%notice that there is an indent on every line that must be removed 
% Details of the adjustment speeds and parameter setting are not discussed here because they have to be `tuned'. The parameter space of the model is large, and cities only exist in a sub-region of the space. For example, if the marginal product of labour is too low, cities cannot exist. If it is too high the city explodes. We have identified a region with plausible values in which city behaviour corresponds to what we know about the evolution of cities. 

{\small\begin{verbatim} 

# LABOUR MARKET AND FIRM PARAMETERS
`subsistence_wage': 40000.,     # psi wage fo rural worker
`seed_population':  400,        # 
`dist':   1, 
`init_wage_premium_ratio': 0.2, # 1.2,
`init_city_extent': 10.,        # initial city extent
`init_F': 100.0,                # initial number  of firms
`init_k': 50000.0,              # initial firm capital stock 
`init_n': 10,                   # initial firm size (number of workers)

# PARAMETERS MOST LIKELY TO AFFECT SCALE
`c': 500.0,                     # transportation costs
`price_of_output': 10,          # received by firms per unit of output
# `A'                           # Scale parameter  in production function. Our central value is 1800
`A_productivity_link': False,   # Flag for link between productivity and A. Eqn is A_base + (1-share) * A_slope, where share = 0 if False CHECK
#       "share" is share of rents capture by investors 
`A_base': 1200,                 # Fixed component of A
`A_slope':600,                  # Part of A from local investment 
`density': 100,                 # nmber of workers peer unit land
`alpha': 0.18,                  # Cobb-Douglas exponent for firm capital
`beta':  0.75,                  # Cobb-Douglas exponent for firm labour
`gamma': 0.003,                 # exponent for agglomeration effect of adjusted city population N
`overhead': .5,                 # extra cost associated with an additional employee 
`mult': 1.2,                    # ratiio of population affecting agglomeration to total number of workers 

# SPEED OF ADJUSTMENT PARAMETERS 
`adjN': 0.015,                  # Adjustment speed firm
`adjk': 0.010,                  # Adjustment speed firm capital
`adjn': 0.075,                  # Adjustment speed firm jworkforce
`adjF': 0.3,                    # Adjustment speed number of firms
`adjw': 0.002,                  # Adjustment speed wage paid by firm
`adjs': 0.2,                    # Adjustment speed aggregate worker supply 
`adjd': 0.2,                    # Adjustment speed aggregate worker Demand 
`adjp': 0.2,                    # Adjustment speed aggregate agglom_pop -Population

# HOUSING AND MORTGAGE MARKET PARAMETERS
`r_prime': 0.05,               # Bank rate, basic interst rate
`r_margin': 0.01,              # additional return bank requires to lend
`r_investor': 0.05,            # Next best alternative return for investor
`ability_to_carry_mortgage': 0.28, # share of total income that may be spent on mortgage
`wealth_sensitivity': 0.1,     # degree of mortgag3e tightness for low wealth
`mortgage_period': 5.0,        # T, in years
`working_periods': 40,         # years in workforce before retirement
`savings_rate': 0.3,           # share of income homeowners save
`discount_rate': 0.07,         # rate at which homeowners discount future income
`housing_services_share': 0.3, # a fraction of rural subsistence spent on housing
`maintenance_share': 0.2,      # b annual fraction of building cost spent on  maintenance
`property_tax_rate': 0.04,     # tau, annual rate, was c
`max_mortgage_share': 0.9,     # m share of property price elligible for mortgage
`cg_tax_per':   0.01,          # Capital gains  tax rate for owner-occupiers  (0-1)
`cg_tax_invest': 0.15,         # Capital gains  tax rate for investors  (0-1)
`investor_turnover': 0.05,     # fraction of investor land holdings put up for sale each period
`investor_expectations': 1.,   # "animal spirits," optimism ofinvestor estimate of capital gains (near 1)

\end{verbatim} }