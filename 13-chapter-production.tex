\chapter{Production Theory} \label{chapter-production}

\section{Neoclassical production theory}
The \gls{neoclassical} approaches to economic analysis built on the classical framework. They added  mathematical modelling and the use of calculus. Calculus made it much easier to derive formal statements of what came to be called `marginal conditions,' or rules for efficient choice. It also allowed economists to derive clear rules for complex cases that were very difficult to state verbally. For example, the utilitarian goal of `the greatest good for the greatest number' which is comprehensible but vague could be translated into a guideline expressed in terms of the sums of individual marginal utilities equalling the marginal social cost for each good. 

%*E THIS PARAGRAPH SEEMS TO MOVE VERY FAST %i THINK YOU NEED TO EXPLIAN MARGINAL CONDITIONS IN MORE DETAIL, REALLY MAKE THAT CLEAR THEN BUILD OUT THE UTILITARIAN GOAL AS MORE OF A FLESHED OUT EXAMPLE. IT SEEMS LIKE THE SKELETON OF WHAT YOU NEED TO INTRODUCE THESE CONCEPTS CLEARLY IS HERE BUT NOT ENOUGH. 

The results of this new style are analytically more precise than the verbal statements they replace, and more easily communicated accurately for those with some mathematical skill. The approach also tend to lay bare the precise condition---the background assumptions- that must hold for results to be true. A cost of this progress is that economics became less accessible for many people. 

It is important to remember that neoclassical economics did not overthrow the insights of the classical economists. It provided increased precision and power, and it shifted attention from questions of class and distribution to deriving efficiency conditions. Economics became less a collection of great thoughts about the economy and more a mathematical edifice that embodied most of those thoughts in a rigorously consistent way and was the path to ever more arcane insights.
% *E WHILE THIS IS INTERESTING AS AN ACCOUNT OF THE RELATIONSHIP BETWEEN ECONOMICS AND THE PUBLIC, I'M NOT SURE HOW IT SETS UP WHERE YOU ARE GOING WITH THIS CHAPTER. CAN YOU BRING IT BACK TO RENT OR DISTRIBUTION, HOW THE NEW APPROACHES RELATE TO PROBLEMS YOU ARE LOOKING AT IN THIS THESIS?

% \section{Neoclassical production theory}
The concept of a production function used by the increasingly mathematical neoclassical economists and  the rapidly developing statistical techniques  naturally led to attempts to identify the precise \gls{functional form} that would describe the contributions of labour, capital, and income to output.
% *E COULD YOU FLESH THIS OUT, ADD EXAMPLES % ALSO MAYBE ADD WHAT WAS HAPPENING AT THE TIME, WHAT WERE THE PEOPLE LOOKING AT WHEN THEY STARTED TO APPLY THESE TOOLS TO THESE QUESTIONS. THIS MIGHT BE A GOOD PLACE TO REFERENCE BACK TO PREVIOUS PARTS ABOUT MARGINALISTS TO SET CONTEXT. 
 
\subsection{Classical rent theory using neoclassical notation}
Ricardo, for example, did not write down a formal production function as later \gls{neoclassical} theorists would,\footnote{He did generate numerical examples to demonstrate comparative advantage, and others, such as the physiocrats, von Th\"unen and Marx developed models still cited today, but these were not central for classical theorizing.} but his verbal production theory can be put in the notation neoclassical economists later developed. In modern notation, Ricardo's production model can be written: 
% *THIS IS CONFUSING DID YOU PUT IT IN THIS FORM OR DID THE MARGINALISTS. NOTE.. %JUST ABOVE YOU SAY THE MARGINALISTS STARTED TO DO THIS. iS THIS AN EXAMPLE? OR YOU DOING WHAT THEY ALSO DID? cLARIFY. OR THIS CONFUSION MIGHT JUST GO AWAY WHEN YOU FILL OUT MORE EXAMPLES ABOVE

\begin{equation} 
Y=F(K,L,N).
\label{eqn-production-ricardo}
\end{equation} 

where  $Y$ is output, $K$ is capital invested, $L$ is labour and $N$  is the natural resource land.\footnote{This makes it a three-factor model of production.  In principle any number of factors can be included.}  
Ricardo does not specify a functional form, but, %like mathematical neoclassical economists, 
he does assume diminishing returns to all factors. 

The  rent, $\mathcal{R}$, that the landlord receives is the total market value of the potatoes produced minus the cost of the capital employed in improving the land and the wage bill for the labour employed: 

\begin{equation} 
\mathcal{R}= PQ-rk-wl
\label{eqn-rent-ricardo}
\end{equation} 
Equation~\ref{eqn-rent-ricardo} makes it clear that the rent is a residual or a \gls{surplus}. The land is not paid for its services. The value of the land is the \gls{present discounted value} of the surplus it generates for the owner, that is what it would be worth paying now, to capture the future rents from that land.

Ricardo's analysis of rents  can be expressed by focusing only on land:

\begin{equation} 
Y=F(L,N).
\label{eqn-production-ricardo-2}
\end{equation} 
while most modern neoclassical treatments of production simplify by omitting land and emphasizing capital:\begin{equation} 
Y=F(K,L).
\label{eqn-production}
\end{equation}  
This makes sense for a number of reasons. The economy has shifted from agriculture to industry and the focus of economic theory has shifted to manufacturing processes.

%Furthermore, according to the Ricardian theory, rent is a surplus above cost. It does not, therefore enter into price. Land is a fixed factor for society as a whole that is not consumed in  the process of production.  Neoclassical treatments of production focus price determination based on the cost of the last unit used, the marginal  unit of input, while rents are generated on all of the inframarginal units, those units used earlier, which are more productive. The marginal unit of land generates no rents. In neoclassical analysis, the rents disappeared from view for this reason. This difference is at the heart of the distinction between classical and neoclassical economic theory. 
% #E this distinction is because of the actual structural difference in the societies? Also because of the differ tools? and apporaches? 

 %The Principles tells us that as cultivation is extended and exchange increases, profits fall while rents increase. 

Leaving land out, however, creates a problem in  the neoclassical growth theories we will examine below. John B. Davis \cite{davisRicardoTheoryProfit1993} noted that ``Questions arise, however, when one turns to exchange between a sector paying rent and one not.'' 
Under the assumption of perfectly competitive goods and factors markets as well as marginal productivity pricing of capital and labour, neoclassical growth requires technical change to be generated outside the model because there are no resources left to innovate if both factors of production are paid their marginal product.\footnote{This follows from Euler’s theorem: if, for a given level of technology $\bar A$ output Y is produced according to a \textbf{constant returns to scale} and twice continuously differentiable function of capital and labour $F(K, L, \bar A)$, Euler’s theorem implies that $F_K K + F_L L=Y$, where $F_i$ is the marginal product of factor $i$. Payments to  capital and labour take up the entire national product and no resources are left to finance the production of technology-improving innovations. are paid their marginal product.} 
If, however, land is reintroduced, as it must be in an urban model, there must be rents and there is therefore a surplus available for innovation.
\footnote{An alternative and common approach is to assume imperfect competition, which may be based on increasing returns to scale, in which case firms with market power may achieve a surplus. ``Although seldom modeled outside the monopolistic competition framework, market incompleteness and imperfect competition are central to the new growth theories'' (Gilles Duranton, Growth and imperfect competition on factor markets: Increasing returns and distribution, European Economic Review, 44-2, 2000, 255-280), Similarly, Sjak Smulders and Theo van de Klundert conclude that ``Growth is higher in a more concentrated market provided that market power of firms is not too high,'' (Imperfect competition, concentration and growth with firm-specific R \& D, European Economic Review, 39-1, 1995,139-160).}

\subsubsection{The Cobb-Douglas Production Function}

In 1928, mathematician Charles Cobb and Economist Paul Douglas came up with a specific and very convenient functional form \cite{cobbTheoryProduction1928}\footnote{The function had apparently previously been used by Knut Wicksell, Philip Wicksteed, and L\'eon Walras.} that captured much of what economists were talking about. The function is just a \gls{generalized arithmetic mean}:
 
 \[Y=AK^\alpha L^\beta\]
 where $A$ is a constant scale factor, now commonly called \gls{total factor productivity}. This function becomes the workhorse of neoclassical growth theory in the second half  of the 20th century. Our urban model is a direct heir to developments in neoclassical growth theory.
 
%The Cobb Douglas function has several convenient features. One is that the sum of the coefficents tells us the degree of returns to scale. If $\alpha+\beta = 1$, we have constant returns to scale,

%Another is that the coefficients of the factors, $\alpha$  and $\beta$ turn out to be the elasticities of output with respect to capital and labour respectively as well as the income share of the factor. These made it relatively easy for economists to combine national data on labour and capital stocks or income with output to test the model.

The Cobb-Douglas form was developed and almost immediately tested against statistical evidence in the USA by Cobb and Douglas between 1927--1947. It was  their widely circulated empirical work seems to have permanently associated this simple function with Cobb and Douglas for economists.


%I have not followed this track down to give references.

% ALSO Imperfect Competition and \gls{total factor productivity} Growth  AZZEDDINE M. AZZAM, ELENA LOPEZ and RIGOBERTO A. LOPEZ. Journal of Productivity Analysis. Vol. 22, No. 3 (November, 2004), pp. 173-184 (12 pages)

%Sjak Smulders and Theo van de Klundert.Imperfect competition, concentration and growth with firm-specific R & D European Economic Review. Volume 39, Issue 1, January 1995, Pages 139-160
% Duranton, Gilles (1997) Essays on growth: imperfect competition, labour supply and local public goods. PhD thesis, London School of Economics and Political Science.  http://etheses.lse.ac.uk/1471/1/U105715.pdf

%\footntoe{Alberto Bucci.  R&D, Imperfect Competition and Growth with Human Capital Accumulation, 2003. Scottish Journal of Political Economy. https://doi.org/10.1111/1467-9485.5004004. This paper studies the long-run consequences of imperfect competition on growth and the sectoral distribution of skills within an R&D-based growth model with human capital accumulation. We find that steady-state growth is driven only by incentives to accumulate skills. In the model imperfect competition has a positive growth effect, while influencing the allocation of human capital to the different economic activities employing this factor input. Contrary to general wisdom, the share of resources invested in R&D turns out not to be monotonically increasing in the product market power and its correlation with the equilibrium output growth rate is not unambiguous.}

%NOTE URBAN COMPETITION PROVIDES INCENTIVES TO UPGRADE SKILLS!!!


% Both the Solow (1956) growth model and its Ramsey-Cass-Koopmans counterpart featuring an endogenous saving rate (Ramsey, 1928; Cass, 1965; Koopmans, 1965) but treat technical change as purely exogenous. In fact, under the assumption of perfectly competitive goods and factors markets as well as marginal productivity pricing of capital and labour, neoclassical growth requires technical change to be generated outside the model because there are no resources left to innovate if both factors of production. 

% 
% assuming that technical progress is labour augmenting (Uzawa, 1961), we can rewrite the production function as $F(K, AL$), where $AL$ is a measure of labour in efficiency units, or effective workers. Let k = K/(AL). Then, output per effective worker is y =Y/(AL)=f (k). Population grows at the constant rate n > 0 and, as we will assume throughout the whole paper, capital does not depreciate. The steady state of the Solow model solves

% $\frac{f(k_{ss}}{k_{ss}} = \frac{n+g_A}{s}kss s$

% Journal of Economic Surveys (2017) Vol. 31, No. 5, pp. 1272--1303 \c ECONOMIC THEORIES 1275


Classical rent re-appears in neoclassical theory as `economic rent' (``a money payment made for a factor of production that is over and above the minimum payment to keep it in its present use,'')  as quasi-or pseudo-rents (non-equilibrium rents that will be competed away in a competitive equilibrium according to Marshall.\footnote{see Lewis Cecil 4 Rent Under the Assumption of Exhaustibility, Quarterly Journal of Economics, May, 1914, Vol. 28, No. 3 (May, 1914), pp. 466-489}),  as consumer  and producer surplus in supply and demand analysis,  as rent profiles or Pseudo-rent curves in urban theory, as a major concern on resource economics, and the theory of rent-seeking. Economic rent is a surplus insofar as its payment is not necessary to ensure a supply of a particular factor of production. 


% HOUSING RENT IN THE NATIONAL ACCOUNTS
%   Owner-occupied housing is included in Peersonal Consumption Expenditure because the National Income and Producgt Accounts (NIPAs) treat the owner-occupant as if it were a rental business, or in other words, a landlord renting to him or herself. That is, BEA imputes a value for the services of owner-occupied housing (space rent) based on the rents charged for similar tenant-occupied housing, and this value is included in GDP as part of personal consumption expenditures. This imputation is necessary in order for GDP to be invariant when housing units shift between tenant occupancy and owner occupancy.

%Ricardo  clearly understood and used the concept of diminishing marginal product. This shows in his use of the terms ``extensive margin'' and ``intensive margin'' to explain the income of the landowner. He focussed on the difference between the cost of production on a unit of land and the revenue generated. The landlord would rent out all the land which generated at least enough to pay all the costs. Anything in excess of the costs could be charged as land rent to a tenant farmer.

%Clearly in his model there are two basic productive factors, land and labour. The landlord  receives the surplus generated by the land and the rest of the value of production goes to labour. 
Recent urban models, on the other hand, tend to ignore the production process and consider the locational implications of land and transportation costs on the location of people. Wealth distribution is often ignored. 

% \subsection{Marginalist dis tribution}
% MARGINALIST DISTRIBUTION
% we've been paying some people less than the market wage so our profits our higher. this is what it would be if we paid everybody

% FOOTNOTE - RELATIONSHIP with marginalist distribution story ******** TODO Does the marginalist approach assume they are not exploited? Is it an experiment in examining the case where production is non-exploitative? 
% In a sense if labour gets the marginal value of their product, are they exploited. It's a matter of interpretation.  -It has an attraction 
%Clark tried to make an ethic of this. if everyone is being paid the marginal product of their labour. We know that's an efficient outcome. If it's efficient, is it also fair
%Is it possible someone's taking out an extra large fair. Yes. Not fair for simple classical reason that labour has been exploited in the past and that the current owner ship is a result of exploitation. The ownership of land introduces a kind of exploitation-- clearly exploitation if you claim that. 
%Lot's of marxists didn't like Henry George making it a locational question, they wanted to keep it located in the factory.
% You could - well what value did they create---in line with those other-- could interpret.. 
%What is the average value, because every worker is not just marginal, they're also average/identical. What is the value created by the whole of the workforce. Should they be paid the marginal value or the average value of their work.
%
%The avg value---declining.. 
%The demand for labour is declining--  
%Every infra marginal worker has been paid less than the avg contribution 
%Every infra marginal workers should - 
%every marginal worker should get the average wage.. that's fair.

%Get to the margin - that's what you pay.. that's what the next worker is worth to the firm. .. 5th' worker is paid more than the 10th. should it be averaged out and paid to all workers? paid to worker, or should the difference between top and the marginal goes to the firm---that's profit.. pay everyone the marginal value and keep the rest as profit.. 
%Effective labour has a higher marginal product.. - even higher - higher for the firm.. - but they don't have to pay the workers that... firms only have to pay enough to get their marginal individual cost down to the wage. The problem there is if they're making more profit they want to expand the workforce, but that wage only supports a certain size of city---they've got off raise the wage a bit.. so they face an upward sloping supply curve for labour=-- that's why you know there's an equilibrium.. declining product and upward sloping supply so they cross.

%(all the profit you earn on the way could be redistributed)

\cite{arvidssonUrbanScalingLaws2023} find that cities’ tails are responsible for 36--80\% of the observed superlinearities across indicators. 


