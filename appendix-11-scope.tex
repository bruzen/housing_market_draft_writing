% \chapter{Scope of the basic model}
\chapter{Simplicity in Modelling}

% TODOO SHOULD THIS ALSO GO IN METHODOLOGY
% defining boundary between the parts?

The goal of modelling in general is to have a accessible and useful description of reality.  Mader et al \cite{maderConstructionVerificationModels2007} suggest a set of properties of a good model. They emphasize that it should have a clearly specified object of modelling and  a clearly specified purpose. They go on to list the  following additional epistemological criteria:
\begin{enumerate}   
\item  Truthful. (The model has to represent the relevant behaviour of the system we are describing.) 
\item  Complete (It must include necessary features of the system.)
\item Simple,  (It must be possible to verify and debug the model.)
\item Understandable. (Elements should be clearly described or derived and documented.)
\item  Traceable.  (It should be obvious what elements of the artefact went into which design decision and are reflected at which point in the model.)
\item  Efficiently constructable and maintainable. 
\end{enumerate}

Frits Vaandrager \cite{} %(2010 \url{http://www.cs.ru.nl/~fvaan/PV/what_is_a_good_model.html})   
augmented their list to produce the following list. A good model: \begin{enumerate}
     \item has a clearly specified object of modelling,
     \item  has a clearly specified purpose
     \item is traceable: each structural element of a model either (1) corresponds to an aspect of the object of modelling, or (2) encodes some implicit domain knowledge, or (3) encodes some additional assumption.
     \item  is truthful: relevant properties of the model should also carry over to (hold for) the object of modelling
     \item is simple (but not too simple)
     \item is extensible and reusable
     \item has been designed and encoded for interoperability and sharing of semantics.
 \end{enumerate}

Simplicity is the key to achieving these goals. % That leads us to a long list features of real urban systems that 
We intentionally leave features out of  our model to focus attention on the features that in our view matter most. 


We have three basic hypotheses:
\begin{enumerate}
    \item Financialization of the urban housing stock extracts wealth produced by human agglomeration 
    \item Financialization of the urban housing stock changes the class structure of urban society
    \item Financialization of the urban housing stock can limit urban productivity growth
\end{enumerate}
Each  of these represents a high-level and general proposition. In considering whether to add a feature to the model, varying housing density for example, we ask: 
\begin{enumerate}
    \item is it necessary to demonstrate the principle? 
    \item would incorporating it result in falsifying our result?
    \item would incorporating it result in a qualitative difference in our result?
    \item would incorporating it result in clarifying our result?
   % \item can it be added at a later point to get more neuanced results? 
\end{enumerate}

For example, % It should be clear that, 
while varying housing density across our model urban system is technically easy to do, it is not necessary to demonstrate any of our three propositions. Including density variation would not undermine our results nor clarify results them. %It is simply not useful to incorporate this feature in our model. 

We can say the same about: 
\begin{enumerate}
    \item adding a detailed production sector with multiple firms,
    \item adding a detailed labour market,
    \item adding households of different sizes,
    \item adding an income distribution,
    \item adding a range of distinct occupations,
    \item incorporating more complex lifestyle choices,
    \item articulating the transmission mechanism from productivity to wages and from wages to population,
    \item incorporating building costs, 
    \item adding developers and,
    \item including zoning regulations. 
\end{enumerate}


Each of these extensions is interesting and would add fine-grained detail to our understanding of how the effects of financialization of the housing market are distributed. None of them is likely to affect the qualitative results or make our argument easier to grasp.
% While making it 
% In the thesis, we have described a stylized model to  establish how the urban system generates rents. Even with its simplifications, the model can describe key features of urban structure and urban history, and addres the core hypotheses.   %In this section, we illustrate some of the insights supported by the model. 
% Extensions can incorporate variations in wages, density, transportation costs, preference, and even building technology and codes. The limitations of the simple, continuous, equilibrium-based versions described above can be overcome using agent-based models to model the evolution of complex and much more realistic urban systems. 
