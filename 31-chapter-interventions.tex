\chapter{Interventions - TO ADD} \label{chapter-interventions}

There is a housing crisis with rapidly increasing cost of living as a result of rising housing prices. 
In the last few years, the need for affordable housing has come into focus as one of the most pressing issues facing Canadians. As more and more Canadians are finding housing unaffordable, the effects are being seen across the board in everything from declining homeownership rates to increasing numbers of Canadians unable to afford housing at all.

...

In principle there are a range of alternative regimes.
The advantages of financial capital have multiple sources, so the result does not require that any one advantage be true all of the time, so it's not sensitive to small or localized changes in policy environments. The analysis tells us that the advantages of financialized capital are robust.

are true across a range of explanations/features - including REITS and formal financial capital, second houses purchased by the investment vehicles by the middle class, and mortgages which represents the banks' share of privately owned homes

...


In the last few years, the need for affordable housing has come into focus as one of the most pressing issues facing Canadians. As more and more Canadians are finding housing unaffordable, the effects are being seen across the board in everything from declining homeownership rates to increasing numbers of Canadians unable to afford housing at all.

Over the course of two years of studying how changes in the housing market affect housing affordability, Social Innovation Canada has emerged with a picture of a system at a historic tipping point. 

This crisis means that we now have a huge but brief window of opportunity to fix the housing system.  Due to this current crisis, the government has moved to establish a National Housing Strategy and make massive investment in housing. This creates the opportunity for profound and lasting change, but unless steps are taken to address the drivers that have led to the crisis, the housing market will likely continue on its trajectory of eroding the middle class and making vulnerable populations worse off.   

This crisis has been developing for years. Canada's housing market reached an important turning point between the 2011 and 2016 census: homeownership rates both nationally and across major cities fell for the first time on record and the roots of this shocking reversal in housing trends goes back to 1993 when the federal government froze federal funding to new public housing projects, transferring public housing authority to provincial, municipal, and local governments. 

From the end of WWII until this point, Canada’s housing strategy ensured housing accessibility and affordability via a growing stock of public housing and cooperatives. These were funded by xx and made home ownership increasingly accessible and affordable. By the 1990s, this post-war housing strategy had built a prosperous and productive home-owning middle class: almost 70% of households were homeowners. 

In 1993, without federal leadership and funds, the social housing sector stalled, setting the Canadian housing market on the path to the current crisis. However, we didn’t see the true effect on homeownership until later because interest rates were dropping in the same period. This meant that home ownership continued to grow despite rising prices and slow growth in real wages. According to the International Monetary Fund and CMHC, these falling interest rates can explain the majority of the growth in house prices relative to incomes in most Canadian cities during this period. 

Meanwhile, low returns for investors in other sectors and rising home prices drove investors into the housing market. Speculative demand drove already rising home prices up further and faster rising prices attracted more speculative investment. Private capital began to take over the housing stock. This is the heart of the process called financialization. 

This trajectory of the market is preventing an increasing number of people in Canada from accessing safe and affordable homes. If financialization proceeds to its logical limit, the home-owning middle class will shrink, vulnerable populations will get poorer and financial institutions will extract more and more value from our cities.

However, real and lasting change can be made through a multi-pronged approach that includes innovative ownership models such as shared equity, co-housing and community land trusts; innovative financial models, products and services geared towards affordable housing; and greater alignment between municipal, provincial and federal programs with affordable housing developers and providers, investors and financial institutions to facilitate the creation of affordable housing units.

SI Canada proposes to establish a fund with a layer of concessionary funding to support deep, perpetual affordability. This could be part of a larger strategy to bring together stakeholders including government, non-profit, incubators, housing developers, community members, and financial institutions to take successful models that have emerged through the national housing strategy and scale them to make them the new normal. This  “all-hands-on-deck” approach brings together all the actors required to act quickly and effectively to design and implement solutions to problems that neither the government, the market or civil society are able to solve alone.  

This housing crisis could be the moment that Canada gets housing right. For a brief moment Canada has an opportunity to change the trajectory of housing affordability in Canada forever.


\section{Systems change}


%[[Nash asumption that everybody else will keep doing what they're doing]]
Bichiari cooperation
like to be somewhere else, like everyone else to make the adjustment at the same time. .

these system have feedback loops with the switching- the circuit resources creates more of itself-- the part's fo the land the water flows deepens the stream - making it more resilient, possibly wearing it out. - a kind of autopoetic quality -

switch is formally like a transistor/switch- also like a kind of tai chi - it has to use the energy of the particular kind of force coming in which is itself directed/complex/ has a mind, and direct that into the new system-- the input and the switch are complex systems and systems with minds.


the reductions to systems models become more not less important in the presence of this richness/complexity because it matters to desegregate the dynamics and the universal patterns of the type


\section{Social engineering}

has to relinqish it's commitment to state. 


\section{Who gets what share?}

As Ricardo said
\begin{quotation}   
 “The produce of the earth - all that is derived from its surface by the united application of labour, machinery, and capital, is divided among three classes of the community; namely, the proprietor of the land, the owner of the stock or capital necessary for its cultivation, and the labourers by whose industry it is cultivated. ...  But in different stages of society, the proportions of the whole produce of the earth which will be allotted to each of these classes, under the names of rent, profit, and wages, will be essentially different; ”  Chapter 1
\end{quotation}

% What changes over time is the share that goes to each group

That's what's at stake here.
The concern of the classical economists is central again, and we have the theoretical tools, to formalize the understanding.

We also have examples from many efforts to shift the patterns towards shared wealth to draw on. 

What are the prospects for this moment?



 Say, for example, the cost of borrowing, $r_i$ for agent $i$ if the base lending rate is $\bar{r}$
\[ r_i = (A + B \frac{\bar{W}}{W_i})\bar r\]
where $\bar{W}$ is mean wealth and $W_i$ is individual wealth. Figure~\ref{Fig:BorrowingCost} illustrates the effect.