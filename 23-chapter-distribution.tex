\chapter{Distribution and Growth} \label{chapter-distribution}

\epigraph{A large body of literature documents the existence of agglomeration economies in developed economies (see Rosenthal and Strange 2004 for a review). The main conclusion of this literature is the finding of scale economies of 3--8 percent (that is, a 10 percent increase in the size of an activity in a city raises productivity in this activity by 0.3--0.8 percent).}{Gilles Duranton \cite{durantonAreCitiesEngines2009}} 

\epigraph{Alternative micro-foundations cannot be regarded as interchangeable contents for the black box. \dots%The micro-foundations of urban agglomeration economies interact with other building blocks of urban models in ways that we cannot recognise unless they are explicitly stated. For instance, the composition of cities typically emerges as a consequence of the scope of different sources of agglomeration economies and their interaction with other aspects of individual behaviour. Third, 
different micro-foundations have different welfare and policy implications. %If we begin building an urban model by postulating an aggregate production function with increasing returns, we can only take this function as given. If instead we derive this aggregate production function from first principles, we may see that its efficiency can be improved upon. The means for achieving such an improvement will depend on the specifics of individual behaviour and technology. Thus, while different assumptions regarding individual behaviour and technology may support similar aggregate outcomes, the normative implications of alternative micro-foundations can differ substantially.
}{Duranton and Puga \cite{durantonMicroFoundationsUrbanAgglomeration2004}}



%WHERE DOS THIS GO - HERE OR IN MODEL? We suggest the distribution of rents has the potential to affect the productivity of cities, and we will then attempt to provide the conceptual tools for an analysis that fills that gap.

%Cities are the engines of economic growth (Jacobs, 1969; Bairoch, 1988). It is in cities that a large share of the innova- tions and entrepreneurship takes place that fosters economic growth in the long run. Spontaneous Orders and the Emergence of Economically Powerful Cities. JOHANNA PALMBERG

A growing body of research on urban scaling has demonstrated increasing returns to population. In our chapter on growth theory we showed that neoclassical growth theory associated these increasing returns to scale at the urban or national level with increasing effective human capital, which grows faster than the labour supply as a result of increased education or agglomeration effects. The two lines of research have settled on what might be termed a  common ``stylized  fact'': population and output are related according to 
\begin{equation}
    Y=AN^\beta,\qquad \beta>1 \label{eqn-population-output}
\end{equation}
% can be expressed as Y (\lambda N)~Z(\lambda,N)Y (N). When the scale factor Z depends only on \lambda, i.e. Z(\lambda,N)~Z(\lambda), equation (2) can be solved uniquely to give the scale-invariant result of equation (1), with Z(l\lambda)~\lambda^\beta.

Equation~\ref{eqn-population-output}  is a high-level generalization---a ``coarse-grained'' model.
% There's a distinction between coarse grained and fine grained models. 
Modelling always faces a trade-off between computational tractability and representing details. Coarse-grained models must capture the stylized facts, as ours does. A great deal of empirical research supports the general relationship represented by Equation~\ref{eqn-population-output}
% where the model is insenstitive to the details, you really want a representation that's small that captures the big pattern. 
% When you move to getting the details of the model, you know you have a model that tracks the observable. 
%Sometimes the model is sensitive to the details, and a more fine grained model is needed. There is an advantate to having a continuum of models that make it posible to represent systems with different levels of nuance/detail. We are focused on modelling a coarse grained model of the production system.

\cite{GET_TerrysDissertation, GET_PaulsBook}


Empirical estimates of $\beta$ vary considerably, however \cite{rosenthalEvidenceNatureSources2004, bettencourtIntroductionUrbanScience2021, loboUrbanScalingProduction2013}. 
McCoskey and Kao \cite{mccoskeyPanelDataInvestigation} show that the impact of urbanization on growth varies greatly across countries in different countries.
The World Bank (2016) reported that every 1\% growth in urban population correlates with an increase in GDP per capita by 13\%, 10\%, and 7\% in India, China, and Thailand, respectively.  However, Indonesia realizes only 4\% GDP growth for every 1\% increase \cite{haryantotriRelationshipUrbanizationEducation2021}. The literature has not yet offered an explanation  of the variation \cite{loboUrbanScalingProduction2013}.

%  MOVE THISAt the national level, Solaki \cite{solakiRelationshipEducationGDP2013} demonstrates a causal relationship between education and growth, and that tertiary Education should be considered as an exogenous variable.  Empirical results for Bangladesh \cite{islam2007relationship}show evidence of bidirectional causality between education and growth.  

% USE THIS
%At the urban level a study by Glaeser and Saiz \cite{glaeserRiseSkilledCity2003} for the US found that human capital predicted population and productivity growth at the city and metropolitan area level as surely as it predicts income growth at the country level. 

%They specifically found that cities with more educated residents have grown more quickly than comparable cities with less human capital. They find that causation running from growth to education seems to be present only in a handful of declining metropolitan areas, and cannot account for much of the relevant effect. Their evidence supports the view that skills induce urban growth.


The source of the variation is a matter of considerable policy importance. If $\beta$ can be affected positively by policy choices, or perhaps negatively by trends such as financialization, governments may be able to significantly increase social wealth and wellbeing.\footnote{Interestingly, the residuals or unexplained components for smaller cities are much larger than they are for large cities, suggesting that clues about potential policies will be found by examining smaller and mid-sized cities and that potential policy impacts may be greater for these cities.}

\begin{figure}
    \centering
    \includegraphics[scale=0.40]{fig/residuals-lobo.png}
    \caption{Residuals from regressing ln(total wages) on ln(population) using data for all 943 urban areas of the United States smoothed over the 2009--2011 period. Source: Lobo et al. \cite{loboUrbanScalingProduction2013}.}
    \label{fig-residuals-lobo}
\end{figure}


 The literature has identified many possible channels that are likely to affect the overall productivity individual cities. 
 While neoclassical growth theory, for example, provides one convincing general explanation in terms of human capital, it does not provide definitive empirical evidence on precisely how the effects of agglomeration are transmitted or what modulates those effects. More generally, researchers have not settled on which  of the possible influences dominate, what channels they work through, or what policies might improve the transmission of positive effects. These gaps are  not surprising. The literature on urban scaling is relatively new.  

  In this chapter, we examine a number of potential  links between urban population and productivity that have been proposed, and discuss how we  might incorporate specific policy-relevant linkages into our model.  Our interest is in how financialization might affect cities, and in this chapter specifically in the channels through which financialization might affect the productivity and growth of cities.


%\footnote{Lobo et al.are careful not to  claim that there is a causal relation between urban scaling and urban productivity. Causality, they say, stems from the ways in which being embedded inside larger agglomerations fundamentally affects how individuals interact with each other. Such micro-processses have not been demonstrated.}

\section{The transmission puzzle}
 In our model the long-term wage is set at the social marginal product of labour, including all external effects on other firms. Wages are set, however, in private markets. It seems unlikely that the agglomeration effect of adding a worker on the productivity of other workers even within a firm  would be attributed to that worker. For the firm there  would be a lagged, unevenly distributed effect that would appear exogenous, and so would not be compensated. 
 We should  assume therefore that firms do not take the positive externalities into account, implying firms will hire fewer than the optimal number of workers. 
 This seems to imply that direct transmission of agglomeration economies to employment and wage growth is unlikely,  a serious challenge to our decision to model the city relying on Equation~\ref{eqn-population-output} and the  Alonzo-Jacobs cycle.

%.  REFERS to a figure that haas been removed
%The challenge to our to directly linking employment to the wage is that the firm will base its hiring decision on the red line, or at best the blue line, but it will not perceive or  respond to externalities incorporated in the green line.  The argument implies  that cities will be chronically inefficient, failing to grow as  much and as quickly as they should because private decision makers fail to take into account the agglomeration externalities. 

The empirical results from the scale literature  make it clear that cities do exhibit increasing returns, however. There is a link, or perhaps many links. The argument firms will not respond internally to agglomeration effects could be correct, but then, since agglomeration is observed, there must  be other channels  transmitting at least some of the gains from scaling to the workforce. The possibility that the linkages are slow, indirect, and partial raises further difficulty in identifying the mechanisms at work.  The existence of these channels, difficult to observe though they may be, raises an exciting prospect, however: it may be possible to improve the  transmission process. 


Modelling or even identifying  the actual transmission processes is beyond the scope of this thesis, but our concern with the impact of financialization demands that we at least consider specific channels through which financialization of the housing market might affect the Alonzo-Jacobs cycle. To do this we have examined a range of hypotheses about possible transmission mechanisms and considered how they might be affected by the financialization of the housing market. Our goal is to identify a subset of those hypotheses that may be relevant in policy formation and that therefore may warrant future research.



\section{The Scale-Adjusted Metropolitan Indicator}

 Lobo et al. \cite{loboUrbanScalingProduction2013} provide an  analysis that is a useful starting point in the search for potential transmission channels and policy levers. They show that the regression model for Equation~\ref{eqn-population-output} can be written with an error term $\xi_i$ that is an explicit  function of specific local deviations.\footnote{Glaeser and Saiz \cite{glaeserRiseSkilledCity2003} find %little evidence for population growth accompanying skill upgrading among growing cities,  but they did  find  
 evidence for skill upgrading that would increase effective labour in declining cities. This result strongly  supports the hypothesis of city-specific  and history-specific transmission from left to right in our figure.}  
They  call these terms ``Scale-Adjusted Metropolitan Indicators'' (SAMIs). The SAMIs they derive depend on local wages and capital costs and they account for the city-specific residuals in Figure~\ref{fig-residuals-lobo} when $\beta$ is estimated using population alone. 
 
 Their model treats Equation~\ref{eqn-population-output} as a production function, as we do in Chapter~\ref{chapter-growth}.  Their preferred form is the one we discussed in Chapter~\ref{chapter-growth}:
 \[Y_i =A N^\beta_ i e^{\xi_i^Y} \eqno  \label{eqn-lobo}\]where $ e^{\xi_i^Y}$ is the city-specific residual for output $Y$ for city $i$ in their regression.\footnote{There may be several SAMIs that enter multiplicatively if data permits. As is usual in this literature, the log of Equation~\ref{eqn-lobo} lets them use  linear regression.}  They work backward from this expression to a Cobb-Douglas production function that depends only on labour and capital inputs.\footnote{In Chapter~\ref{chapter-growth} we show how the neoclassical growth theorists, in effect, worked forward from the Cobb-Douglas production function to an expression equivalent to the result from the scale literature.} They conclude that any  variables used to explain the residuals in estimates  of $\beta$ for cities must be expressed in terms of their  contribution to  the wage and capital shares of income or to  the magnitude of wage and capital inputs. 

 
We can question whether the two-factor model in Lobo et al.is an adequate representation. Neoclassical growth models incorporate human capital, which is itself a complex of technical knowledge, learned  skills, and social capital. In other words,  the Lobo et al. model can be seen as incorporating multiple factors conceptually while   suppressing them notationally. This is similar to the way the two-factor model suppresses land and resources  while the classical model explicitly include them.  Simplifications of this sort have been tremendously productive, in part because they allow us to focus on the large-scale and common features of the  system. For our work  Lobo et al. provide a reasonable starting point.



 In Figure~\ref{fig-impact-channels} we illustrate how this argument relates to financialization. On the right are the variables that  Lobo et al. \cite{loboUrbanScalingProduction2013} argue feed directly into determining $\beta$, which then, in our model, sets the magnitude of the \gls{Alonzo-Jacobs cycle} at the top right.\footnote{This can be seen as modifying  the parameter in  the difference equation linking population to wage.} A fine-grained model would locate these variables within the Alonzo-Jacobs cycle. It is worth noting that, however finely it is modeled, it is  a relatively slow cycle. We are considering processes that act slowly on productivity and city size. Any interventions will take a long time to take effect and a long time to observe. They may also be difficult to reverse. 



{\newpage\thispagestyle{empty}
\vspace{-1.5cm}
\begin{figure}
\vspace{-1cm}
\caption{Impact Channels}
% \label{figure-impact-channels}
\begin{adjustwidth}{-0.24\textwidth}{-0.24\textwidth}
    \centering
    \includegraphics[scale=.45, angle=90]{fig/impact_channels_revised.png}
    \label{fig-impact-channels}
%\pagestyle{headings}
% \usetikzlibrary{positioning}
%\begin{tikzpicture}[remember picture,overlay,shift={(current page.north east)}] \node[anchor=north east,xshift=-1cm,yshift=-1cm]{\includegraphics[width=1cm]{example-image-a}};\end{tikzpicture}
\end{adjustwidth}
\end{figure}
}


 
 
 On the left in Figure~\ref{fig-impact-channels} is the self-reinforcing financialization--price-increase-- speculative-gain--financialization cycle. This is by comparison a rapid cycle. It is unclear if it is reversible. 
 
 Between the two cycles are the fine-grained differences between cities that actually explain the residuals in the estimates for specific cities. These finer features are, we suspect, historically determined and highly variable. 
The intermediate channels of impact we are attempting to tease out % in  this chapter, and that appear in the empirical work as mere residuals, 
might be understood as factors that have been suppressed because they are simply too fine-grained, variable, and slow-acting  to deal with at this stage in the development of urban research.
% on the right side we have variables that introduce the variation in beta for individual cities, which means that for any particular case, all the stuff in the middle is for a particular city.

% Over on the left we have the financialization, price increase, speculation cycle, its another self reinforcing city when it kicks in it feeds in through the dynamic of a particular city to affect the city
% but the mechanism in between is not a coarse grained one, it's a fine-grained one for each city.



\subsection{Influences on the scale parameter}

We will begin with potential effects of financialization  that work through the factor supply channels suggested by Lobo et al. Most  can be represented in Figure~\ref{fig-impact-channels}

 
\begin{enumerate}
\item financialization may make reduce the amount of local capital available, raising the cost of capital in the city. 
    
\item since rising rents reduce disposable income, inflation coming from the right  side of the figure will reduce  the effective urban wage, which Lobo et al.find is one of the determinants of growth.

\item rent capture might influence the concentration of educated personnel by reducing the diversity and amenity of cities, making urban living less attractive, reducing labour supply or raising its cost.

\item rent capture leading to reduced local capital of reinvestment in human capital  might reduce the adaptive capacity of a city's population.

\item Rising housing costs might make it more difficult to attract or hold people with the specific adaptive skills Glaeser and Saiz identified.

\item  Rising housing costs might squeeze low-wage workers out of the city, reducing the opportunity for upgrading the urban human capital efficiently. Glaeser and Saiz found   evidence for skill upgrading  in declining cities, which suggests efficient investment in less skilled workers is a key adaptive/growth mechanism. 

\item Rising housing costs at the centre of a city would tend to push low-wage workers to the edges, increasing their transportation costs, putting further upward pressure on wages while increasing the costs of all amenity services that rely on lower-cost workers.

\item General financialization may have reduced the bargaining power of labor, as Tomaskovic-Devey and Lin  argue\cite{tomaskovic-deveyFinancializationCausesInequality2013}, reducing wages.\footnote{They point to a shift in behaviour of non-finance firms away from production and non-financial services and toward financial investments and services. This shift, they  argue,  has and led to lower employment, income transfers to executives and capital owners, and increased inequality among workers \cite{tomaskovic-deveyFinancializationCausesInequality2013}}.
\end{enumerate}
All of these potential channels weaken the link between productivity and wages or actually inhibit population growth.


\section{Linkages and spillovers}
Three theories have dominated the discussion of how agglomeration increases productivity. The Marshall-Arrow-Romer (MAR) theory concerns knowledge spillovers between firms in an industry.  Marshall (1890) described how the concentration of an industry in a city helps knowledge
spillovers between firms and, therefore, the growth of that industry and of that city. Porter (1990) argued that local competition within concentrations of industry drives innovation and growth. Both assume that technological spillovers occur within an industry. In the Jacobs model,\footnote{Glaeser et al. \cite{glaeserGrowthCities1991} refer to the Jacobs model as the Jacobs-Rosenberg\cite{rosenbergTechnologicalChangeMachine1963}-Bairoch \cite{bairochCitiesEconomicDevelopment1988} model.} unlike the MAR and Porter models, suggests that the most important knowledge transfers come from outside the core industry through cross-fertilization among a variety  of  industries and even occurs between individuals. 


Glaeser, Kallal, Sheinkman and Shleifer \cite{glaeserGrowthCities1991} used data on 170 of the largest US cities which industries in which cities grew fastest between 1956 and 1987 and why. %They found that  industries grow slower in cities in which they are more heavily over-represented and faster where the firms in the industry are smaller than the national average  and when the city is less specialized. 
This evidence was ``negative on MAR, mixed on Porter, and consistent with Jacobs.'' Scherer \cite{schererInterindustryTechnologyFlows1982} presented systematic evidence indicating that around 70 percent of inventions in a given industry are used outside that industry, generally supporting the Jacobs hypothesis that knowledge spills over across industries because cities bring together people from different walks of life and foster the transmission of ideas. The debate about which theory is the best description  continues. Grilliches and Lichtenberg also examined R\&D spillovers but concluded the evidence about which channels are effective remains tenuous \cite{grilichesInterindustryTechnologyFlows1984}. Beaudry and Schiffauerova proposed that levels of  aggregation together the choice of performance measures are the main causes of the lack of resolution in the debate  \cite{beaudryWhoRightMarshall2009}.


Duranton and Puga \cite{durantonMicroFoundationsUrbanAgglomeration2004} focus more directly on channels rather than theories. ``Urban agglomeration economies are commonly classified into those arising from labour market interactions, from linkages between intermediate- and final-goods suppliers, and from knowledge spill-overs.'' This classification loosely follows the three main examples provided by Marshall \cite{marshallPrinciplesEconomics1890} in his discussion of the sources of agglomeration economies.  Duranton and Puga go on to offer an alternative based on three types of micro-foundations for agglomeration, based on:
\begin{enumerate}
\item sharing,
\item  matching, 
\item  learning mechanisms.
\end{enumerate}
In this section,  we try to identify ways that  financialization might affect one or more of these processes. 

\subsubsection{Sharing}
For Duranton and Puga, sharing gains arise because a larger final-goods industry will support a larger pool of input supplier who are able to offer the gains form specialization. This is an applications of one of Adam Smith's main insights. It is not obvious that financialization will significantly affect the scale of industry, or the number of suppliers.\footnote{Rosenthal and Strange \cite{rosenthalEvidenceNatureSources2004} find the effect of sharing a common base of suppliers  are weak relative to other motives for agglomeration, while 
 Overman and Puga \cite{overmanLaborPoolingSource2010} find strong support for input sharing as a motive for agglomeration at the sector level.}

A possible indirect mechanism is that increasing costs and rising inequality induced by financialization might limit the diversity of consumption choices which functions indirectly as a component of the effective urban wage. Increasing costs and rising inequality might also limit the diversity of the labour pool, reducing the potential gains from specialization. Duranton and Puga  demonstrate that specialization by workers can generate increasing returns. It follows that any force that reduces the diversity of the workforce could eventually reduce $\beta$.  Both of these channels are incorporated in Figure~\ref{fig-impact-channels}.

\subsubsection{Matching}
Matching mechanisms create increasing returns by cutting search costs for producers and by increasing the quality of matches. As the workforce grows and the number of firms increases, the average worker is able to find an employer that is a better match for the worker's skill set. This appears as an improvement in effective labour. Matching occurs in a developed labour market with a specific set of informational institutions. It would probably require  quite large changes in the labour supply or churn to cause a noticeable difference in matching speed or success rates. It is unclear how financialization could affect the matching process.

\subsubsection{Learning}
Similarly, a variety of learning-based mechanisms have been suggested, but none that might obviously  be affected by financialization. 

Puga \cite{pugaMagnitudeCausesAgglomeration2010} concluded in 2010 that the literature had been so far unsuccessful at distinguishing between the possible sources of agglomeration  despite broad agreement on the magnitudes. 
At this stage it is difficult to see how the industry-based mechanisms he considered might be directly affected by  increasing financialization.


% fromt he skilled  city
% yield standard results in the regional literature: (1) increases in urban productivity will cause increases in the population, average wages and the price of non-traded goods (i.e. housing), (2) increases in the fixed factor of production will likewise increase population, wages and the price of non-traded goods, (3) increases in the consumption amenity will raise population, lower wages and raise housing prices and 
% /\(4) increases in the endowment of non-traded goods will increase the population, decrease wages and decrease the if a variable is increasing population and prices, but not wages, this implies that the variable is increasing consumption amenities.price of the non-traded good.
%...if a variable is increasing population and prices, but not wages, this implies that the variable is increasing consumption amenities.

% The intuition behind this result is a pure price effect. For cities that specialize in the skilled, their primary form of labor has become more expensive and as a result they grow less.

%Moretti (2003) extends Rauch (1993) and identifies human capital externalities by using instrumental variables related to human capital but plausibly exogenous to wages.11 He finds that, after controlling for the private returns to education, a 1 percentage point increase in the share of the college educated in a metropolitan area raises average wages by 0.6%-1.2%.

%the coefficient on schooling in housing price growth regressions is extraordinarily robust statistically when we control for initial housing price.

%Panel A in Table 5 certainly seems to make it clear that higher levels of education increase both the population of metropolitan areas and the price that this population is paying for the privilege of living in the area. In Panel B (Table 5), we examine housing pr

% what is the precise channel through which such advantages operate? Is it because a larger labor market improves matching between employers and employees? Or is it because large concentra- tions of employment iron out idiosyncratic shocks and improve establishments' ability to adapt their employment to good and bad times? Or perhaps because larger markets allow workers to specialize in a narrower set of activities and improve their performance? And how important are these advantages relative to alternative sources of agglomeration economies not operating through the labor market? To answer such questions we need good models that formalize the microeconomic foundations of urban agglomeration economies, as well as detailed empirical work able to identify and quantify the precise mechanisms at work. This is an area where there has also been much recent progress. However, as we shall discuss in detail below, there are substantive open questions that forthcoming research ought to address.

%One of the fundamental results in spatial economics is Starrett's \cite{starrettMarketAllocationsLocation1978} spatial impossibility theorem. This states that, once we abstract from the het- erogeneity of the underlying space, and without indivisibilities or increasing returns, any competitive equilibrium in the presence of transport costs will feature only fully autarchic locations where every good will be produced at small scales (see Ottaviano and Thisse, 2004, for a detailed discussion). Thus, substantial localization or spatial concentration of economic activity may be seen as a sign of agglomeration economies \cite{pugaMagnitudeCausesAgglomeration2010}.


%%%%.   Wage Premium
%Glaeser and Mare ́ (2001) ... controlling for observable skills, instrumenting for urban res- idence using parental background, and finally exploiting the panel dimension of the data to include individual worker fixed-effects, ... find that, even after all these corrections, there is significant wage premium associated with living and working in dense cities, although smaller in magnitude than before taking unobserved ability into account \cite{pugaMagnitudeCausesAgglomeration2010}.


%For workers, higher wages make them better off whereas higher rents make them worse off. Thus, greater consumption amenities in a city will make workers willing to accept both lower wages and higher rents. For firms, both higher wages and higher rents mean increased costs. Thus, localized productive advantages will make firms willing to accept higher wages and higher rents  \cite{pugaMagnitudeCausesAgglomeration2010}.
%.  This helps disentangle the consumption amenities from the productive advantages of big cities.

%The first influential modern study to do this, by Sveikauskas (1975), regressed log output per worker in a cross-section of city-industries on log city population and found that a doubling of population increases output per worker by about 6 percent. After dealing with various con- cerns discussed below, more recent productivity studies suggest that a doubling of city size increases productivity by between 3 and 8 percent for a large range of city sizes (Rosenthal and Strange, 2004)\cite{pugaMagnitudeCausesAgglomeration2010}

%mechanisms to explain the existence of urban agglomeration economies
%First, a larger market allows for a more efficient sharing of local infrastructure and facilities, a variety of intermediate input suppliers, or a pool of workers with similar skills. Second, a larger market also allows for a better matching between employers and employees, buyers and suppli- ers, or business partners. This better matching can take the form of improved chances of finding a suitable match, a higher quality of matches, or a combina- tion of both. Finally, a larger market can also facilitate learning, for instance by promoting the development and widespread adoption of new technologies and business practices \cite{durantonMicroFoundationsUrbanAgglomeration2004}.


Glaeser and Saiz \cite{glaeserRiseSkilledCity2003} considered three explanations for why people increasingly ``crowd around the most skilled:'' 
\begin {enumerate} 
\item ``cities are increasingly oriented around consumption amenities, and skilled neighbors are an attractive consumption amenity. This suggests an effect on amenity or on the effective wage which can be defined as  including the  amenities of urban life.'' 
\item  ``cities exist to facilitate the flow of ideas and the skilled specialize in ideas.'' This is the argument that agglomeration facilitates learning and suggests that, for financialzation to have an effect, it must reduce inhibit the networking capacity of  the city resulting in less effective labour being available.
\item ``cities survive only by adapting their economies to new technologies,'' and human capital enables people to adapt well to change. 
\end{enumerate}
Rather than increasing firm productivity the first and third operate at  different levels than the three channels examined in detail by Puga. 
Their evidence supports the third hypothesis, that human capital only matters in potentially declining places, which  supports the third hypothesis: skills are valuable because they help cities adapt  to negative economic shocks. 

%Based on this kind of research, we should expect that cities with universities will grow more quickly and that financialization might have either positive or negative effects on the growth of local colleges and universities. 

\subsection{Effects through innovation}

%Belderbwhich ideas How quickly trom person to person.  %%% ???


%Jacobs (1969, 1984) argued that interactions between people in cities help them get ideas and innovate, a view of cities that fits nicely with the work (Romer 1986; Lucas 1988)on economic growth that views externalities  externalities associated with knowledge spillovers as driving growth. Griliches (1979) surveyed the empirical literature on the role of knowledge spillovers. 


%\subsection{Spillovers}
%Belderbos et al. examine the simultaneous effects of spillovers due to research and development by universities and by firms \cite{belderbosWhatSpilloversUniversities2022}. %Rising urban productivity in Japan are significant. 



%Bairoch, P. (1988)\cite{Condit1990CitiesAE}. Cities and Economic Development: From the Dawn of History to the Present. Chicago, IL: University of Chicago Press.



%Henderson 1986 presented evidence that output per labour hour is higher i when firms in the same industry are clustered.

Henderson also described ``urbanization externalities'' that lead different firms to locate together. In this view urbanization increases final market demand and diveristy of products, but Glaeser et al.suggest that this effect does not drive growth

Lobos et al. examine the simultaneous effects of spillovers due to research and development by universities and by firms \cite{belderbosWhatSpilloversUniversities2022}.

Glaeser et al. \cite{glaeserGrowthCities1991} tested a model  of employment growth (not productivity growth) in an industry in a city as a function of the specialization of that industry in that city, local competition in the city-industry, and city diversity. None of the their results  support the importance
of within-industry knowledge spillovers for growth. If such spillovers. %This may be usefull for us because it rules out a  difficult channel to examine.
%\section{Network accounts}
Network accounts of increasing returns have drawn a good deal of attention.
Johansson and Quigley \cite{johanssonAgglomerationNetworksSpatial} in \cite{floraxFiftyYearsRegional2004} consider the parallel developments in the economics of agglomeration and the economics of networks, making a distinction between public and private capital in generating efficiencies.

\begin{quotation}
``the formation and efficiency of agglomeration arise from its character as public capital; households and firms in the same agglomeration share its benefits in common. In contrast, an economic network is private capital shared primarily by the network participants. Agglomerations also rely on public institutions, which aggregate individual decisions. In contrast, economic networks arise from a collective decision by group members, generating a private institution. Networks are clubs in which exclusion is possible and price discrimination is the norm. Agglomerations cannot exclude economic actors from receiving benefits nor can they price these benefits efficiently.''
\end{quotation}
They point out, drawing on Dixit and Stiglitz \cite{AvinashK.Dixit1977MCaO},  Fujita \cite{fujitaMonopolisticCompetitionModel1988}, Stiglitz and Venables and \cite{fujitaSpatialEconomyCities1999} that economies of scale  can arise with the diversification of the consumer market or of the input market, even though all individual competitors and firms earn normal profits. In such cases, it is not necessary for firms to internalize the effects of increasing their workforce, because they enjoy growing external economies.

 It is a remarkable result, but can the financialization of the housing market affect this class of  gains? It would appear unlikely unless the changing population mix induced by financialization led to a thinning of the local market for consumer goods. 

 The second mechanism they identify is at the firm level and forward and backward linkages among agents. These linkages may be of a pure-market type or may involve transaction links. They yield  gains from proximity or localization rather than urbanization specifically, although urbanization may be the force leading to localization. Can the financialization of the housing market affect this class of  gains? One feature of financialization has been the acquisition of firms, in many cases followed by layoffs, management changes, and consolidations focused on rapid profits and stock gains. Arguably these changes can inhibit urban growth but these effects are unrelated to the housing market.
 

\section{Effects through inequality}
Barro \cite{barroInequalityGrowthInvestment1999} found that inequality had a negative effect on growth in poorer countries but no significant effect for the richer countries. Grigoli et al. \cite{grigoliInequalityGrowthHeterogeneous2016} find  that the effect of income inequality on economic growth can be either positive or negative, and that at levels  of inequality  represented by a Gini coefficient below about 27  to be exact inequality hurts economic development.\footnote{Canada's Gini Coefficient Index was 66.7 in 2017. } %PUZZLE

 ``On the one hand, a higher concentration of income in the hands of a few is reflected in reduced demand by a larger share of poorer individuals, which would invest less in education and health and grow a sense of social and political discontent, jeopardizing human capital and stability. Moreover, more inequality can exacerbate households' leverage to compensate for the erosion in relative income, empower the influence of the richer population on the legislative and regulatory processes, and motivate redistribution policies that are often blamed for slowing growth, especially when aggressive. On the other hand, a certain level of inequality endows the richer population with the means to start businesses, as well as creates incentives for individuals to increase their productivity and invest their saving, hence promoting economic growth. Across income levels, only the findings for emerging markets indicate that more inequality slows economic growth. The only country groups for which we find evidence of a significant negative effect are the Middle East and Central Asia, the Western Hemisphere, and emerging markets (3/8).

 %But Canada is in the western hemisphere, with inequality more like that of ERurope 
 
 Other theories propose a positive relationship. These are based on the argument that inequality can rather provide incentives for innovation and higher productivity (Lazear and Rosen, 1981; Okun, 2015), foster saving and investment to the extent that rich people have a higher propensity to save (Kaldor, 1957), and endow richer individuals with the minimum capital and education needed to start some economic activity (Barro, 2000)
Bivens, reporting on the USA, argues that inequality is reducing growth by reducing the aggregate demand of the population below the 90$^{th}$ percentile in income \cite{bivensInequalitySlowingUS2017}.

\section{The state of the art}
In a discussion of the  future of the new economic geography, Fujita and Krugman suggested that there are three important directions for future work: enlarging the theoretical menu, buttressing the approach with empirical work, and addressing the welfare and policy implications of the whole approach. Much the same could be said for our project. We have extended the treatment of rent in urban modes and begun to the impact of financialization on urban distribution and growth. In this chapter, we have begun the process of identifying the channels through which  that impact is likely to occur. This is a crucial step toward  addressing the welfare and policy implications of financialization in the urban housing market.
 
Buttressing our approach with empirical work remains. Urban productivity can only evolve slowly, so Equation~\ref{eqn-population-output} is clearly a long-term relationship. Cross-sectional studies to date are simply not yet able to identify the long-term effects of all of  the variables that might explain the varied results for  $\beta$. Empirical tests will wait on the development of extended time series on urban output.
Studies that link city size to a variety of variables, such as patent production, total factor productivity, skill mix, industry growth, and industry diversity have made considerable progress in exploring individual factors. Theoretical models that make $\beta$ depend on directly policy variables of interest are in their infancy. There are to date  none that tell us what the effects of the increasing financialisation of the urban housing market are likely to be.  
 


% The notion that your labour force is on average more productive when there are more people around is pretty dramatic and it's not part of the basic model that we use. Our starting point is that's the fundamental feature of cities, and what does that do with financial capital and what does that do to distribution and that's not been explored. 

% The concentration of wealth is associated with slower econ growth.  There's a possibility that somehow the city will grow less.
% %{\color{red} We can speculate on how the distribution. of wealth directly affects the scaling parameter $\beta$. This probably calls of a discussion of the paths through which agglomeration works } 




% \subsection{Financialization and productivity}

% For urban theory and policy formation it is important to distinguish between financial instruments that enable production of real assets, and instruments like  mortgages that primarily facilitate the transfer of real assets or rights to real estate income. Housing developers borrow to purchase land for development and builders borrow to finance construction. While important, the financial instruments involved are not driving the financialization of housing.  The size of the loans involved is affected by the amount of land purchased and the potential rents earned by that land, but the degree of non-occupant ownership is not affected. *** CLARIFY

% Although inflows of financing through financialization, can fund development, financialization of the does not add to the housing stock in it's own right. Financialization refers the process of transferring ownership of existing housing units. 

% When  a productive asset is acquired as a financial asset it remains productive.  The financial instrument is separate form the real asset, at least in principle, and is traded in different markets. Why then is financialization an issue?  Theory suggests that financialization is positive.  The major argument is that finacialization enables real investment. In theory then, financialization of the housing market therefore  to more housing production. It is striking that after 40 years of growing financialization across we face a housing crisis , increasing homelessness and falling rates of home ownership.  Palley \cite{palleyFinancializationWhatIt2007} says that 

% \begin{quotation}At the macroeconomic level the era of financialization has been associated with generally tepid economic growth.\dots  In all countries except the U.K., average annual growth fell during the era of financialization that set in after 1979. Additionally, growth also appears to show a slowing trend so that growth in the 1980s was higher than in the 1990s, which in turn was higher than in the 2000s. \end{quotation}

% African land or land in Northern Ontario 
% Land acquired by holding companies may even be made more productive. The theory is that 
% the goal of such investments, however, is generally to achieve a capital gain over time. Financial analysis is essentially about rates of return on financial capital invested. The opportunity for capital gains  attracts financial capital to the housing market.%Financial managers have no interest is n in assets that are not expected to increase in value. 

% ***ILLUSTRATE AND CLARIFY
% If you see it as just supply and demand .. 
% Supply demand with fixed product and everything's neat
% Agglomeration changes everything. firms are underestimating each time they add a worker, the value that's going to be produced. They benefit from an agglomeration effect and that's where they interesting dynamics are coming from..
% we know that there is a marginal product of labour for a firm that it should be able to figure it out.. can the person on the shop floor figure out whether it's worth hiring another person.. we can talk about it, add details etc. We have a declining marginal product of labour. Because of transportation costs, we have a rising cost of getting labour so they cross and there is an equilibrium. There are adjustment questions like which adjusts quickly, how fast people move in, how fast firms decide to hire etc, but we know that there is in principle and equilibrium and that it is in principle a stable equilibrium (DIAGRAM STROGATS) although there are complications with this-- some argue these market equilibria never make sense- true in lots of way, but useful for analysis. 
% The question is then, what happens in our city? Do you get a growth dynamic? What seems to be the case is that if all the firms add workers then the marginal value of the product of all the workers they have goes up, which means they are making more profit which means if they are making more profit they want to hire more workers? Does it ever converge? Likely eventually, but it's got a powerful dynamic.  If you add other features like more products being created in the city, which is part of this agglomeration process you can start seeing, if you exhaust one source of growth, we know that there are others, that simplification is just firms of the same sort hiring workers of the same sort is wrong. so we need to add the local service sector, we need to add the possibility of creating new products and those depends on the number of workers and so depend on further agglomeration effects. What does this mean? For the purpose of the model, we'd want to strengthen the agglomeration effect relative to what they are for specific firms or industries..



% MOVE?  YES

%  Notice that this formulation implies it is possible to have increasing returns to scale for the urban economy even with a production function at the firm level with decreasing returns to scale: the return to the total economy $\alpha + \beta(1 + \gamma)$ can be greater than one, even if $\alpha +\beta$ is less than one. %. \label{Fn-PSI}}  (CITE Appendix: Excess Returns)
 
% The \gls{agglomeration effect} means that although individual firm returns to scale are declining, the city can experience \gls{increasing returns to scale} in the utilization of human capital. % NEED? Individual firms have decreasing returns, but the presence of agglomeration economies external to firms but internal to the city gives the urban economy as a whole increasing returns to scale .}.
%\footnote{The formulation is consistent with the discussion of endogenous growth theory in Chapter~\ref{chapter-growth}, which demonstrated that increasing returns at the macro level is consistent with decreasing returns at the micro level. The effects are not limited to the individual firm. \Gls{spillover effects} can be large, see Section~\ref{section-spillover}. }






% \begin{quotation}`` high productivity cities show invariably high wages and high levels of employment relative to their size expectation. Conversely, low productivity cities show both low wages and employment.''

% \dots Urbanization and growth go together: no country has ever reached middle- income status without a significant population shift into cities \cite{annezUrbanizationGrowthSetting2009}\end{quotation}




% %\cite{rosenthalEvidenceNatureSources2004}
% This gives 
% A language to talk about a whole class of problems that are important/of rising importance in the literature but that this class of formal models has lacked the infrastructure/language/conceptual toolkit to explore. ** MAYBE ALSO MENTION THIS IN DOC INTRO/CONCLUSION
% NEED TO SHOW A CONNECTION BETWEEN GROWTH RATE AND DISTRIBUTION - IN THE GROWTH CHAPTER..---OR RIGHT AFTER AS A NEW CHAPTER..



% \epigraph{Widespread urbanization is a recent phenomenon. In 1900 just 15 percent of the world's population lived in cities. The 20th century transformed this picture, as the pace of urban population growth accelerated rapidly in about 1950. Sixty years later, it is estimated that half of the world's people lives in cities.}{Annez and Buckley\cite{annezUrbanizationGrowthSetting2009}}


% ****  Urban infrastructure has high social but low private returns. It is unlikely private capital will flow into urban infrastructure projects with substantial public support. That is the motive for ``\gls{Public-Private Partnerships}'' (PPPs) that involve private capital financing government projects and services up-front, and then subsidizing the private investors  from taxpayers out of the expected social returns over the course of the PPP
% Annez and Buckley\cite{annezUrbanizationGrowthSetting2009} observe that 
% \begin{quotation}
% Britain's cities were cleaned up only when the central government stepped in to alleviate the binding financial constraint in cities. In this story lies an important lesson about building urban infrastructure, especially those lumpy discrete investments in networks that expand the limits at which congestion costs outweigh agglomeration benefits. Neither the municipal finance systems that worked before the urban transition nor those suitable for cities in a demographic steady state will necessarily generate finance for investments in local public goods that more than pay for themselves in economic terms. 
% \end{quotation}




%``As Quigley points out in chapter 4, the fundamental question in urban economics is why people voluntarily live in close proximity to one another when there are costs to competing for land. The simple answer has two parts: efficiency gains and consumption benefits. Recent theoretical and empirical work provides a sense of the nature and significance of these gains.'' Annez and Buckley\cite{annezUrbanizationGrowthSetting2009}% P  13 %see their list



%Evidence from a broad panel of countries shows little overall relation between income inequality and rates of growth and investment. However, for growth, higher inequality tends to retard growth in poor countries and encourage growth in richer places. The Kuznets curve-whereby inequality first increases and later decreases during the process of economic development-emerges as a clear empirical regularity.


% Policy tools unrelated to financialization can be examined in the frame we use here. Cutting transportation costs in a city is equivalent to raising the effective wage, which would  make it easier to attract labour and would reduce pressure on producers to raise wages. It would also reduce the cost of education for commuters, increasing the supply of augmented labour.



%  We can mimic the effects of financialization as it works through reducing the effective wage by introducing a gap between the value of the marginal product of labour and the wage paid. 

\footnote{Alberto Bucci.  R\&D, Imperfect Competition and Growth with Human Capital Accumulation, 2003. Scottish Journal of Political Economy. https://doi.org/10.1111/1467-9485.5004004. This paper studies the long-run consequences of imperfect competition on growth and the sectoral distribution of skills within an R\&D-based growth model with human capital accumulation. We find that steady-state growth is driven only by incentives to accumulate skills. In the model imperfect competition has a positive growth effect, while influencing the allocation of human capital to the different economic activities employing this factor input. Contrary to general wisdom, the share of resources invested in R\&D turns out not to be monotonically increasing in the product market power and its correlation with the equilibrium output growth rate is not unambiguous.}