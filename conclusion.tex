

\chapter{Conclusion}


The economics is clear that this is what's at stake is productivity of cities, the distributive features of the economy and the impact of the middle class.

In a passage that can be seen as a direct precursor to our analysis of urban land rent, Ricardo  wrote in his 1817 Principles of Political Economy, %Early theorists like Ricardo described something can be thought of as describing a three-factor, three-class model with great precision but without the use of mathematics.   

Adding 2 things 1. rent extraction and 2. power law scaling of productivity, we find rent is the breaks on the engine of wealth creation

INTEGRATE- AS THIS SUGGEST ALL GOES TO LANDLOARDS?
What changes over time is the share that goes to each group:

As Ricardo said

\begin{quotation}   
 “The produce of the earth - all that is derived from its surface by the united application of labour, machinery, and capital, is divided among three classes of the community; namely, the proprietor of the land, the owner of the stock or capital necessary for its cultivation, and the labourers by whose industry it is cultivated. ...  But in different stages of society, the proportions of the whole produce of the earth which will be allotted to each of these classes, under the names of rent, profit, and wages, will be essentially different; ”  Chapter 1
\end{quotation}

what's changed is xyz

Ricardo was ``the first writer to take the industrial phenomenon of city life and to create an economy based upon those characteristics.''  \footnote{Simon N. Patten,  The Quarterly Journal of Economics, volume 7, Issue 3, April 1893, Pages 322–352, https://doi.org/10.2307/1884006 }  

Our focus is land rents, but in the context of an urban economy. 

{Ricardo concluded that % , ``It follows then, that 
``the interest of the landlord is always opposed to the interest of every other class in the community.'' }



%Strikingly, Ricardo concluded\footnote{An Essay on the Influence of a low Price of Corn} that, “It follows then, that the interest of the landlord is always opposed to the interest of every other class in the community.” Rents approriated by the land owner are not available to the worker  or the capitalist for re-investment. This is a view that Henry George would later pick up and it is one that we return to in our study, since the issue of urban rents is of great policy importance. 