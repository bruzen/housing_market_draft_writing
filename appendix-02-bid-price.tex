\chapter[Bidding]{Calculating Bid Price}
\label{appendix-bid-price}
\epigraph{One of the very important components in the urban and agricultural land use model is the so-called \gls{bid-rent curve}. Regional and urban economists, city planners, and economic geographers have used this curve extensively as an analytical device.}{Yeung-Nan Shieh\cite{shiehWilhelmLaunhardtBidRent2004}}


This chapter links the theoretical discussions of the previous chapters with an agent-based simulation model. The chapter is intended to explain the code and justify the modeling decisions that link the theory to the actual output of the simulations.

 We compute the bid price home buyers or speculators are willing to pay for an property, beginning with the value and the return on investment for an urban property, and modelling each agent as %'ll assume that all agents are
 speculating on potential \glspl{capital gain} as well as on the \gls{use value} they get from living in a property. %We assume that the use value is captured by the stream of rental values. %, whether a home is owner-occupied or held by an investor as a financial asset. 
 % For simplicity, we consider a one-period investment.  %To keep the analysis simple without loss of generality 

 The agent purchases a house for a price, $P_0$ %a down payment, $D$, 
 and receives the increased price $P_T = (1 + \dot P)P_0$, back after a period $T$. 
% The value of the investment is the net present value of buying and then selling after one period:
The value of the investment is the capital gain, $\mathcal{C}$ plus the rents, net of operating costs and taxes, $\mathcal{R}_N$.

% minus the mortgage, repaid with interest, plus rents, minus any operating costs and taxes,\footnote{We have applied this model to explore the effect of a vacancy tax in Beirut.  For that analysis we  added a use-value, $U$ in place of rent for expatriate owners to represent using the property - say one month a year - when they are not renting the property and a \textbf{vacancy tax}, $T$ at rate $t$ to affect the speculator's  decision.} %\cite{Al-Shihabi}


% OLD equations from working out the above. Could change symbols
% \begin{eqnarray}
% V  	&=& capital\ gain - Interest\ due  	+ Rent  - operating\ cost -taxes \\
% 	&=& \delta P_T-D \qquad \qquad \quad - (1+\delta r)M \quad	 + R  	-C\\
% 	&=& \delta P _T \qquad-(P_0-M) \quad- (1+\delta r)M 	 + R  	-C\\
% 	&=& \delta (1+\dot P)  P_0 -(P_O -M)  -(1+\delta r)mP_0  + R  -C\\
% 	&=& \delta (1+\dot P)  P_0 -P_O + M \qquad -(1+\delta r)mP_0  + R -C\\
% 	&=&( \delta (1+\dot P)-1)  P_0  + mP_0 \quad -(1+ \delta r)mP_0  + (\rho-\kappa)P_0 \\	
% 	&=& \left(  \delta (1+\dot P)-1    + m \quad - m(1+\delta r)  + (\rho-\kappa)\right)P_0 \\'
% 	&=& \left(  \delta (1+\dot P)-1    + m \quad - m-\delta rm  + (\rho-\kappa)\right)P_0
% \end{eqnarray}
 Where $\mathcal{R}$, $\mathcal{O}$, $\mathcal{T}$, and $M$ are total rent, operating costs, tax payments, and mortgage borrowed, as net present values at the end of the period. The interest rate is $r$, and the discount factor $\delta$. % I don't like this. I think it makes sense for them to be total payments over the period, but I now think it would be more intuitive to compute them as net present values at the start of the period, since that is when the mortgage is borrowed and the decision is made.

 %For ease of calculation, ..To get the return on investment, 
 This may be written in terms of shares of the purchase price:
\begin{eqnarray}
V &=& \delta \left((1+\dot P) P_0 - (1+r)mP_0\right) + \rho P_0 - \theta P_0 - \tau P_0 \nonumber \\
  &=& \left(\delta \left(1+\dot P - (1+r)m   \right) + \rho     - \theta     - \tau\right) P_0.
\label{eqn-property-investment-value2}
\end{eqnarray}
 Where $\rho$, $\theta$, $\tau$, and $m$ are rent, operating costs, taxes, and mortgage shares, respectively. % Where $\phi$ is a fraction that takes into account taxes and operating costs. 
 % of price for rents, operating costs, and taxes. %The discount factor is $delta$, $P_0$ is the property price at the time of sale. %$r$ is the interest rate paid, and $m$ is the share of the price taken out as a mortgage.
 %It has  seven  parameters, $\delta, \dot P, r, m, \rho, \kappa$ and $t$. The first four, $\delta, \dot P, r$ and $m$ are exogenous for the investor while $\rho$, $\kappa$ and $t$  are    %Operating revenue and costs $\rho, \kappa$ and $t$ are expressed as  present values. 

%Agents borrow a share of the purchase price, $P$. The amount borrowed is the mortgage, $M$. This is a share of the total purchase price $mP = M$. The \gls{mortgage term}, $T$, is the period it takes to pay down the mortgage.

\section{Return on investment}

The rate of return on funds invested, $r_{return}$, is the value divided by the size of the down payment, $D$: 
%The rate of return is $v = \frac{V}{D}$. %For expat investors, we get a \textbf{decision rule}:
%\begin{enumerate}
%\item  if $v \geq a$ (with some private use?) with no rent,  don't bother renting. 
%\item If $v(no\ rent\ and\ tax) < a\geq v(with\ rent)$,  then  rent. 
%\item If $ v(with\ rent) \le a $,  then sell 
%\end{enumerate}\
\begin{eqnarray}
r^{return} 
  &=& \frac{V}{D}  \nonumber \\
  &=& \left(\delta \left(1+\dot P - (1+r)m\right) \ + \rho - \theta - \tau \right) \frac{P_0}{D}        \nonumber \\
  &=& \left(\delta \left(1+\dot P - (1+r)m\right) \ + \rho - \theta - \tau \right) \frac{P_0}{P_0-mP_0} \nonumber \\ 
  &=& \frac{\delta \left(1+\dot P - (1+r)m\right) \ + \rho - \theta - \tau }{1-m}.
\label{eqn-property-investment-return1}
\end{eqnarray}
Equation~\ref{eqn-property-investment-return1} provides a criterion for investors. Agents invest if if their expected return is greater than the target return, they are seeking:
\begin{equation}
r^{return} \geq r^{target}. 
\label{eqn-property-investment-return2}
\end{equation}



\section{Maximum bid} % given the rate of return}

To model the market transactions when speculative motives are in play we need an expression for  how much financial investors will bid. More specifically, we need to find the maximum price that they are willing to pay. As Horowitz \cite{horowitzBiddingModelsHousing1986} notes, a prospective buyer  considering knows a vector of attributes of the house, the seller’s asking price and the property taxes, transaction costs, and financing costs at a specified price. The  potential buyer also is likely to have estimates of the maintenance costs and resale value of the house, although these may be highly subjective. For our bidding model, therefore, we need find how much financial investors will bid in terms of these variables.



We anchor our housing price expectations in bid rents, which, based on the literature can reasonably be seen as the equililbrium toward which prices will evolve.  %This is the maximum price that satisfies the criterion $r^{return} \geq r^{target}$.  The bid price is used in the price determination process in our model. %Where the return is the value over the down payment:
% Equation \ref{eqn-property-investment-return1} expresses the expected return on investment in terms of the  market price, $P_0$: 
% \begin{eqnarray*}
%  r^{return} 
%    &=& \frac{V}{D} \\
%    &=& \frac{\delta \left(1+\dot P - (1+r)m\right) \ + \rho - \theta - \tau}{1-m}.
% \end{eqnarray*}
%which we define as the ``bid price'',% $P^{max}_{bid}$. 
%We return to Equation~\ref{eqn-property-investment-value3} %
We start by writing the value of the investment, from Equation \ref{eqn-property-investment-value1}, with the net rent. %, which can be written a share, $\phi$ of total rent $\mathcal{R}_N = \phi \mathcal{R}$, %The net rent can also be written as a share, $\phi$, of the rent 
%: %(originally from Chapter \ref{chapter-financialization} on Financialization. Since the rent is a known quantity at the beginning of the term, independent of the current, so we reformulate net rent,  $\mathcal{R}$, $\mathcal{O}$, $\mathcal{T}$ as a fraction of total rent: 

\begin{eqnarray}
V &=& \delta \left(P_T - (1+r)M\right) +      \mathcal{R}_N   \nonumber \\
  % &=& \delta \left(P_T - (1+r)M\right) + \mathcal{R}_N   \nonumber \\
  % &=& \delta \left(P_T - (1+r)m P_0\right) + \mathcal{R}_N \nonumber \\
  &=& \delta \left((1+\dot P)P_0 - (1+r)m P_0\right)     + \mathcal{R}_N \nonumber \\
  &=& \delta \left((1+\dot P)    - (1+r)m    \right) P_0 + \mathcal{R}_N
\label{eqn-property-investment-value3}
\end{eqnarray}


Dividing by the down payment, $D = (1-m)P_0$:
\begin{eqnarray}
r_{return} = \frac{\delta \left(1 + \dot P - (1+r)m\right)}{1-m} + \frac{\mathcal{R}_N}{(1-m)P_0}
\end{eqnarray}




% +++++++++++++++ JUST REPLACED BETWEEN THESE WITH THE ABOVE. CHECK IT'S OKAY
% % We also have to assume that $P^{bid}=P_0$, which we can justify as an equilibrium condition - investors believe they are paying the market price. The result is 
% % \[V= \delta(P^T- (1+r)M)   +\phi r P^{bid}\]

% % The price at the end of the term $T$, $P_T$, is a predicted value for the investor \textit{ex-ante}, so we can replace $P^T$ in Equation~\ref{eqn-property-investment-value3} with the predictor, $(1+\dot P)P^{bid}$. 
% % We can replace $P^T$ in Equation~\ref{eqn-property-investment-value3} with $(1+\dot P)P_0$.
% \[V= \delta \left((1+\dot P)P_0 - (1+r)m P_0\right) +\phi \mathcal{R}\] 
% Then we replace $\dot P$ with an estimate, $L(P)$, representing an estimated function of the lagged values of $P$ and any  other relevant data. We imagine the potential investor informed estimates by information from  real estate agents or analysts.  The result is 
% \[V= \delta \left((1+L(P))P^{bid}- (1+r)mP^{bid}\right) +\phi \mathcal{R}\]
% Combining terms:
% \[V= \delta \left((1+L(P))- (1+r)m \right) P^{bid} +\phi \mathcal{R}\]

% WHY BIG BRACKETS NOT SHOWING UP

% The rate of return $r_{return}=V/D=V/(1-m)P^{bid}$ is then

% \[r_{return}= \frac{\delta \left((1+L(P) - (1+r)m\right)}{1-m} + \frac{\phi \mathcal{R}}{(1-m)P^{bid}}\]

% %++++++++++++++++

% \begin{align}  
%   v  & =  \frac{\delta ((1+\dot P)  - (1+r)m)\  + \psi r}{1-m}\label{eq:RULE}\\
%   & =  \frac{\delta ((1+\dot P)  - (1+r)m)\  + \psi r}{1-m}\label{eq:RULE}
% \end{align}

% \begin{eqnarray}
% v %&=& \delta(P^T- (1+r)M) \qquad \qquad \qquad 	 + \mathcal{R}_N \nonumber\\
% % &=&\delta\left( (1+\dot P)P^{bid} - (1+r)mP^{bid} \right)  + \mathcal{R}_N  \nonumber\\
%   &=&\delta\left( (1+L(p)) - (1+r)m \right) P^{bid} + \mathcal{R}_N  \nonumber
% \end{eqnarray}

Agents bid if the RHS is larger than the target rate of return, %$ $r^{target}$ %, not not if it is smaller'', 
as stated in Equation \ref{eqn-property-investment-return2}. % The maximum bid is the bid that makes $r^{target} = r^{return}$. % two sides equal in the following equation, the bid gives a return greater than or equal to the target return rate. 
Replacing $r_{return}$ with $r_{target}$, $P_0$ with $P^{bid}$, and $\dot P$ with an estimator for $\dot P$, $L(P)$, we get: TODO REPLACE PDOT - MAYBE WITH TILDE AND DOT..:  %$\dot \tilde{A}$ $\wwtilde{\dot{u}}$

\begin{eqnarray}
r_{target} \ge \frac{\delta \left((1+ \dot P - (1+r)m\right)}{1-m} + \frac{\mathcal{R}_N}{(1-m)P^{bid}}
\end{eqnarray}

Solving for $P_{bid}$:


\begin{align}
r^{target} &\ge \frac{\delta \left(1 + \dot P - (1+r)m\right)}{1-m}   +\frac{\mathcal{R}_N}{(1-m)P^{bid}}. \nonumber \\
(1-m)r^{target} &\ge \ \ \delta \left(1 + \dot P - (1+r)m\right) + \frac{\mathcal{R}_N}{P^{bid}} \nonumber \\ %\delta(1+L(P))- (1+r)m%
(1-m)r^{target} - \delta \left(1 + \dot P - (1+r)m\right)  &\ge  \frac{\mathcal{R}_N}{P^{bid}} \nonumber\\
P^{bid} & \le    \frac{\mathcal{R}_N}{(1-m)r^{target}-\delta \left(1 + \dot P - (1+r)m\right)} 
% P^{bid} & \le    \frac{\mathcal{R}_N}{(1-m)r^{target}-\left[ \delta(1+L(P)- (1+r)m\right]} \nonumber 
\label{eqn-bid-price1}
\end{align}
The denominator is an adjusted rate of return for capitalizing net rents, analogous to the value of $r$ in Equation~\ref{eq:Capitalization}. 

Each agent has their own interest rates, discount rates, mortgage share .. information, and expectations, so individual bids can differ:

\begin{align}
P_i^{bid} \le   \frac{\mathcal{R}_N}{(1-m_i)r_i^{target} - \delta_i \left(1 + L(P) - (1+r_i)m_i \right)}.
\label{eqn-bid-price2}
\end{align}

% \begin{eqnarray}%. OLD VERSION: WRONG
% %r^{target}&=& \delta\left( (1+L(p)) - (1+r)m \right) P^{max}_{bid} + \mathcal{R}_N  \nonumber\\
%    P^{max}_{bid} &=&\frac{r^{target} - \mathcal{R}_N}{\delta\left((1+L(p)) - (1+r)m \right)} %\label{EqBidPrice2} 
% \end{eqnarray}

% \section{Finding bid price}
% We start with Equation~\ref{B2}. for convenience, replace $\rho -\kappa - \sigma $ with $\mathcal{R}_N$ (net Rent). 

% Replace $P^T$ with $(1+\dot P)P^{bid}$ assuming that the bidder is bidding the equilibrium market price for the period.

% Then replace   $\dot P$ with $L(p)$ representing some (estimated function ($\tilde{\dot P}$)) of the lagged values of $P$ that incorporates other data. 

% \begin{eqnarray}
% v&=& \delta(P^T- (1+r)M) \qquad \qquad \qquad 	 + \mathcal{R}_N \nonumber\\
%  &=&\delta\left( (1+\dot P)P^{bid} - (1+r)mP^{bid} \right)  + \mathcal{R}_N  \nonumber\\
%   &=&\delta\left( (1+L(p)) - (1+r)m \right) P^{bid} + \mathcal{R}_N  \nonumber
% \end{eqnarray}

% So I want to use this relationship to find the maximum bid price for the bank. The rule is, ``Bid if the RHS is larger than the target rate of return, $r^{target}$, and do not bid if it is smaller.''  The maximum bid  is the bid that makes the two sides equal. 

% {\color{red}
% \begin{eqnarray}
% r^{target}&=& \delta\left( (1+L(p)) - (1+r)m \right) P^{max}_{bid} + \mathcal{R}_N  \nonumber\\
%    P^{max}_{bid} &=&\frac{r^{target}-\mathcal{R}_N}{\delta\left( (1+L(p)) - (1+r)m \right)} \label{EqBidPrice} 
% \end{eqnarray}}
% %(What makes. this work is that I do not use an identity to get $\dot P$, which made the system of equations singular.)
% \newpage

\section{Other factors affecting price determination}
% \subsection{On bidding}

In practice, potential investors will make an  initial  bid that is lower than this value and subsequent bargaining will settle of a price between the initial bid and the seller's asking price.

In the market, the initial bid should be smaller than the bid price calculated, which is a maximum that can earn the target rate of return, The bank will definitely go this high. 

If there is a  max bid among competing buyers, the second highest max bid should be the sale price, but the bidder with the highest max bid wins the property. This makes sense because in a bidding war the final bid only has to be a very small amount higher than that of the last competitor left.
That is the highest price that a seller can get.

It might be simplest to treat every seller as a bidder. All bid above her own are considered. The seller chooses the second highest bid. This  seems realistic enough and is very simple to implement. It should producer a path that is indistinguishable form any more complex approach. 

Will persons retiring who would leave the city invest in an urban rental?

Bargaining between buyers and sellers differs. Buyers bid low and sellers ask high. {\color{red}We need to calculate a minimum selling price for the seller}.

% But what is $D$? Does the bank have unlimited funds? Isn't D just a fraction of P?  If so it cancels out
% r is the prime rate- that the bank pays? 

\section{The period and time value of money}

In developing the model we introduced a number of rates, such as $r_i$, the rate that individual $i$ pays for a single borrowing period. The payment calculation is made for a period of length $T$, which we refer to as a mortgage term.
%This means the actual interest paid is a compounded interest rates (VARIABLES LIST FOR THIS).

%In developing the theoretical model we introduced a number of rates, such as $r_i$, the rate that  individual $i$ pays for a single period of borrowing. We assume the calculations are for a  period of length  $T$, which we refer to as a mortgage term. This means that the rates in equations such XXXXX are actually compounded rates.
Taking a simplified example, if the rate is $x$/year, interest payments are $xM$, a share of the total mortgage amount borrowed, $M$, for each of $T$ years. 
If the interest payments are all made at the end of the mortgage term, the lender will require interest on the deferred interest, so agent $i$'s payment at the end of the period $T$ will be:

\begin{align*}
\text{Payment} &= (1+r_i)M                                            \\ 
    &= M + xM(1+x)^{T-1}+ xM(1+x)^{T-2}\dots + xM(1+x)^{0} \\
    &= M\left(1+ x\sum_{z=T-1}^0(1+x)^{z}\right).          \\ 
\end{align*}
Therefore, the interest payment is:
\begin{align*}
r_i.   &=x\sum_{z=T-1}^0(1+x)^{z}.
\end{align*}

For the sake of notational simplicity and clarity of exposition,  we omit the compounding formula throughout our discussion. This means that, while banks may quote a per period interest rates, the equations use a compounded rate. In the computational model we employ the appropriately compounded values. All %the rates are compounded in this way to provide a per-period rates, including all 
 the interest rates, the $r$'s, are compounded in this way, because they require annual or monthly payments at their stated rates.
 It does not affect the discount rates, the $\delta$'s, or the borrowing ratio $m_i$, because they are initially calculated for the term and don't require the same period payments.
 The discount factor $\delta_i$ is always a compounded version of $x$:
 \[\delta_i=\left(\frac{1}{1+x}\right)^T\].
%is a feature of the individual, so it is not affected in this way.
%The appropriate compunding expressions for other rates are displayed below.

The {mortgage term}, $T$, is the period it takes to pay down the mortgage. We work with a mortgage terms for two reasons. First, in the theoretical analysis, by transforming the multi-period transaction to a single period, we simplify the comprehensibility of the analysis. Second, it reflects the fact that in practice agents  likely consider the profitability of a purchase for a finite term longer than one period. The term period lets them consider the time cost of money in their analysis in a natural way. Parameters for their discounting rates and the term considered can be used to explore the effect of borrowing costs and their personal discounting rates on their decisions. Future interest rates are also not fully known. In the future, we can also vary the compound interest rates to explore the impact of uncertainty, given agents' guesses about the future, and their level of risk aversion.


\section{WHERE DOES THIS GO? Calculate rate of return for an investor}

 The warranted price of the property is the present value of the \gls{warranted rents}% *** Note the warranted price may be rising. This is a problem for  us. WHY?
% NOTE Some of the taxes and most of the repairs/maintenance costs are for the building and not the for the locational value of the land. These two have correlated values. (we can't separate the amount of house and quality form our locational decision ) 

%**** We would like to  somehow ignore the building and push it into the subsistence wage. Can we do this? EXPLAIN
% How do we ignoor buildings - when we talk about the land rents, we're calculating on the basis of the warranted rent wage premium- put people are paying for buildings.. - assume it comes out of the subistence wage- since this is just a theoretical treatment.-- but if you think about how much capital gain will you get, we'll also get capital gainso on the house piece which is not locational- just gains on that asset they happen to have.. --  looking at things coming from close toe the center ignoring what else. - tied together so market spreads capital gain over both of them, and the tax does the same too-- for locational advantage and services..

% Since the present value of rental income (OR NET RENTAL INCOME?) is the value of the house, i.e.,\\

In principle, the  sale price of a unit of land considered is the present discounted value of the stream of net rents. If $\mathcal{R}_N$ is one period rent, which is conventionally calculated as:
\begin{equation}
  P=\frac{\mathcal{R}_N}{r}  
\label{eq:Capitalization}
\end{equation}

%\[v= \delta(P^T- (1+r)M)    +\phi\mathcal{R}\]

*** CLARIFY WHAT WE GET FROM INCLUDING $\psi$, and define, maybe compare with other shares like $\phi$.

\noindent so we can write 

\begin{align}  
    V & = \delta \left((1+\dot P)  P_0- (1+r)mP_0 \right)  + \psi rP_0  \\
      & =\left( \delta ((1+\dot P)  - (1+r)m)\  + \psi r\right)P_0
\end{align}
%there are potetntilaly two interest ratex hereee - one used for capitaliΩing rents to price, and one  is $r_i$. It may be sensibe to use the same value for both. Person I might be looking at the actual rents and using them to calculate the prtice it should sell for. If there is a real estate agent providing price information the tory may be different. SWe are really using htis for the expected value of the present value of net rent, and 

This can be expressed in terms of the rate of return on the down payment, $V/D$
\begin{align*}  
v & = \frac{V}{D} \\
  & = \left( \delta ((1+\dot P) - (1+r)m)\ + \psi r\right)P_0/D \\
  & = \left( \delta ((1+\dot P) - (1+r)m)\ + \psi r\right) \frac{P_0}{P_0-mP_0} \\
  & = \frac{ \delta \left((1+\dot P) - (1+r)m\right)\ + \psi r}{1-m}
\label{eq:RULE}
\end{align*}



\chapter{Initial offer calculations}
\section{Code Discussion}

% \begin{verbatim}
% \end{verbatim}

\[\delta(t)=\left(\frac{1}{1+r}\right)^t\]


    %\href{https://www.google.com/url?sa=t&rct=j&q=&esrc=s&source=web&cd=&ved=2ahUKEwiOmNPUvIL9AhUUmokEHX-5C9oQFnoECBIQAQ&url=https%3A%2F%2Fwww.greatersudbury.ca%2Fcity-hall%2Ftax-services%2Fpdf-documents%2F2022-tax-rates%2F&usg=AOvVaw2XEdfcC5z-5AqfOeH5t-eN}{Sudbury values}

    
\section{Equation}
Equation~\ref{eqn-bid-price2}:


\begin{align}
P_i^{bid} \le   \frac{\mathcal{R}_N}{(1-m_i)r_i^{target}- \delta_i \left(1+L(P)- (1+ r_i)m_i\right)}.
\end{align}

Notation

$delta(0)=1$  for funds received now

Lets calculate the
\begin{align}
delta (1)   &= 1/(1+r) \\
delta (t)   &= (1/(1+r))^t \\
delta (\infty)   &= \sum_0^\infty\left(1/(1+r)\right)^t\\ 
Let\ a=1/(1+r)&<1\\
delta (\infty)   &= \sum_0^\infty a^t\\ 
S               &= \sum_0^\infty a^t\\ 
               &= 1+a+a^2+a^3+a^4 \dots\\ 
S-aS             &= 1\\ 
S             &= 1/(1-a)\\ 
S             &= \frac{1}{1-\frac{1}{1+r}}\\ 
             &= \frac{1}{1-\frac{1}{1+r}}\\ 
             &= \frac{1}{\frac{1+r-1}{1+r}}\\ 
              &= \frac{1}{\frac{r}{1+r}}\\ 
             &= \frac{1+r}{r}
\end{align}
This is the case where you get paid at the beginning of the first period. Rent might be paid in advance for each month. .

For  the case where you get pay at the end of the first period. Interest on a loan  might be paid this way.  summing from t=1 this time, we get $S-aS =a$, and $S = a/(1-a)= \frac{1}{r}$. 

 

%\[\delta(t)= \delta(1)^t =\left(\frac{1}{1+r}\right)^t\]

Finally, we need an expression for the sum of a finite  series to T.  Doing the derivation to check my logic, notice that  omega, psi, c  and a are constants that can be factored out in the following
\begin{align}%
    tax&= \sum_{t=0}^{T-1} \delta(t) \left(c*\omega + c*a*\psi \right)\\
        &= c(\omega + a\psi)\sum_{t=0}^T  \delta_t\\
        &= c(\omega + a\psi)(\delta_1+\delta_2 \dots \delta_T)\\
        &= c(\omega + a\psi) \left(\sum_0^\infty \delta_t-\sum_{T}^\infty \delta_t\right)\\
        &= c(\omega + a\psi) \left(\frac{1+r}{r}  - \left(\frac{1}{1+r}\right)^{T+1} \left(\frac{1+r}{r} \right) \right)\\
        &= c(\omega + a\psi) \frac{1+r}{r}\left(1  - \left(\frac{1}{1+r}\right)^{T+1} \right)
%      &=& c(\omega + a\psi)\delta_T\\
\end{align}




\newcommand{\ee}[1]{\color{red}#1 \color{black}}

\subsection{REVISING and  Initializing the model}
\subsubsection{housing prices}

A house offer two kinds of services - home services proper and locational services - access to the central city job. The value of a house then is the present discounted value of the flow of those services

Locational services are, on an annual basis $\omega- trans * d$.

Housing services absorb about 30\% of income  we will use than number  as an approximation,  $a\psi$, where $a$ is the share of the subsistence wage and $\psi$


A property at $d$ offers the combined annual services of 
\[h=\omega- trans * d + a\psi\]
and the value of the house is then the discounted value of the stream of services,
\[H=\frac{h}{r}\]

We will use this value as the price for the first period. 
 $P_0=\frac{\omega- trans * d + a\psi}{r}$

This is what we can think of a a ``warranted price'', which exactly covers the actual value of the services. The warranted price  is not the same as the market price.  \textbf{The market price can depart from the warranted price} when there is price speculation based on expected future values or if agents make errors based on market information.

Annual rent charged initially is  equal to the value of services, $h$:
\[\mathcal R^{charged}= h= \omega- trans * d + a\psi\]



\subsubsection{Annual Taxation}
Municipal taxes are based on the price of the house, $P_0=\omega- trans * d + a\psi$ - usually with a lag, 

In each price a period a `mill rate'. Each mill is  expressed as  1/1,000 of the value is determines - The annual tax is initially

 $(taxrate)*P_0=(millrate)*\frac{\omega- trans * d + a\psi}{r}$

This has to be compounded for decisions made over a period f more than one year.


\subsection{Distinguishing market rental price and locational rent}



Making these distinctions, we find that Equation C3 is wrong.

\begin{align}
\mathcal{R}_N &= \mathcal{R} - \mathcal{O} - \mathcal{T}
\end{align}

Why?  A tenant should expect to pay a market price for the combination of locational services and the services of land and buildings that they use.  It is an equilibrium market rent, not the economic locational rent economic rent. (The market can cause rents to diverge from this value. ) 

We should distinguish the locational rent from the $\mathcal{R}^l = \omega - trans*d$ from the Warranted rental cost of housing services. (Maybe use a capital L?) 

Market  rental price $\mathcal{R}^w$ should  include the services of the house and land.  $\mathcal{R}^w =\mathcal{R}^l+\mathcal{R}^h$ 

There is another distinction: The warranted value is base on the true value of the services, while the market value can diverge. We do the calculation in terms of warranted values because we will use these to initialize the model. It might be better to use the subscript $m$ for `market'. 

For an investor, it is Net Warranted market  rental price that is relevant in decision making. For now, define (decide on the symbols you really want).
\begin{quotation}
a  =  share of subsistence wage  used for land and building e.g. 0.3

b  = share of share of subsistence wage  used on maintenance e.g. 0.2

c  = annual tax rate on rent and home  e.g. 12 mills = 0.012

\end{quotation}
Now Net Warranted market  rental price is
\begin{align}
\mathcal{R}_N^w &= \omega - trans*d + a\psi -  \mathcal{O} - \mathcal{T} \\
\end{align}

It is also Net Warranted market  rental price that is relevant in taxation and maintenance. 

\begin{align}
\mathcal{O} &\equiv  ba\psi\\%Maintenance?
 \mathcal{T} &\equiv ca\psi %taxes?
\end{align}
so we can write
 \begin{align}
\mathcal{R}_N^w &= \omega - trans*d- + a\psi -   ba\psi - ca\psi \\
\end{align}

All of these are in annual values. We will use the present values  for the appropriate period  T in computations. (See notes on the present value calculations to use)

\textbf{These terms can all be used as initial values.}

When a property is rented, the rent charged will be recorded and used in \textbf{future rent calculations}.   (i,e., with a lag)

Taxes will be updated after the period T (i,e., with a lag)

\subsection{NEW WORK STARTS - How does this affect the bid price?}

I now return to Equations B6 to compute the value of a proposed purchase


\begin{align*}
V &= \delta \left(P_T - (1+r)M\right) +      \mathcal{R}^w_N  \tag{B6} \\
  &= \delta \left((1+\dot P)    - (1+r)m    \right) P^{bid} + \mathcal{R}^w_N%\label{eqn-property-investment-value1}
\end{align*}
Similarly, Equation B7 becomes

\begin{align*}
r_{return} = \frac{\delta \left(1 + \dot P - (1+r)m\right)}{1-m} + \frac{\mathcal{R}^w_N}{(1-m)^{bid}}\tag{B7}
\end{align*}
And following the same sequence of steps to derive the bid price 

\begin{align*}
P_i^{bid} \le &   \frac{\mathcal{R}^w_N}{(1-m_i)r_i^{target} - \delta_i \left(1 + L(P) - (1+r_i)m_i \right)} \tag{B10}\\
		 \le &   \frac{ \omega - trans*d + a\psi -   ba\psi - ca\psi}{(1-m_i)r_i^{target} - \delta_i \left(1 + L(P) - (1+r_i)m_i \right)}	\\	
   		 \le &   \frac{ \omega - trans*d + a(1-b-c)\psi}{(1-m_i)r_i^{target} - \delta_i \left(1 + L(P) - (1+r_i)m_i \right)}	
\end{align*}

Our conclusion about the advantage that the bank and the wealthy have can be read out of this equation. 
That may let us simplify the exposition in the financialization chapter.







