\chapter[Bidding]{Calculating Bid Price}
\label{AppendixBidPrice}

\section{Calculate rate of return for an investor}
% \section{Investment decision and bid price}
The value of a property purchase over the course of a mortgage term is the \gls{capital gain} minus the financing cost plus the \gls{present value} of the net rental income, $\mathcal{R}_N$. 



\begin{align}
V  	& = Net\ Capital\ Gain & +\ & \mathcal{R}_N \\
    & =  Capital\ gain - Interest\ due      &+\ &   Rent  - Operating\ cost - Taxes \nonumber \\
    &= \delta(P^T- (1+r)M)  & +\ & \mathcal{R} -C - T  \\
    &= \delta(P^T- (1+r)M)  & +\ & \phi\mathcal{R}\label{eq:Value}
\end{align}

Where $\phi$ is a fraction that takes into account taxes and operating costs. 

% ___ NOT SURE WHY WE NEED $\phi$. and don't love writtent name

 The warranted price of the property is the present value of the \gls{warranted rents}
 
% *** Note the warranted price may be rising. This is a problem for  us. WHY?
% NOTE Some of the taxes and most of the repairs/maintenance costs are for the building and not the for the locational value of the land. These two have correlated values. (we can't separate the amount of house and quality form our locational decision ) 

%**** We would like to  somehow ignore the building and push it into the subsistence wage. Can we do this? EXPLAIN
% How do we ignoor buildings - when we talk about the land rents, we're calculating on the basis of the warranted rent wage premium- put people are paying for buildings.. - assume it comes out of the subistence wage- since this is just a theoretical treatment.-- but if you think about how much capital gain will you get, we'll also get capital gainso on the house piece which is not locational- just gains on that asset they happen to have.. --  looking at things coming from close toe the center ignoring what else. - tied together so market spreads capital gain over both of them, and the tax does the same too-- for locational advantage and services..

% Since the present value of rental income (OR NET RENTAL INCOME?) is the value of the house, i.e.,\\

In principle, the  sale price of a unit of land considered is the present discounted value of the stream of net rents. If $\mathcal{R}_N$ is one period rent, which is conventionally calculated as
\begin{equation}
  P=\frac{\mathcal{R}_N}{r}  \label{eq:Capitalization}
\end{equation}

%\[v= \delta(P^T- (1+r)M)    +\phi\mathcal{R}\]

\noindent so we can write
\begin{align}  
    V & = \delta \left(1+\dot P)  P_0- (1+r)mP_0 \right)  + \psi rP_0  \\
    & =\left( \delta ((1+\dot P)  - (1+r)m)\  + \psi r\right)P_0
\end{align}

This can be expressed in terms of the rate of return on the down payment, $V/D$
\begin{align}  
   v & = \frac{V}{D}
   \\
    & = \left( \delta ((1+\dot P)  - (1+r)m)\  + \psi r\right)P_0/D
   \\
    & = \left( \delta ((1+\dot P)  - (1+r)m)\  + \psi r\right) \frac{P_0}{P_0-mP_0}
    \\
    & =  \frac{\delta ((1+\dot P)  - (1+r)m)\  + \psi r}{1-m}\label{eq:RULE}
\end{align}


\section{Agent's maximum bid given the rate of return}
% \section{Market behaviour}
Equation~\ref{eq:RULE} provides a criterion for investors: invest if $v \geq r^{target}$. The rule has qualitative implications for the ownership of housing (Discussed in chapter on financialization)

Equation~\ref{eq:RULE} is expressed in terms of a given market price, $P_0$. It cannot be solved for the $P_0$, however.

To find how much investors will bid, we need to find the maximum price that satisfies this criterion, which we define as the ``bid price'', $P^{max}_{bid}$.   The bid price is used in the price determination process in our model. We return to Equation~\ref{eq:Value} %replacing $\mathcal{R}$ with $rP_0$. 

\[V= \delta(P^T- (1+r)M)   +\phi  \mathcal{R}\]
% We also have to assume that $P^{bid}=P_0$, which we can justify as an equilibrium condition - investors believe they are paying the market price. The result is 
% \[V= \delta(P^T- (1+r)M)   +\phi r P^{bid}\]

We assume that the present value of $\mathcal{R}$ is known to an investor in advance. We can imagine the investors' accountant having information on the rent that the market will bear or on prior rents and including this information in the calculation of $P^{bid}$.

The  price at the end of the term $T$,  $P^T$, is a predicted value  for the investor \textit{ex-ante}, so we have to replace $P^T$ in Equation~\ref{eq:Value} with a predictor, $(1+\dot P)P^{bid}$. 
\[V= \delta((1+\dot P)P^{bid}- (1+r)M)   +\phi  \mathcal{R}\]

Then we replace $\dot P$ with and estimate, $L(P)$, representing an estimated function of the lagged values of $P$ and any  other relevant data. We imagine the potential investor drawing on common  knowledge or information from the real estate agents or analysts.  The result is 
\[V= \delta((1+L(P))P^{bid}- (1+r)mP^{bid})   +\phi \mathcal{R}\]
Combining terms
\[V= \delta((1+L(P))- (1+r)m)P^{bid}   +\phi \mathcal{R}\]

The rate of return $v=V/D=V/(1-m)P^{bid}$ is then

\[v= \frac{\delta((1+L(P))- (1+r)m)}{1-m}   +\frac{\phi \mathcal{R}}{(1-m)P^{bid}}\]

% %++++++++++++++++

% \begin{align}  
%   v  & =  \frac{\delta ((1+\dot P)  - (1+r)m)\  + \psi r}{1-m}\label{eq:RULE}\\
%   & =  \frac{\delta ((1+\dot P)  - (1+r)m)\  + \psi r}{1-m}\label{eq:RULE}
% \end{align}

% \begin{eqnarray}
% v %&=& \delta(P^T- (1+r)M) \qquad \qquad \qquad 	 + \mathcal{R}_N \nonumber\\
% % &=&\delta\left( (1+\dot P)P^{bid} - (1+r)mP^{bid} \right)  + \mathcal{R}_N  \nonumber\\
%   &=&\delta\left( (1+L(p)) - (1+r)m \right) P^{bid} + \mathcal{R}_N  \nonumber
% \end{eqnarray}

The rule is, ``Bid if the RHS is larger than the target rate of return, $r^{target}$, and do not bid if it is smaller.''  The maximum bid  is the bid that makes the two sides equal. 

\begin{align}
r^{target} &= \frac{\left[\dots \right]}{1-m}   +\frac{\phi \mathcal{R}}{(1-m)P^{bid}}. \\
%
(1-m)r^{target} &= \ \  \left[\dots\right] + \frac{\phi \mathcal{R}}{P^{bid}}\\%\delta(1+L(P))- (1+r)m%
%
(1-m)r^{target}-\left[\dots\right]  &=  \frac{\phi \mathcal{R}}{P^{bid}}\\
%
P^{bid} &=    \frac{\phi \mathcal{R}}{(1-m)r^{target}-\left[ \dots\right]}\\
%
P^{bid} &=    \frac{\phi \mathcal{R}}{(1-m)r^{target}-\left[ \delta(1+L(P)- (1+r)m\right]}
\end{align}

The denominator is an adjusted rate of return for capitalizing net rents, analogous to the value of $r$ in Equation~\ref{eq:Capitalization}. 


% \begin{eqnarray}%. OLD VERSION: WRONG
% %r^{target}&=& \delta\left( (1+L(p)) - (1+r)m \right) P^{max}_{bid} + \mathcal{R}_N  \nonumber\\
%    P^{max}_{bid} &=&\frac{r^{target} - \mathcal{R}_N}{\delta\left((1+L(p)) - (1+r)m \right)} %\label{EqBidPrice2} 
% \end{eqnarray}
In practice, potential investors will make an  initial  bid that is lower than this value and subsequent bargaining will settle of a price between the initial bid and the seller's asking price.


% \section{Finding bid price}
% We start with Equation~\ref{B2}. for convenience, replace $\rho -\kappa - \sigma $ with $\mathcal{R}_N$ (net Rent). 

% Replace $P^T$ with $(1+\dot P)P^{bid}$ assuming that the bidder is bidding the equilibrium market price for the period.

% Then replace   $\dot P$ with $L(p)$ representing some (estimated function ($\tilde{\dot P}$)) of the lagged values of $P$ that incorporates other data. 

% \begin{eqnarray}
% v&=& \delta(P^T- (1+r)M) \qquad \qquad \qquad 	 + \mathcal{R}_N \nonumber\\
%  &=&\delta\left( (1+\dot P)P^{bid} - (1+r)mP^{bid} \right)  + \mathcal{R}_N  \nonumber\\
%   &=&\delta\left( (1+L(p)) - (1+r)m \right) P^{bid} + \mathcal{R}_N  \nonumber
% \end{eqnarray}

% So I want to use this relationship to find the maximum bid price for the bank. The rule is, ``Bid if the RHS is larger than the target rate of return, $r^{target}$, and do not bid if it is smaller.''  The maximum bid  is the bid that makes the two sides equal. 

% {\color{red}
% \begin{eqnarray}
% r^{target}&=& \delta\left( (1+L(p)) - (1+r)m \right) P^{max}_{bid} + \mathcal{R}_N  \nonumber\\
%    P^{max}_{bid} &=&\frac{r^{target}-\mathcal{R}_N}{\delta\left( (1+L(p)) - (1+r)m \right)} \label{EqBidPrice} 
% \end{eqnarray}}
% %(What makes. this work is that I do not use an identity to get $\dot P$, which made the system of equations singular.)
% \newpage

\section{Other factors affecting price determination}
% \subsection{On bidding}
In the market, the initial bid should be smaller than the bid price calculated, which is a maximum that can earn the target rate of return, The bank will definitely go this high. 

If there is a  max bid among competing buyers, the second highest max bid should be the sale price, but the bidder with the highest max bid wins the property. This makes sense because in a bidding war the final bid only has to be a very small amount higher than that of the last competitor left.
That is the highest price that a seller can get.

It might be simplest to treat every seller as a bidder. All bid above her own are considered. The seller chooses the second highest bid. This  seems realistic enough and is very simple to implement. It should producer a path that is indistinguishable form any more complex approach. 

Will persons retiring who would leave the city invest in an urban rental?

Bargaining between buyers and sellers differs. Buyers bid low and sellers ask high. {\color{red}We need to calculate a minimum selling price for the seller}.

% But what is $D$? Does the bank have unlimited funds? Isn't D just a fraction of P?  If so it cancels out
% r is the prime rate- that the bank pays? 
