\chapter{Methodology} \label{chapter-methodology}

In this chapter we discuss the methodological contribution that the thesis makes, as well as a number of modelling decisions.

The main methodological innovations, as we see it, are:

\begin{enumerate}
    \item We incorporate classical rent theory in an agent-based model and allow the creation and distribution of rents to influence urban growth, productivity and   population structure.  
    \item we construct an ABM that is consistent with neoclassical decision-making and neoclassical growth theory,
    \item we incorporate current research on urban scaling into the  core spatial urban model.  
    \item We employ an ABM to examine how financial markets impact the urban,  and test for hysteresis
    \item 
\end{enumerate}

Ours is a  model of a multi-agent economic system. We 
% Formal economic modelling is still largely characterized by on analysis of systems in equilibrium. It has been highly productive, providing tractable models.  equilibrium analysis of marginal effects, and representative agents which hid distributional effects, as well as spaceless economic models of markets made it difficult to capture the richer spacial dynamics of urban rents, and the details of the ways economic forces play out for individual actors.

% More recent work has looked at agent-based modeling, looking at the dynamics, space and individual work. 

\textbf{We offer three developments/extensions building on that work.}

First they've not tended to build in, in a sophisticated way classical economic theory,
In general, classical economic theory has not been developed in agent based modeling work
Agent based models have begun with simple models, using and relaxing neoclassical assumptions, and building from first principles. This is absolutely the place to start. As the field matures, it makes sense to introduce theory in a more nuanced way, that connects with classical theory/the history of thought, etc

Second, relatively little agent based modelling work integrates with the neoclassical economic work in a way that makes the relation clear/holds the advantages. ABM work tends to both reject neoclassical approaches and rely on neoclassical assumptions.
More generally only a few agent models (e.g. spruce budworm) connect the analytic and agent models in a clear rigorous way. We focus on holding in addition to the relation with classical theory, a close connection with the many advances made withing neoclassical modelling

- this connection will make it easier to incorporate in teaching and for mainstream economists to engage on and build with.


This work builds, first, a simple conceptually clear model tightly integrated with the core economic modelling traditions, that builds on the theory of rent.

% ALSO (Agent modelling also tents to model individuals- we also take some steps to agent based modelling beyond the individual, and to the work developing model in a mode ideas)

Third, econ lacks resilience analysis and models, yet hysteresis clearly present in the relation between the built environment and econ activity. Although there's been work on dynamics and individual effects, there has been little work looking at the resilience dynamics in economic models, we take that approach looking at the resilience of community and individual wealth, and the relationship between that wealth and productivity. 

- This puts resilience dynamics at the center of economic analysis.

The resilience analysis looks at the dynamics of rent in economic boom and bust cycles.
There is a ratchet effect, achieved through hysteresis in the system, in which sucks wealth out of communities on the way up and on the way down. % DETAIL ONCE DRAFTED.


\section{Other notes - to sort}

% TEMP - here are some other notes we may want to reference or bring in. [[non individualistic modeling of agents]] [[generalizability in agent vs classical econ models]]

The emphasis is on clarity and connecting with the equilibrium in economics, and systematically relaxing each, to connect with the analytic tradition of economic modelling
The clarity of intuition of the neoclassical tradition with the deeper root of distribution theory rooted in classical economics and the breath and rigor possible with new tools from the study of complexity and statistical physics.

Methodological questions: 

    - agent models (integrating theory more completely into agent models)
    
    - rent theory

Core model

    - static version
    
    - dynamic version

Simulations
Result---> hysteresis
Policy