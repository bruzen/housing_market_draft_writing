\chapter{Methodology} \label{chapter-methodology}

In this chapter we discuss the methodological contribution that the thesis makes, as well as a number of modelling decisions.

The main contibutions%methodological innovations
, as we see it, are:

\begin{enumerate}
    \item we incorporate classical rent theory into an agent-based urban model 
    \item we allow the creation and distribution of rents to influence urban growth, productivity and  population structure. 
    \item we incorporate current research on urban scaling into the  core spatial urban model.   
    \item we construct an   Urban ABM that is consistent with neoclassical growth theory,
    \item we integrate financial capital into a standard spatial model of the urban system
    \item we integrate financial capital into an overlapping generations population model of the urban system
    \item we employ the ABM to examine how financial markets impact the urban land markets 
    \item we test for hysteresis  resulting from the business cycle  in the urban system 
    \item we build a model that is easily extended to explore a wide range of issues
    \item we provide a model that we believe can be used  to evaluate urban policies
\end{enumerate}


Each of the items above requires us to  integrate 


\begin{enumerate}
    \item we incorporate classical rent theory into an agent-based urban model 

This requires a how Ricardian rent theory - a theory of distribution based on an agricultural economy - applies in an economy driven by human capital agglomeration effects within the urban system. 

    \item we allow the creation and distribution of rents to influence urban growth, productivity and  population structure. 
    
This requires us to articulate the links between the wealth production of cities and  urban systems evolve 
    

    \item we incorporate current research on urban scaling into the  core spatial urban model.  

We show how understanding how complex systems scale provides an elegant way to model a an urban system with production

    \item we construct an   Urban ABM that is consistent with neoclassical growth theory,

Although many  who employ ABMs to analyse urban systems are deeply skeptical of neoclassical assumptions we demonstrate in this context how easily and productively the neoclassical frame work can be implemented in n the ABM framework

    \item we integrate financial capital into a standard spatial model of the urban system

We are not aware of any urban simulation models that introduce capital as we do to examine the outcomes that concern us.
    
    \item we integrate financial capital into an overlapping generations population model of the urban system

Financial assets are central to most overlapping generations models. Our innovation is to articulate the movement of financial capital though the urban land market.
    
    \item we employ the ABM to examine how financial markets impact the urban land markets 

We began with with the fact that there is growing policy concern about the ``Financialization of  Housing''. We have produced a formal simulation model that illustrates the process. 

    \item we test for hysteresis resulting from the business cycle   in the urban system 

There are two types of resilience questions when any system is shocked, does it return to an equilibrium state - the stability question - and does it return to the same kind of equilibrium - the hysteresis question. We focus on the latter question.  


    \item we build a model that is easily extended to explore a wide range of issues

The model combines clear and explicit theoretical assumptions with careful and transparent implementation of the logic in flexible Python code.

    \item we provide a model that we believe can be used  to evaluate urban policies


\end{enumerate}


% 

% More recent work has looked at agent-based modeling, looking at the dynamics, space and individual work. 

\textbf{Some limitations of preceding work}%We offer three developments/extensions building on that work.}

Ours is a  model of a complex multi-agent economic system. We draw on techniques and insights from several fields. In doing so we are attentive to the strengths and weaknesses of  the  approaches that we are adopting. 

In Economics, for example, the most familiar approach is the  analysis of systems in equilibrium. It is a highly productive methodology that simply bypasses the complex process of adjustment by focusing on conditions that must be true if a particular situation is to persist. The classic example is the ubiquitous supply and demand model. Each of the curves represents the plausible behaviours of a class of agents. A  situation is not likely to persist if either of the classes of agents is not satisfied with the combination of price and quantity. The conclusion   is that the only combinations that can persist for long are  those that satisfy the behavoural intentions of both classes - i.e., are on both curves. If the two curves can be described mathematically, equilibrium prices and quantities  can be derived solving the two-equation system.

We use  equilibrium arguments in our model to describe hiring decisions by producers and wage demands by workers. Equilibrium locational by commuters determines the extent and even population of the city. they also determine the pattern of land rents. 

We are focused on the evolution of a city, however. It is never in equilibrium. At best the various agents in the city are adjusting in sensible ways to the changing parameters of the city they inhabit. Some adjustments are slow and some are fast. 

At the same time, because our  

that we 

providing tractable models.  equilibrium analysis of marginal effects, and representative agents which hid distributional effects, as well as spaceless economic models of markets made it difficult to capture the richer spacial dynamics of urban rents, and the details of the ways economic forces play out for individual actors.

First they've not tended to build in, in a sophisticated way classical economic theory,
In general, classical economic theory has not been developed in agent based modeling work
Agent based models have begun with simple models, using and relaxing neoclassical assumptions, and building from first principles. This is absolutely the place to start. As the field matures, it makes sense to introduce theory in a more nuanced way, that connects with classical theory/the history of thought, etc

Second, relatively little agent based modelling work integrates with the neoclassical economic work in a way that makes the relation clear/holds the advantages. ABM work tends to both reject neoclassical approaches and rely on neoclassical assumptions.
More generally only a few agent models (e.g. spruce budworm) connect the analytic and agent models in a clear rigorous way. We focus on holding in addition to the relation with classical theory, a close connection with the many advances made withing neoclassical modelling

- this connection will make it easier to incorporate in teaching and for mainstream economists to engage on and build with.


This work builds, first, a simple conceptually clear model tightly integrated with the core economic modelling traditions, that builds on the theory of rent.

% ALSO (Agent modelling also tents to model individuals- we also take some steps to agent based modelling beyond the individual, and to the work developing model in a mode ideas)

Third, econ lacks resilience analysis and models, yet hysteresis clearly present in the relation between the built environment and econ activity. Although there's been work on dynamics and individual effects, there has been little work looking at the resilience dynamics in economic models, we take that approach looking at the resilience of community and individual wealth, and the relationship between that wealth and productivity. 

- This puts resilience dynamics at the center of economic analysis.

The resilience analysis looks at the dynamics of rent in economic boom and bust cycles.
There is a ratchet effect, achieved through hysteresis in the system, in which sucks wealth out of communities on the way up and on the way down. % DETAIL ONCE DRAFTED.


\section{Other notes - to sort}

% TEMP - here are some other notes we may want to reference or bring in. [[non individualistic modeling of agents]] [[generalizability in agent vs classical econ models]]

The emphasis is on clarity and connecting with the equilibrium in economics, and systematically relaxing each, to connect with the analytic tradition of economic modelling
The clarity of intuition of the neoclassical tradition with the deeper root of distribution theory rooted in classical economics and the breath and rigor possible with new tools from the study of complexity and statistical physics.

Methodological questions: 

    - agent models (integrating theory more completely into agent models)
    
    - rent theory

Core model

    - static version
    
    - dynamic version

Simulations
Result---> hysteresis
Policy