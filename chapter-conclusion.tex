
\chapter{Conclusion}

Adding 2 things 1. rent extraction and 2. power law scaling of productivity, we find rent is the breaks on the engine of wealth creation

The economics is clear that this is what's at stake is productivity of cities, the distributive features of the economy and the impact of the middle class.

The analysis in this dissertation makes clear that in addition to distributional consequences, the housing crisis has productivity impacts. Specifically, the analysis in this thesis concludes that, given the ongoing financialization of the housing market,

\begin{enumerate}
\item the financial system will eventually extract all net urban land rents through investment in urban property
\item housing accessibility will become increasingly challenging for disadvantaged groups
\item housing will be largely eliminated as a saving mechanism and asset fr middle income Canadians,  resulting in a systematic decline in the `middle class'
\item that the quality of urban life will decline
\item the economic growth and development of cities is threatened by this financialization
\end{enumerate}



% These implications should be considered in policy.

Repeat from introduction contributions
- not sensitive to any advantage being true all of the time. etc