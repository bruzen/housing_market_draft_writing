\begin{center}\textbf{Abstract}\end{center}
% \begin{center}\textbf{Urban Productivity and the Financialization of the Housing Market}\end{center}
% \begin{center}\textbf{Financialization of the Housing  Market: A Contribution to Modern
% Urban Rent Theory}\end{center}
% \begin{center}\textbf{Lost Productivity: Financialization of Housing in the Urban System}\end{center}
% \begin{center}\textbf{From Owners to Tenants: Financialization in anAgent Based Urban Model with Agglomeration Effects}\end{center}


% MOTIVATE BETTER: E.G. from introduction
% \epigraph{Cities are the crucible of civilization, the hubs of innovation, the engines of wealth creation and centers of power, the magnets that attract creative individuals, and the stimulant for ideas, growth, and innovation.}{Geoffrey West \cite{westScaleUniversalLaws2017}}

% Since the industrial revolution, widely distributed property ownership, particularly in North America, has contributed to the wide and relatively equal distribution of {wealth} \cite{pikettyCapitalTwentyfirstCentury2014, harrisGrowthHomeOwnership1977, chevanGrowthHomeOwnership1989, andrewsEvolutionHomeownershipRates2011}.\footnote{The role of housing in wealth accumulation is well known. After a detailed  examination of data from the US Survey of Consumer Finance (SCF) and the Panel Study of Income Dynamics (PSID), Herbert et al. concluded that ``there continues to be strong support for the association between owning a home and accumulating wealth.'' The effect persists even among minorities and lower-income households. In lower-income minority households, on average, renters do not see any gains in wealth \cite{herbertHomeownershipStillEffective2013}.} However, property ownership is now going through a great transformation in structure, with financial capital coming to own a larger share of the urban land and housing stock \cite{farhaReportFinancializationHousing2017, palleyFinancializationWhatIt2007}. For the first time in Canadian history the rate of homeownership has declined, falling from 69\%  in 2011 to 66.5\% in 2021 \cite{statisticscanadaBuyRentHousing2022}. % BLACKSTONE
% In Ontario, 70\% of new condo units are owned by investors not residents \cite{pickelInvestorsOwn772023} and asking rent in Canada increased by 10\% over last year, as of February 2024 \cite{urbanationNationalRentReport2024}. 
% % https://kitchener.ctvnews.ca/investors-own-77-per-cent-of-new-condos-in-waterloo-region-1.6273766 % Ontario numbers in email from DiR

% INTRODUCE Cities drive the creation of value,  1. Cities drive the creation of value.. WEALTH AND VALUE IS CREATED IN CITIES THROUGH the productivity of cities AND 2. PROPERTY IN HOUSE OWNERSHIP IS IMPORTANT TO DISTRIBUTION OF WEALTH. 

% Cities are at the heart of how people create value. 

% and ownership of the land and properties in cities is important to how that wealth is distributed.
% As financial actors buy up more of the property near cities, it can affect both distribution and the capacity of cities to build human capital and create value. 


A great deal of wealth is produced through the economic activity of cities. There is a gap, however, in the formal apparatus in standard economic theory for analyzing the distribution of this enormous value created in cities. 

In the context of a widely-felt housing crisis, we explore how the capture of urban value by financial actors through the financialization of the housing market affects ownership patterns in urban areas, and the ultimate implications of these processes for urban productivity.  We hypothesize that financialization induces a shift towards tenancy among the urban workforce that is likely to result in decreased urban productivity through a range of channels. To examine this hypothesis, we construct an agent-based model with a land market and production sector in which productivity scales superlinearly with city population. This work brings together urban agglomeration effects, Ricardian rent theory and a spatially explicit land market model in a novel way. 

In our model, transportation costs determine the size of the city, and the available locational rents. Rising productivity increases wages and urban land values, so the value of increased productivity is transferred to land owners. Investors attempt to capture these productive gains by purchasing land. These financial actors can bid against residents to purchase urban land. The interaction of agents determines the distribution of property ownership,  city size, and wages. City size and wages provide a measure of urban productivity. The evolving pattern of property ownership tells us how residents are distributed between the tenant class and the owner class. 

We then explore a range of channels through which financialization might result in decreased urban productivity. When we add this link in the model, we see that financialization not only transforms the class structure of the city and the distribution of urban wealth, it disrupts the relationship between population growth and productivity, reducing the wealth and resilience of the urban system. To illustrate the uses of this kind of computational model for economic policy analysis, we run six policy experiments with and without the productivity link.  

\textbf{Contributions of this work include:} \textit{integrating classical rent theory into an agent-based urban model; linking urban rent dynamics with urban productivity, and population growth; incorporating urban scaling literature into the model framework; examining the impacts of financialization on wealth distribution and urban productivity; creating a framework for a broader understanding of public policies in an urban system; and examining the qualitative effects of various public policies on wealth distribution, productivity, and class.}


%We describe the channels through which financialization is likely to affect urban productivity and explore the implications of the link. 