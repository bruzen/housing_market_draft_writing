The dissertation posits two core hypotheses: firstly, that financialization induces a shift towards tenancy among the urban workforce. Secondly, that this results in decreased urban productivity due to the redirection of spatial rents away from local investments financialization disrupts the relationship between population growth and productivity, reducing the wealth and resilience of the urban system.

DONE WITH MODELS INCLUDE LIMITATIONS

Contributions of this dissertation include: integrating classical rent theory into an agent-based urban model; linking urban rent dynamics with urban productivity and population growth; incorporating urban scaling literature into the model framework; and examining the impacts of financialization on wealth distribution and urban productivity.


## Contributions:

0. Having understood what you mean by rent/by the implications of rent in an urban situation
1. First we put financialization into a model an agent based urban model that is well constrained and well understood, and show that in a plausibly specified model, the financialization is driving the change in the class structure. 
2. The economic literature does not examine the class impact. Economists don't talk about. Sociologists talk about it, but they don't model it.
3. We model boom bust dynamics/response. how it responds to changes in the outer world - fact you could get a change with an interest rate. what happens when the model is repeatedly struck. What are the impact effects and how do they play out?
4.  We ask the question about if there's a spillover to productivity. There's speculation about the link in the left win urban literature but we haven't found any modeling.

This is what's done in the climate literature. The dynamics and resilience effects are not obvious from the equations and must be modelled. The modeling is important because empirical dynamics are really hard to establish. This is a 1 off world. We can't run 50 versions. You can, however model 50 versions. 

The basic fiancialization result is clear from looking at the model. The equations only make it obvious once you put it together, however. They way they are combined is not obvious.  It is important to note also that the results are not a consequence of building a model that will give any result, but only a consequence of building a model that behaves urban theory and the economic theory predicts. None of the equations were constructed to give a result. Each is our best about how it works. Each step was a best guess at how the process works and based on a knowledge of the literature and the economics. The base theory in neither literature clearly predicts the class effects (there is some work in the literature e.g. Jacobs, reinvesting in Henry George literature - a few places where the economists have gone in that direction) We implemented a version of the best of the urban growth theories.

## Hypotheses
1. The financial sector affects the ownership of housing and the class structure of society, 
2. There are dynamic/resilience features of this model that make the effects worse
- Boom bust - pump wealth on/out of city on the boom and on the bust. 'There are sharks in the water' 1. Higher bid can amplify up swings - have to compete with speculators 2. can't get it back on down swing since outside finance offers a stable floor- buys up on way down (we expect larger effects as people are 1. displaced on the way up and then 2. evicted on the way down - can't use their spaces)
- Hysteresis - perturb, doesn't come back - e.g. interest rates go up.
- The way it aligns with long run changes in the landscape e.g. tech changing local info changes vulnerability to these shocks -- Depth of the basin changes - erodes systemic resilience - together these changes the capacity to hold value in landscape. (links to the productivity feedback - much will/can they invest in increasing their productivity/supporting kids/good food to grow brains. Education to increase productivity is the feature that makes productivity increases resilient to de-industrialization). This has implications for landscape/system/class structure.
3. This shift in ownership may have implications for urban productivity. (can actually displace productive uses - empty store fronts) - who can/will enter, how , who can rent spaces, speculative value may keep it empty (work spaces or living space - lowering pop), reduces consumption
