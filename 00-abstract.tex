% \begin{center}\textbf{Abstract}\end{center}
\begin{center}\textbf{Class and  Productivity: Financialization of Housing transforms the Urban System}\end{center}
\begin{center}\textbf{Lost Productivity: Financialization of Housing transforms an Urban System}\end{center}

% WHAT IS SPATIAL RENT
In the context of a widely-felt housing crisis, we explore how the capture of urban value for financial actors through the financialization of the housing market affects ownership patterns in urban areas, and the ultimate implications of these processes for urban productivity.
% We have a model of productivity to change. 


%\textbf{We are in a housing crisis.  These are things that can illuminate this particular.. **}
%We examine how urbanization and financialization interact through urban housing markets. % The dissertation examines a pair of linked hypotheses about the evolution of an urban system: firstly, 
We hypothesize that financialization induces a shift towards tenancy among the urban workforce. To examine this hypothesis, we construct an agent-based model with a land market and production sector in which productivity scales superlinearly with city population. This work brings together urban agglomeration effects, Ricardian rent theory and a spatially explicit land market model in a novel way. 
We then explore how financialization is likely to result in decreased urban productivity through a range of channels. 


%To test the 
%We built an agent based model to explain tenatization.
% models for each class of agent in the system and then allow agents to interact. 
% Firms hire workers if it is profitable. Firm productivity and wages rise with population due to agglomeration effects. 
In our model, transportation costs determine the size of the city, and the available locational rents. Rising productivity increases wages and urban land values, so increased productivity is transferred to land owners. Investors attempt to capture these productive gains by purchasing land. % Rural workers near the city apply for urban jobs if the urban wage justifies travelling to the city center to work. Homes come up for sale when workers retire New entrants buy homes if they can afford homes, otherwise they become tenants. Investors purchase or sell properties if the net rent and capital gains exceed the return on alternative investments. % this paragraph describes an agent-based approach and establishes the independence of the agents 
We therefore introduce financial actors who can bid against residents to purchase urban land in a %spatially explicit urban land market. To illustrate the uses of this kind of computational model for economic policy analysis, we run six policy experiments with and without the productivity link.  

% The most complex part of the model is the financial bidding for property by different agents. % Firm technology and the price of output are held constant. Investors represent the supply of global capita and are by assumption non-resident.
The interaction of agents determines the distribution of property ownership,  city size, and wages. City size and wages provide a measure of urban productivity. Property ownership is determined by competitive bidding. The evolving pattern of property ownership tells us how residents are distributed between the tenant class and the owner class. % It also allows us to see how the locational rents generated by the city is distributed between investors or residents.  % This paragraph says that results emerge from the interaction of independent agents./
By purchasing property within the model, financialized actors capture a share of urban productivity. % is captured through land rents. 
% By capturing rents, financialized actors are able to capture a share of urban productivity. 
This also results in increasing tenantization of the city. 



We then describe the channels through which financialization is likely to affect urban productivity and explore the implications of the link. %They don't live in the city, they can't reinvest, they don't belong to the class of those who own property and can save through homeownership. 
This linkage implies that financialization not only transforms the class structure of the city and the distribution of urban wealth, it disrupts the relationship between population growth and productivity, reducing the wealth and resilience of the urban system. 
%In the model, this affected the systemic effects of the shifting ownership of property in cities on the distribution of wealth and the potential that this shift can actually change the ability of cities to grow, thrive, and produce wealth. 
% Big chunks simplified but the guts are in financial behaviour. 

\textbf{Contributions of this work include:} \textit{integrating classical rent theory into an agent-based urban model; linking urban rent dynamics with urban productivity and population growth; incorporating urban scaling literature into the model framework; examining the impacts of financialization on wealth distribution and urban productivity, and creating a framework for a broader understanding of public policies in an urban system. Since agents respond to publicly set parameters, the model allows ups to examine the qualitative effects of  various public policies on both the wealth distribution, productivity, and class.}
% It is helpful for economic policy in complex environment. 
% The model allows us to test qualitative effects of various policies.


\section{Other stuff}

% ## Contributions:

0. Having understood what you mean by rent/by the implications of rent in an urban situation

1. First we put financialization into a model an agent-based urban land ownership model that is well-constrained and well-understood, and show that in a plausibly specified model, financialization drives a change in the class structure. 

2. The economic literature does not examine the class impact. Economists don't talk about. Sociologists talk about it, but they don't model it.

3. WE CAN We model boom bust dynamics/response. how it responds to changes in the outer world - fact you could get a change with an interest rate. what happens when the model is repeatedly struck. What are the impact effects and how do they play out?

4.  We ask the question about if there's a spillover to productivity. There is speculation about such a link in the left-wing urban literature but we haven't found any modelling.

This is what's done in the climate literature. The dynamics and resilience effects are not obvious from the equations and must be modelled. The modelling is important because empirical dynamics are really hard to establish. This is al1 off world. We can't run 50 versions. You can, however, model 50 versions. 

The basic fiancialization result is clear from looking at the model. The equations only make it obvious once you put it together, however. The way they are combined is not obvious.  It is important to note also that the results are not a consequence of building a model that will give any result, but only a consequence of building a model that behaves urban theory and the economic theory predicts. None of the equations were constructed to give a result. Each is our best about how it works. Each step was a best guess at how the process works and based on a knowledge of the literature and the economics. The base theory in neither literature clearly predicts the class effects (there is some work in the literature e.g. Jacobs, reinvesting in Henry George literature - a few places where the economists have gone in that direction) We implemented a version of the best of the urban growth theories.

% ## Hypotheses

1. The financial sector affects the ownership of housing and the class structure of society, 

2. There are dynamic/resilience features of this model that make the effects worse

- Boom bust - pump wealth on/out of city on the boom and on the bust. 'There are sharks in the water' 1. Higher bid can amplify up swings - have to compete with speculators 2. can't get it back on down swing since outside finance offers a stable floor- buys up on way down (we expect larger effects as people are 1. displaced on the way up and then 2. evicted on the way down - can't use their spaces)

- Hysteresis - perturb, doesn't come back - e.g. interest rates go up.

- The way it aligns with long run changes in the landscape e.g. tech changing local info changes vulnerability to these shocks -- Depth of the basin changes - erodes systemic resilience - together these changes the capacity to hold value in landscape. (links to the productivity feedback - much will/can they invest in increasing their productivity/supporting kids/good food to grow brains. Education to increase productivity is the feature that makes productivity increases resilient to de-industrialization). This has implications for landscape/system/class structure.

3. This shift in ownership may have implications for urban productivity. (can actually displace productive uses - empty store fronts) - who can/will enter, how , who can rent spaces, speculative value may keep it empty (work spaces or living space - lowering pop), reduces consumption

\section{Thesis abstract}

\section{Conference abstract}


\section{STUFF FOR CONCLUSION}


% 'Economic Policy in Complex Environments. computationally intensive methods for decision and policy analysis in complex and uncertain environments. Particular focus lies on the application of such methods in the domains of climate change and innovation.'
% INTRODUCE SPATIAL RENTS
% INTRODUCE CLASS
% CLEAR THERE IS NOT MULTIPLE FIRM.. MAYBE LAND MARKET FIRST, THEN SAY REP FIRM
% PRODUCTIVITY - NOT A HYPOTHESIS- WE LOOK AT EFFECT OF ADDING IT TO THE MODEL

%to explore financialization in the context of housing markets and 
% Consider the implications of the capture of spatial rents. 
% Ricardian rent.

% We're looking at financialization  via the housing market. 
% This is what we've put together to explore that. 
% These are the linked hypotheses 
% This is what we get 
