\begin{center}\textbf{Abstract}\end{center}
% \begin{center}\textbf{Urban Productivity and the Financialization of the Housing Market}\end{center}

In the context of a widely-felt housing crisis, we explore how the capture of urban value by financial actors through the financialization of the housing market affects ownership patterns in urban areas, and the ultimate implications of these processes for urban productivity.  We hypothesize that financialization induces a shift towards tenancy among the urban workforce that is likely to result in decreased urban productivity through a range of channels. To examine this hypothesis, we construct an agent-based model with a land market and production sector in which productivity scales superlinearly with city population. This work brings together urban agglomeration effects, Ricardian rent theory and a spatially explicit land market model in a novel way. 

In our model, transportation costs determine the size of the city, and the available locational rents. Rising productivity increases wages and urban land values, so increased productivity is transferred to land owners. Investors attempt to capture these productive gains by purchasing land. We therefore introduce financial actors who can bid against residents to purchase urban land. The interaction of agents determines the distribution of property ownership,  city size, and wages. City size and wages provide a measure of urban productivity. The evolving pattern of property ownership tells us how residents are distributed between the tenant class and the owner class. 

We then explore how financialization is likely to result in decreased urban productivity through a range of channels. We describe the channels through which financialization is likely to affect urban productivity and explore the implications of the link. This linkage implies that financialization not only transforms the class structure of the city and the distribution of urban wealth, it disrupts the relationship between population growth and productivity, reducing the wealth and resilience of the urban system. To illustrate the uses of this kind of computational model for economic policy analysis, we run six policy experiments with and without the productivity link.  

\textbf{Contributions of this work include:} \textit{integrating classical rent theory into an agent-based urban model; linking urban rent dynamics with urban productivity, and population growth; incorporating urban scaling literature into the model framework; examining the impacts of financialization on wealth distribution and urban productivity, and creating a framework for a broader understanding of public policies in an urban system. Since agents respond to publicly set parameters, the model allows us to examine the qualitative effects of various public policies on wealth distribution, productivity, and class.}