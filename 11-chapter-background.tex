\chapter{COMBINE WITH RENT Background} \label{chapter-background}

% There are two dominant stories of \gls{distribution} in economics. The first and oldest is based on the classical concept of rent as explained  by David Ricardo \cite{ricardoEssayInfluenceLow1815}, in which owners of an asset are able to extract a value beyond what they contribute based on their ownership of a scarce resource. 
% The second is the marginalist approach, developed by Clark and others, in which workers and other factors  in competitive markets receive the \gls{marginal value-product} of their contribution to production. 

% Both developed in response to the specific social and economic conditions of the periods in which they emerged. Both attempt to explain where the output of society ended up within society. they are, at their heart, stories of who  claims what share of production.  Classical rent theory  emphasized the distribution of the social surplus, the part of production  over and above what was needed to reproduce society. This included only land rents initially, but was extended to the distribution of profit profits.   These are exactly the types of surplus income that are not explained by the neoclassical theory, which is  awkward because profits and rents form a substantial part of national income (20–25 percent) in the world the neoclassical model describes. In this thesis we identify classical rents intn the urban systems and exxamine their distribution.. 
% % ***E FIRSTNAME Ricardo was a INSERT NATIONALITY classical economist  in the WHICH Century. 

%Classical economics emerged in the 16$^{th}$ and 17$^{th}$ centuries at a time of exploding colonial wealth. In this period,  European economies were still structured largely around agricultural production in which farmers OR peasants work the land  owned by the descendants of feudal aristocrats. Urban economies based around factories were just beginning to emerge as significant drivers of wealth.\footnote{The Industrial Revolution is usually described as beginning around 1760 and having significantly transformed society by about 1820–1840.} % It was only after 1650 that living standards in the UK did start to increase for a sustained period. Before the modern era of economic growth, the economy worked very differently. Not technological progress, but the size of the population determined the standards of living.} 
In this economic context, the agricultural surplus is captured by feudal aristocrats. 

%As WHAT LED TO THE EXPLOSION... CHANGES IN UNIVERSITIES OR THE ECONOMY, there was an explosion of thinking about social and economic structures and systems.  This was a period where a range of thinkers explored many ideas to explain the shifting social and economic conditions. 

%Distribution, or who got what share of production was a central concern. This focus came to the forefront in the 1600s and 1700s as colonial expansion of land holdings, factories and supplied change concentrated wealth- funding investment in the arts and sciences, great fortunes - harsher conditions for farmers, pushed of their land with enclosures, and concentrated in urban factories, almost exclusively poor with - illness bad working conditions % 90 rural,-- population snap shot % (unprecedented in northern Europe, a relative backwater) --  %gt (new ineuqlity- jusxtabosition- fortuioned unpreciendet ont hat content to rivla hsotori great- wons- woth workers dying, kids in factories- Dickens period, pesantry.. - both were new- the eilsure- )
% and concerned with inequality, which suited a time in which the labourers both on the land, and in the emerging factories struggled with poverty, at a time of rapidly growing -
% -the intense interest spiked as wealth grew in colonial Europe. %(reformation another answer to this)
% NOTABLY THIS WAS AN URBAN TRANSITION- FARMS TO SUBURBS- POWER GROWTH, AN INFLECTION - LIKE COVID REVEALED THE NEW FORM WITH A RAPID MOBILITY CHANGE..
% The new form is the urban
% dense exchange - experience in person density can compete with screen experience -

%In this context, Ricardo refined the classical concept of rent.  

% ***E Today the word rent is usually used to refer to payments made by a tenant to a landlord to for the right to occupy a property, but this is not the traditional economic concept of  rent.  

% ***E As a concept, it's more closely related to profit / surplus than to rent paid for properties. 


% ***E NEED TO ADD HOW DOES THIS CONTRIBUTE OR EQUATE TO A THEORY OF DISTRIBUTION... WHAT DOES IT EXPLAIN...
%Ricardo's concept of rent accounted for the way that.... WHAT IT DESCRIBED. 
%This remains important to understanding distribution in general because.....


%The marginalist account describes how workers receive the marginal value of their contribution to production. THIS MEANS.....
%It formalizes what classical economist Adam Smith describes in the story of the pin factory. (EXPLAIN PIN FACTORY)

%The marginalist account of distribution gave a story of production that seemed to align with the rising fortunes of workers following WWII, in the 1950s and 1960s when it came to dominate. 

%This narrative dominates in neo-classical economics, particularly in the United States, and formed the basis of conventional micro-economics training. (EVEN NOW OR WHEN)
%It was also influential in shaping public discourse and policy in the early 20th century. For example, because it implies WHAT THAT MEANS MONOPOLY IS BAD... it provided the intellectual foundation for anti-monopoly political movements in the early 20th century.

\textbf{These two stories each reflect a distinct dominant mode of production}, period in society, and methodological set of tools available.From the Classical to Neo-Classical economic approaches, we also see development in the methodological approaches to describing economies. In Ricardo's era, economics was characterized by a free flowing descriptive approach characterized by wide debate, and many concepts explored. By the the time the marginalist approach emerged, new mathematical approaches were beginning to emerge OR dominate the economic discourse. 

SUMMARIZE THE DIFFERENCE IN WHAT EACH MODEL TELLS US ABOUT DISTRIBUTION. WHAT IS OVERALL DIFFERENT, WHAT DIFFERENT INSIGHTS EMERGE. 

\section{This Work}

\textbf{This thesis contributes to a third theory} % ***E IS THIS RIGHT?? I CHANGED IT TO THEORy OF DISTRIBUTION FROM THEORY OF MODERN URBAN RENT WHICH IS NOT THE CATEGORY WE ARE DISCUSSEING THEORIES FOR SO IT DOESNT MAKE SENES
of distribution by developing a theory of modern urban rent which integrates the classical descriptive work on rent and the neoclassical marginalist approach, with modern work on the scaling of wealth in cities. 

Just as the classical and neo-classical approaches \textbf{responded to and theorized the dominant mode of production in their time,} the theoretical framework emerging now, responds to the economic and social factors that are defining this period.  The early stories of production from thinkers like Ricardo centred on exploring who claimed the surplus from agricultural production Over time, the story moved to industrial production. Now the center of production is increasingly  WHAT KIND OF WORK (% ***E SAYING IT IS URBAN CHANGES THE SUBJECT TO LOCATION FROM TYPE OF WORK).
Industrialization shifted the centres of wealth to cities and since then the economic importance of cities has only grown. (%***E IS THIS SO?).  
% CUT : The social wealth of cities/human capacities developed in cities is central in new world where the production of value by people in cities 

NEED A CLEAR STATEMENT HERE ABOUT WHAT CITIES ARE ACTUALLY DOING. 
WHAT IS TANGIBLY HAPPENING IN CITIES THAT MAKES THEM SO IMPORTANT? WHAT KIND OF WORK? 
THIS IS THE PLACE TO SKETCH OUT A CLEAR AND RECOGNIZABLE PICTURE OF URBAN ECONOMIES IN THIS HISTORICAL MOMENT. ACCOUNT FOR TYPES OF WORK, COVID, ETC. 

Cities are the centres of social wealth and a wealth production. 
WHAT DOES WEALTH MEAN? 
WHAT DOES SOCIAL WEALTH MEAN?


A theory of production that doesn't center/involve the relation between cities and production/urban space and human capacity simply can't explain the creation of value in the modern world. It will miss WHAT SPECIFICALLY 

This is why is useful to draw in urban theory and spatial models (OR WHAT)

NOW SUMMARIZE IN BRIEF THE THREADS BEING BROUGHT TOGETHER. 
While econimists have focused on, URBAN THEORY has developed accoutns of WHAT????
WHAT ARE THE MAOR ACCOUNTS YOU ARE DRAWING ON? 
Meanwhile, while economists have focused on WHAT, Urban Theory has explored a different set of questions. 
WHAT DOES URBAN THEORY STUDY? WHAT KINDS OF QUESTIONS. 

This work weaves traditional economic approaches together with urban theory to contribute to a picture of distribution that accounts for how modern modern social and economic trends play out in space (FIX THIS SENTENCE)
ALSO ADD: 
key thinkers - is anyone else building the broader picture of this third story?? if not say something like "the key insights to this work have been developed separately in these different fields, this thesis pulls them into an singular account of distribution in the the current period. 


This approach also draws on the advancing\textbf{ methodological tools available}. The early classical work could reflect rich dynamic stories with many dimensions, using qualitative methods % ***E CAN YOU SAY QUALITATIVE METHODS??? I WANT TO EMPHASIZE THE DIFFERENCE IN METHODS MORE
to develop theories that responded to and referred to many individual stories. The neo-classical work was able to incorporate increasingly mathematical approaches% ***E (BE MORE SPECIFIC) MAYBE LIKE: THEY DID SOMETHING using the modelling systems available before large complex system modelling OR WHAT?? was possible.
Since then, computational capacity and approaches have progressed. This work uses ABM... COMPLEX SYSTEMS that enable us to work with large data sets and complex stories. % ***E BE SPECIFIC HERE ABOUT THE METHODS YOU USE) 
agent-based models and complex stories
-- big data sets, describe those concepts get back to..
cities - land wealth is the key stone
% #NEED TO FILL OUT HIS SUMMARY OF METHODS AND PROBABLY ADD A CONCLUDING PARAGRAPH THAT SUMMARIZES THE SHiFTS DESCRIBED ABOVE AND WHY THIS WorK IS IMPORTANT. CAN REWORK THE FOLLOWING:
Changes in social and economic structures are explains by different models and emerging methods that allow us to bring the complexity that could previously only be accounted for using narrative and qualitative approaches into math-based models. 
lassical Accountsfirst mostly farmers, very poor workers
then marginalist  'gave a story of production that seemed to align with the rising fortunes of workers following WWII, in the 1950s and 1960s when it came to dominate. - peoples fortunes were increasing, the new emerging production seemed central.
now inequality has risen sharply, it seems there is a need for a model that explains what has changed. This thesis contribute to developing that new understanding of how distribution functions in the world as it is today. 

% E CUT OR MOVE ALL THIS, I THINK: 
% They reflect

% WHAT IS THIS - what fits...
% neoclassical tradition emerged with high ..
% - elauition -- pro- vastly àmaterial realism. etc..

% we uses as a common threat to put these in the same language of formal function functions, tracing from early models of cobb doulgas fucntion

% we then tell the history..  
% descriptive, analytic and embedded in a complex syste

% The complexity - allows for tracing the paths of individual- what happens for whom under a far broader range of conditions

The clarity of pedagogical models- bottom up and top up both have illustrative cases e.g. edge worth box or the Schelling's/birds models.
But true theory integrates in something that moves between scales fluidly, makes it possible for the distinct scale based approaches to come together.


\section{CUT/EDIT Very rough notes on the three stories of distribution}

ABOVE WE DISCUSS THE CHANGE IN THE PRODUCTION SYSTEM, THE THREE STORIES ALSO REFLECTS METHODOLOGICAL EVOLUTION, AND THE POLITICAL AND SOCIAL CONTEXT.

They are also %These stories are, at their heart, 
stories of who claims what share of production. 

These stories evolved with changes in the system of production.
%They evolved within an evolving theory of production. 
The early stories of production thinkers like Ricardo focuses on were agricultural. Who claimed the surplus from agricultural production? Over time, the story moved to industrial production.
Now the center of production is increasingly urban- with the social wealth of cities/human capacities developed in cities dominating. 
new world where the production of value by people in cities is the center..
A theory of production that doesn't center/involve the relation between cities and production/urban space and human capacity can't explain the creation of value in the modern world. % Later thinkers including Smith and Marx%leaving aside purely inherited weath- as that becaumse caught in this same circuit of capital transforming from production, to money and back. 

They reflect
first mostly farmers, very poor workers
then marginalist  'gave a story of production that seemed to align with the rising fortunes of workers following WWII, in the 1950s and 1960s when it came to dominate. - people's fortunes were rising, the new emerging production seemed central.
now inequality has risen sharply, it seems there is a need for a model that explains what has changed.




*** METHOD ***
ADDING SOME ROUGH NOTES ON THE EVOLUTION OF METHOD AND POLITICAL CONTEXT FOR EACH OF THESE 3 THREADS (CLASSICAL, NEOCLASSICAL, AND WHATEVER WERE DOING WITH THE URBAN)..

A third thread is the advancing methodological tools available.
The early classical work could reflect rich dynamic stories with many dimensions - individual stories.

The stories also follow the methodological development of the discipline from the descriptive work of the classic descriptive work. ..
to central complex systems, large data sets
Methodological-- early descriptive theories told rich layered stories with different
The excitement concentrated  calculus.. in the classical distributional dynamic.

cutting edge technique focused on formalized.. 


The complexity of agent base models - allows for tracing the paths of individual- what happens for whom under a far broader range of conditions



*** SOCIAL AND POLITICAL CONTEXT ***

Early descriptive work
This explosion of formal rigour - focused attention.. 
And the political context..


Monopoly- political pressure real explosion of wealth creation-- economic success of political efforts to break up monopolies.
And a dynamic- lots of worker power- expanded equality-- workers seemed strong, 
As well as the political environment in the US during the cold war, older stories rooted- marx- repression, economists perhaps created an environment in which economists
a side of the economics

-- mythological center moved to the point where descriptive and historical approaches barely taught.

Samuelson-- successful exciting-- formal-
a generation
created micro, macro
-- at the moment of the baby boom- departments founded in this moment of exuberance. raised in it, taught according to this framework.
Polarized in the periodo of the cold war - the discussion of the market-- perhaps a tendency to avoid the distributionl.. Revolution and drama.

Computers took over from calculus -Brian Arthur
Cities took over from industries - concentrated value-- finance- and law main power centers.. - eigenvalue centrality.

Crisis in 2008 -- reintroduced descriptive methods and and an openness to new formal methods including computational and agent modelling beyond the 
Methodologial

Cities-- power law dist. rising debt and inequality. -- unstable and financialized
Increasing inequality, rising debt. - worker power expanding wages and equality, a story that explained- vs subsistence.


Exactly what those pattnersnew methods are so succesful are what was lef tout..

---


With Clark
A second great theory of distribution
The result is much of the theory of rent was lost. 
time

While Marx emphasized the tendency towards consolidation and exploitation in markets, Clark saw the tendency to increased competition. 

This allocation— dynamic quality of how wages evolved
They are bidding- and it will converge 
What share do workers get- subsistence wages- get 
But as output grows, and as firms compete for labour, particularly skilled labour, is that a sufficient experience.



Three drivers
Calculus had limits.
The political moment of expanding wages with a labour sector in a position to negotiate as the economy rebuilt following WWII and destruction of old wealth— dynamic time. 
Following WWII with growing demand for labour labour could bargain, 
Following WWII in the period— subsistence waves tending— when labour could bargain,
Following break up some of the largest monopolies like in steak— general steal


Also coincided with the political movement McCarthiesm perhaps led scholars to de-emphasize the connections of their work with the classical socialist literature.
Mathematical economics became an exciting and dynamic area.

Until this point the theory was largely descriptive..
xx Cobb working with Douglass developed a formulation — exponential, in economics their names have remained associated with the xyz formulation. 

Clark made a case it was just- became problematic.

OUR CASE IS THAT IT FALLS OFF A TOTALY DIFFERENT CLIFF



Economics had theories with rich dynamics, concerned 
Classical economics was concerted with ownership and wealth. But they were largely descriptive.
But the new calculus struggled to deal with stocks and with dynamics. 
(Came back with forester and other systems theory, as well as complexity etc.)

The French Engineers in the school of bridge end road used calculus early .. followed by xyz
Technical development and intellectual excitement aligned
Became very exciting dynamic, had many success - took over the discipline. 
Tied with political successes breaking up big monopolies — seemed to offer a path forward

US opposed soviet ideas and an intellectual environment that may have led academics to dephasize the aspects of their thinking connected with classical socialists thought. 

In this environment a particular approach became dominat— also at a moment when schools and departments were growing— the baby boom came to universities at the moment of Samuelson’s peace micro-macro divide gave a tool kit to a whole generation of economists— 

Embedded at the heart of micro- the satisfyingly precise formal structure of calculus.. the marginalize appraoche— 
Thus came to define a new disciplien— a formalization of Econ.. 
Extensions from that base became the defining advances of a generation of American economists..
Attracted math- a feedback loop.

Less emphasis on intellectual history, how changing- heterodox.. all the full range of thought

Including the much more exiting new techniques of complexity and systems- opening in 2007 an explosion of these techniques in the economies. 


CITIES

But cities matter more and more
Jacobs theory of wealth and value as fundamentally social.. 
Combined with xysz. Jacobs did

Now complexity and scaling theory revealing the universality of those principles advanced by Jacobs..

This requires a different formulation of rent… - and production wealth is inherently social what are the implicaitons— what does that mean.. 



In our model, land comes in implicitly through the demand for labour. 



