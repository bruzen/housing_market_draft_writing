\chapter[Model Implementation]{Model Implementation}
\label{appendix-model-implementation}

% \section{Urban wage premium}

% $\omega$ is the urban wage premium. It is a share of the urban agglomeration effect. 

% I think of this as worker income, $\psi + \omega + r_prime*savings$ 

% The wage income  $\psi + \omega$ part has to be related to the marginal productivity of workers. The urban output function from Lobo et al \cite{loboUrbanScalingProduction2013} is  
% \begin{equation}Y=AN^\beta\label{LoboEqn2}\end{equation}

% Where $\beta$  is the scaling exponent, with a value of,  for example, 1.13  \`a la 
% Lobo. $A$ is called the ``scale factor.''\footnote{Much of the analysis assumes scale invariance of  $A$.}  The \textbf{total urban marginal productivity of a worker} is  
% \[UMPL=\beta AN^{\beta-1}=\frac{\beta Y}{N} =\]
% This is not the same as the \textbf{firm-level marginal productivity of a worker}. The worker total share in Lobo et al. is \[W= (1-\alpha)Y \] 
% so the individual share, which should be the competitive wage, is
% \[W= \frac{(1-\alpha)Y}{N} \] 
% where $(1-\alpha)=0.8$ is a common estimate. If we assume that this sets the rural wage,$\phi$, then $\omega$ has to come out of the  urban surplus per worker,

% \[surp= \frac{\beta -(1-\alpha)Y}{N} \] 

%  so set a fraction $\lambda$ of the surplus a, and 
%  \[\omega= \lambda\frac{\beta -(1-\alpha)Y}{N}= (1.13-.8) \frac{Y}{N} \] 

%  Since capital expects 0.2 as its payment and labor 0.8, the surplus available to share has to be taken out of the 0.13. The easiest formulation then is probably 
%  \[\omega= \lambda(\beta -1) \frac{Y}{N} =\lambda(\beta -1) \frac{AN^\beta}{N} \] 
 

% $(\beta -1)$ is agglom and  $\lambda(\beta -1)$ is the workers' share of the surplus over and above the \gls{constant returns to scale} (CRS) case.   $\lambda(\beta -1)$ is 

\section{Wage calculation}
Firms will make as hiring decision faced with a market wage. This is typical in \glspl{competitive market} where firms are price-takers. They do not make  wage offers in this model. That happens when firms have a degree of market power.)


When aggregate firm labour demand exceeds N, wages rise 
% Common to think of the variations in local wages as variations around a mean, in this context - perfect labour market.
% We're using this number at the agregate level, raise the market wage to a level that attracts that many. Divide available workers among the firms, then they want to raise the wage again. It's more transparent to do it at the agregate level than at the firm level, but we still have the rising supply curve. A more detailed agent model could implement the hiring and wage adjustment mechanisms for firms dirrectly, but the side effect of increased ocmplication is it also can obscure the clarity of the rent results, our focus wit this work. 
\begin{lstlisting}
# TODO FIX TOTALY - MOVE THIS TO THE MODEL SECTION NOW #
# Firm step function updates wage, omega
def step(self):
    prefactor  = self.model.prefactor
    agglom     = self.model.agglomeration_ratio
    population = self.model.agglomeration_population
    wage_share = self.model.wage_share  
    wage_premium = wage_share * (agglom-1) * prefactor * population**agglom # omega # ****** 
    self.wage = wage_premium + self.model.psi
    # k thought # self.wage_premium = (wage_share * prefactor * population**agglom)/ population # omega    
    # note surplus is: (beta - 1) * (prefactor * population**agglom)
\end{lstlisting}

Where wage share is a parameter input to the model.

\section{Bidding}
\subsection{Subjective discounting}
Growth rate= rt
Growth factor =($1+r)^t$
discount rate= r
discount factor = $1/(1+r)^t$

\gls{discount rate} vs
\gls{discount factor}
   A factor may be a compounded rate.
    It is the present value of one dollar in one year 
    Turns one dollar in one period into dollars of present value.
    sum\_delta is sum of the infinite series 
    minus discounted infinite series after mortgage\_period years
    It is the present value of annual payments from one to 
    mortgage\_period years e.g. of mortgage payments or rent received
    delta\_mortgage\_period was called   delta\_period\_T
    
\begin{lstlisting}
def get_discounting(self): RENAME TO
def get_discount_factor(self):
    """
    Delta is the subjective individual discount rate for agent
    after one year. This will be close to r_i
    EXLAIN WHAT A FACTOR IS HERE
    """    
    delta = self.r_prime # if constant 
    delta_period_1 = 1 / (1 + delta) 
    delta_mortgage_period = delta_period_1**self.mortgage_period
#   sum_delta = delta_mortgage_period * (1 - delta_mortgage_period)
    sum_delta = (1 - delta_mortgage_period)/delta # corrected  ****
    # Note delta_mortgage_period is subtracted to subtract the long tail. 1/delta gives the PV of an infineite series of payments
    return sum_delta
\end{lstlisting}

Delta could also depend on wealth. For example,  use the bank rate, which is the rational rate but people who are poor typically have higher rates.  It would not change as the central bank changes r-pirme
% delta could be wealth based typically higher for poor.

\begin{lstlisting}
# A version with delta depending on wealth
wealth = self.wealth
delta =
\end{lstlisting}
% savings =(sum(0-age)((1+r)**age)*savings_rate*subsistence)
\subsection{Maintenance costs}
\begin{lstlisting}
    def get_maintenance(self):
        """Maintenance share of property service (a*b*psi summed and discounted)
        OR IS IT TOTAL maintenance COST OVER THE MORTGAGE PERIOD?
        """
        a   = self.housing_services_share
        b   = self.maintenance_share
        psi = self.subsistence_wage
        sum_delta = self.sum_delta # CALCULATE PER PERSON
        return (a * b * psi) * sum_delta
\end{lstlisting}

\subsection{Taxes}
\begin{lstlisting}
    def get_tax(self):
        """ 
        THIS DOES NOT CHANGE WITH INCREASING WAGES?
        BUT THAT IS THE MAIN WAY TO FUND A CITY

        WHAT TO CALL THIS WEHRE DOES IT GO. WHERE DO WE USE THIS VS TAU
        Just for initialization? - warranted price. 
        Use warranted prices as initialization
        Tax costs for the mortgage period, T. 
        (Example of rate for an  multiperiod annual rate)
        tax_T= tau*(omega-c*d + a*psi) * sum_delta_T
        This is assuming taxes are paid at the end of each year for T years
        tau_T       = tau * sum_T_delta 
        #  present value of the tax rate over T years        
        """
        tau   = self.model.property_tax_annually
        omega = self.firm.wage_premium # FIXED
        psi   = self.model.subsistence_wage
        a     = self.model.housing_services_share
        c     = self.model.transport_cost_per_dist # RENAME
        d     = self.distance_from_center
        sum_delta = self.model.sum_delta # TODO - make individual  - this would have to be average discounting - THIS TAKE SUM DELTA OUT - AND PUT WITH LARGER CALCULATION.. - CACLULATE FOR A PERSON/PROPERTY COMBINATION..
        return tau * (omega - c*d + a*psi) * sum_delta
\end{lstlisting}


\subsection{TODO Warranted price}
\begin{lstlisting}
@property
def warranted_price(self):
    # USELESS PLACEHOLDER - GET CALCULATION
    return self.model.firm.wage/(self.transport_cost + 1) 
\end{lstlisting}

\subsection{TODO Maximum mortgage calculation}

\textbf{wealth-based  mortgage maximum} 
 \[max\ m_i = 9-\left(\frac{W_i}{\bar W}\right)^{0.1} \]

% **Source**: Ch:model line 580, page 87.. I have done some fiddling Wealth $W_i = P-M+S.  for i - real estate agents estimated price wealth of a property owner as assessed by the bank

\textbf{income-based  mortgage maximum} of 

\[M^{max}_Yi = \frac{0.28*(\omega+w)}{r_i}\] It is the maximum the bank will let you pay.

\textbf{Combined  mortgage maximum}
\[ M_i^Y{max} = min \{9-\left(\frac{W_i}{\bar W}\right)^{0.1}P,  \frac{0.28*(\omega+w)}{r_i} \}\]

\begin{lstlisting}
# Max mortgage
wealth = property_value - mortgage + savings
mean_weath = sum(wealth)/number_of_people

def get_max_mortgage(self, applicant):
    max_mortgage =  ...
    
    return max_mortgage
\end{lstlisting}

% wealth = property_value + 
wealth $W_i = P_e-M+S$.  

- Also need mean wealth. $\bar W$ , which you have to calculate from the sums for property values total mortgages issued, and individual savings. The bank could keep these values
- Individual borrowing rate 
$r_i = (A + B \frac{\bar{W}}{W_i})\bar r=(.1 + B \frac{\bar{W}}{W_i})\bar r$.
The value .1 can be seen as the bank's share of the prime rate set by the Bank of Canada. this is an easy place to insert that value. We should discuss this detail. An alternative is
$r_i = (0 + B \frac{\bar{W}}{W_i})(\bar r_i+ bank\ margin)$.

- Maximum M  from wealth constraint = $(9-(W_i/\bar W)^{0.1}P$
  Check if $(W_i/\bar W)0.9P$ will work. 
- Maximum M  from income = $M^{max}_Yi = \frac{0.28*(\omega+w)}{r_i}$ 
% - Maximum M  $M= min(0.28*(omega+phi)/r_i,  0.8P$,  (9-(W_i/\bar W)^{0.1}P,  \frac{0.28*(\omega+w)}{r_i}   } $


 
\subsection{Net rent based on}
Tenant willing to pay, vs what it is worth for a company to buy a property.

\begin{lstlisting}
def get_net_rent(self, property):
    """Compute the rent for a land parcel, or what someone could afford
    to pay to live there. 

    Rent depends on the urban wage premium over and above the subsistence
    wage, and on transportation costs and the distance to the
    central business district. Applies with a single wage. Adjust for
    differential urban wages.

    :param property: the land parcel to get rent information for.
    """
    a     = self.model.housing_services_share
    b     = self.model.maintenance_share
    c     = self.model.transport_cost_per_dist # RENAME
    d     = property.distance_from_center 
    tau   = self.model.property_tax_annually 
    # property_tax_rate # IS THIS FOR THE MORTGAGE PERIOD
    psi   = self.model.subsistence_wage
    omega = self.model.workers_share
    return omega - c*d - a*psi - b*a*psi - tau*a*psi
    # urban_wage = self.model.firm.wage
\end{lstlisting}


\subsection{Net rent based on ..}

\begin{lstlisting}
def get_net_rent(self, property):
    """Compute the rent for a land parcel, or what someone could afford
    to pay to live there. 

    Rent depends on the urban wage premium over and above the subsistence
    wage, and on transportation costs and the distance to the
    central business district. Applies with a single wage. Adjust for
    differential urban wages.

    :param property: the land parcel to get rent information for.
    """
    a     = self.model.housing_services_share
    b     = self.model.maintenance_share
    c     = self.model.transport_cost_per_dist # RENAME
    d     = property.distance_from_center 
    tau   = self.model.property_tax_annually 
    # property_tax_rate # IS THIS FOR THE MORTGAGE PERIOD
    psi   = self.model.subsistence_wage
    omega = self.model.workers_share
    return omega - c*d - a*psi - b*a*psi - tau*a*psi
    # urban_wage = self.model.firm.wage
\end{lstlisting}

\subsection{Max bid}

Calculate max desired bid for an agent
\begin{lstlisting}
    def get_max_bid(self, property, bidder):
        net_rent = self.get_net_rent(property)
        r        = self.model.r_prime   
        r_target = r + self.model.r_premium
        m        = 0.8 # TODO FIX - ADD WEALTH
        # I can't do delta_T. It reads as delta_transpose to me.
        sum_delta    = self.model.sum_delta 
        p_dot    = 0.01 # TODO - estimate rate of price change
        return net_rent/((1 - m)*r_target - sum_delta*(1 + p_dot - (1 + r)*m))
\end{lstlisting}

Agent will bid the min of the desired bid or the max allowed mortgage
\begin{lstlisting}
max_mortgage = self.bank.get_max_mortgage(self)
min_downpayment = self.bank.min_down_payment_share * max_mortgage
downpayment = min(min_downpayment, self.savings)
max_allowed_bid = max_mortgage + downpayment
for sale_property in (self.model.realtor.sale_listing):
    # max_bid = self.bank.get_max_bid(sale_property, self)
    # TODO Fix
    max_desired_bid = self.model.bank.get_max_bid(sale_property, self)
    max_bid = min(max_allowed_bid, max_desired_bid)
\end{lstlisting}

\section{Negotiation Process}

Bidding.

There is a problem in that they bid on all properties as a short cut. If the number of bids structures the negotiation process, we need to limit their bids or do something much more iterative. (see above section)


\section{Individual Accounting}

\begin{lstlisting}
# FIX - NEED TO ADD THIS
# Update savings
self.savings += self.model.savings_per_step

# TODO pay costs for any properties owned
# if self.residence in self.properties_owned:
#     # TODO pay mortgage if needed pay costs
#     pass
# else:
#     self.savings -= self.rent # TODO check this is right rent
\end{lstlisting}