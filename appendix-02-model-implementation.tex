\chapter[Model Implementation]{Model Implementation}
\label{appendix-model-implementation}


\section{Parameter values}

\begin{lstlisting}
"""Combining a labour market and land market.

:param width: Width of the spatial grid. Each property ocupies one unit.
:param height: Weight of the spatial grid. Each property ocupies one unit.
:param transport_cost_per_dist: The cost of traveling a distance of one
    grid space. :meth:`transport_cost` cancluates travel costs using
    euclidean distance.
:param subsistence_wage: The value of the subsistence wage which workers
    could achieve working in the countryside. Workers will work in the
    city if it offers a wage premium above this subsistence wage.
:param working_periods: The number of time steps from when a worker
    begins work till retirement.
:param savings_rate: The share of the subsistence wage a worker can save
in each time step, with undifferentiated labour and a uniform savings rate.
:param init_wage_offer: Initial urban wage offer. TODO consider replacing
:param init_interest_rate: TODO make this prime interest rate?
:param prefactor: Multiplier for the wage aglomeration effect 
:param agglomeration_ratio: Exponent for the aglomeration wage.
:param wage_share: Share of the agglomeration effect that goes to workers.
:param property_tax_annually: Property tax rate.
:param mortgage_period: The period for the mortgage. Number of years.
:param housing_services_share: Share of subsitance wage going to housing 
sevices, a.
:param maintenance_share: Share of housing sevices going to maintenance and 
sevices, b.
:param r_prime:  Expected return on alternative investments.
:param r_premium_person: Person's desired return over and above 
alternative investments.
:param r_premium_bank: Bank's desired return over and above alternative 
investments
TODO wage should have an adjustment rate
TODO update for new parameter values, prefactor, scaling factor etc
"""
\end{lstlisting}

\begin{lstlisting}
def __init__(self, width = 50, height = 1,
             transport_cost_per_dist  = 0.1,    # c
             subsistence_wage         = 40000., # psi
             working_periods          = 40,     # in years
             savings_rate             = 0.3,
             init_wage_offer          = 10.,
             init_interest_rate       = 0.03,
             prefactor                = 0.2, # 251., 
             agglomeration_ratio      = 1.2,  # was scaling_factor
             wage_share               = 1.0,
             property_tax_annually    = 0.04, # tau, was c
             mortgage_period          = 5.0,  # T, in years
             housing_services_share   = 0.3,  # a
             maintenance_share        = 0.2,  # b
             r_prime                  = 0.05,
             r_premium                = 0.00,
             ):
\end{lstlisting}

\section{Computations}

\subsection{Worker's share}
\begin{lstlisting}
def get_workers_share(self):
    """Share of wage premium that goes to workers, omega """
    psi        = self.subsistence_wage
    wage_share = self.wage_share
    agglom     = self.agglomeration_ratio
    return wage_share * agglom * psi
\end{lstlisting}

\subsection{Subjective discounting}
\begin{lstlisting}
def get_discounting(self):
    """
    Delta is the subjective individual discount factor for agent
    after one year. 
    A factor may be a compounded rate.
    It is the present value of one dollar in one year 
    Turns one dollar in one period into dollars of present value.
    sum_delata is sum of the of infinite series 
    minus discounted infinite series after mortgage_period years
    It is the preasent value of annual payments from one to 
    mortgage_period years e.g. of mortgage payments or rent received
    delta_mortgage_period was called   delta_period_T
    """
    delta = self.r_prime
    delta_period_1 = 1 / (1 + delta) 
    delta_mortgage_period = delta_period_1**self.mortgage_period
    sum_delta =  delta_mortgage_period * (1 - delta_mortgage_period)
    # TODO: why minus?
    return sum_delta
\end{lstlisting}

\subsection{Maintenance costs}
\begin{lstlisting}
    def get_maintenance(self):
        """Maintenance share of property service (a*psi summed and discounted)
        OR IS IT TOTAL maintenance COST OVER THE MORTGAGE PERIOD?
        maintenance_T = (b*a*psi) * sum_delta_T
        """
        a   = self.housing_services_share
        b   = self.maintenance_share
        psi = self.subsistence_wage
        sum_delta = self.sum_delta # TODO: rename?
        return (a * b * psi) * sum_delta
\end{lstlisting}

\subsection{Taxes}
\begin{lstlisting}
    def get_tax(self):
        """ 
        THIS DOES NOT CHANGE WITH INCREASING WAGES?
        BUT THAT IS THE MAIN WAY TO FUND A CITY

        WHAT TO CALL THIS WEHRE DOES IT GO. WHERE DO WE USE THIS VS TAU
        Just for initialization? - warranted price. 
        Use warranted prices as initialization
        Tax costs for the mortgage period, T. 
        (Example of rate for an  multiperiod annual rate)
        tax_T= tau*(omega-c*d + a*psi) * sum_delta_T
        This is assuming taxes are paid at the end of each year for T years
        tau_T       = tau * sum_T_delta 
        #  present value of the tax rate over T years        
        """
        tau   = self.model.property_tax_annually
        omega = self.model.workers_share
        psi   = self.model.subsistence_wage
        a     = self.model.housing_services_share
        c     = self.model.transport_cost_per_dist # RENAME
        d     = self.distance_from_center
        sum_delta = self.model.sum_delta
        return tau * (omega - c*d + a*psi) * sum_delta
\end{lstlisting}

\begin{lstlisting}

\end{lstlisting}
\section{Urban wage}

\begin{lstlisting}
# Firm step function updates wage
def step(self):
    prefactor  = self.model.prefactor
    agglom     = self.model.agglomeration_ratio
    population = self.model.agglomeration_population
    wage_share = self.model.wage_share
    self.wage  = wage_share * prefactor * population**agglom
\end{lstlisting}


\section{Warranted price}
\begin{lstlisting}
@property
def warranted_price(self):
    # USELESS PLACEHOLDER - GET CALCULATION
    return self.model.firm.wage/(self.transport_cost + 1) 
\end{lstlisting}

\section{Maximum mortgage calculation}

\begin{lstlisting}
# Max mortgage
def get_max_mortgage(self, applicant):
    max_mortgage =  ...
    return max_mortgage
\end{lstlisting}


% wealth = property_value + 
wealth $W_i = P_e-M+S$.  

- Also need mean wealth. $\bar W$ , which you have to calculate from the sums for property values total mortgages issued, and individual savings. The bank could keep these values
- Individual borrowing rate 
$r_i = (A + B \frac{\bar{W}}{W_i})\bar r=(.1 + B \frac{\bar{W}}{W_i})\bar r$.
The value .1 can be seen as the bank's share of the prime rate set by the Bank of Canada. this is an easy place to insert that value. We should discuss this detail. An alternative is
$r_i = (0 + B \frac{\bar{W}}{W_i})(\bar r_i+ bank\ margin)$.

- Maximum M  from wealth constraint = $(9-(W_i/\bar W)^{0.1}P$
  Check if $(W_i/\bar W)0.9P$ will work. 
- Maximum M  from income = $M^{max}_Yi = \frac{0.28*(\omega+w)}{r_i}$ 
% - Maximum M  $M= min(0.28*(omega+phi)/r_i,  0.8P$,  (9-(W_i/\bar W)^{0.1}P,  \frac{0.28*(\omega+w)}{r_i}   } $
- 

\section{Net Rent}

\begin{lstlisting}
def get_net_rent(self, property):
    """Compute the rent for a land parcel, or what someone could afford
    to pay to live there. 

    Rent depends on the urban wage premium over and above the subsistence
    wage, and on transportation costs and the distance to the
    central business district. Applies with a single wage. Adjust for
    differential urban wages.

    :param property: the land parcel to get rent information for.
    """
    a     = self.model.housing_services_share
    b     = self.model.maintenance_share
    c     = self.model.transport_cost_per_dist # RENAME
    d     = property.distance_from_center 
    tau   = self.model.property_tax_annually 
    # property_tax_rate # IS THIS FOR THE MORTGAGE PERIOD
    psi   = self.model.subsistence_wage
    omega = self.model.workers_share
    return omega - c*d - a*psi - b*a*psi - tau*a*psi
    # urban_wage = self.model.firm.wage
\end{lstlisting}

\section{Max Bid}

Calculate max desired bid for an agent
\begin{lstlisting}
    def get_max_bid(self, property, bidder):
        net_rent = self.get_net_rent(property)
        r        = self.model.r_prime   
        r_target = r + self.model.r_premium
        m        = 0.8 # TODO FIX - ADD WEALTH
        # I can't do delta_T. It reads as delta_transpose to me.
        delta    = self.model.sum_delta 
        p_dot    = 0.01 # TODO - estimate rate of price change
        return net_rent/((1 - m)*r_target - delta*(1 + p_dot - (1 + r)*m))
\end{lstlisting}

Agent will bid the min of the desired bid or the max allowed mortgage
\begin{lstlisting}
max_mortgage = self.bank.get_max_mortgage(self)
min_downpayment = self.bank.min_down_payment_share * max_mortgage
downpayment = min(min_downpayment, self.savings)
max_allowed_bid = max_mortgage + downpayment
for sale_property in (self.model.realtor.sale_listing):
    # max_bid = self.bank.get_max_bid(sale_property, self)
    # TODO Fix
    max_desired_bid = self.model.bank.get_max_bid(sale_property, self)
    max_bid = min(max_allowed_bid, max_desired_bid)
\end{lstlisting}

\section{Negotiation Process}

Bidding.

There is a problem in that they bid on all properties as a short cut. If the number of bids structures the negotiation process, we need to limit their bids or do something much more iterative. (see above section)




\section{Individual Accounting}

\begin{lstlisting}
# FIX - NEED TO ADD THIS
# Update savings
self.savings += self.model.savings_per_step

# TODO pay costs for any properties owned
# if self.residence in self.properties_owned:
#     # TODO pay mortgage if needed pay costs
#     pass
# else:
#     self.savings -= self.rent # TODO check this is right rent
\end{lstlisting}