\chapter[Model Implementation]{Model Implementation}
\label{appendix-model-implementation}

\subsection{Max mortgage calculation}

\begin{lstlisting}
# Max mortgage
import javax.swing.JApplet;
import java.awt.Graphics;

public class Hello extends JApplet {
    public void paintComponent(Graphics g) {
        g.drawString("Hello, world!", 65, 95);
    }    
}
\end{lstlisting}


% wealth = property_value + 
wealth $W_i = P_e-M+S$.  

- Also need mean wealth. $\bar W$ , which you have to calculate from the sums for property values total mortgages issued, and individual savings. The bank could keep these values
- Individual borrowing rate 
$r_i = (A + B \frac{\bar{W}}{W_i})\bar r=(.1 + B \frac{\bar{W}}{W_i})\bar r$.
The value .1 can be seen as the bank's share of the prime rate set by the Bank of Canada. this is an easy place to insert that value. We should discuss this detail. An alternative is
$r_i = (0 + B \frac{\bar{W}}{W_i})(\bar r_i+ bank\ margin)$.

- Maximum M  from wealth constraint = $(9-(W_i/\bar W)^{0.1}P$
  Check if $(W_i/\bar W)0.9P$ will work. 
- Maximum M  from income = $M^{max}_Yi = \frac{0.28*(\omega+w)}{r_i}$ 
% - Maximum M  $M= min(0.28*(omega+phi)/r_i,  0.8P$,  (9-(W_i/\bar W)^{0.1}P,  \frac{0.28*(\omega+w)}{r_i}   } $
- 
