\chapter{Introduction}

Cities are a central feature of human society - Human beings are increasingly an urban species. Cities are one of the primary sources of technological development and increasing wealth. Behind these observations is a fundamental feature demonstrated in the recent literature on scaling laws: the productivity of cities increases super-linearly in population. Cities are the locus of a positive feedback loop: rising populations raises productivity, rising productivity attracts more people and resource,

Cities are where people live and work, where a great deal of production is concentrated, in addition to being where wealth is created and accumulated, cities are also where income is actually distributed. 

In Canada, there is a housing crisis. In the last few years, the need for affordable housing has come into focus as one of the most pressing issues facing Canadians. As more and more Canadians are finding housing unaffordable, the effects are being seen in everything from declining home ownership rates to an increasing number of Canadians unable to afford housing at all.

There has been extensive work on the drivers of the crisis, including supply shortages, stagnating incomes, and the finacialization of housing ownership.

There's been less work on the implications for productivity. The housing crisis raises the question of whether Canadian cities can continue to attract people and accumulate wealth for its residents and industries, whether in fact it can even sustain their growth.

This thesis presents a spatial model of the city that incorporates distributional issues and financialization and allows us to examine the productivity implications of the housing crisis. The model that incorporates the scaling of productivity in cities within a standard urban model. 
The urban model is based on those developed in geography, planning and urban economics. The organizing principle in  the spatial models of all three disciplines is an economic variable, land rent, which is the link to distribution, financialization and continuing productivity. (another sentence on why this is great)

The analysis makes clear that in addition to the recognized distributional consequences, the housing crisis has  productivity impacts that should be considered in developing urban and housing policy. 


\subsection{OVERVIEW OF DOCUMENT}
\color{blue}

\textbf{In chapter XXX}  we link classical rent theory, neoclassical production theory, neoclassical growth theory, the scaling literature, and urban spatial models.
To show how our model is directly connected with this broad collection of linked theories, we use the Cobb-Douglas function, which is used across this entire range of literature 

After we develop the mathematical description of the relationship among these will discuss  in more detail, rent theory and our contribution, scalinbg laws, ......  and otheer issues in the literature that draw on parts of this model and 

???  apply to the specific situation we're in why rent theory is related to discussions of exploitation why it might lead the inefficiencies, whether or not this links with other important models in the literature.

\textbf{In chapter XXX} we  provide a description of finacialization and show it is a a form of rent-seeking in the housing market and ?? the potential consequences of fiancialization in the housing market. 


\textbf{In chapter XXX} we  describe an illustrative agent-based model of the urban system. Most of the analysis of urban systems has employed analytical models with roots that go back to von Thunen () and more recently Alonzo. These models are extremely useful, but necessarily abstract from the concrete  and variable individual behaviour and  the details  of dynamics that make real cities path-dependent. XXX (Dawn) have shown that agent-based models can reproduce the features of the analytical models, at least in simple cases. ABMs can be run multiple times to produce distributions of expected outocomes, which makes them valuable in planning exercises.  Our model is intended to be elaborated  for such use. 

After we develop the mathematical description of the relationship among these will discuss in more detail, various relevant applications, and issues in the literature that draw on parts of this model and apply to the specific situation we're in why rent theory is related to discussions of exploitation why it might lead the inefficiencies, whether or not this links with other important models in the literature.


\color{red}
Because we draw on a wide range of methods and literatures, we discuss the relevant literature and  nethodologies in the chapters where they apply 

\color{black}


Methodological questions: 

    - agent models 
    
    - rent theory

Core model

    - static version
    
    - dynamic version

Simulations

Result---> hysterisis

policy

\subsubsection{OR (rougher):}

1. the core model and analysis - do a model of the endogenous dynamics of the model.

2. the resilience analysis -
but this is coupled with a larger system. we're interested in how it is coupled..

low interest rates have been key to financialization 
now they're going up?

we drive the system with signals to see how. and look at the external driving variables.

but what happens with changing interest rates? to explore we drive the system with external signal to explore how it is coupled with the larger economy and get an interesting resilience result, that it is actually a kind of ratchet pumping wealth out of communities on the upswing and on the downswing.

add interest, get a result which is hysteresis, which has policy implications
We get predictions about the implications of rising interest rates.

3.  policy analysis - finally we take a second step out to position the model within a larger dynamical system and do a systems analysis of the model and suggest policy implications. 

----------

% ?Ideally, an analysis of the current state of cities will incorporate supply, 
what do we want for the long introduction and background


\color{red}
\Chapter{POSSIBLE CHAPTER OUTLINE}% MAKE THIS A CHAPTER
In this chapter, we link classical red theory, neoclassical, production theory, neoclassical, growth theory, the scaling literature,  and urban spatial models. 

We use the Cobb-Douglas function, which is used to cross this entire range of literature to show how our  model is directly connected with this broad collection of linked theories. Our model connects to the results in this chapter at four points:
\begin{enumerate}
    \item to introduce  the productive nature of cities we basically assume the presence of scaling. Given  that the scaling literature gives us an estimate of the economies of scale in a production function this allows us to simplify the model and focus on the features of the urban system rather than on fully specifying a production system. In our model, the city  exhibits economies of scale with respect to population directly. 

     \item  productivity of the city to generates an economic values of land that gives rise to rents

    \item  the rental value of land structures the spatial structure of the city

    \item we exploit the rent model and transport costs to get  distributional consequences
\end{enumerate}


\section{Classical production and distribution theory}
This model has three stages: first, a production function, modeling how urban regions generate wealth,  second a spatial model of an urban housing market, and third, an analysis of distribution within that model 

In this section, we introduce the basic structure of the production side and connect it to the literature on urban scaling. The basic scaling result at the level of the city allows us to incorporate the effect of agglomeration in a standard  circular-city model in a simple way, avoiding the need to explicitly model labour markets and firms.\footnote{Explicitly modeling labour markets and firms is a natural way to specify the model more completely, but it would require introducing many ancillary assumptions and selecting among alternative models of agglomeration, when when we want to focus on distributional and growth-affecting features of the system.}


 \section{Ricardo}



Modern neoclassical production theory can be seen as having one of its origins in Ricardo's famous 1815 discussion of capital and  land rent \footnote{An Essay on the Influence of a low Price of Corn on the Profits of Stock; shewing the Inexpediency of Restrictions on Importation: With Remarks on Mr Malthus' Two Last Publications: "An Inquiry into the Nature and Progress of Rent," and "The Grounds of an Opinion on the Policy of restricting the Importation of Foreign Corn"} in which he considers three-factor, three-class model with great precision but without the use of mathematics.   According to Simon N. Patten, 
% Simon N. Patten, The Interpretation of Ricardo. The Quarterly Journal of Economics, volume 7, Issue 3, April 1893, Pages 322–352, https://doi.org/10.2307/1884006  
Ricardo was ``the first writer to take the industrial phenomenon of city life and to create and economy based upon those characteristics.'' 

In economics, rent is a surplus value\footnote{``By rent I always mean the remuneration given to the landlord for the use of the original power of the land.'' David Ricardo corn laws note 7.}. The great social question that Ricardo addressed  was who gets the surplus.  The question was pressing because it appeared that landlords were capturing the surplus without contributing to production while many peasants were very poor. 

 
We begin with Ricardo because we eventually bring the focus back to land rents, but in the context of an urban economy. Ricardo uses the model to explain the distribution of the product of the earth among the “three classes of the community”  to the owners of which is to say, the owners of land, labour, and capital. In a passage that can be seen as a direct precursor to our analysis of urban land rent, Ricardo  wrote in his 1817 Principles of Political Economy,

\begin{quotation}   
 “The produce of the earth - all that is derived from its surface by the united application of labour, machinery, and capital, is divided among three classes of the community; namely, the proprietor of the land, the owner of the stock or capital necessary for its cultivation, ad the labourers by whose industry it is cultivated. ...  But in different stages of society, the proportions of the whole produce of the earth which will be allotted to 
each of these classes, under the names of rent, profit, and wages, will be essentially different; ”  Chapter 1
\end{quotation}



\subsection{A more explicit treatment}
Ricardo developed a theory of land rent. He did not write down a production function, but he quite clearly understood and used the concept of diminishing marginal product. In modern notation, Ricardo's model is written


\begin{equation} 
Y=F(K,L,N).
\label{Eqn:Prod1}
\end{equation} 
where $K$ is capital, $L$ is land and $N$  is labour.\footnote{In principle any number of factors can be included.} 
Ricardo does not specify a functional form, but, like mathematical neoclassical economists, he does assume diminishing returns to all factors. The landlord  receives the surplus generated by the land and the rest of the value of production goes to labour and any capital employed in improving the land. 

Most modern neoclassical treatments of production simplify by omitting land because it is a fixed factor  for society as a whole and because the focus is on the capitalist organization of production. 

\begin{equation} 
Y=F(K,L).
\label{Eqn:Prod1}
\end{equation} 



%Ricardo  clearly understood and used the concept of diminishing marginal product. This shows in his use fo the terms ``extensive margin'' and ``intensive margin'' to explain the income of the landowner. He focussed on the difference between the cost of production on a unit of land and the revenue generated. The landlord would rent out all the land which generated at least enough to pay all the costs. Anything in excess of the costs could be charged as land rent to a tenant farmer.

%This excess, or surplus, he identified as the income of the landlord. The landlord captures the surplus by ownership of the natural resource land. 

%Clearly in his model there are two basic productive factors, land and labour. The landlord  receives the surplus generated by the land and the rest of the value of production goes to labour. 

Ricardo essentially assumes that the wage is  just sufficient to reproduce the labouring class.\footnote{ ``In the natural advance of society, the wages of labour will have a tendency to fall, as far as they are regulated by supply and demand; for the supply of labourers will continue to increase at the same rate, while the demand for them will increase at a slower rate.''  This is  basically Malthus.} He has explained the distribution of the fruits of the land among the main classes of the economy.

\section{Marx}
 Marx examined  manufacturing economy in which the owners contributed the machinery, buildings, and even working capital to fund the workers until the product can be sold. %This contribution must be accumulated from their profits in the preceding cycle of production,  and has to be reinvested once the revenues of the current round have come in and the bills have been paid. Marx actually describes a circuit of capital from its form as money to its form as physical capital. 
As in Ricardo, labour is in surplus and capital is scarce. As in Ricardo the scarce factor owned by a special class - now the capitalists, is able to appropriate the is able to capture the surplus value. Like Ricardo,  Marx saw the appropriation of surplus as without moral justification - 

Marx also pointed to a new dynamic in capitalist systems - that productive capital is not fixed as land is, but  expands as surplus is reinvested. Hew famously suggested that the expansion will eventually outrun the expansion of demand and the rate of return will fall, leaving capitalists unwilling to invest and creating a crisis,.


 
\section{Henry George} 
  Henry George returned to land rent with a new insight based on the emergence of the capitalist city.\footnote{Progress and Poverty: An Inquiry into the Cause of Industrial Depressions and of Increase of Want with Increase of Wealth: The Remedy is an 1879 book by social theorist and economist Henry George.} In George, the concept of land extends to all natural resources, everything ``that is freely supplied by nature.''  
  
  Since land rent is unearned income he argued that it should be seen a social income - that it could be used to pay for all the needs of the community. This is the basis of the `single tax' movement. He clearly looks back to Ricardo early rent theory, but also forward to urban models. His analysis would be formnalized much later urban models with the proof of the `Henry George Theorem" in... by .... It demonstrated that if it was some public good that attracted people to a city, the optimal level of the good was just the amount that could be paid for from the increment in land value.
  
For George the owners of urban land extract surplus in exactly the same way that owners of agricultural land in Ricardo's analysis. Where Marx saw  the extravagant productivity of capital  as the source of capitalist crises, George saw the extraction of wealth by land speculators as the mechanism that would bring on crises

%Wikipedia expresses the dynamics this way: ``The tendency of speculators to increase the price of land faster than wealth can be produced to pay has the result of lowering the amount of wealth left over for labor to claim in wages, and finally leads to the collapse of enterprises at the margin, with a ripple effect that becomes a serious business depression entailing widespread unemployment, foreclosures, etc. '

The classical economists agreed that rents are unearned income. They did not observe, as George did, that land rents arise from proximity to urban populations. This is easily seen by considering a population at the carrying capacity of the land with individuals supporting themselves using purely local resources. There can be no land rent in this case. If a city rises that must be supplied form those still on the land, land close enough to the city will generate land rent. The value of the land is created by proximity to the city.


\subsection{Rent seeking}
  Rent-seeking is the act of growing one's existing wealth without creating new wealth by manipulating the social or political environment. Rent-seeking activities have negative effects on the rest of society. They result in reduced economic efficiency through misallocation of resources, reduced wealth creation, lost government revenue, heightened income inequality,
  
\section{Neoclassical theories of production and distribution}
 
 The neoclassical revolution opened the use of formal functional mathematics and calculus.  Charles Cobb and Paul Douglas (notably the mathematician Cobb) came up with a specific and very convenient functional form\footnote{Cobb, C. W.; Douglas, P. H. (1928). "A Theory of Production" t. American Economic Review. 18 (Supplement): 139–165. JSTOR 1811556. Retrieved 26 September 2016.} that captured much of what economists were talking about:\footnote {apparently previously used by Knut Wicksell, Philip Wicksteed, and L\'eon Walras.}. The function is just a generalized arithmetic mean:
 
 \[Y=AK^\alpha L^\beta\]
 where $A$ is a constant scale factor, commonly called `Total Factor Productivity'. The Cobb–Douglas form was developed and tested against statistical evidence  in the USA by Cobb and Douglas between 1927–1947. It was  their widely circulated empirical work seems to have permanently associated this simple function with Cobb and Douglas for economists.\footnote{ A 2021 meta-analysis of 3186 estimates concludes that "the weight of evidence accumulated in the empirical literature emphatically rejects the Cobb-Douglas specification."Gechert, Havranek, Irsova, Kolcunova (2021), "Measuring capital-labor substitution: The importance of method choices and publication bias", Review of Economic Dynamics, doi:10.1016/j.red.2021.05.003, S2CID 236400765}

The Cob-Douglas form captured  important regularities in the cross-sectional data but the estimates showed a systematic bias with time series. Essentially the value of the $A$ seemed to rise over time.
 \[Y=A(t)K^\alpha L^{1-\alpha}\]

  \section{Neoclassical growth theories}  

 \subsection{The Solow-Swann growth model}
In 1956 Robert Solow\footnote{A Contribution to the Theory of Economic Growth,  Robert M. Solow, The Quarterly Journal of Economics, Vol. 70, No. 1 (Feb., 1956), pp. 65-94. Stable URL: http://www.jstor.org/stable/1884513} provvided a possible explanation, opening the field for a further series of refinements  in an enterprise that became known as ``growth theory.'' 

Solow argued ``As a result of exogenous population growth the labor force increases at a constant relative rate n,'' so
  \[L(t)= L_0e^{nt}\] 
If we stick this into the production function 
 \begin{eqnarray}
 Y&=cK^\alpha (L_0e^{nt})^{1-\alpha}\\
    &=c(e^{nt})^{1-\alpha}K^\alpha L^{1-\alpha}\\
  %  &=A(t)K^\alpha L^{1-\alpha} \label{Eq:Solow}
 \end{eqnarray}
and we see that $A$ becomes
 \[A(t)=c(e^{nt})^{1-\alpha}\]
It is no surprise that adding a variable allowed the model to track the data better. The estimated model explained 78\% of variation in income across countries.% the estimates of $\beta$ implied that\textbf{ human capital's external effects on national income are greater than its direct effect on workers' salaries.}%(\url{https://en.wikipedia.org/wiki/Solow\%E2\%80\%93Swan_model)}.  Theodore Breton provided an insight that reconciled the large effect of human capital from schooling in the Mankiw, Romer and Weil model with the smaller effect of schooling on workers' salaries. He demonstrated that the mathematical properties of the model include significant external effects between the factors of production, because human capital and physical capital are multiplicative factors of production.[20] The external effect of human capital on the productivity of physical capital is evident in the marginal product of physical capital:
%    \[ MPK={\frac {\partial Y}{\partial K}}=\frac {\alpha A^{1-\alpha }(H/L)^{\beta }}{(K/L)^{1-\alpha} }\]

Arrow's Learning by Doing model is recognized as a forerunner of recent models of endogenous growth, 
 
 
\subsedction{endogenous growth models}
Endogenous growth theories make the increase in total factor productivity depend on an optimizing decisions about human capital investment, invention, investment in technology improvement made by firms.  It results from an active search process for innovations in which the ability to appropriate profits determines the resources devoted to innovative activity (OECD, 1992, Crafts, 1996). Growth depends on the incentives to in-vest in improving technology.% https://link.springer.com/chapter/10.1007%2F978-1-349-26732-3_13
 


  \subsection{John Bates Clark and neoclassical distribution theory}
  Classical theories of distribution elucidated the way that ownership of a scarce and non-produced factor, land, was the  basis of rent extraction by the class of landowners. Profits were a bit puzzling in this context - Capital also earns its return from scarcity. Marshall pointed outm however, that any scarcity profits (i.e., rent) would be subject to competition and depreciation.  He used the term `pseudo-rents' for these temporary  unearned incomes.\footnote{Alvin Saunders Johnson. Rent in Modern Economic Theory: An Essay in Distribution. AEA 3rd Series, Vol. 3, No. 4 (Nov., 1902), pp. 1-129 (129 pages)}

  
  Another socialist like George, John Bates Clark was also one of the pioneers of marginalism. By 1986 he was praising the dynamical process of competition partly in opposition to the single tax movement George had initiated.  His (1891). ``Distribution as Determined by a Law of Rent,'' argued that, given  competition and homogeneous factors of production labor and capital, the division of the social product will be according to the productivity of the last physical input of units of labor and capital.\footnote{Responding to the "indictment that hangs over society" that it involves "exploiting labor," Clark wrote:

    It is the purpose of this work (his 1899 'Distribution of Wealth) to show that the distribution of the income of society is controlled by a natural law, and that this law, if it worked without friction, would give to every agent of production the amount of wealth which that agent creates. However wages may be adjusted by bargains freely made between individual men (i.e., without labor unions and other "market imperfections"0, the rates of pay that result from such transactions tend, it is here claimed, to equal that part of the product of industry which is traceable to the labor itself; and however interest (i.e., profit) may be adjusted by similarly free bargaining, it naturally tends to equal the fractional product that is separately traceable to capital.} 
 

%\section{Rent}In economics, rent is a surplus value, i.e. the difference between the price at which an output from a resource can be sold and its respective extraction and production costs, including normal return (DFID, 2003; Luchsinger \& M\:uller, 2003; Sharp, 2003; Stoneham et al., 2005).

%More briefly, rent is a surplus value after all costs and normal returns have been accounted for. Normal costs include  payment of all the factors of production at their market rate.(Labour at the going wage, Capital at the interest rate, supplies at their normal price). 



%It is convenient in this model to use a Cobb-Douglas utility function that has the property that a fixed fraction of income is spent on housing.  We can start with the assumption that earnings are fixed for the lifetime at the one-period wage, $w$. Then total spending on housing is $\beta Y, \beta <1$ and $ Y=w$. Let the transportation cost for a specific location $l$ be $T(l)$. The  equilibrium price at that location will be $P(l)= \beta Y-T(l)$.

N. Gregory Mankiw, David Romer, and David Weil created a human capital augmented version of the Solow–Swan model that can explain the failure of international investment to flow to poor countries.

    \[Y(t)=(A(t)K(t)^\alpha H(t)^\beta L(t))^{1-\alpha -\beta} \]
    
    From the Solow example we can see that if all the time functions are exponential we end up with equation~\ref{Eq:Solow} again.
    
 

(((((   scale Bettancourt does this at the city level with N. he adds the empirical demonstration of the fun ctiohn. We  simply use the estimat to    

The model implicitly has a Solow-Swan style production model incorporating Jacobs-style labor-augmenting agglomeration economies (\textbf{\color{blue} maybe the discussion of agglomeration theories goes here? }) 









\color{black}
