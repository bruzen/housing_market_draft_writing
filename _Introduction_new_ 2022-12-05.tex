t\chapter{Introduction}

Cities are a central feature of human society - Human beings are increasingly an urban species. Cities are one of the primary sources of technological development and increasing wealth. Behind these observations is a fundamental feature demonstrated in the recent literature on scaling laws: the productivity of cities increases super-linearly in population. Cities are the locus of a positive feedback loop: rising populations raises productivity, rising productivity attracts more people and resource.

Cities are where people live and work, where a great deal of production is concentrated, in addition to being where wealth is created and accumulated, cities are also where income is actually distributed. 

In Canada, there is a housing crisis. In the last few years, the need for affordable housing has come into focus as one of the most pressing issues facing Canadians. As more and more Canadians are finding housing unaffordable, the effects are being seen in everything from declining home ownership rates to an increasing number of Canadians unable to afford housing at all.

There has been extensive work on the drivers of the crisis, including supply shortages, stagnating incomes, and the finacialization of housing ownership.

There's been less work on the implications for productivity. The housing crisis raises the question of whether Canadian cities can continue to attract people and accumulate wealth for its residents and industries, whether in fact it can even sustain their growth.

This thesis presents a spatial model of the city that incorporates distributional issues and financialization and allows us to examine the productivity implications of the housing crisis. The model that incorporates the scaling of productivity in cities within a standard urban model. 
The urban model is based on those developed in geography, planning and urban economics. The organizing principle in  the spatial models of all three disciplines is an economic variable, land rent, which is the link to distribution, financialization and continuing productivity. *** (another sentence on why this is great)

The analysis makes clear that in addition to the recognized distributional consequences, the housing crisis has productivity impacts that should be considered in developing urban and housing policy. 


\subsection{OVERVIEW OF DOCUMENT}
\color{blue}

\textbf{In chapter XXX}  we link classical rent theory, neoclassical production theory, neoclassical growth theory, the scaling literature, and urban spatial models.
To show how our model is directly connected with this broad collection of linked theories, we use the Cobb-Douglas function, which is used across this entire range of literature 

After we develop the mathematical description of the relationship among these will discuss  in more detail, rent theory and our contribution, scaling laws, ......  and other issues in the literature that draw on parts of this model and 

???  apply to the specific situation we're in why rent theory is related to discussions of exploitation why it might lead the inefficiencies, whether or not this links with other important models in the literature.

\textbf{In chapter XXX} we  provide a description of finacialization and show it is a a form of rent-seeking in the housing market and ?? the potential consequences of fiancialization in the housing market. 



\textbf{In chapter XXX} we  describe an illustrative agent-based model of the urban system. Most of the analysis of urban systems has employed analytical models with roots that go back to von Thunen () and more recently Alonzo. These models are extremely useful, but necessarily abstract from the concrete  and variable individual behaviour and  the details  of dynamics that make real cities path-dependent. XXX (Dawn) have shown that agent-based models can reproduce the features of the analytical models, at least in simple cases. ABMs can be run multiple times to produce distributions of expected outocomes, which makes them valuable in planning exercises.  Our model is intended to be elaborated  for such use. 

After we develop the mathematical description of the relationship among these will discuss in more detail, various relevant applications, and issues in the literature that draw on parts of this model and apply to the specific situation we're in why rent theory is related to discussions of exploitation why it might lead the inefficiencies, whether or not this links with other important models in the literature.


\color{red}
Because we draw on a wide range of methods and literatures, we discuss the relevant literature and  nethodologies in the chapters where they apply 

\color{black}


Methodological questions: 

    - agent models (integrating theory more completely into agent models)
    
    - rent theory

Core model

    - static version
    
    - dynamic version

Simulations

Result---> hysterisis

policy

\subsubsection{OR (rougher):}

1. the core model and analysis - do a model of the endogenous dynamics of the model.

2. the resilience analysis -
but this is coupled with a larger system. we're interested in how it is coupled..

low interest rates have been key to financialization 
now they're going up?

we drive the system with signals to see how. and look at the external driving variables.

but what happens with changing interest rates? to explore we drive the system with external signal to explore how it is coupled with the larger economy and get an interesting resilience result, that it is actually a kind of ratchet pumping wealth out of communities on the upswing and on the downswing.

add interest, get a result which is hysteresis, which has policy implications
We get predictions about the implications of rising interest rates.

3.  policy analysis - finally we take a second step out to position the model within a larger dynamical system and do a systems analysis of the model and suggest policy implications. 

----------

% ?Ideally, an analysis of the current state of cities will incorporate supply, 
what do we want for the long introduction and background

\textbf{Ricardo concluded, significantly, that, ``It follows then, that the interest of the landlord is always opposed to the interest of every other class in the community.'' }


Our model has three stages: first, a production function, modeling how urban regions generate wealth,  second a spatial model of an urban housing market, and third, an analysis of distribution within that model 

In this section, we introduce the basic structure of the production side and connect it to the literature on urban scaling. The basic scaling result at the level of the city allows us to incorporate the effect of agglomeration in a standard  circular-city model in a simple way, avoiding the need to explicitly model labour markets and firms.\footnote{Explicitly modeling labour markets and firms is a natural way to specify the model more completely, but it would require introducing many ancillary assumptions and selecting among alternative models of agglomeration, when when we want to focus on distributional and growth-affecting features of the system.}

\begin{enumerate}
    \item to introduce  the productive nature of cities we basically assume the presence of scaling. Given  that the scaling literature gives us an estimate of the economies of scale in a production function this allows us to simplify the model and focus on the features of the urban system rather than on fully specifying a production system. In our model, the city  exhibits economies of scale with respect to population directly. 

     \item  productivity of the city to generates an economic value for land that gives rise to rents

    \item  the rental value of land structures the spatial structure of the city

    \item we exploit the rent model and transport costs to get  distributional consequences
\end{enumerate}

\color{red}
\chapter{Antecedents of modern Urban Rent theory}
In this chapter, we link classical rent theory, neoclassical, production theory, neoclassical, growth theory, the scaling literature,  and urban spatial models. 

We use the Cobb-Douglas function %, which is used to cross this entire range of literature 
to show how our  model is directly connected with this broad collection of linked theories. Our model connects to the results in this chapter at four points:



\section{Classical production and distribution theory}


 \subsection{Ricardo}



Modern neoclassical production theory can be seen as having one of its origins in  Classical economic theory going back to the Physiocrates. The physiocratic school of economics was the first to see labor as the sole source of value but, for the physiocrats, in the context of the prevalent European rural society of the time, only agricultural labor created a surplus. Ricardo's famous 1815 discussion of capital and  land rent \footnote{\href{http://la.utexas.edu/users/hcleaver/368/368RicardoOnCornLaws.html}{An Essay on the Influence of a low Price of Corn on the Profits of Stock}; shewing the Inexpediency of Restrictions on Importation: With Remarks on Mr Malthus' Two Last Publications: "An Inquiry into the Nature and Progress of Rent," and "The Grounds of an Opinion on the Policy of restricting the Importation of Foreign Corn" } 
provides the cannonical reference. He considers three-factor, three-class model with great precision but without the use of mathematics.   
In classical economics, land rent is a surplus value\footnote{``By rent I always mean the remuneration given to the landlord for the use of the original power of the land.'' David Ricardo corn laws note 7.}. 
and as Saunders pointed out in 1902, ``Of the concrete forms of income that have usually been classed as surplus, the rent of land was the earliest to be defined; and so prominent a position has been given to it that the terms " rent" and " surplus" have come to be used interchangeably.''\footnote{Rent in Modern Economic Theory: An Essay in Distribution Author(s): Alvin Saunders Johnson, Publications of the American Economic Association, Nov., 1902, 3rd Series, Vol. 3, No. 4 (Nov., 1902), pp. 1-129}. Ricardo and

The great social question that Ricardo addressed  was who gets the surplus. % The question was pressing because it appeared that landlords were capturing the surplus without contributing to production while many of those who worked that land were very poor. 
We begin with Ricardo and the classical economists because our focus is land rents, but in the context of an urban economy. According to Simon N. Patten, \footnote{Simon N. Patten,  The Quarterly Journal of Economics, volume 7, Issue 3, April 1893, Pages 322–352, https://doi.org/10.2307/1884006 }  Ricardo was ``the first writer to take the industrial phenomenon of city life and to create an economy based upon those characteristics.'' %Ricardo uses his model to explain the distribution of the product of the earth among the “three classes of the community” which is to say, to the owners of land, labour, and capital. 
In a passage that can be seen as a direct precursor to our analysis of urban land rent, Ricardo  wrote in his 1817 Principles of Political Economy, %

\begin{quotation}   
 “The produce of the earth - all that is derived from its surface by the united application of labour, machinery, and capital, is divided among three classes of the community; namely, the proprietor of the land, the owner of the stock or capital necessary for its cultivation, ad the labourers by whose industry it is cultivated. ...  But in different stages of society, the proportions of the whole produce of the earth which will be allotted to each of these classes, under the names of rent, profit, and wages, will be essentially different; ”  Chapter 1
\end{quotation}

Strikingly, Ricardo concluded\vootnote{An Essay on the Influence of a low Price of Corn} that, “It follows then, that the inter-
est of the landlord is always opposed to the interest of every other class in the community.” This is a view that Henry george would later pick up and it is one that we approach in our study, since the issue of urban rents is of great policy importance


\subsection{A more explicit treatment}
Ricardo expounded a theory of land rent, but he did not write down a formal production function as later neoclassical theorists would. In modern notation, Ricardo's model can be written


\begin{equation} 
Y=F(K,L,N).
\label{Eqn:Prod1}
\end{equation} 
where $K$ is capital, $L$ is labour and $N$  is the natural resource  land.\footnote{In principle any number of factors can be included.} 
Ricardo does not specify a functional form, but, like mathematical neoclassical economists, he does assume diminishing returns to all factors. The landlord  receives the surplus generated by the land and the rest of the value of production goes to labour and any capital employed in improving the land. 

The value of the land is the present discounted value of the surplus it generates

Most modern neoclassical treatments of production simplify by omitting land: 

\begin{equation} 
Y=F(K,L).
\label{Eqn:Prod1}
\end{equation} 

Leaving land out of the model makes sense for a variety of reasons. According to the Ricardian theory, rent is a surplus above cost. It does not, therefore enter into price. Land is a fixed factor for society as a whole that is not consumed in  the process of production.  Furthermore, neoclassical treatments of production  focus price determination and on the capitalist organization of production. 

% HOUSING RENT IN THE NATIONAL ACCOUNTS
%   Owner-occupied housing is included in Peersonal Consumption Expenditure because the National Income and Producgt Accounts (NIPAs) treat the owner-occupant as if it were a rental business, or in other words, a landlord renting to him or herself. That is, BEA imputes a value for the services of owner-occupied housing (space rent) based on the rents charged for similar tenant-occupied housing, and this value is included in GDP as part of personal consumption expenditures. This imputation is necessary in order for GDP to be invariant when housing units shift between tenant occupancy and owner occupancy.


%Ricardo  clearly understood and used the concept of diminishing marginal product. This shows in his use of the terms ``extensive margin'' and ``intensive margin'' to explain the income of the landowner. He focussed on the difference between the cost of production on a unit of land and the revenue generated. The landlord would rent out all the land which generated at least enough to pay all the costs. Anything in excess of the costs could be charged as land rent to a tenant farmer.



%Clearly in his model there are two basic productive factors, land and labour. The landlord  receives the surplus generated by the land and the rest of the value of production goes to labour. 
Recent urban models, on the other hand, tend to ignore the production process and consider the locational implications of land and transportation costs location of people. Wealth distribution is often ignored. 

\subsection{Marx}

Ricardo, agreeing with Malthus, essentially assumes that the wage is  just sufficient to reproduce the labouring class.\footnote{``In the natural advance of society, the wages of labour will have a tendency to fall, as far as they are regulated by supply and demand; for the supply of labourers will continue to increase at the same rate, while the demand for them will increase at a slower rate.''} He then explains the distribution of the fruits of labour on the land among the main classes of the economy.

 Marx shifted attention to a  manufacturing economy in which the owners contributed the machinery, buildings, and even working capital to fund the workers until the product can be sold. %This contribution must be accumulated from their profits in the preceding cycle of production,  and has to be reinvested once the revenues of the current round have come in and the bills have been paid. Marx actually describes a circuit of capital from its form as money to its form as physical capital. 
As in Ricardo, however, labour is in surplus and capital is scarce. As in Ricardo the scarce factor owned by a special class - now the capitalists, is able to appropriate the is able to capture the surplus value. Like Ricardo,  Marx saw the appropriation of surplus as without moral justification. 

Marx pointed to a new dynamic in capitalist systems - that productive capital is not fixed as land is, but  expands as surplus is reinvested. He famously suggested that the expansion will eventually outrun the expansion of demand and the rate of return will fall, leaving capitalists unwilling to invest and creating a crisis.


 
\subsection{Henry George} 
  Henry George, an influential American political economist,\footnote{Progress and Poverty: An Inquiry into the Cause of Industrial Depressions and of Increase of Want with Increase of Wealth: The Remedy (1879) book by social theorist and economist Henry George.}  returned to land rent with a new insight based on the emergence of the capitalist city: ``With the growth of population, land grows in value, and the men who work it must pay more for the privilege.'' For George the owners of urban land extract surplus in exactly the same way that owners of agricultural land in Ricardo's analysis. Where Marx saw  the extravagant productivity of capital  as the source of capitalist crises, George saw the extraction of wealth by land speculators as the mechanism that would bring on crises.
  
  Since land rent is not created by its owners, George argued that land rent should be seen as a social income - that it could be used to pay for all the needs of the community. The clearest statement of this view is found in Progress and Poverty: "We must make land common property." The same view was expressed by the Physiocrats who concluded  that ``ground rents'' should be the source of most or all taxes. They defined ground rent as that portion of all rent which is attributable only to the size and location of the parcel. George's analysis the `single tax' movement, which sought to shift all taxation to land  and resource rents.   
  
  In 1977, Joseph Stiglitz  showed, using a standard urban model, identified the conditions in which Henry George's "single tax" is  the only tax necessary to finance public expenditures.\footnote{Arnott, Richard J.; Joseph E. Stiglitz (November 1979). "Aggregate Land Rents, Expenditure on Public Goods, and Optimal City Size" (PDF). Quarterly Journal of Economics. 93 (4): 471–500. doi:10.2307/1884466. JSTOR 1884466. S2CID 53374401 }   The logic is fairly simple: if the public good increases productivity or the attractiveness of a city, attracting more people or businesses, land rents rise, and investment in the public good should proceed until the marginal cost of the public good is equal to the increase in land rent it brings. The result is now called the `Henry George theorem.'


The classical economists agreed that rents are unearned income. They did not emphasize, as George did, that land rents arise from labour's proximity to urban population and production.\footnote{To be fair, it was not lack of understanding, that the omission reveals, but rather lack of interest in explicitly examining urban land rent from residential or even industrial purposes.}% Ricardo von Thunen, Marx, Cantillon all grasped the notion of proximity to the market as part of the source land rent. The discussions seem to not gone farther than discussions of diffeerential and rents, however.  I just am not aware of them explicitly examining urban land rent for residential or even industrial purposes. 

%The need to be near a market or prodduction center is easily seen by considering a population at the carrying capacity of the land with individuals supporting themselves using purely local resources. There can be no land rent in this case. If a city rises that must be supplied from those still on the land, land close enough to the city will generate land rent. The value of the land is created by proximity to the city.



%  no separate and comprehensive data are provided on the amounts of land rents and subsoil rents charged and earned, because they are not officially regarded as part of value-added, and consequently are not included in the calculation of GDP (except for the value of productive lease contracts)     https://en.wikipedia.org/wiki/Differential_and_absolute_ground_rent#Rent_in_macro-economics    \href{https://en.wikipedia.org/wiki/Differential_and_absolute_ground_rent#Rent_in_macro-economics}{Wikipediat article on differential rent}



  \subsection{John Bates Clark and neoclassical distribution theory}
  Classical theories of distribution showed that ownership of a scarce and non-produced factor, land, was the  basis of rent extraction by the class of landowners. Profits were a bit puzzling in this context - Capital also earns its return from scarcity. Marshall pointed out, however, that scarcity profits (i.e., rent) would normally be competed away  as entrepreneurs entered the market in pursuit of those `excess' profits. He used the term `pseudo-rents' for these unearned but temporary incomes.\footnote{Alvin Saunders Johnson. Rent in Modern Economic Theory: An Essay in Distribution. AEA 3rd Series, Vol. 3, No. 4 (Nov., 1902), pp. 1-129 (129 pages)}

  
 John Bates Clark was one of the pioneers of marginalism and the neoclassical theory of  distribution.  The marginalist approach emphasized the rational decisions of economic agents in allocating their resources would lead them to allocate resources according to the value of the marginal product of he resource in production.  Initially a socialist like George, by 1986 he was praising the dynamical process of competition partly in opposition to the single tax movement George had initiated.  His (1891) ``Distribution as Determined by a Law of Rent,'' argued that, given  competition and homogeneous factors of production labor and capital, the division of the social product will be according to the productivity of the last (or marginal) physical input of units of labor and capital.\footnote{Responding to the "indictment that hangs over society" that it involves "exploiting labor," Clark wrote:

    It is the purpose of this work (his 1899 'Distribution of Wealth) to show that the distribution of the income of society is controlled by a natural law, and that this law, if it worked without friction, would give to every agent of production the amount of wealth which that agent creates. However wages may be adjusted by bargains freely made between individual men (i.e., without labor unions and other "market imperfections", the rates of pay that result from such transactions tend, it is here claimed, to equal that part of the product of industry which is traceable to the labor itself; and however interest (i.e., profit) may be adjusted by similarly free bargaining, it naturally tends to equal the fractional product that is separately traceable to capital.} 
 
Clark's analysis of income distribution does not contradict the classical view of rents, it simply displaces the analysis to the point where a competitive equilibrium prevails, and shifts attention away from the distribution of land rents. Rents are not earned by the marginal unit of land and  

\section{Neoclassical production theory}
The concept of a production function used by increasingly mathematical neoclassical economists and  rapidly developing statistical techniques  naturally led to attempts to identify the precise functional form that would describe the contributions of labour, capital and income to output.
 
Mathematician Charles Cobb and Economist Paul Douglas came up with a specific and very convenient functional form\footnote{Cobb, C. W.; Douglas, P. H. (1928). "A Theory of Production"  American Economic Review. 18 (Supplement): 139–165. JSTOR 1811556. Retrieved 26 September 2016. The function had apparently previously been used by Knut Wicksell, Philip Wicksteed, and L\'eon Walras.} that captured much of what economists were talking about. The function is just a generalized arithmetic mean:
 
 \[Y=AK^\alpha L^\beta\]
 where $A$ is a constant scale factor, commonly called `Total Factor Productivity'. 

%The Cobb Douglas function has several convenient features. One is that the sum of the coiefficents tells us the degree of returns to scale. If $\alpha+\beta = 1$, we have constant returns to scale,

%Another is that the coefficients of the factors, $\alpha$  and $\beta$ turn out to be the elasticities of output with respect to capital and labour respectively as well as the income share of the factor. These made it relatively easy for economists to combine national data on labour and capital stocks or income with output to test the model.

The Cobb–Douglas form was developed and almost immediately tested against statistical evidence in the USA by Cobb and Douglas between 1927–1947. It was  their widely circulated empirical work seems to have permanently associated this simple function with Cobb and Douglas for economists.

The Cobb-Douglas form captured  important regularities in the cross-sectional national data,\footnote{ A 2021 meta-analysis of 3186 estimates concluded that "the weight of evidence accumulated in the empirical literature emphatically rejects the Cobb-Douglas specification."Gechert, Havranek, Irsova, Kolcunova (2021), "Measuring capital-labor substitution: The importance of method choices and publication bias", Review of Economic Dynamics, doi:10.1016/j.red.2021.05.003, S2CID 236400765. More sophisticated models  such as the CES and translog functions have been developed  since.} 
but the estimates soon showed a systematic bias with time series. Essentially the value of the $A$ seemed to rise over time. Something that was not captured in the initial model  contributed to productivity over time: 
 \[Y=A(t)K^\alpha L^\beta\]

  \section{Neoclassical growth theories}  

 \subsection{The Solow-Swann growth model}
In 1956 Robert Solow\footnote{A Contribution to the Theory of Economic Growth,  Robert M. Solow, The Quarterly Journal of Economics, Vol. 70, No. 1 (Feb., 1956), pp. 65-94. Stable URL: http://www.jstor.org/stable/1884513} provided a possible explanation, opening the field for a further series of refinements in an enterprise that became known as ``growth theory.'' 

{\color{blue}  Both Solow and Denison were attempting to account for the main features of U.S. economic growth, not to provide a theory of economic development,  Denison's 1961 monograph, \textit{The Sources of Economic Growth in the United States}.  from R.E. Lucas, Jr., On the mechanics of economic development.}

Solow argued ``As a result of exogenous population growth the labor force increases at a constant relative rate n,'' so
  \[L(t)= L_0e^{nt}\] 
If we stick this into the production function 
 \begin{eqnarray}
 Y&=cK^\alpha (L_0e^{nt})^\beta\\
    &=c(e^{nt})^{\beta}K^\alpha L^\beta\\
  %  &=A(t)K^\alpha L^{1-\alpha} \label{Eq:Solow}
 \end{eqnarray}
we see that $A$ becomes
 \[A(t)=c(e^{nt})^\beta\]
The time-dependent term that allowed the model to track the data better. More than half  of the cross-country variation in income can be explained by per capita saving and population growth alone. However, the effects of saving and population growth on income are too large. To understand the relation between saving, population growth, and income, one must go beyond the textbook Solow model.t

%???       It is no surprise that adding a variable allowed the model to track the data better. More  interesting is that the appearance of term $1-\alpha}$ in the scale factor $A$ suggests a spillover effect of human capital on the productivity of other factors.\footnote{Breton, T. R. (2013). "Were Mankiw, Romer, and Weil Right? A Reconciliation of the Micro and Macro Effects of Schooling on Income" (PDF). Macroeconomic Dynamics. 17 (5): 1023–1054. doi:10.1017/S1365100511000824. hdl:10784/578. S2CID 154355849.}  

%The estimated model explained 78\% of variation in income across countries.
% the estimates of $\beta$ implied that\textbf{ human capital's external effects on national income are greater than its direct effect on workers' salaries.}%(\url{https://en.wikipedia.org/wiki/Solow\%E2\%80\%93Swan_model)}.  Theodore Breton provided an insight that reconciled the large effect of human capital from schooling in the Mankiw, Romer and Weil model with the smaller effect of schooling on workers' salaries. He demonstrated that the mathematical properties of the model include significant external effects between the factors of production, because human capital and physical capital are multiplicative factors of production.[20] The external effect of human capital on the productivity of physical capital is evident in the marginal product of physical capital:
%    \[ MPK={\frac {\partial Y}{\partial K}}=\frac {\alpha A^{1-\alpha }(H/L)^{\beta }}{(K/L)^{1-\alpha} }\]


Solow's 1956 paper stimulated a vast literature in the 1960s, exploring many variations on the original one-sector structure. % (per Lucas on mechanics) , See Burmeister and Dobell (1970) for an excellent introduction and survey. 
in these models the population growth rate determines the growth trend of the economy.

Robert E. Lucas observed  in 1988 that, ``It seems to be universally agreed that the model ... is not a theory of economic development.   \dots while it is not exactly wrong to describe these differences (in GDP  growth rates, ) by an exogenous, exponential term like A(t) neither is it useful to do so. We want a formalism that leads us to think about individual decisions to acquire knowledge, and about the consequences of these decisions for productivity.''\ footnote{Lucas,  Robert E. On the Mechanics of Economic Development. Journal of Monetary Economics 22, 1988 3-42}

\subsection{endogenous growth models}

Endogenous growth theories make the increase in total factor productivity depend on optimizing decisions about human capital investment, invention, and investment in technology improvement. Firms  actively search for innovations, and the ability to appropriate profits determines the resources devoted to innovative activity  %See (OECD, 1992, Crafts, 1996). Growth depends on the incentives to in-vest in improving technology.% 


Kenneth Arrow's 1962 paper\footnote{The Economic Implications of Learning by Doing, Kenneth J. Arrow, The Review of Economic Studies
Vol. 29, No. 3, 1962, pp. 155-173 }  is recognized as a forerunner of recent models of endogenous growth. It proposed `what he called `an endogenous theory of the changes in knowledge which underlie intertemporal and international shifts in production functions.'' Human capital, $H$ is acquired is acquired through learning by doing. 



In 1992, N. Gregory Mankiw, David Romer %(not to be confused with Paul M. Romer, mentioned above and below) 
and David N. Weil analyzed Solow’s Model in their paper “Contribution to the Empirics of Economic Growth” and  showed that %Solow correctly predicts the directions of saving and population growth, but not the orders of magnitude. Furthermore they pointed out that, 
if the model was augmented by the factor of human capital $H$, it would fit the data even better (Mankiw et al. 1992). The model became\footnote{Because they work with time series, all the quantities are dated. We omit the time marker for notational simplicity.}
\[Y=H^\beta K^\alpha (AL)^{1-\alpha-\beta\]
As a proxy for  $H$ the authors use the percentage of the working-age population in secondary school.  With Human capital included, the Solow model explains up to  78\% of the cross-country variation in income. 

Models of this sort have been described as employing only passive learning because the proxy for learning is not a decisions variable: it is still exogenous to the firm. Newer growth theories postulate that technology is endogenous because it relies on the decision to invest in research and development and diffusion


%\section{Rent}In economics, rent is a surplus value, i.e. the difference between the price at which an output from a resource can be sold and its respective extraction and production costs, including normal return (DFID, 2003; Luchsinger \& M\:uller, 2003; Sharp, 2003; Stoneham et al., 2005).

%More briefly, rent is a surplus value after all costs and normal returns have been accounted for. Normal costs include  payment of all the factors of production at their market rate.(Labour at the going wage, Capital at the interest rate, supplies at their normal price). 



%It is convenient in this model to use a Cobb-Douglas utility function that has the property that a fixed fraction of income is spent on housing.  We can start with the assumption that earnings are fixed for the lifetime at the one-period wage, $w$. Then total spending on housing is $\beta Y, \beta <1$ and $ Y=w$. Let the transportation cost for a specific location $l$ be $T(l)$. The  equilibrium price at that location will be $P(l)= \beta Y-T(l)$.

N. Gregory Mankiw, David Romer, and David Weil created a human capital augmented version of the Solow–Swan model that can explain the failure of international investment to flow to poor countries.

    \[Y(t)=(A(t)K(t)^\alpha H(t)^\beta L(t))^{1-\alpha -\beta} \]
    
    From the Solow example we can see that if all the time functions are exponential we end up with equation~\ref{Eq:Solow} again.
    
 

(((((   scale Bettancourt does this at the city level with N. he adds the empirical demonstration of the fun ctiohn. We  simply use the estimat to    

The model implicitly has a Solow-Swan style production model incorporating Jacobs-style labor-augmenting agglomeration economies (\textbf{\color{blue} maybe the discussion of agglomeration theories goes here? }) 



\color{black}

\color{green}
\subsection{Rent seeking}
  Rent-seeking is the act of growing one's existing wealth without creating new wealth by manipulating the social or political environment. Rent-seeking activities have negative effects on the rest of society. They result in reduced economic efficiency through misallocation of resources, reduced wealth creation, lost government revenue, heightened income inequality,
\color{black}