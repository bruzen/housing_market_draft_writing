\chapter{Analysis of Results from the Core Model} \label{chapter-analysis}
% Get banks 'We are the biggest innovators in history quote'
%``Financial engineering has created a rentier class, a modern feudal system, and the biggest beneficiaries of all that extra debt have been the bankers.'' Times, Sunday Times (2016) 


% Repeat from introduction contributions
% Note multiple drivers - 
NOTE \textbf{not sensitive to any advantage being true all of the time. etc } IN THE STRUCTURE.
% can be thought of as regimes containing tipping points, the feduciary duty of governance is to sustain the regimes society chooses, to make that process of choosing tractibel labs, Kahane.

% MAIN PLOTS DIRECT ANALYSIS
% RESILIENCE ANALYSIS 
% DISCUSSION AND EXTENSIONS

In the base model, we only allow the radius of the city to change, and not the density and firm hiring behaviour.\footnote{Complex city form, density patterns, and alternative hiring patterns, can be easily incorporated in the computation model. We expect that they would not significantly affect our conclusions.}  
% model rules, and individual 
%and the job search process. 
% In practice, many other factors impinge on these choices. %additional decision factors, agents, driving variable, and \gls{stochastic} elements enter the model. 


\section{Financialization: implications of the bidding rule}

There are some implications of the financialization mechanism:
\begin{enumerate}
\item A large $m$ magnifies the return. (The downpayment is smaller as a fraction of the price, increasing the investor's leverage). 
Given the  common rule that mortgage payments cannot exceed some fraction of disposable income, the wealthy will be able to borrow larger amounts and at lower interest rates than the less wealthy. At any distance from the centre they will be able to make a higher bid.

\item A lower mortgage interest rate increases the return by lowering interest payments. The cost of capital is known to differ for rich and poor.  The wealthy can generally borrow  at lower interest rates than the less wealthy. 

\item A lower discount rate $\delta$ reduces the subjective rate of return.  Poverty in assets and cash liquidity constraints are correlated with higher rates of time preference  \cite{carvalhoPovertyTimePreference2010}\cite{holdenPovertyMarketImperfections1998}. If agents discount at their borrowing rate, wealthier agents may have a lower subjective rate of time preference and therefore value properties more highly. 

\item Higher expected price appreciation increases the attractiveness of an investment. Financial corporations and the wealthy are likely to have better price forecasts than  the occasional home buyer.

\item Higher rents make the unit more profitable. Higher expected  rents may result from expecting greater price appreciation  leading to raising rents for tenants. Lower discount rates may give future rent increases greater present value.

\item Lower maintenance costs increase profits. There may be scale economies in the maintenance  of rented housing. 

\item Lower tax rates decrease holding costs and increase the value of the investment. There may be opportunities to shelter income with land held for investment (speculative) purposes. Tax treatment of income and capital gains as well as interest deductibility may also provide advantyages for institutional buyers and investors.%\footnote{Case and Schiller \cite{LOST_CaseandSchiller} observe that (source?) `` \dots increases in real per capita income all are positively related to excess returns or price changes over the subsequent year.''} 
\end{enumerate}

Some  of these conditions (1-3) hold generally for wealthier actors. Others (4-7) may be available only to institutional investors.  Financial corporations in particular may have advantages relative to individual investors, making it  reasonable to expect that financial corporations increasingly dominate urban land 
markets.\footnote{Fr\'ed\'erick Demers \cite{demersModellingForecastingHousing2005} found that the response of housing investment to interest rates has become more pronounced over time. This suggests a rising share of financial investors relative to buyers focused on housing services. Case and Schiller \cite{caseThereBubbleHousing2003} observe that `` \dots increases in real per capita income all are positively related to excess returns or price changes over the subsequent year.''}  

Since interest rates are lower for those with higher wealth, the analysis implies, consistent with the empirical evidence, that net returns for investment are increasing with wealth. Large wealth holders will get higher expected and actual rates of return on land than those with lower wealth holdings. Managers of large pools of capital will have an even greater   advantage. Overall, Equation~\ref{eqn-bid-price} implies  sales generally go to the richest participant.
 
%  \footnote{Case and Schiller \cite{LOST_CaseandSchiller} observe that (source?) 
%  `` \dots increases in real per capita income all are positively related to excess returns or price changes over the subsequent year.''} 

% The conclusion that we draw from the analysis above is that  financialization of urban housing benefits a rentier class of urban landholders. There is evidence that it benefits a globally distributed class of rentiers.  



\section{Financialization as system change} \label{section-system}
We established that, at least in theory,  financial institutions and the wealthy are likely to own increasing shares of the housing stock. the theoretical conclusions is consistent with what has already happened in the Canadian Housing market. Recent data from Statistics Canada \cite{fontaineResidentialRealEstate2023} suggests people who own more than one property in Ontario make up more than 25\% of buyers in the province. (The proportion of investors among owners varied from 20.2\% in Ontario to 31.5\% in Nova Scotia.)
Just under one in five properties overall was used as an investment.
In Ontario 41.9\% of condominium apartments are investment properties \cite{statisticscanadaBuyRentHousing2022}.

The immediate social implications are fairly obvious. As Statistics Canada points out, these trends might limit the number of properties available to buyers who intend to use it as a primary place of residence  \cite{fontaineResidentialRealEstate2023}. Statistics Canada reports that latest census release, two-thirds of Canadians owned a home in 2021, down from a peak of 69 per cent a decade earlier. The decline is was higher for younger members of the population. 

When the homeownership rate goes down, the rental rate goes up. The 2021 Canadian Housing Survey reported that the number of renter households increased  at over twice the pace of owner households, pushing down the homeownership rate in Canada. If the trends continue, Urban Canada will gradually change from a society dominated by homeowners to a predominantly tenant society. Since wealthier buyers are advantaged in the market, the younger and poorer parts of the community will be increasingly excluded from ownership. Financialization will increase income and asset inequality in cities.

Combined with rising housing prices the effect will be to squeeze lower-wage households closer to what we have termed to subsistence level and make it harder for low-wage workers to live in the city. The city requires low-wage workers for many of the services, so labour shortages are a possibility. Labour shortages will squeeze some activities out of the city and are likely to reduce productivity. Labour shortages may push up wages, but in rental markets, landlords can capture much of any increase in wages. 

The incentive structure in our model was derived purely from the point of view of an individual investor. Examination shows that investment incentives favour the wealthy and institutional buyers, but that does not necessarily imply that the process of financialization will drive social transformations. Individual choices are at most  a link in the chain. Modelling  allows us to identify which parameters are most influential. 

A question that is especially important is whether the process of financialization and tenentization our micro model suggests is reversible:  high levels of home ownership we have seen throughout the 20$^{th}$ century, as Purdy \cite{purdyPropertyOwningDemocracyHome1993} suggests, may be ``a transitory phenomenon of the 20th century.''

% .#https://www.jstor.org/stable/j.ctt80wdt Housing the North American City
% MICHAEL DOUCET
% JOHN WEAVER
% Copyright Date: 1991
% Published by: McGill-Queen's University Press
% Pages: 608
% https://www.jstor.org/stable/j.ctt80wdt

 %how financialization might affect  the housing market as a system and some consequences for society in general. At that point we can introduce our specific hypotheses and how we intend to test them.
%***E IF YOU WANT TO INTRODUCE ANEW USE OF THE TERM, CONTEXTUALIZE IN TERMS OF HOW YOU ARE USING THE WORD. IS THIS AN EXTENSION OF HOW YOU USE IT? THIS SEEMS RELATED TO THE MACRO VERSION YOU MENTION ABOVE. MAKE THE RELATIONSHIP CLEAR. I ALSO WONDER IF THIS WOULD BE USEFUL TO MOVE UP. IT FEELS LIKE IT MAKE BELONG WITH THE BROADER CONTEXT AT TEH BEGINNING OF THE CHAPTER? GOOD IDEA I will try it

*e tHIS IS CLEAR BUT COULD BE HELPED BY RESTATING HOW THEY MOVE IT ONTO FINANCIAL MARKETS. JUST A QUICK SUMMARY STATEMENT OF HOW THEY ARE FINACIALIZATION. %EVEN JUST something like \dots They put the ownership of housing onto financial markets. just to keep us oriented in what we are talkign about.

*E I FEEL LIKE THERE IS SOMETHING MISSING BETWEEN THESE TWO PARGRAPHS. %PERHAPS JUST FLESHING OUT WHAT FINANCIALIZATION LOOKS LIKE TECHNICALLY. MAYBE ALSO INSTRODUCE POSIBLE EFFECTS \dots LIKE EVEN JUST INTRODUCE THEM AS QUESTIONS? IT'S BEEN SUGGESTED OR SHOWN THAT ITS CONTRIBUTING TO THE HOUSING CRISIS. tHIS WOULD ALSO BE A GOOD PLACE TO EXAPLIN WHAT YOU MEAN BY ``aS SYSTEMS CHANGE \dots BECAUSE i THINK THAT IS A BIG PART OF WHY YOU SAY NEXT THAT IT NEEDS TO BE UNDERSTOOD \dots BECAUSE IT HAS SUC BROAD EFFECTS

We need to understand the economics of financialization.
% \section{Literature on theory and evidence} % PROVIDE EVIDENCE 	mention theories?
There is substantial evidence that the financialization of urban housing is underway in Canadian cities..

Two questions arise when we observe the growing participation of global capital in the urban housing system: 
\begin{enumerate}
\item How far will the financialization of urban land go? 
\item That are the implications for the urban economy and the welfare of the urban population? 
\end{enumerate}

We can demonstrate that in the absence of policy interventions, differential access to finance capital ensures that capital owners acquire an increasing share of urban land % over time
and therefore capture the growing land rents from urban productivity growth. 

With this insight, growing wealth inequality emerges within a simple, widely accepted model of the urban land market. In the limit, urban residents are tenants, and new residents without capital no longer receive any of the increases in rents arising from the growing productivity of the city. 

%The first question, therefore, is reduced to which capital holders will increase their share of urban land and whether there is any reason to expect the process of financialization process to stop or reverse itself.

% \section{The incentives for financialization}
%Instead, drawing on the ideas of Jane Jacobs, Lucas proposes the city as the unit of analysis. Lucas, Robert (1990), ``Why Doesn't Capital Flow from Rich to Poor Countries?,'' American Economic Review Papers and Proceedings v. 80, no. 2 (May) pp. 92-96.  
%Jacobs, Jane  (1969), The Economy of Cities (New York: Random House).  
% The Death and Life of Great American Cities \cite{jacobsDeathLifeGreat1961}

MOVE The mortgage share and interest rate are functions of the agents wealth %Both the  share of the price  that can be mortgaged, $m$, and the interest rate and the interest rate paid, $r$, are functions of the agent's wealth. 
The discounting factor may be correlated with wealth as well. 

%%%%%%%%  VVVVVVVVVVVVVVVVVVVVVVV   This section May 18 to cut?  V
%%%%%%%%  ^^^^^^^^^^^^^^^^^^^^^^^   This section May 18 to cut?  V
% TODO - add interest rate discussion - (borrowing rates drive land prices up, even if there is no development or improvements, simply because it makes it worth a larger--the effect of low rates, especially for institutional actors have driven a large effect)
%\begin{enumerate}
%
%\item  the buyer and seller calculate the value of the property  differently. 
%
%\item  the  buyer and seller may have different expectations of the path of prices and therefore the stream of rents.
%%There are two standard ways that expectations are modeled
%%	\begin{enumerate}
%%	\item \textbf{Adaptive expectations.} Expectations are largely based on what has happened in the past. 
%%	Under normal conditions most people  have relatively weak incentive to get forecasts about inflation correct and lack the resources and time to purchase expert advice. 
%%	Recent price trends are easily available and likely to be the main source of  information.
%%	\item     \textbf{Rational expectations.} Expectations are based on a model of the future economy. 
%%International investors and banks employ economists and other experts to  forecast prices, exchange rates, and trends in the economy.
%%	\end{enumerate} 
%\end{enumerate}
% Why would  discount rates differ between identical workers? Buyers and sellers are not identical in wealth, . 
%%We could implement the first  explanation either by generating expectational errors based on functional class or wealth. 


\section{Distributional consequences of the analysis}
The analysis in this dissertation makes clear that in addition to distributional consequences, the housing crisis has productivity impacts. Specifically, the analysis in this thesis concludes that, given the ongoing financialization of the housing market:

\begin{enumerate}
\item the financial system will eventually extract all net urban land rents through investment in urban property
\item housing accessibility will become increasingly challenging for disadvantaged groups
\item housing will be largely eliminated as a saving mechanism and asset fr middle income Canadians,  resulting in a systematic decline in the `middle class'
\item that the quality of urban life will decline
\item the economic growth and development of cities is threatened by this financialization
\end{enumerate}


\section{Model dynamics}
How quickly various the housing stock, the population and the rate of rent extraction adjust are key parameters determining the dynamics of the system. We have parameterized each in a simple way in order to conduct sensitivity analysis