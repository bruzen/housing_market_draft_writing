
\section{Introduction}
QUOTE THAT THERE'S A housing prices--

NEED INTRO
we know there is a housing crisis, CUT  the devastating effects are clear 

This crisis is about affordability, about having enough housing, 
and we can see the effects being more and more visible %CUT at the surface -- SLIDE TENT CITY 

in this thesis we are looking at the urban and economic dynamics that are driving it and how  these dynamics, and the  crisis they lead to, are going to effect the larger system.

In particular, we look at one major driver of the crisis. The financialization of housing.. 
Financialization is the capture of flows of surplus by financial actors. And it's achieved through the creation of financial instruments that allow the capture of that surplus value. 

So to look at the financialization of housing,  we are essentially going to take a step back %CUT from narrowly looking at the housing market 
and look and the housing market in the context of a bigger picture of a city and of the production and capture of economic value.

There's two key insights about cities that underlie our approach: 
First, research demonstrates that cities  generates a surplus so there are important economic questions about where this surplus is going. 
Second, cities are both economic engines and spatial entities and that duality is also key to housing markets which are fundamentally a spatial kind of market. So to understand what's happening in these markets we need to take into consideration both economic production of flows of value and how they are organized within space. 

Financialization %CUT in general, 
is about capturing surplus, and %CUT specifically 
in the housing market the surplus that can be captured is embedded in this spatial economic structure which we introduce through the concept of Ricardian rents. 

So our challenge has been to bring together the logic of the spatial relationships,  with economic logics of production and finance that are generally modelled as spaceless,  in a representation of the processes and effects of  financialization to explore  what is happening.

To do this,  We produce a model that captures the dynamic with two components::
%CUT To do this, we create a model 

 1. First, how value is created in cities %CUT there's all this value created in cities.  CUT the productivity of cities . ...  Cities drive the creation of value.. WEALTH AND VALUE IS CREATED IN CITIES THROUGH AND 
 2. Second how property markets play a role in distributing that value. %CUT PROPERTY IN HOUSE OWNERSHIP IS IMPORTANT TO DISTRIBUTION OF WEALTH. 

So on one side there is a model of production, on the other there is a model of the housing market-- 

In between them, there is a link that models the institutional mechanisms by which some actors capture that value through housing markets.


This model allows us to explore  how financialization affects % the breakdown ownership ratios of housing between speculative investors and owner occupiers.
who owns property.

By looking at how ownership ratios change, under different policy regimes, 
we get at this question we've raised about who is capturing the value produced by the city through ownership of land and property.  Who can claim the rents from the land?

And ultimately, we can look at the effects of financialization on urban productivity.  
Bringing these 3 pieces together, we capture  some of what's happening in the housing market.

%CUT happens through the housing market.
%CUT and how does it affect urban productivity? We look at financialization in specifically relation to urban productivity.
%CUT how- this is a model of the institution/legal/financial structure to extent. . linked with the theory of Ricardian rent, and a detailed if abstract model of finance, capital, and expectations. 


%CUT CONTEXT/THIS IS IMPORTANT BECAUSE..
\subsection{Outline}

So for this talk, we  
1. start by going through the key ideas that form the theoretical foundation for this thesis. 
 2. describe how we've brought these things together in the actual model %CUT /how the model functions/and the results we got from the model,
3. describe the results that we got from the model,
4. and then we talk about the questions this suggests for research and discuss specific extensions to explore with future work. 

\section{Theoretical foundations}

So to begin with the theoretical foundations, we need these three concepts: 

1. financialization
2. spatial rents and
3. growth

\subsection{Financialization}

% This leaves one piece, who gets those locational rents?

%know here's an interesting phenomena that's happening to the city and that is financialization.

Financialization is at the heart of this. To introduce how we're thining about financialization, we’re gonna talk about what financial is specifically, %CUT and how it is created in. It is done accomplished in the housing market like 
how it is accomplished
and the specifics of what financialization looks like in the housing market.


We have a process underway in which accumulated wealth capital goes looking for streams of revenue and one way to get to invest is to buy land and capture the future stream of agglomeration economies the wealth created by the city being bringing people together.

That is essentially what financialization is doing. %CUT It is buying future streams of revenue and in this case, the streams of revenue arise from 
in the case of the city the productivity of bringing people together in the city. 

Land values come from the productivity of agglomeration agglomerating people, okay, so you can imagine global capital buying up all the land.
And this was the idea of the heart of the whole project tree, the beginning, what if financialized institutions and capital actually do buy more and more of the land? What does that do to the social structure? What does it do to the productivity of the city?

We make an agent based model of this process. 


\subsection{Spatial rents}

To understand this process, we then  go into classical economics to borrow this theory of urban rents, which illuminates how  spatial relationships are translated into value 
% which make a key relationship between space and economics because economics is about money and productivity and space is something where when you add transportation cost in you have it you have this urban rents give you this connection between how far things are away and how and how much it costs, so so we’ll get into that because that makes this key connection rents the key link 

rents are the key link between spatial between how far things are away from the centre and the economic questions about productivity %  that we’re gonna talk about then we’re gonna talk about space which brings in urban theory because there’s urban rents and develops them through urban theory, and ways that are useful because it is relevant already so we’re talking about how urban rents were taken developed in further in urban theory and then we’re gonna go back into 
%CUT So that is how value is created in the city?  how is it distributed? We formalute an approach to modeliing the linkage,  drawing on Ricardian rent theory.

fundamental to the nature of cities is something called locational rent %CUT from economic history, this is essentially 
which is a version of Ricardian land rent.

Ricardo developed rent in the context of an agricultural economy in the 1800s. 
The model that we use this, the Alonzo model, it's actually has several other names attached to it and Muth Mills came out of the early 60s. - urban model of locational rent- that brought this into the city -to understand spatial urban land rents and become a workhorse model for urban economics.

Although it's stayed central in planning, rent and space has dropped out of standard economics and finance, in part as production has shifted from actulture towards first forms of production that require space less explicitly - factories, and offices.

A contribution here is we're introducing space in a new way.

To make the relationships clear, we  % It's really simple, really neat, and 
we use a really simple version, a circular city with uniform transportation costs.

Here's the standard picture jobs. are at the center people have to pay to get the work so distance matters to them. There's a maximum distance that it's worth traveling to get to work so that determines how big the city is.

COULD PLUG IN ANY MAP HERE. 

And the  interesting thing here is if you live close to the city, you don't have to travel. So you really like to, you don't have to pay that cost. So you'd like to be close to the city and that means you'll pay more for properties close to the center of the city in this model.

The addition in our model is that that means rents can be higher- those rents can be claimed by whoever owns the land.

Okay, but if you're paying more that means that whoever owns the land is extracting some of the wage that you're earning in the city.

The wealth of the city created by bringing people together at the center is being captured by that's what land owners okay if the landowners happen to be the the people who are working, then the wealth of the city is being distributed among the people of the city but if the wealth if the land is owned by a separate class of landowners than they are expropriating the wealth they're expropriating, the locational rent rents that arise from agglomeration.

So specifically, what is it that makes people want to move to the city? Well, the city are more productive so the firms pay more than  firm's outside the city do.  We call this an urban wage premium. It's, it's well established in the literature. PLOTS
And the wage premium is basically a measure of the extra productivity of the city workers in the city.
It's the wage premium that the landlords are capturing. 


This is the observation that I found really fascinating. 
BRIDGES SCALING LITTERATURE AND JANE DJACOBS URBAN PRODUCTION WITH the classical notion of rent, and the formulation of class Roemer developed an excelent treatmen of class rooted in the source of income. 


\subsection{Growth}
So part of what we need to do is model how value is created in the city. 
We know a lot about that. 

one simple fact that comes out of this that very reliable and that is 
productivity of cities increases with population. It's been measured empirically. The relationship is a power law relationship.
In  standard economic terms, economists would say cities show increasing returns to scale. (PLOT)

So what is this  scaling?
Economists have thought about this for more than two centuries. Smith talked about specialization giving rise to gains from scale. 
Marshall noticed that firms and in industrial districts tend to concentrate and scale up, that is they showed increasing returns at the level of the industrial district. 
and Growth theorists looked at longitudinal performance of countries and found increasing returns over time.

Jane Jacobs, observed that increasing returns applied at the city level and that's the scale that we work with here CUT what we want to deal with.

The reason for this scale effect may be technology, networks, growth of human capital is a great deal of literature here, and the causes almost certainly interact. . - WHATEVER CAUSE, there are these agglomeration effects. 

Here's the equation that's been estimated and as long as that beta is greater than one, that city gets bigger as the population increases.

So how to model this?

So part of what we need to do is model how value is created in the city. 
WE DO THIS WITH STRONG CONNECTION WITH THE ECONOMICS LITTERATURE SO EACH PIECE CAN BE BUILT BACK. -- 
TALK A BIT ABOUT HOW IT CONNECTS WITH OTHER LOGICS OF FIRM- AXTEL WAS HERE, HE'S MODELLING FIRMS EXPLICITLY. 
his history of firm modelling reasonin review..
we take theo most classic and mainline formulation of the thoery of the firm- to make watertight connections iwth the litterature. note modular enough could just subsitute out pieces. 

So economists struggled to do deal with increasing returns to scale, because the feature of the economy, that most of the models are built around and require for stability, is decreasing returns at the firm level. CHECK That's why firms don't grow forever.  At some point it makes more sense for a new firm to enter.
he neoclassical model is our fundamental tool and it has a problem in this context. Context, because it exhibits diminishing returns to scale as the basic logic.

This specific formulation we use is pretty standard, the Cobb Douglas production function.  It looks like this.

You might notice that this is kind of similar to the equation I just showed you. There's the small n is the population of the firm in this case, and capital is included. But other than that, it looks like the same model. 

In the thesis, I actually make a mathematical connection between the two formulations.

So what we have to solve, in a sense is how you put firms with diminishing returns to scale, into an economy with increasing returns. (the bigger it gets, the more it grows..)

That is the normal features of a firm into a city that has increasing returns.

One of the intermediate steps is something called the marginal product of labor. Economists tell us that if you are a profit maximizing firm you hire until the marginal value product of your labor the next unit of labor, is equal to the wage the cost of the labor to you. Now that's a rule we build it in. It takes into account profit maximizing rational firms. It's a simple, neat model from economics.

THIS IS THE FUNDAMENTAL NEOCLASSICAL ASSUMPTION- distributional implications- neat about with this model is we see it with this assumption, and then could relax that assumtion. 

Now, among the economists, this problem that that the macro economy seemed to have increasing returns and they thought that all the pieces of it had Decreasing Returns was a problem. And in the 60s, you started to get what's called growth theory, neoclassical growth theory emerging. And they basically wanted to find some way that that neoclassical production function could be turned into one with increasing returns over time, without actually changing the logic at the firm level.


The first trick was to make that a in the formula depend on time. It got bigger over time.

Over time  growth theorists came to the conclusion it had something to do with human capital, network relations, and so on. So the modern version actually starts to converge with what the scaling litterature has been finding. 

This is where Jane Jacobs came in, because she put on the table the idea that just bringing people together, makes them more productive. It wasn't just technology. It had to do with the gains from having a lot of people cooperating,  what we call  network effect now, 

so growth theory doesn't explain how firms switch decreasing returns at the micro level to a macro level, there's still something missing. 
And for us, it is the structure of the city and Jacob's general idea. - about the rising agglomeration as people come together. 


The math is agnostic to us, but that is in the background and in the theory. 


\subsection{Productivity linkages}

That gives us the ownership story. We can connect these things and as what happens to the structure of ownership in the city. 

BUT THERE'S IS A FOURTH PIECE
But is it possible there's another link that goes the other way?

it's  possible that this effect not only takes money away from people in the city, it's possible that financialization also feeds back and  affects the productivity of the city. One mechanism clearly as people don't have as much money, people living in the city don't have as much money, they don't invest as much in the city so it's not as productive. 
There are multiple mechanisms that might drive this. SLIDE
One of the things we do with the model is say okay if there was an effect, given the very plausible linkages in the literature, what would the implications of that be in our model?

\section{Model}

So what we got is spatial rents being captured by capital, and that transforming the city so how do we put all these things together now we're getting into last three years work.
First, we take the Allonzo model, which just gives us a relationship between the wage and the size of the city and their for the population.

Second, we take the Jane Jacobs observation.

As the population grows, the city gets more productive. So now we got a dynamic model or one in which growing population increases productivity which increases population, which increases productivity, this is you have a we have a slide for this case.

It could be described as two differential equations. I wanted to build an agent based model so rather than simply using the equations I have agents making decisions about what they will do on a, on a myopic, very local level. And we see how that goes together. That's the characteristic of an agent based model.

The next thing that we have to do that gives us a essentially a production center sector that produces a city.

What we don't have in this is any distribution mechanism, what we don't have in this as any ownership of land. So in our model, my model I introduce property owners, they can buy the land and the banks the capitalists can buy land if they think it's profitable. Now this is the heart of the model here is really a set of financial decisions we got to win is a piece of property worth buying for an owner occupier, when is it worth buying for a capitalist inventor and we just let them compete in the market.

So the structure then is on top of this production, urban model, which we call the Alonso Jacobs model for shorthand.

16:55
We have to have agents entering the city and trying to buy property or becoming tenants and we have investors trying to buy land to capture what they expect are the capital gains from that purchase two sets of decisions that's the guts of the model really.

I will just work through those three diagrams as the basic structure of the model


So I've just described the Alonzo That's not called the financial the Alonzo

Jacobs model, and that was the first thing I described. The second thing I'd like to describe is the relationship between
the housing market, the bank and buyers and essentially, both in our model, workers have to get mortgages, investors get loans. They decide if a particular property when it comes up is worth purchasing and we have a mechanism where it's inspired a little bit by some discussion in game theory about who wins in these bidding games. A little bit of bidding theory affected this. But that's our housing market, bank supply capital we have parameters what the various costs of capital are, how willing they are to lend the workers, how willing they are to lend to investors, and so on.

So that's our financialization component. And the whole story is really almost as simple as this: do the investors have an advantage at some point over the worker buyers? If the investors have an advantage, then they'll buy an increasing fraction of the housing stock over time as people retire and as new housing comes into the market?

That's a dynamic question. Just as the previous one was a dynamic question, 
and this is where we have the most thoroughly articulated agents in the model.

THIS GIVES THE OWNERSHIP RESULT

WE HAVE TWO PROPOSITIONS.

1
2

Now let's discuss what is the second piece

Finally, let's just discuss a little bit what the housing market is doing. We've got two kinds of buyers as I've said.

The market does two things. It allocates the housing between owner occupiers and tenants but it also allocates the locational rents between owner occupiers in a different social class called rentiers. Now, this is standard economic terminology. rentiers are people who are living off the avails of their capital rather than their labor.

They get rent, not because they do work, but because they own the property.

So that means that they are extracting rents from the urban system.

And that could have an effect on growth. So we have two fundamental outputs of this model, the production or sorry, the population structure structure.

And that is essentially whether you have a city of all owner occupiers and that's where we start start for the runs we consider or whether we have some tenants or all tenants. And it makes a difference, because these are, in economic terms, different classes with different opportunities.

Canada has been a nation of owner occupiers almost, well, certainly, almost all of the time that Canada has existed and an enormous effort was made to expand the housing system the housing supply, after the Second World War. And that carried on right until about 1991. The federal government pulled out over the following 30 years. What we have seen is supply growing supply shortage.

Increasing tenant diarization and increasing financialization of the market. Very interesting.

In the process, of course, the wealth created by the city is increasingly transferred to capitalists from workers. Very simple story.
So that's the basic model.

Look at the efect of a link with productivity.

----

- integrated with the equilibrium reasoning the model of the firm.
 similar appraoch in that we use relatively standard reasoning, but carefully built from the ground up. 

And again it Note highly modular. Both the model of the city and the institutional link can be replaces with different models, better data brough in etc. - urbansim - model of data-Dawn working with him for Toronto. I've hosted conversations with 4 or 5 local developers developing proforma for a Waterloo context.  can link. 
- can build of Dawn, Urbansim, Corsica group, workshop also had collaborators from Doynes group at Oxford join. you may know better the current work at that group. But dawn and Corrsica collaborators, and we are working on one on the SLUCE market that will feed into that. interesting features - linking rental markets etc. .
LAND MARKET MODELLING.
also transporation etc.

In each case, built a carefully grounded .. similar appraoch in that we use relatively standard reasoning, but carefully built from the ground up. 
relatively simple, but constrained tightly to the core purpose.

Fits with the purpose of this model 'theoretical exposition' - extension beyond, keep it simple.

THEREA RE DISTINCT INTERVENTIONS
change in the routines resource authority flows -- change in deep structures
Meadows- maps of types of interventions.. lots more.. 

DEFINE SOCIAL INNOVATION - 
Institutional mechansim - one among many- ostrum talked about land rens



## Results


What results that this produces:

So, in the thesis we actually report 84 distinct results.

26:50
Some of them aren't very interesting. But when we say 84 distinct results, it's because for each run, we report graphically seven results like that. So here's our base run at show a picture of the output.

27:11
The data is all in data files, but there is just so there are so many numbers that we've summarized that with I've summarized it with the seven figures.

27:23
Capital wage wage is a measure really of how much people are being paid but since we have fixed this is a modeling trick. We've fixed a subsistence wage, everything above the subsistence wage is what is the urban premium. So we just take the wages subtract the subsistence wage, we get the urban premium that determines the size of the city so here's the wage here's the size of the city what is is determined the size of the city because people only move in if it's worth traveling. 

We've made the city given the the city a uniform distribution of property sizes for simplicity. We've given the city that's not a necessary feature, by the way, it's actually technically really easy to change. It just it just makes everything more complicated to describe. We have made the transportation cost the same for everybody. We've made all the workers the same. None of these are necessary.

28:32
Restrictions except for clarity.

28:35
We have also eliminated any amenities, any features and neighborhoods that might be attractive. Again, it's easy to add back in is an attribute of each piece of property in the agent model.

28:53
But we have followed that dictum from whoever it is that we want as simple as possible model that you want to take away everything your model is complete when you've taken away everything you can.

29:07
So this is really bare bones, but extremely standard theory

29:18
so we have wage, city size.

29:25
Well, little bursts restriction here has meant that we keep the firm size the same to make comparisons a little bit easier, which means that firm numbers have to increase as the population increases. So you're seeing that happening. So firm numbers mimic population.

29:44
We can change that a little bit by letting firms respond. But again, it's a complication that you don't really that doesn't make it easier to understand the important things that are happening.


\section{Results}

30:01
Because it's a standard economic model, we want to make sure that the capital stock behaves as we expect.

30:08
This court this item in the upper corner is the size of the capital stock. And as productivity rises, we expect it to go up. So what you're seeing here is the dynamic adjustment of a city from an initial size with an initial wage to something like stability, and we're modeling it over a period of 150 years, which is a reasonable term. Obviously, we've had to play with the parameters to get this the basic economics of the city to behave properly. Once we do that, what we're most interested in is what happens with the financialization and that's this bottom graph. And the single indicator we're using here. I I am using here is the ownership ratio. How much does the bank own? How much does the what fraction do owner occupiers own? Now this is a bit of a proxy because we could actually talk in terms of the share of ownership right now this is we haven't carefully separated.

31:33
We are not reporting I should say the amount of the rent that's going to each class we're just looking at this owner occupier ratio, they convert pretty directly so it just haven't bothered with the distinction.

Okay, so we have these seven, seven items. And what we what we do is we report on six distinct policy interventions that we thought might affect the evolution.

32:05
Those are capital gains taxes for owners and for for investors, property taxes.

32:14
Here's the list.
That gives us seven results times six or 42 and then we have two cases. 

And this is this is the sort of this is my was my long term objective, which is to see what happens if financialization also affects the productivity if somehow it feeds into the Alonzo Jacob's cycle that we described earlier. And here's a graph of the possible ways that it could feed in and now there's a lot we've talked, we actually have thought I've thought carefully. 

various people looked at the literature and nobody tracks the effects in these sub sectors of the the model all the way through to the Alonso J. Cole. Note there isn't any good reporting on that at this point. The question we're asking is not whether this particular effect is passed through. But whether overall if financialization does affect the cycle, what would happen? Okay, so are we're demonstrating I'm demonstrating that this is important. I'm not demonstrating that any one of these possible channels really is operating regularly.

And this is a this is an indication if this put introducing this linkage actually affects important variables in our 42 experiments.

That we need more research on which linkages are strong, how much they matter.

The model tells us whether or not this is important. This is research that should be done.

So we repeat all 42 that gives us 84 results.

Now I'm just going to summarize the results now.

\section{Results}

First, if there aren't any linkages, it turns out that only then city regulations and transportation cost have much effect.

And the capital gains tax can be important, but the effects are relatively small in the presence of linkages. We believe that potentially important result is that in the presence of linkages, many of the policy instruments are effective.
There are large effects on the urban system. So we'll show you with one comparison. And this is where I would show a case with a linkage in it. And we've got that graph. It's one of the early ones that shows all seven things just with the linkage and no policy variation. 

Okay, second big result is capital gains and speculative investment in housing as a very powerful and a very powerful effect on our indicator variable. And city size when the linkages are there.


So if there are these linkages as we postulate, they actually reduce the productivity of the city and its size, and capital gains taxes can undo that. CUT While capital gains taxes are highly political issue.


But the most effective one is a capital gains tax on investors.


And since investment in buying up housing is non productive investment in some sense, this is an obvious policy direction, CUT politically acceptable except to capital owners.

Okay, the cost of capital has surprisingly little effect on any of the variables. 

I mentioned that transportation has a strong effect, as predicted by the basic Allonzo model. 

increasing density increases the population doesn't affect wages interestingly or city extent by itself. So density is not going to affect the wages or the size of the city, but it will increase the population and per permit, further productivity growth.


To our surprise, restrictive restrictive mortgage market doesn't have much effect.

That's it doesn't it doesn't affect the production sector. And it doesn't actually change the housing mix much properly. 

There's a bit of a surprise for us that increasing property taxes actually reduced the number of owner occupiers, but the effect was trivial if there was no linkage it depresses wages. However, a property tax increasing the property tax, will depress wages if there's a linkage and this is another interesting result in our mind. 

Okay, I've listed the roughly eight results that jumped out at us. The big one, there are two that in my mind are most important one is if linkages exist, they're extremely important and therefore research is needed.

And secondly, perhaps the most powerful instrument for reducing the penetration of cities and increasing the productivity in cities may be a capital tax. On Investment, ownership of property.

'So to summarize feature work three variations on the agents as they stand and three extensions into new domains. 

\section{Future work}

Three pieces of the model 
- The model of urban production
- The model of urban rents in the city
- And the institutioanal link between them, by which the value produced in distributed through land markets.

There are substantial opportunities to extend this work in any of the three dirrectiosn

Can extend either of them,
- sensitivity- study non l inear dynamcis cysters, characterise introduce data.
- And can use them with people. - in lab as boundary objects for communcitaion. to make ideas explicity - LAB GROUP POHOTO
#### Firms
- Firms- explicit firm models.. Axtel sat 129 million firms- a platform - very good time to work with that kind of data
- central bank expanding how ti does this kind of modelling.- from representative firm models- - 2-3 countries beginning to explicitly model firms.
### City
- Working with dawn- city - partnership proforma California lang group. met with developers sharing pro-forma. - review of 5 main land models- publishing with Dawns. - Rental markets, development etc.
- Also in practice - some work with the Devco- and HIRT model, and with Maryam - national stratgety for a new normal pattern of development. Ian Bird.
Institutional linkages
- Much more explicit - SOCIAL INNOVATION IS - change in routine resource flows
- in practice - a fund with a concessionary layer of finance for non-market- get at core-- but also actually creating a different class of actors- different flows and relationship.. all the non-market actors in Hamilton conveined so it's a 500m-2b fund - vs xx billions of federal investment -- but not just trying to move through the market- actually using that as part of an implicit model with relationships, actors, routines relationships and paradigm -- prioritizing not just the state but the resilience of the state and set of relationships.
- ** this is part of it because - there are a small set of canonical models- holding in relation with that- integrated with the advantages of precision and long run trajectories-- with distributional effects - over people but an extensible- people do things believe things - rules can stop-- and the model is shareable like a toy.

\section{Conclusion}


The housing market is increasingly financialized and this thesis is about what are the effects are. 

As financial actors buy up more of the property near cities, it can affect both distribution and the capacity of cities to build human capital and create value. z
so it is important to understand the mechanism.

THIS IS IMPORTANT BECAUSE WHILE THERE IS ALL THIS VALUE CREATED IN CITIES. - # FOR HOW MUCH, PLOT OF HOW IT'S INCREASED - AND QUOTE FROM WEST -- WE LACK IN ECONOMIC THEORY TO UNDERSTAND THE DISTIRBUTIONAL implications OF THIT.

To summarize the contributions

- we haven't found any model that has both production and financialization in the same model - 
- -- very little formal modeeling space linking any of the study of production to distribuiton, to who it goes to in, any way-- 
- rare to tightly integrate neoclsasicla economic- equal distribution within an agent-base modelling. -- importatn- let's you carse grain.
- ...
In this talk, we first took you through the the theoretical foundations for the work, then we we introduces the model financialization of housing in urban centers with production. Then shared some results, and  discussed possibilities for future work. 

So let me thank you there and I'll take questions.'
