\documentclass[]{article}
\usepackage{tikz}
\usetikzlibrary{shadings, shadows, shapes, arrows, calc, positioning, shapes.geometric}
\usepackage{pgfplots}
\pgfplotsset{compat=1.16}
\usepackage{mathtools,amssymb}
%\input{Preamble}
%\input{(SpecialDistanceCourse}
%\input{FRAMES}
\begin{document} 
\section{Notes}
Obsidian






\subsection{Financializatin}
However, property ownership is now going through a great transformation in structure, with financial capital coming to own a larger share of the urban land and housing stock \cite{farhaReportFinancializationHousing2017, palleyFinancializationWhatIt2007, ADD_FINANCIALIZATION_BOOK-maybe-tomaskovic-deveyFinancializationCausesInequality2013}. For the first time in Canadian history the rate of home ownership has declined, falling from 69\%  in 2011 to 66.5\% in 2021 \cite{statisticscanadaBuyRentHousing2022}. % BLACKSTONE
In Ontario, 70\% of new condo units are owned by investors not residents \cite{pickelInvestorsOwn772023} and asking rent in Canada increased by 10\% over last year, as of February 2024 \cite{urbanationNationalRentReport2024}. 
% https://kitchener.ctvnews.ca/investors-own-77-per-cent-of-new-c

\section{Introduction}
We know we are in a housing crisis, housing is becoming increasingly expensive relative to income, and more people lack access to adequate housing. In this thesis we are looking at the urban and economic dynamics that are driving the crisis and how these dynamics are going to effect the larger system.

In particular, we look at one major driver of the crisis. The financialization of housing. Financialization is the capture of flows of surplus by financial actors. And it's achieved through the creation of financial instruments that allow the capture of that surplus value. 

So to look at the financialization of housing,  we take a step back and look and the housing market in the context of a bigger picture of a city and of the production and capture of economic value.

There are two key insights about cities that underlie our approach: 
1. First, cities generates a surplus. So there are important economic questions about where this surplus is going. 
2. Second, cities are both economic engines and spatial entities and that duality is key to housing markets which are fundamentally a spatial kind of market, and space plays a role in who can claim the surplus. So to understand what's happening in these markets, we need to take into consideration both the production value and how it is organized within space. 
So our challenge has been to bring together the logic of the spatial relationships,  with economic logics of production and finance that are generally modelled as spaceless,  in a representation of the processes and effects of  financialization, in order to to explore what is happening in the urban housing market. % AWK

To do this,  we produce a model with two components:
 1. On one side, how value is created in cities 
 2. On the other side, how spatial urban land rents play a role in distributing that value. 

In between, there is a link that models the institutional mechanisms by which some actors capture the value through housing markets.

This model allows us to explore how financialization affects who owns property, and get at this question of who is capturing the value produced in the city. CUT? Ultimately, it allows us to look at the effects of financialization on urban productivity.  

Bringing these 3 pieces together, we capture some of what's happening in the housing market.

\subsection{Outline}
So for this talk I will: 
1. Start by going through the key ideas that form the theoretical foundation for the work.
2. Describe how we've brought these elements together in the model 
3. And discuss the results that we get.

\section{Theoretical foundations}
For the theoretical foundations, we need three concepts: 

1. financialization
2. spatial rents and
3. growth

\subsection{Financialization}

So first, financialization is is the capture of flows of surplus by financial actors. 

It's achieved through the creation of financial instruments, like REITs, and mortgages,  that allow the capture of that surplus value.

In the case of the city, the surplus it's capturing is the productivity that we understand as coming from bringing people together in the city. 

We argue that this mechanism of claiming surplus through rents, is at the heart of what's happening specifically in financializatin in urban centers. 
However we haven't found a model that has production and the financilization of the housing market in one model - or one that explicitly formalizes the relationship between the two. 
Putting these two pieces together in one model makes it possible to explore how the pattern plays out over time, and explore the effect of policy interventions. 

\subsection{Spatial rents}
In order to model the capture of surplus through land markets, we need to understand how value is distributed through space.

To understand this process, we go to classical economics to borrow the theory of rent.   

Classical economics developed during a period in which agriculture still dominated economic production.

Ricardo introduced the theory of rent in xx with his corn GET source.  IN ORDER TO UNDERSTAND THE RELATIONSHIPS between space, distribution and prices.

In this model, the carter purchases products at the farm gate, and there is cost to transport them to the market. Both the cost of production, and the value of the land thus depend on the distance to the market.  

Ricardo was working at a time with rising inequality as Colonial expansion began to dramatically increase inequality in Europe. - so he was particularly concerned with the distributional effects in his...  rents. DEFINE - dsitrubiton functioned

Later, with industrialization, economic theory shifted to focusing on production with space excluded.

In the newer economics, there is a fundamentally different formulation of distributin, that displaces the old spatial understanding centered on rents. 
Workers get the marginal value of their contribution to output, this gives the wage. 
% This is the marginalist approach, in which workers are understood as getting the marginal value of their contribution to production.
- put it forward as a copetitor.
- makes a case for it's resonablness, subtly shifts attention from questions of distribution - the problem is then output/or productivity, outside- vs sharing the value here in space. 
- the marginal production and it has certain advantages as well because it the calculus allows you to create these very Elegant understandings of whatever whatever, and you get and equilibrium, and being able to calculate long run trend  which was espeically impoortant before computers.. 

CUT which is important because it allows you to understand certain things about the relationships and how they find equilibrium whatever
-didn’t have computers so they could do fairly sophisticated modelling without the capacity to do agent base modelling 

Rent came back in through the Alonzo model, where the value of urban land depends on the distance from the workplace. 

Notice how similar this picture is to the Riccardo story of the carter.

In  the standard picture. There are jobs at the center and at people have to pay to get the work so  there's a maximum distance that it's worth traveling to work. That determines how big the city is. It also shapes land prices since workers will pay to be close to work.

The details of the map, the mechanism, the relationship to space can vary, but this basic approach to understanding rent in urban systems underlies a whole bunch of work in urban economics.

WE BUILD FROM - THE SIMPLE CANNONICAL MODLE TO BEST SEE THE RELATIONSHIP IN THIS....

What hasn't been done is modelling the dynamic mechanism by which cities produce value, in a model with land markets and a range of actors to see how that value is distributed in space. 



\subsection{Growth}
So how do cities produce value?

% Rents give the spatial structure of the distribution of surplus.
% The instruments of financialzation enable particular actors to claim those rents.

% The last piece we need is to model the source of the rents, how value is created in the city. 

one of the most robust empirical regularities is that 
productivity of cities increases with population. It's been measured empirically. The relationship is a power law relationship. In  standard economic terms, economists would say cities show increasing returns to scale. (PLOT)

So what is this  scaling?
Economists have thought about this for more than two centuries. Smith talked about specialization giving rise to gains from scale. 
Marshall noticed that firms and in industrial districts tend to concentrate and scale up, that is they showed increasing returns at the level of the industrial district. 
and Growth theorists looked at longitudinal performance of countries and found increasing returns over time.
Jane Jacobs, observed that increasing returns applied at the city level and that's the scale that we work with here CUT what we want to deal with.

The reason for this scale effect may be technology, networks, growth of human capital is a great deal of literature here, and the causes almost certainly interact. . - WHATEVER the CAUSE, there are these agglomeration effects. 

Here's the equation that's been estimated and as long as that beta is greater than one, that city gets bigger as the population increases.

So how to model this?
growth thoerists addressed this using a formulation that' s become pretty standard

the Cobb Douglas production function.  

So what we have to solve for with this equation, in a sense is how you put firms with diminishing returns to scale, into an economy with increasing returns

And they basically wanted to find some way that that neoclassical production function could be turned into one with increasing returns over time, without actually changing the logic at the firm level.

The first trick was to make that a in the formula depend on time. It got bigger over time.

Over time  growth theorists came to the conclusion it had something to do with human capital, network relations, and so on. So the modern version actually starts to converge with what the scaling litterature has been finding. 

Jane Jacobs brought in the idea that bringing people together, makes them more productive. It wasn't just technology. It had to do with the gains from having a lot of people in communities,  what we call  network effect now,


What we're doing with this model is putting an urban model of production together with, a formulation of urban rents, and land market with particular mechanisms by which financial and other actors can gain the benefits of agglomeration effects in cities. 

We're putting space into the model of rents in a different way. 

The willingness to pay of investors depends on the rents they expect to get from the land.
 The rent depends on urban production, and thus it depends on the aglomation. 
 
That means that whoever owns the land is extracting some of the wage workers are earning in the city.
And there's this structural link between place, productivity and distribution. 

The wealth of the city comes from bringing people together and  what investors are doing is  claiming a share of the value from bringing people together. 

There is a second piece which is the investment in production. Investors in property may also invest in building new housing, upgrading properties, etc. 
This process of claiming rents  is orthogonal to the process of investment and development.
one of the contributions of this work is to carefully desegregate the two in this work so we can consider creating value for the city, and taking value out of the city separately. 
So policy can more effectively encourage the kind of investments that create value. 


\section{Model}

So that is the theoretical background.

Now we have to put these things together in a model.
First, we take the Alonzo model, which just gives us a relationship between the wage and the size of the city and their for the population.

As the population grows, the city gets more productive. So now we got a dynamic model or one in which growing population increases productivity which increases population, which increases productivity, this is you have a we have a slide for this case.

I wanted to build an agent based model so rather than simply using the equations I have agents making decisions about what they will do on a, on a myopic, very local level. And we see how that goes together. That's the characteristic of an agent based model.

The next thing that we have to do that gives us a essentially a production center sector that produces a city.

What we don't have in this is any distribution mechanism, what we don't have in this as any ownership of land. So in our model, my model I introduce people choosing whetehr to work in the city., they can buy  land and investors can buy land. What's here here is really a set of financial decisions we got to win is a piece of property worth buying for an owner occupier, when is it worth buying for a capitalist inventor and we just let them compete in the market.

So the structure then on one side is the urban production and aglomeration effects, drawing on Jacobs work, growth theoru, and the scaling litterature. On the other is the spatial strucutre of rents drawing on Ricardo, clasical rent theory and Alozon's model of rent in the urban center. 

It could modelled as two differential equations. WE model a spatially explicit land market to more explicitly model individual decision making and the distributional implications of urban procution and finacialization. 

We have to have agents entering the city and trying to buy property or becoming tenants and we have investors trying to buy land to capture what they expect are the capital gains from that purchase two sets of decisions that's the guts of the model really.


2 The second thing I'd like to describe is the relationship between
the housing market, the bank and buyers and essentially, both in our model, workers have to get mortgages, investors get loans. They decide if a particular property when it comes up is worth purchasing and we have a mechanism where it's inspired a little bit by some discussion in game theory about who wins in these bidding games. A little bit of bidding theory affected this. But that's our housing market, bank supply capital we have parameters what the various costs of capital are, how willing they are to lend the workers, how willing they are to lend to investors, and so on.

So that's our financialization component. Essentially,  do the investors have an advantage at some point over the worker buyers? If the investors have an advantage, then they'll buy an increasing fraction of the housing stock over time as people retire and as new housing comes into the market?



3 Finally, let's just discuss a little bit what the housing market is doing. We've got two kinds of buyers as I've said.

The market does two things. It allocates the housing between owner occupiers and tenants but it also allocates the locational rents between owner occupiers in a different social class called rentiers. Now, this is standard economic terminology. rentiers are people who are living off  rents than any action they take to create value for others.

So that means that they are extracting rents from the urban system.

And that could have an effect on growth. So we have two fundamental outputs of this model, the production or sorry, the population structure structure.

And that is essentially whether you have a city of all owner occupiers and that's where we start start for the runs we consider or whether we have some tenants or all tenants. And it makes a difference, because these are, in economic terms, different classes with different opportunities.

Increasing tenant diarization and increasing financialization of the market. Very interesting.

In the process, of course, the wealth created by the city is increasingly transferred to capitalists from workers. Very simple story.
So that's the basic model.

Look at the efect of a link with productivity.

----

- integrated with the equilibrium reasoning the model of the firm.
 similar appraoch in that we use relatively standard reasoning, but carefully built from the ground up. 

And again it Note highly modular. Both the model of the city and the institutional link can be replaces with different models, better data brough in etc. - urbansim - model of data-Dawn working with him for Toronto. I've hosted conversations with 4 or 5 local developers developing proforma for a Waterloo context.  can link. 
- can build of Dawn, Urbansim, Corsica group, workshop also had collaborators from Doynes group at Oxford join. you may know better the current work at that group. But dawn and Corrsica collaborators, and we are working on one on the SLUCE market that will feed into that. interesting features - linking rental markets etc. .
LAND MARKET MODELLING.
also transporation etc.

In each case, built a carefully grounded .. similar appraoch in that we use relatively standard reasoning, but carefully built from the ground up. 
relatively simple, but constrained tightly to the core purpose.

Fits with the purpose of this model 'theoretical exposition' - extension beyond, keep it simple.

THEREA RE DISTINCT INTERVENTIONS
change in the routines resource authority flows -- change in deep structures
Meadows- maps of types of interventions.. lots more.. 

DEFINE SOCIAL INNOVATION - 
Institutional mechansim - one among many- ostrum talked about land rens



## Results


What results that this produces:

So, in the thesis we actually report 84 distinct results.

26:50
Some of them aren't very interesting. But when we say 84 distinct results, it's because for each run, we report graphically seven results like that. So here's our base run at show a picture of the output.

27:11
The data is all in data files, but there is just so there are so many numbers that we've summarized that with I've summarized it with the seven figures.

27:23
Capital wage wage is a measure really of how much people are being paid but since we have fixed this is a modeling trick. We've fixed a subsistence wage, everything above the subsistence wage is what is the urban premium. So we just take the wages subtract the subsistence wage, we get the urban premium that determines the size of the city so here's the wage here's the size of the city what is is determined the size of the city because people only move in if it's worth traveling. 

We've made the city given the the city a uniform distribution of property sizes for simplicity. We've given the city that's not a necessary feature, by the way, it's actually technically really easy to change. It just it just makes everything more complicated to describe. We have made the transportation cost the same for everybody. We've made all the workers the same. None of these are necessary.

28:32
Restrictions except for clarity.

28:35
We have also eliminated any amenities, any features and neighborhoods that might be attractive. Again, it's easy to add back in is an attribute of each piece of property in the agent model.

28:53
But we have followed that dictum from whoever it is that we want as simple as possible model that you want to take away everything your model is complete when you've taken away everything you can.

29:07
So this is really bare bones, but extremely standard theory

29:18
so we have wage, city size.

29:25
Well, little bursts restriction here has meant that we keep the firm size the same to make comparisons a little bit easier, which means that firm numbers have to increase as the population increases. So you're seeing that happening. So firm numbers mimic population.

29:44
We can change that a little bit by letting firms respond. But again, it's a complication that you don't really that doesn't make it easier to understand the important things that are happening.


\section{Results}

30:01
Because it's a standard economic model, we want to make sure that the capital stock behaves as we expect.

30:08
This court this item in the upper corner is the size of the capital stock. And as productivity rises, we expect it to go up. So what you're seeing here is the dynamic adjustment of a city from an initial size with an initial wage to something like stability, and we're modeling it over a period of 150 years, which is a reasonable term. Obviously, we've had to play with the parameters to get this the basic economics of the city to behave properly. Once we do that, what we're most interested in is what happens with the financialization and that's this bottom graph. And the single indicator we're using here. I I am using here is the ownership ratio. How much does the bank own? How much does the what fraction do owner occupiers own? Now this is a bit of a proxy because we could actually talk in terms of the share of ownership right now this is we haven't carefully separated.

31:33
We are not reporting I should say the amount of the rent that's going to each class we're just looking at this owner occupier ratio, they convert pretty directly so it just haven't bothered with the distinction.

Okay, so we have these seven, seven items. And what we what we do is we report on six distinct policy interventions that we thought might affect the evolution.

32:05
Those are capital gains taxes for owners and for for investors, property taxes.

32:14
Here's the list.
That gives us seven results times six or 42 and then we have two cases. 

And this is this is the sort of this is my was my long term objective, which is to see what happens if financialization also affects the productivity if somehow it feeds into the Alonzo Jacob's cycle that we described earlier. And here's a graph of the possible ways that it could feed in and now there's a lot we've talked, we actually have thought I've thought carefully. 

various people looked at the literature and nobody tracks the effects in these sub sectors of the the model all the way through to the Alonso J. Cole. Note there isn't any good reporting on that at this point. The question we're asking is not whether this particular effect is passed through. But whether overall if financialization does affect the cycle, what would happen? Okay, so are we're demonstrating I'm demonstrating that this is important. I'm not demonstrating that any one of these possible channels really is operating regularly.

And this is a this is an indication if this put introducing this linkage actually affects important variables in our 42 experiments.

That we need more research on which linkages are strong, how much they matter.

The model tells us whether or not this is important. This is research that should be done.

So we repeat all 42 that gives us 84 results.

Now I'm just going to summarize the results now.

\section{Results}

First, if there aren't any linkages, it turns out that only then city regulations and transportation cost have much effect.

And the capital gains tax can be important, but the effects are relatively small in the presence of linkages. We believe that potentially important result is that in the presence of linkages, many of the policy instruments are effective.
There are large effects on the urban system. So we'll show you with one comparison. And this is where I would show a case with a linkage in it. And we've got that graph. It's one of the early ones that shows all seven things just with the linkage and no policy variation. 

Okay, second big result is capital gains and speculative investment in housing as a very powerful and a very powerful effect on our indicator variable. And city size when the linkages are there.


So if there are these linkages as we postulate, they actually reduce the productivity of the city and its size, and capital gains taxes can undo that. CUT While capital gains taxes are highly political issue.


But the most effective one is a capital gains tax on investors.


And since investment in buying up housing is non productive investment in some sense, this is an obvious policy direction, CUT politically acceptable except to capital owners.

Okay, the cost of capital has surprisingly little effect on any of the variables. 

I mentioned that transportation has a strong effect, as predicted by the basic Allonzo model. 

increasing density increases the population doesn't affect wages interestingly or city extent by itself. So density is not going to affect the wages or the size of the city, but it will increase the population and per permit, further productivity growth.


To our surprise, restrictive restrictive mortgage market doesn't have much effect.

That's it doesn't it doesn't affect the production sector. And it doesn't actually change the housing mix much properly. 

There's a bit of a surprise for us that increasing property taxes actually reduced the number of owner occupiers, but the effect was trivial if there was no linkage it depresses wages. However, a property tax increasing the property tax, will depress wages if there's a linkage and this is another interesting result in our mind. 

Okay, I've listed the roughly eight results that jumped out at us. The big one, there are two that in my mind are most important one is if linkages exist, they're extremely important and therefore research is needed.

And secondly, perhaps the most powerful instrument for reducing the penetration of cities and increasing the productivity in cities may be a capital tax. On Investment, ownership of property.

'So to summarize feature work three variations on the agents as they stand and three extensions into new domains. 

\section{Future work}


IN THE DISSERTATION, WE ALSO EXPLORE

% BUT THERE'S IS A FOURTH PIECE
But is it possible there's another link that goes the other way?

it's  possible that this effect not only takes money away from people in the city, it's possible that financialization also feeds back and  affects the productivity of the city. One mechanism clearly as people don't have as much money, people living in the city don't have as much money, they don't invest as much in the city so it's not as productive. 
There are multiple mechanisms that might drive this. SLIDE
One of the things we do with the model is say okay if there was an effect, given the very plausible linkages in the literature, what would the implications of that be in our model?



We lay out some possible channels through which that connection or that affect might be happening.

%%%%%%%%%%%%%

\
Three pieces of the model 
- The model of urban production
- The model of urban rents in the city
- And the institutioanal link between them, by which the value produced in distributed through land markets.

There are substantial opportunities to extend this work in any of the three dirrectiosn

Can extend either of them,
- sensitivity- study non l inear dynamcis cysters, characterise introduce data.
- And can use them with people. - in lab as boundary objects for communcitaion. to make ideas explicity - LAB GROUP POHOTO
#### Firms
- Firms- explicit firm models.. Axtel sat 129 million firms- a platform - very good time to work with that kind of data
- central bank expanding how ti does this kind of modelling.- from representative firm models- - 2-3 countries beginning to explicitly model firms.
### City
- Working with dawn- city - partnership proforma California lang group. met with developers sharing pro-forma. - review of 5 main land models- publishing with Dawns. - Rental markets, development etc.
- Also in practice - some work with the Devco- and HIRT model, and with Maryam - national stratgety for a new normal pattern of development. Ian Bird.
Institutional linkages
- Much more explicit - SOCIAL INNOVATION IS - change in routine resource flows
- in practice - a fund with a concessionary layer of finance for non-market- get at core-- but also actually creating a different class of actors- different flows and relationship.. all the non-market actors in Hamilton conveined so it's a 500m-2b fund - vs xx billions of federal investment -- but not just trying to move through the market- actually using that as part of an implicit model with relationships, actors, routines relationships and paradigm -- prioritizing not just the state but the resilience of the state and set of relationships.
- ** this is part of it because - there are a small set of canonical models- holding in relation with that- integrated with the advantages of precision and long run trajectories-- with distributional effects - over people but an extensible- people do things believe things - rules can stop-- and the model is shareable like a toy.

\section{Conclusion}


The housing market is increasingly financialized and this thesis is about what are the effects are. 

As financial actors buy up more of the property near cities, it can affect both distribution and the capacity of cities to build human capital and create value. z
so it is important to understand the mechanism.

THIS IS IMPORTANT BECAUSE WHILE THERE IS ALL THIS VALUE CREATED IN CITIES. - # FOR HOW MUCH, PLOT OF HOW IT'S INCREASED - AND QUOTE FROM WEST -- WE LACK IN ECONOMIC THEORY TO UNDERSTAND THE DISTIRBUTIONAL implications OF THIT.

To summarize the contributions

- we haven't found any model that has both production and financialization in the same model - 
- -- very little formal modeeling space linking any of the study of production to distribuiton, to who it goes to in, any way-- 
- rare to tightly integrate neoclsasicla economic- equal distribution within an agent-base modelling. -- importatn- let's you carse grain.
- ...
In this talk, we first took you through the the theoretical foundations for the work, then we we introduces the model financialization of housing in urban centers with production. Then shared some results, and  discussed possibilities for future work. 

So let me thank you there and I'll take questions.'



SORT 
Land values come from the productivity of agglomeration agglomerating people, okay, so you can imagine global capital buying up all the land.
And this was the idea of the heart of the whole project tree, the beginning, what if financialized institutions and capital actually do buy more and more of the land? What does that do to the social structure? What does it do to the productivity of the city?

We make an agent based model of this process. 

\end{document}