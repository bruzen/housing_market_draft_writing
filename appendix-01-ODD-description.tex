\chapter{OOD Documentation of the Agent-Based Model} \label{appendix-odd}
The core model is documented below using the Overview, Design concepts, and Details (OOD) standard \cite{grimmODDProtocolReview2010a}. Following Grimm et al. \cite{grimmODDProtocolDescribing2020} this ODD consists of seven elements (Figure 1).  1) purpose, 2) state variables and scales, 3) process overview and scheduling, 4) design concepts, 5) initialization, 6) input, and 7) submodels. Conceptually the elements fall into the three categories ``Overview,'' (1-3) ``Design concepts'' (4, with subcategories)  and ``Details''; hence the acronym ODD.

Chapter~\ref{chapter-methodology}, on methodology, gives more detail on the modelling purpose, strategy, and type, The formulae used in the model are described and explained in Chapter~\ref{chapter-model}, the model chapter. Appendix~\ref{appendix-parameters} provides parameter values.  The exposition in this chapter is strictly descriptive and what the model includes and does.


\section{Purpose}
% TODO add in purposes of modelling work - The purpose of this model is to understand ..Following xyz, xyz gives x purposes for modelling.
The purpose of this model \gls{theoretical exposition}. More specifically, we seek is to understand the links between financialization and  the distribution of housing ownership, the capture of rents, and  urban productivity. %, through financialization of the urban housing market. 
Our  qualitative positive results on these issues provide valuable directions for further quantitative research.



\section{State variables and scales}
\subsection{State variabes}
In the ODD framework, state variables refer to the attributes of each kind of entity. The entities in our model are of two sorts, agents and functions .

The agents are persons, properties, a representative urban firm, and an investor. The major functions are a bank, and a realtor. %  who may choose to work at an urban firm, properties, each occupying a an urban grid space. %, 

State variables for the system are: number of workers (population), individual worker savings, number of firms, workers per firm, capital stock per firm, city area, housing stock owned by workers, housing stock owned by investors, and wage level.

The state variable (attributes) for persons are property location, ages, ownership status, savings and mortgages payments or rental payments. Each person occupies one property. 
Persons can choose to work or not work in the city, depending on wages and transportation costs. If the wage justifies travel to the city center, persons with properties at the edge of the city join the urban workforce. 
Retirees are replaced by newcomers from the rural region who bid on properties. If they succeed they become owner occupiers. If they fail, they become tenants.  Individual have a 40 period working life.  

Properties have owners, residents, prices, tax liabilities, maintenance costs,  and locations which yield  locational rents. 

The bank function sets maximum mortgage share and credit-worthiness using person attributes and bank parameters. It sets lending rates for persons and investors based on the prime rate. Mortgages payments are calculated for a 5-period term.
 
Investors and owner-occupiers take out mortgages at rates set by the bank to purchase properties. 

The realtor function keeps lists of properties for sale and for rent in each period, conducts bidding and allocates properties to the highest bidder.

A representative firm has a production function, a workforce, capital stock and a wage 


\subsection{Scales}

The computational field is a grid of properties, typically set to 130 properties by 130 properties, which is sufficient to contain the city and still leaving properties in the `rural' state in every direction after a typical .
100 period run. 100 model cycles are intended to represent 100 chronological years of city development. Longer or shorter runs are often convenient. 


Each property has one resident. (This is a computational convenience that exploits our decision to have a single person type and a uniform density. At the cost of additional computation properties can have different densities and diverse person-types.) 

The city has a density parameter which multiplies the the number of properties by a density number to get the basic city population. 

The population determines the magnitude of the agglomeration effect that enters the firm's production function.



% There are x types of agents. They have the state variables outlined in the tables. 

% variables can include behavioral attributes and model parameters.
% Variables include the model’s entities, their state variables (possibly including behavioral attributes and model parameters), and the model’s spatial and temporal scales.

% Environment variables include XXX

% Overview of process, parameters and default values for the xx model

\section{Process overview and scheduling}

The model proceeds in discrete steps. Each time step represents one year.  Agents use state variables from the prior time step. Agents of a particular kind execute their step function in randomized order.  Figure~\ref{fig:computational-sequence} illustrates the sequence in which function are called. Figure~\ref{fig:information-flows}illustrates the major information flows between agents and functions.

\begin{figure}
\centering \vspace{-1.5cm}
\begin{tikzpicture}[node distance=1.5cm]
\node (init) [startstop] {Initialization};
\node (interventions) [process, below of=init] {Model applies interventions};
% \node (record) [process, below of=interventions] {Record ownership share and reset counters};
\node (mainloop) [startstop, below of=interventions] {Main loop};
\node (firmupdate) [process, below of=mainloop] {Firms update wages based on number of workers};
\node (land) [process, below of=firmupdate] {Land records data and computes price forecast};
\node (actions) [process, below of=land] {People choose to work based on wages, retire, and list properties};
\node (investors) [process, below of=actions] {Investors list properties};
\node (newcomers) [process, below of=investors] {Newcomers bid on properties};
\node (bid) [process, below of=newcomers] {Investors bid on properties};
\node (realtors_sell) [process, below of=bid] {Realtors sell homes};
\node (realtors_rent) [process, below of=realtors_sell] {Realtors rent properties};
\node (store) [process, below of=realtors_rent] {Model stores data};
\node (advance) [process, below of=store] {Model advance time step};

\draw [arrow] (init) -- (interventions);
% \draw [arrow] (interventions) -- (record);
% \draw [arrow] (record) -- (mainloop);
\draw [arrow] (interventions) -- (mainloop);
\draw [arrow] (mainloop) -- (firmupdate);
\draw [arrow] (firmupdate) -- (land);
\draw [arrow] (land) -- (actions);
\draw [arrow] (actions) -- (investors);
\draw [arrow] (investors) -- (newcomers);
\draw [arrow] (newcomers) -- (bid);
\draw [arrow] (bid) -- (realtors_sell);
\draw [arrow] (realtors_sell) -- (realtors_rent);
\draw [arrow] (realtors_rent) -- (store);
\draw [arrow] (store) -- (advance);

\draw [arrow] (advance.south) -- ++(0,-.5) -- ++(7,0) |- (mainloop);
% % Custom arrow path
% \draw [arrow] ($(advance.south) + (0,-0.5)$) -- ++(0,-1) -- ($(mainloop.south) + (-2,-1)$) -- ($(mainloop.south) + (-2,0)$) -- (mainloop);

\end{tikzpicture}
\caption{Computational sequence}
% \label{fig:computational-sequence}
\end{figure}


Persons initially have locations. They choose whether to work or not work, based on the wage premium and transportation costs, thus determining if their location is in the commuter-shed. This step determines the extent of the city and its population. Persons remain in the workforce until they reach retirement age. If people are above the retirement age, they retire and list any properties for sale if they are moving out of the city. 


 % The sequence of the code means agents do not use information from, or interact with agents of the same type during their step function, so the order doesn't matter.

In each time step, the representative firm computes the value of the marginal worker's value product at current prices, chooses a new target workforce and a new target wage. It follows a partial adjustment rule in choosing wage, labour force, capital stock for the following period. Firm size is constrained by diminishing returns to scale. New firms enter when population grow beyond what existing firms target. 


The the wage premium is calculated from the wage.

Each unit of land (lot) is an `agent.' Each uses the wage premium and its own location (plus any other attributes) to compute its locational value (warranted price) and potential rental earnings. 

The Bank function determines mortgage availability for newcomers 

Newcomers and investors bid on properties listed for the period. They use an expected price includes information about how the market has behaved, so it is possible for price bubbles and expectations to feed back into the dynamics. 

A Realtor function takes the list of bids and mimics a bargaining process to reach a final price. if possible. If the owners will not occupy their own houses, properties are rented to newcomers who are included in the tenant count.



% Overview of process, parameters and default values for the xx model
% - description of the model’s schedule that is detailed and precise enough to allow the model to be re-implemented.
% - schedule descriptions based on pseudo-code most useful.


\section{Design concepts}

% Basic principles
% Adaptation - adaptive traits - rules for  how changing in response to changes in environment or themselves. do they seek to increase some measure
% Objectives - what is the objective and how is it measured
% No - Learning - change adaptive traits based on experience
% Prediction - the anticipate based on past prices


We combine two distinct approaches to modelling social systems: agent-based and equilibrium modelling. At the core is  agent-based migration and an agent-based housing market. We use equilibrium conditions for competitive labour markets to bypass the complex and partially understood wage-setting process. We also use equilibrium arguments to compute land rents from the transportation cost and urban wage premium that drive urban locational decisions. Chapter~\ref{chapter-methodology}, on methodology details the decisions about model type.

For simplicity, we present the model for the case where individuals have the same preferences, employment opportunities and transportation costs. The principles explored include the relationship between urban agglomeration effects and the financialization of the urban housing market. The model also expresses several other concepts as described below.

\subsection{Emergence}
There are two possible pure regimes, An owner-occupier regime and a tenant regime which investors own all properties. We start our experiments with a pure owner occupier regime in which those who own houses and work can build equity.  

The city may then transitions toward the investor-owned regime as investor-agents purchase properties based on local expected financial return calculations.  

% Ownership, the regime, emerges, advantage are amplified leading to a different end state.

\subsection{Adaptation, learning and prediction}
There is no learning in the sense that agents do not change their traits based on experience. 
Agents do make predictions, anticipating prices based on the rate at which those prices have increased in the past $\dot P$.

\subsection{Sensing}
All agents have information about the wage, warranted price, expected and their own borrowing costs. They make their decisions based on what is best for them. 

\subsection{Interaction}
Agents get information from the firm about wages. They independently make decisions to work or retire, but because of agglomeration effects, their decision to work and it's effect on the size of the labour market feed back to affect wages in the next step. 

\subsection{Stochasticity}
Stochasticity comes into the main model two ways: through the range of initial agents and savings, and through randomising the order in which which agents act.

% For testing the model's sensitivity to parameters, we use a version that shortcuts the land market and bidding process by directly calculating equilibrium population.  This version lacks the stochastic elements built into the land market sub-model. It is much faster computationally.

\subsection{Feedback loops}
Feedback loops are not part of the ODD standard, but an important concept in this work. 
% https://en.wikipedia.org/wiki/James_J._Kay
% https://www.researchgate.net/scientific-contributions/James-J-Kay-2162967174
% https://uwaterloo.ca/systems-design-engineering/about-systems-design-engineering/department-history
There are two main \glspl{feedback loop} in the model: the productivity-wage, population-productivity loop that we call the Alonso-Jacobs cycle, and the speculative investment-price, inflation-investment cycle that may produce price bubbles. 
% Our model incorporates two important feedback loops. One, driven by agglomeration, we call the \Gls{Alonzo-Jacobs cycle}. The other is a price-financialization feedback that directly changes the ownership pattern in the urban housing market. 
The two loops are linked. Rising productivity raises wages which then works through two paths. It can raise rents, effectively transferring productivity gains to landowners, and it draws more workers into the city workforce, enhancing the \Gls{Alonzo-Jacobs cycle}.

A rapid increase in housing prices may choke off urban population growth and cause the \Gls{Alonzo-Jacobs cycle} to stall. Rapid expansion of the housing stock should have the opposite effect. Much depends on the speed of response of the housing stock and the rate of transmission of agglomeration effects to wages. Our base model allows the housing stock to respond instantly and automatically increments the wage with a small lag. In the base model, financial flows are unrestricted but the rate of financialization is limited by the rate of turnover of ownership. We then parameterize the rate of adjustment for each of the stocks in a simple way in order to conduct sensitivity analysis.\footnote{Feedback loops are a fundamental feature of almost all systems. They have probably been recognized by theorists for centuries. Marx, to take one relatively modern example, identified the growth dynamic of the capitalist system  as a feedback loop, with capital investments producing a surplus that was fed back into investment, growing the stock of capital. Marx claimed that this loop produced dynamic instability and a great deal of subsequent work has supported his insight \cite{dumenilStabilityInstabilityDynamic1986} \cite{schumpeterInstabilityCapitalism1928}. More recently, the Keynesian multiplier is a result of feedback in macro models between expenditure and income. That loop produces a stable equilibrium. Neoclassical growth theory built on that mechanism to explore the determinants of economic growth using differential and difference equations. % Forrester, the creator of system dynamics computer simulation modeling, argued that change over time is caused by the process of accumulation CITE.
% The feedback concept formally entered the social sciences through two channels: cybernetics, pioneered by Nobert Wiener  and the participants of the Macy Foundation Conferences, and the servomechanism/control engineering thread championed by Jay W. Forrester and others. Both threads were picked up and applied by prominent economists.\footnote{Richardson \cite{richardsonFeedbackThoughtSocial1991} mentions Oscar Lange (1970), Kenneth Boulding, and Alfred Eichner, Phillips,  R. G. D. Allen (1956), and Axel Leijonhufvud.} There is now a niche sub-discipline in economics called ``Feedback Economics'' \cite{radzickiIntroductionFeedbackEconomics, cavanaFeedbackEconomicsEconomic2021}. %  and a great deal of work in The servomechanism/control engineering thread is the one most closely related to \gls{system dynamics} modeling and to ideas used in this thesis.
}


\subsection{Observation}
The model records the urban wage, bid prices, realized price, population, which people worked and who owns which property.

\section{Initialization}
Initial values for the model are detailed in Appendix~\ref{appendix-parameters} on initial values.

\section{Input}
To explore the model dynamics interventions change key parameters mid-run to represent shocks to the system, to model labour market and price shocks and cyclical patterns. % time varrying data. 

\section{Submodels}
Three submodels are the spatial structure of the model, with each land unit having a transportation cost based on its distance from the urban center, second the firm model that produces the urban wage, and finally the land market model that allows homeowners and financialized investors to bid on properties in order to capture the rising rents due to agglomeration.  


\section{DIAGRAMS AND FLOW}

Initialization
Apply interventions
Record ownership share and reset counters

Main loop
Firms update wages based on how many people choose to work in the city
Land records locational rents and calculates price forecast
People decide whether to work, retire, and list properties to sell
Investors list properties to sell
Create newcomers to purchase properties for sale
Investors bid on properties
Realtors sell homes
Realtors rent properties
Store data
Advance model

\begin{figure}
\centering
\begin{tikzpicture}[scale=.3,node distance=1.5cm]
\node (init) [startstop] {Initialization};
\node (interventions) [process, below of=init] {Model applies interventions};
% \node (record) [process, below of=interventions] {Record ownership share and reset counters};
\node (mainloop) [startstop, below of=interventions] {Main loop};
\node (firmupdate) [process, below=1cm of mainloop, text width=6cm] {Firms update wages, number of workers, capital  based on prior wage};

% 
\node (land) [process, below=.5cm of firmupdate, text width=6cm] {Land records data and computes price forecast};
    \node (actions) [process,  left=1cm of land, text width=5cm] {People choose to work based on wages, retire, and list properties};
    \node (investors) [process, right=1cm of land] {Investors list properties};

\node (bank) [process, below = .5cm of land] {Bank calculates mortgage availability};

\node (realtors_sell) [process, below = .5cm of bank] {Realtors sell homes};
\node (newcomers) [process, below=2cm of actions, left=2cm of realtors_sell] {Newcomers bid on properties};
\node (bid) [process, right=2cm of realtors_sell] {Investors bid on properties};

\node (realtors_rent) [process, below of=realtors_sell] {Realtors rent properties};
\node (store) [process, below of=realtors_rent] {Model stores data};
\node (advance) [process, below of=store] {Model advance time step};

\draw [arrow] (init) -- (interventions);
% \draw [arrow] (interventions) -- (record);
% \draw [arrow] (record) -- (mainloop);
\draw [arrow] (interventions) -- (mainloop);
\draw [arrow] (mainloop) -- (firmupdate);
\draw [arrow] (firmupdate) -- (actions);
\draw [arrow] (firmupdate) -- (land);
\draw [arrow] (land) -- (investors);

\draw [arrow] (land) -- (bank);
\draw [arrow] (bid) -- (realtors_sell);
\draw [arrow] (newcomers) -- (realtors_sell);
 \draw [arrow] (bank) -- (newcomers);
% \draw [arrow] (investors) -- (newcomers);
% \draw [arrow] (newcomers) -- (bid);
% \draw [arrow] (bid) -- (realtors_sell);
 \draw [arrow] (realtors_sell) -- (realtors_rent);
 \draw [arrow] (realtors_rent) -- (store);
\draw [arrow] (store) -- (advance);

\draw [arrow] (advance.south) -- ++(0,-.5) -- ++(30,0) |- (mainloop);
% % Custom arrow path
% \draw [arrow] ($(advance.south) + (0,-0.5)$) -- ++(0,-1) -- ($(mainloop.south) + (-2,-1)$) -- ($(mainloop.south) + (-2,0)$) -- (mainloop);

\end{tikzpicture}
\caption{Information flows}
% \label{fig:information-flows}
\end{figure}


