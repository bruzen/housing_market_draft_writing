\chapter{Talk}

\section{transcript 2024-03-18}

I'll be talking about the financialisation of housing, housing shortages and income distribution. these issues that are very much on the political agenda right now.  I've developed a model that I think is useful for understanding these issues.  Before we get to the model, we're going to need ten ``vocabulary items'', If I can put it that way. I'll develop them as I go.

list: 

When we have talked about these, we will have the pieces of the model. 
Then I can describe how we brought them together into an actual model.
Then I'm going to talk about the results we got from that model. 
And finally, I'll talk about the research these projects point to future work and finally conclude. 

So let's go through the key ideas. Each of them by the way, has a model attached to it and there's a whole theoretical structure behind each that we're not going to spend a lot of time on. I'm just assembling a toolkit. 

So the first thing  I want to say to say is about cities. 

I want to quote Geoffrey West:
``cities are the crucible of civilization, the hubs of innovation, the engines of wealth, creation, and the centers of power the magnets that attract creative individuals, cities are important.''


And what we have right now is a set of crises in the cities, so that's what we want to look at. 

Scaling effects. This is an  observation:

Now, underlying the importance of cities is one simple fact that relatively recent research has formalized for us and that's the scaling literature. And what they found in that literature is that the producttivity of cities, increases with population. It's a power law here 
In standard economic terms, cities show increasing returns to scale.


I would show the equation at that point. 
$Y=AN^\beta$

So our question, our next question is why this scaling?

	Agglomeration effects
 
It's something that economists have thought about Smith talked about specialization giving rise to these gains from scale. 
Marshall, noticed that for firms and industrial districts that ]
Growth  theorists looked at it for countries, 
Jane Jacobs, who we want to follow here,  observed that it applied to cities and that's what we want to explain.


The reason fr this scale effectmay be technology, it may be networks, it may be the growth of human capital. We can say more about that if you want there's a huge amount of theory. In any case, we essentially are using the Jacobs explanation. That power law formula I showed you, is completely consistent with what Jacob said.

Neoclassical production function

So we do have to connect this to those people who were actually thinking about economic production. The economics of the neoclassical model is our fundamental tool there. 

Except it has a problem because it exhibits diminishing returns to scale.

We're going to be using (as so many people do) the Cobb-Douglas production function which looks like this and you notice is sort of similar to that one up above.

(Not all the models economists have some imperfect competition models.) 

But what we have to solve is how you put firms that exhibit diminishing returns to scale into a city and end up with increasing returns.

the marginal product of labor

An intermediate step here is just the notion of the marginal product of labor and the behavior of firms from microeconomics. In neoclassical theory, firms will keep hiring until the last workers addition to sales is equal to the  wage. 

So we have a link made by  profit-maximizing rational  firms 
from production technology through to the  wage. (this is elementary economics. It's pretty much first year, I understand.)


Growth theory

But there was a problem and this is where neoclassical growth theory came in. Macro-economic economies were not following this diminishing returns pattern. 

The productivity of economies was increasing over time. And they had to fiddle with that funny thing that I would say at this point. There is a lot of theory here and a lot of it's pretty technical. But the very first exercise was simply making the eight in that equation.

Unknown 5:17
increase over time, and then the growth theorists settled down to provide stories about how that would happen.

Unknown 5:25
This is where Jane Jacobs came in.

Unknown 5:29
Because she said, it has to do with just bringing people together. It wasn't just technology or it had to do with the gains from having a lot of people cooperating.

Unknown 5:43
In more modern terms, we probably call that network effect.

Unknown 5:50
So network effects may be the resolution to the contradiction in these two theoretical frameworks we've just looked at.

Unknown 6:03
This doesn't actually relate those individual firms, which are fairly micro, micro economic agents to the macro economy with increasing returns. There's something missing in here and for us, it is the city as Jacobs suggested.

Unknown 6:24
But if you're going to look at production in a city, then you have to look at the spatial structure of the city. It isn't a spacious space free entity.

Unknown 6:37
So we bring the Alonso model in here or it might be called the Alonso move model over the Alonzo Muth Mills model of the 60s, early 60s. And this is a really simple and neat model. The simplest one as a circle, circular city with uniform transportation costs, you throw a picture at them they at this point, jobs are at the center.

Unknown 7:02
People have to pay to get the work so distance matters to them.

Unknown 7:07
There's a maximum distance that they'll they'll decide it it's worth traveling all the way to work, which limits the size of the city.

Unknown 7:20
So there's an advantage to being closer because you don't have to pay so much. We will assume everybody gets the same wage that needs to people close to the city. Most of the middle have more money to pay. And guess what happens? The landlords there can raise the charge for housing close to the center because it's so attractive. And we call that a locational rent.

Unknown 7:44
Now what's really interesting and this is important is that what's attracting people to the city is a wage premium that comes from its productivity. Now you'll see I haven't filled this in yet.

Unknown 8:02
The wage premium is basically the measure of the benefit of being close to the center of the city.

Unknown 8:15
And if landlords can raise the price they charge, then that wage premium becomes converted to essentially land returns.

Unknown 8:26
But that means that the increased productivity of the city which is coming from bringing people together

Unknown 8:38
can be exposed created or appropriated by the people who own the land. This is interesting. It's a problem at all. In this sort of modern Canadian society where people run homes, that means homeowners get the land rents. They get a share of the productivity in the city.

Unknown 9:06
However, we have a process underway, which is financial capital. It's just free floating purchasing power. If you want.

Unknown 9:17
Looking at places where it can buy future streams of income.

Unknown 9:22
And since urban property gives rise to a future stream of income

Unknown 9:30
that global lack of capital might lead to buy up urban land to capture the locational rents. In other words, and this was the idea at the heart of this for me at the very beginning was realizing that if financial institutions capital owners can buy land and capture the ribs, they are capturing the productivity of the city.

Unknown 10:04
I want to repeat that. land owners capture the productivity of the city in financialization capital, take over more and more of the land and therefore capture the productivity of the city generated by the people but but because they can buy the land they can capture it. So financialization is

Unknown 10:35
a process that affects who gets the increase in productivity in the city. We can kind of measure the degree using a variable that we find kind of interesting which is the ownership rate ratio. So our ownership ratio is roughly pointing to how much of the value the net surplus produced by the city is going to residents and how much is going to show we call them capitalists the people with the capital.

Unknown 11:09
It's a very simple proposition.

Unknown 11:12
And then we have one more idea. And I didn't write this down yet. Okay. And that is, well what if this process of financialization not only captures income, which of course is going to impoverished people somewhat, but it's not obviously going to affect the housing stock or the growth of the city. But what if financialization spills over into what's invested? In the wcity? What if the transformation from a city of owners to a city of tenants makes the city less productive? That was the big question that caught my attention.

Unknown 11:53
That's as far as I've gotten.




\section{Go through with DiR 2024-03-18, 3:52 PM Mon}

INTRODUCTION

Theres a lot of talk about housing shortages, inequality, and financialization of housing, that's what I'm talking about today..

Maybe cities

I define F in a pretty standard way: Financialization the capture value (of flows of surplus) from productive activities. It is achieved through the creation of financial instruments that allow the ownership of the income generated- in the cities that is the land rent.. 

The housing market is increasingly financialized and this thesis is about what are the effects are. How does that financialization affect ownership , and how does it affect urban productivity?

We’re going to start by going through the key ideas that form the theoretical foundation of this thesis.


To do this we’re going to start by going through the theoretical basis for understanding these things
1. financialization
2. spatial rents
3. growth
4. and linkages between finacialization and productivity.
 
 2 then I'm going to describe how we've brought these things together in the actual model/how the model functions/and the detailed model.
 
 3 Then I'm going to talk about the results that we got from the model in particular around these key questions of ownership distribution and urban productivity, explore the effect of six policy experiments in the model. 
 4 Finally I'll talk a little about the research that this projects points to -future work and conclude.



1. financialization, what financialization is  specifically  and  the specifics of what financialization looks like in the housing market.

2. spatial rents: what is - distance to city, to understand the value at the center of the city - how proximity is related, spatial rents - We'll go into classical economics to look at rent theory which illuminates the value/how spatial relationships are translated into value. Which make a key relationship between space and economics because econ is about money and productivity and space is .. here once you add transportation cost in, it gives you this connection between how far things are away from the work location and what it the owners can charge , These locational rents are the key link between how far things are away from the center and the economic questions about productivity  that we're going to talk about

3. Then we'll talk about more commonly used economic models for productivity

The xubtle point is the relationship between the productivity of labour and locational rents.
rent will make the connection with the spatial. 
these are models of productivity-- making this connection between space and economics so we need these two things.
when you leave the rent they leave the spatial

Well use elements from marginalism neoclassical theory of the firm to get an urban wage premium

 ways to understand space: because what we’re doing is bringing together ways of modelling economically. and ways of modelling spatial to understand the relationship between financialization and urbanization **
 
then we'll go back into spatial to see how rents were developed into urban theory, which is farther from economics but still has important additions for the spatial element. 

Space chapter takes up urban rents and develops them through urban theory in ways that .. talk about how rents were developed further in urban theory.

Financialization
Urban rents

4. tools to understand productivity and ways to understand space. but this bring pre-existing approaches together to understand this specific question I've introduced.


Then we'll go into theories of growth because productivity scale with productivity growth in cities, so we'll look at agglomeration effects which are key feature of the model- introduce that and why it's important

5. Finally  we'll add a finally little piece before we get  into the model,  which is about the relationship between the financialiation of housing and productivity

So the base model gives us the first oof our questions which is the ownership piece, then we're going to be adding an extension which allows us to ask how it affects productivity

to do that we need this piece about how the fin of housing which is the transfer of value into financial actors, is affecting productivity, so we'll lay out some possible channels through which that effect may be happening.

Then  after introducing the theoretical background, 
1. then I'll describe how we've brought these things together in an agent based model, and how the model functions. 
2. Then I'll talk about the results that we got from the model, in particular around these key questions of ownership, distribution, and urban productivity.  (The results we get  are around ownership. then we have the extension result around productivity) 
3. With and without the productivity link, we'll discuss the results of modelling 6 policy interventions. 
4. Finally I'll talk briefly about future work and conclude. 





FINANCIALIZATION
I’m trying to think how to like financial in order we’re gonna talk about what financial is specifically and how it is created in. It is done accomplished in the housing market like the specifics of what financial looks like in the housing market.

RENT
And then we’ll go into classical economics to borrow this theory of urban rents, which illuminates not urban, but you know rent to look at rent theory, which illuminates the value, or how spatial how distance how spatial relationships are translated into value in ways, like which make the relationship which make a key relationship between space and economics because economics is about money and productivity and space is something where when you add transportation cost in you have it you have this urban rents give you this connection between how far things are away and how and how much it costs, so so we’ll get into that because that makes this key connection rents the key link Sent this in a few different ways, but rents are the key link between spatial between how far things are away from the centre and the economic questions about productivity that we’re gonna talk about then we’re gonna talk about space which brings in urban theory because there’s a I forget what you talk about in space I never noticed space chapter but space chapter is about about takes up urban rents and develops them through urban theory, and ways that are useful because it is relevant already so we’re talking about how urban rents were taken developed in further in urban theory and then we’re gonna go back into But you already talked about in you talked about marginal marginal that’s in the wrench chapter

GROWTH
Just wanna put in the right order that you have it OK so before you say, bring the urban theory and you would say you would say then we’re gonna talk about more commonly used economic models to for productivity, which are important, so rents gonna make the connection with the spatial and but these are our models of productivity that are relevant because we’re again I think the key thing you’re gonna keep hitting because it’s a verbal presentation is we’re making this connection between space and economics And so we need these things so the economic things kind of when you move out of the rent they lose track of they they leave behind the spatial and we’re gonna but we’re gonna use some of these elements from marginal so we’ll introduce that then we’re gonna go back into the spatial and see how urban rents were developed into urban theory, which is less about the economics, but still has important additions for the spatial element

Should be all fairly fast, and then we’ll go into theories of growth, because population productivity skills because there’s a very important element with which productivity scales with population growth so we’re gonna look at agglomour ation effects which are going to be a key element of the model so we’re gonna look at the theory of that and it’s important and then we’re gonna add a final little little piece before we get into the model which is about the relationship between finalization of the housing market and productivity. So the base model that we’re talking about is really about gives us the first of our questions which is the ownership piece and then we’re going to be adding an extension which allows us to look at how it affects productivity and in order to do that we need this piece about what are the channels by which financial ation of the housing project house the housing market which is the transfer of value into the hands of financial actors is affecting productivity and we’re gonna talk about channels through that so we’re gonna lay out some possible channels through which that connection or that affect might be happening and then we’re gonna go into the model and The model is an agent based model something something I don’t know say some details about the model and the results we get are around ownership, and then we have the extension results around this which will take you through when we get to that point and then I’ll talk a little bit very briefly about future worker you can leave that out of the defence

RENT
Now you’re gonna need a slide that says classical rent
This is what I think you wanna say by class so classical economics developed during during the colonial. Moving into industrialization was concerned with distribution. Economies were still largely based on the feudal model with landowners, owning the land and farmers working.
OK so all that matters here is that the structure of the economy in Europe, which they were looking at was based on landowners, owning land and farmers working the land and then they would sell the product that’s what’s key to this red concept this was also developed in the context of increasing a quality as colonialism brought in flows of wealth so there was a Ricardo was writing in the context of a still largely feudal structure with colonialism shifting equality, so there was this increasing focus on inequality, but the system he was looking at was not quite yet it was soon to be transformed, but not quite yet transformed by industrialization
Classical rent theory was developed by Ricardo to talk about the value of the land, which was extracted by these landowners, so at the time landowners were paid to work the land or something so at the time landowners would own the value of the land, and the value of the land was set according to how far it was from the centre so rent theory was about how then you can just take like a description I think you’re fairly succinct we just fill in the description so basically fill in the description very precisely here of land value being like, and it’s based on the story of the car and the car has to transport at the marginal you have the extensive margin and whatever the intensive margin just fill in these things OK

SPACE
Then you go after filling in the details of the car or whatever whatever you say what this tells us is that the value of the land is related to its closeness so this gives us a spatial relationship to city centre that we can use when we’re thinking about economics in an urban context and that’s really really important to what we’re doing here, right
Later economics developed as as the economies change, and land wasn’t the main thing because we shifted from feudalism structuring to industrialization and ultimately capitalism

So we want to say that the feature of capitalism is that land is no longer the main the basis of a wealth inequality in an in such an extreme way, and it’s no longer the basis of what because if you’re not working the land you’re putting what you’re doing instead as you’re putting into production, so the economic model changes with industrialization to capitalism, because where you get these great returns are when you put the capital into production and then you’re making these things
And what that means is that the economic theory from this point shifts from having that spatial element to focussing on production with space excluded
And so at this time, what you get our models of production, which are also very important to how we’re understanding production in the cities in the model that we’re doing so you get marginal ism now you explain marginal ism, and it’s about whatever the marginal production and it has certain advantages as well because it the calculus allows you to create these very Elegant understandings of whatever whatever, and you get and equilibrium, which is important because it allows you to understand certain things about the relationships and how they find equilibrium whatever you wanna say about equilibrium sure you can figure this in so but basically the Keypointe here are it’s giving us the calculus allowed for certain things which also you might want to say didn’t have computers so they could do fairly sophisticated modelling without the capacity to do agent base modelling at this time They could also understand certain things about how production was functioning, and how value like the marginal value allowed certain certain insights, which is important and important to understanding production as opposed to land.

Of course, in cities, even though production in itself, is not so fundamentally spatial as landownership and working the land, and then bringing the land to the city centre those have these fundamentally faced spatial things built into the structure of the economy. Space doesn’t go away and you have this alternate tradition in urban theory that has developed out of some of the insights from Ren theory but without focussing so much on economics and that’s where we have these models the Alonzo model we have Jane Jacobs and here are you feeling? What are they doing And they develop their own tradition that allows you to understand cities and space use and how cities are growing and that’s really important because again what we’re doing is bringing together the economics, and the spatial in a way that allows us to understand the financialization of the housing market, ie the economic effects within a spatial system

GROWTH
And there’s an additional piece that’s really important how cities grow and are functioning in terms economically, which is about how they grow and this is why we bring in agglomerulation effects to the model because what we see is that city scale with population city productivity scales in cities with population super linearly right this is something that has become a key stylized fact in economics whatever you wanna say.

You add in some any extra things about growth you wanna add here like how has Ben looked at? This has been looked at this way and this is where we’re drawing from like maybea little bit of the moth like fairly simple but like the key salient pieces about growth.

PRODUCTIVITY LINKAGE
And then you say, finally before we go into the model, There’s one last piece to introduce so far everything that we’ve talked about has been very important to the base model …  building this relationship between tween models of economic productivity and space these go into the base model in a way that allows us to look at how financial is ation. The capture of value within the housing market is going to affect ownership patterns because it Hass to do with how expensive it is to live near the core of the city i.e. spatial rents but we’re doing that with this model that has equilibrium in it as well and on urban theory whatever but then when we want to approach the question, how do these things affect productivity i.e. feedback into productivity in the city or feed into productivity in the city? We have this additional thing because that’s not part of this initial system, so we’ve build an extension onto the model, which will describe soon or I’ll describe soon but to do that they we have to understand that there sum possible various possible channels in which financial ation and the shift ownership and the shift from owners to tenants might affect urban productivity so we identify a few or these number of possible channels, which include list the channels the potential channels which we can explore through the extension to the model



MODEL
This is what we built 
- an agent based model
- map of features
- the details

RESULTS
OK so with this model we looked at first of all our first hypothesis which is that financialization would affect ownership and here’s the results we got around ownership.

And then results we got with the productivity linkage included.

FUTURE WORK
And then we get to the park you really want to talk about which is future work and you can say so this model has these things these basic results it can be used for policy it can be used for these things here some questions you could look at and there’s a number of extensions both to things that like you extend the model or you do this and then you go through the thing, right

- research agenda most excited about
- potential for connecting with other models - explicit firm behaviour - axtel,  transportation, development/city typology
- plumbing to get there - data, sensitivity analysis - enriched agent assumptions. -- 





CONCLUSION

WHAT WE DID

So what we've done is reviewed the theoretical foundation for this work, described the  model and results, with and without the productivity linkages, and with six policy interventions, and then discussed the implications and where this work leads.

So to conclude, what we’ve presented is fundamentally a contribution to bringing space and economic, modelling back together with with new tools. 

If you look at classical economics, space was built in because of the structure of the pre-industrial agricultural economy, but space is always part of the economy and part of what we’re saying is that space really matters, and that this is helpful for understanding the housing crisis. It’s also helpful for understanding cities and how economies of cities work because they are cities are spatial entities. If we understand finance and urban production as spaceless, then we’re missing something key that helps us to understand 1. the distribution of ownership and thus of wealth and 2. urban productivity itself. So bringing together the space and ownership with a model of urban production allows us to understand the structural effects of financialization in housing a deeper way. 

And it also allows us to explore policy questions that otherwise we would miss.  

How cities thrive is fundamentally a question of how they are structured specially because cities are spatial units, so if you wanna understand how they thrive, you have to understand the spatial structure of both how they produce and how they distribute value. If we bring those elements together, then we can under then we can approach this question in a more sophisticated, and possibly more accurate way, which can ultimately inform better policy, as well as better academic understanding.



\section{New Pass 2024-03-16, 3:06 PM Sat}

INTRODUCTION

Maybe cities

Financialization capture value (of flows of surplus). It is achieved through the creation of financial instruments that allow the capture of the surplus value. 

The housing market is increasingly financialized and this thesis is about what are the effects are. How does that financialization affect ownership , and how does it affect urban productivity?

We’re going to start by going through the key ideas that form the theoretical foundation of this thesis.


To do this we’re going to start by going through the theoretical basis for understanding these things
1. financialization
2. spatial rents
3. growth
4. and linkages between finacialization and productivity.
 
 2 then I'm going to describe how we've brought these things together in the actual model/how the model functions/and the detailed model.
 3 Then I'm going to talk about the results that we got from the model in particular around these key questions of ownership distribution and urban productivity, explore the effect of six policy experiments in the model. 
 4 Finally I'll talk about future work and conclude.



1. financialization, what financialization is  specifically  and  the specifics of what financialization looks like in the housing market.

2. spatial rents: what is - distance to city, to understand the value at the center of the city - how proximity is related, spatial rents - We'll go into classical economics to look at rent theory which illuminates the value/how spatial relationships are translated into value. Which make a key relationship between space and economics because econ is about money and productivity and space is .. here once you add transportation cost in, it gives you this connected between how far things are away and what it costs, rents are the key link between how far things are away from the center and the economic questions about productivity  that we're going to talk about

3. Then we'll talk about more commonly used economic models for productivity
rent will make the connection with the spatial. 
these are models of productivity-- making this connection between space and economics so we need these two things.
when you leave the rent they leave the spatial
Well use these elements from marginalism

 ways to understand space: because what we’re doing is bringing together ways of modelling economically. and ways of modelling spatial to understand the relationship between financialization and urbanization **
then we'll go back into spatial to see how rents were developed into urban theory, which is farther from economics but still has important additions for the spatial element. 
Space chapter takes up urban rents and develops them through urban theory in ways that .. talk about how rents were developed further in urban theory.
Financialization
Urban rents

4. tools to understand productivity and ways to understand space. but this bring pre-existing approaches together to understand this specific question I've introduced.


Then we'll go into theories of growth because productivity scale with productivity growth in cities, so we'll look at agglomeration effects which are key feature of the model- introduce that and why it's important

5. Finally  we'll add a finally little piece before we get  into the model,  which is about the relationship between the financialiation of housing and productivity

So the base model gives us the first oof our questions which is the ownership piece, then we're going to be adding an extension which allows us to ask how it affects productivity

to do that we need this piece about how the fin of housing which is the transfer of value into financial actors, is affecting productivity, so we'll lay out some possible channels through which that effect may be happening.

Then  after introducing the theoretical background, 
1. then I'll describe how we've brought these things together in an agent based model, and how the model functions. 
2. Then I'll talk about the results that we got from the model, in particular around these key questions of ownership, distribution, and urban productivity.  (The results we get  are around ownership. then we have the extension result around productivity) 
3. With and without the productivity link, we'll discuss the results of modelling 6 policy interventions. 
4. Finally I'll talk briefly about future work and conclude. 





FINANCIALIZATION
I’m trying to think how to like financial in order we’re gonna talk about what financial is specifically and how it is created in. It is done accomplished in the housing market like the specifics of what financial looks like in the housing market.

RENT
And then we’ll go into classical economics to borrow this theory of urban rents, which illuminates not urban, but you know rent to look at rent theory, which illuminates the value, or how spatial how distance how spatial relationships are translated into value in ways, like which make the relationship which make a key relationship between space and economics because economics is about money and productivity and space is something where when you add transportation cost in you have it you have this urban rents give you this connection between how far things are away and how and how much it costs, so so we’ll get into that because that makes this key connection rents the key link Sent this in a few different ways, but rents are the key link between spatial between how far things are away from the centre and the economic questions about productivity that we’re gonna talk about then we’re gonna talk about space which brings in urban theory because there’s a I forget what you talk about in space I never noticed space chapter but space chapter is about about takes up urban rents and develops them through urban theory, and ways that are useful because it is relevant already so we’re talking about how urban rents were taken developed in further in urban theory and then we’re gonna go back into But you already talked about in you talked about marginal marginal that’s in the wrench chapter

GROWTH
Just wanna put in the right order that you have it OK so before you say, bring the urban theory and you would say you would say then we’re gonna talk about more commonly used economic models to for productivity, which are important, so rents gonna make the connection with the spatial and but these are our models of productivity that are relevant because we’re again I think the key thing you’re gonna keep hitting because it’s a verbal presentation is we’re making this connection between space and economics And so we need these things so the economic things kind of when you move out of the rent they lose track of they they leave behind the spatial and we’re gonna but we’re gonna use some of these elements from marginal so we’ll introduce that then we’re gonna go back into the spatial and see how urban rents were developed into urban theory, which is less about the economics, but still has important additions for the spatial element

Should be all fairly fast, and then we’ll go into theories of growth, because population productivity skills because there’s a very important element with which productivity scales with population growth so we’re gonna look at agglomour ation effects which are going to be a key element of the model so we’re gonna look at the theory of that and it’s important and then we’re gonna add a final little little piece before we get into the model which is about the relationship between finalization of the housing market and productivity. So the base model that we’re talking about is really about gives us the first of our questions which is the ownership piece and then we’re going to be adding an extension which allows us to look at how it affects productivity and in order to do that we need this piece about what are the channels by which financial ation of the housing project house the housing market which is the transfer of value into the hands of financial actors is affecting productivity and we’re gonna talk about channels through that so we’re gonna lay out some possible channels through which that connection or that affect might be happening and then we’re gonna go into the model and The model is an agent based model something something I don’t know say some details about the model and the results we get are around ownership, and then we have the extension results around this which will take you through when we get to that point and then I’ll talk a little bit very briefly about future worker you can leave that out of the defence

RENT
Now you’re gonna need a slide that says classical rent
This is what I think you wanna say by class so classical economics developed during during the colonial. Moving into industrialization was concerned with distribution. Economies were still largely based on the feudal model with landowners, owning the land and farmers working.
OK so all that matters here is that the structure of the economy in Europe, which they were looking at was based on landowners, owning land and farmers working the land and then they would sell the product that’s what’s key to this red concept this was also developed in the context of increasing a quality as colonialism brought in flows of wealth so there was a Ricardo was writing in the context of a still largely feudal structure with colonialism shifting equality, so there was this increasing focus on inequality, but the system he was looking at was not quite yet it was soon to be transformed, but not quite yet transformed by industrialization
Classical rent theory was developed by Ricardo to talk about the value of the land, which was extracted by these landowners, so at the time landowners were paid to work the land or something so at the time landowners would own the value of the land, and the value of the land was set according to how far it was from the centre so rent theory was about how then you can just take like a description I think you’re fairly succinct we just fill in the description so basically fill in the description very precisely here of land value being like, and it’s based on the story of the car and the car has to transport at the marginal you have the extensive margin and whatever the intensive margin just fill in these things OK

SPACE
Then you go after filling in the details of the car or whatever whatever you say what this tells us is that the value of the land is related to its closeness so this gives us a spatial relationship to city centre that we can use when we’re thinking about economics in an urban context and that’s really really important to what we’re doing here, right
Later economics developed as as the economies change, and land wasn’t the main thing because we shifted from feudalism structuring to industrialization and ultimately capitalism

So we want to say that the feature of capitalism is that land is no longer the main the basis of a wealth inequality in an in such an extreme way, and it’s no longer the basis of what because if you’re not working the land you’re putting what you’re doing instead as you’re putting into production, so the economic model changes with industrialization to capitalism, because where you get these great returns are when you put the capital into production and then you’re making these things
And what that means is that the economic theory from this point shifts from having that spatial element to focussing on production with space excluded
And so at this time, what you get our models of production, which are also very important to how we’re understanding production in the cities in the model that we’re doing so you get marginal ism now you explain marginal ism, and it’s about whatever the marginal production and it has certain advantages as well because it the calculus allows you to create these very Elegant understandings of whatever whatever, and you get and equilibrium, which is important because it allows you to understand certain things about the relationships and how they find equilibrium whatever you wanna say about equilibrium sure you can figure this in so but basically the Keypointe here are it’s giving us the calculus allowed for certain things which also you might want to say didn’t have computers so they could do fairly sophisticated modelling without the capacity to do agent base modelling at this time They could also understand certain things about how production was functioning, and how value like the marginal value allowed certain certain insights, which is important and important to understanding production as opposed to land.

Of course, in cities, even though production in itself, is not so fundamentally spatial as landownership and working the land, and then bringing the land to the city centre those have these fundamentally faced spatial things built into the structure of the economy. Space doesn’t go away and you have this alternate tradition in urban theory that has developed out of some of the insights from Ren theory but without focussing so much on economics and that’s where we have these models the Alonzo model we have Jane Jacobs and here are you feeling? What are they doing And they develop their own tradition that allows you to understand cities and space use and how cities are growing and that’s really important because again what we’re doing is bringing together the economics, and the spatial in a way that allows us to understand the financialization of the housing market, ie the economic effects within a spatial system

GROWTH
And there’s an additional piece that’s really important how cities grow and are functioning in terms economically, which is about how they grow and this is why we bring in agglomerulation effects to the model because what we see is that city scale with population city productivity scales in cities with population super linearly right this is something that has become a key stylized fact in economics whatever you wanna say.

You add in some any extra things about growth you wanna add here like how has Ben looked at? This has been looked at this way and this is where we’re drawing from like maybea little bit of the moth like fairly simple but like the key salient pieces about growth.

PRODUCTIVITY LINKAGE
And then you say, finally before we go into the model, There’s one last piece to introduce so far everything that we’ve talked about has been very important to the base model …  building this relationship between tween models of economic productivity and space these go into the base model in a way that allows us to look at how financial is ation. The capture of value within the housing market is going to affect ownership patterns because it Hass to do with how expensive it is to live near the core of the city i.e. spatial rents but we’re doing that with this model that has equilibrium in it as well and on urban theory whatever but then when we want to approach the question, how do these things affect productivity i.e. feedback into productivity in the city or feed into productivity in the city? We have this additional thing because that’s not part of this initial system, so we’ve build an extension onto the model, which will describe soon or I’ll describe soon but to do that they we have to understand that there sum possible various possible channels in which financial ation and the shift ownership and the shift from owners to tenants might affect urban productivity so we identify a few or these number of possible channels, which include list the channels the potential channels which we can explore through the extension to the model



MODEL
This is what we built 
- an agent based model
- map of features
- the details

RESULTS
OK so with this model we looked at first of all our first hypothesis which is that financialization would affect ownership and here’s the results we got around ownership.

And then results we got with the productivity linkage included.

FUTURE WORK
And then we get to the park you really want to talk about which is future work and you can say so this model has these things these basic results it can be used for policy it can be used for these things here some questions you could look at and there’s a number of extensions both to things that like you extend the model or you do this and then you go through the thing, right

- research agenda most excited about
- potential for connecting with other models - explicit firm behaviour - axtel,  transportation, development/city typology
- plumbing to get there - data, sensitivity analysis - enriched agent assumptions. -- 





CONCLUSION

WHAT WE DID

So what we've done is reviewed the theoretical foundation for this work, described the  model and results, with and without the productivity linkages, and with six policy interventions, and then discussed the implications and where this work leads.

So to conclude, what we’ve presented is fundamentally a contribution to bringing space and economic, modelling back together with with new tools. 

If you look at classical economics, space was built in because of the structure of the pre-industrial agricultural economy, but space is always part of the economy and part of what we’re saying is that space really matters, and that this is helpful for understanding the housing crisis. It’s also helpful for understanding cities and how economies of cities work because they are cities are spatial entities. If we understand finance and urban production as spaceless, then we’re missing something key that helps us to understand 1. the distribution of ownership and thus of wealth and 2. urban productivity itself. So bringing together the space and ownership with a model of urban production allows us to understand the structural effects of financialization in housing a deeper way. 

And it also allows us to explore policy questions that otherwise we would miss.  

How cities thrive is fundamentally a question of how they are structured specially because cities are spatial units, so if you wanna understand how they thrive, you have to understand the spatial structure of both how they produce and how they distribute value. If we bring those elements together, then we can under then we can approach this question in a more sophisticated, and possibly more accurate way, which can ultimately inform better policy, as well as better academic understanding.





\section{Recording}
So what I’m gonna do now is introduce the key ideas that are arranged into like the foundation or something of this thesis that is a theoretical background and then I’m going to describe how we brought these things together in the actual model, and how the model functions and the details of the model, and then I’m going to talk about the results That we got in the models in particular around these key questions of ownership distribution and urban productivity.

I’m trying to think how to like financial in order we’re gonna talk about what financial is specifically and how it is created in. It is done accomplished in the housing market like the specifics of what financial looks like in the housing market.

And then we’ll go into classical economics to borrow this theory of urban rents, which illuminates not urban, but you know rent to look at rent theory, which illuminates the value, or how spatial how distance how spatial relationships are translated into value in ways, like which make the relationship which make a key relationship between space and economics because economics is about money and productivity and space is something where when you add transportation cost in you have it you have this urban rents give you this connection between how far things are away and how and how much it costs, so so we’ll get into that because that makes this key connection rents the key link Sent this in a few different ways, but rents are the key link between spatial between how far things are away from the centre and the economic questions about productivity that we’re gonna talk about then we’re gonna talk about space which brings in urban theory because there’s a I forget what you talk about in space I never noticed space chapter but space chapter is about about takes up urban rents and develops them through urban theory, and ways that are useful because it is relevant already so we’re talking about how urban rents were taken developed in further in urban theory and then we’re gonna go back into But you already talked about in you talked about marginal marginal that’s in the wrench chapter
Just wanna put in the right order that you have it OK so before you say, bring the urban theory and you would say you would say then we’re gonna talk about more commonly used economic models to for productivity, which are important, so rents gonna make the connection with the spatial and but these are our models of productivity that are relevant because we’re again I think the key thing you’re gonna keep hitting because it’s a verbal presentation is we’re making this connection between space and economics And so we need these things so the economic things kind of when you move out of the rent they lose track of they they leave behind the spatial and we’re gonna but we’re gonna use some of these elements from marginal so we’ll introduce that then we’re gonna go back into the spatial and see how urban rents were developed into urban theory, which is less about the economics, but still has important additions for the spatial element

Should be all fairly fast, and then we’ll go into theories of growth, because population productivity skills because there’s a very important element with which productivity scales with population growth so we’re gonna look at agglomour ation effects which are going to be a key element of the model so we’re gonna look at the theory of that and it’s important and then we’re gonna add a final little little piece before we get into the model which is about the relationship between finalization of the housing market and productivity. So the base model that we’re talking about is really about gives us the first of our questions which is the ownership piece and then we’re going to be adding an extension which allows us to look at how it affects productivity and in order to do that we need this piece about what are the channels by which financial ation of the housing project house the housing market which is the transfer of value into the hands of financial actors is affecting productivity and we’re gonna talk about channels through that so we’re gonna lay out some possible channels through which that connection or that affect might be happening and then we’re gonna go into the model and The model is an agent based model something something I don’t know say some details about the model and the results we get are around ownership, and then we have the extension results around this which will take you through when we get to that point and then I’ll talk a little bit very briefly about future worker you can leave that out of the defence

Now you’re gonna need a slide that says classical rent
This is what I think you wanna say by class so classical economics developed during during the colonial. Moving into industrialization was concerned with distribution. Economies were still largely based on the feudal model with landowners, owning the land and farmers working.
OK so all that matters here is that the structure of the economy in Europe, which they were looking at was based on landowners, owning land and farmers working the land and then they would sell the product that’s what’s key to this red concept this was also developed in the context of increasing a quality as colonialism brought in flows of wealth so there was a Ricardo was writing in the context of a still largely feudal structure with colonialism shifting equality, so there was this increasing focus on inequality, but the system he was looking at was not quite yet it was soon to be transformed, but not quite yet transformed by industrialization
Classical rent theory was developed by Ricardo to talk about the value of the land, which was extracted by these landowners, so at the time landowners were paid to work the land or something so at the time landowners would own the value of the land, and the value of the land was set according to how far it was from the centre so rent theory was about how then you can just take like a description I think you’re fairly succinct we just fill in the description so basically fill in the description very precisely here of land value being like, and it’s based on the story of the car and the car has to transport at the marginal you have the extensive margin and whatever the intensive margin just fill in these things OK

Then you go after filling in the details of the car or whatever whatever you say what this tells us is that the value of the land is related to its closeness so this gives us a spatial relationship to city centre that we can use when we’re thinking about economics in an urban context and that’s really really important to what we’re doing here, right
Later economics developed as as the economies change, and land wasn’t the main thing because we shifted from feudalism structuring to industrialization and ultimately capitalism

So we want to say that the feature of capitalism is that land is no longer the main the basis of a wealth inequality in an in such an extreme way, and it’s no longer the basis of what because if you’re not working the land you’re putting what you’re doing instead as you’re putting into production, so the economic model changes with industrialization to capitalism, because where you get these great returns are when you put the capital into production and then you’re making these things
And what that means is that the economic theory from this point shifts from having that spatial element to focussing on production with space excluded
And so at this time, what you get our models of production, which are also very important to how we’re understanding production in the cities in the model that we’re doing so you get marginal ism now you explain marginal ism, and it’s about whatever the marginal production and it has certain advantages as well because it the calculus allows you to create these very Elegant understandings of whatever whatever, and you get and equilibrium, which is important because it allows you to understand certain things about the relationships and how they find equilibrium whatever you wanna say about equilibrium sure you can figure this in so but basically the Keypointe here are it’s giving us the calculus allowed for certain things which also you might want to say didn’t have computers so they could do fairly sophisticated modelling without the capacity to do agent base modelling at this time They could also understand certain things about how production was functioning, and how value like the marginal value allowed certain certain insights, which is important and important to understanding production as opposed to land.

Of course, in cities, even though production in itself, is not so fundamentally spatial as landownership and working the land, and then bringing the land to the city centre those have these fundamentally faced spatial things built into the structure of the economy. Space doesn’t go away and you have this alternate tradition in urban theory that has developed out of some of the insights from Ren theory but without focussing so much on economics and that’s where we have these models the Alonzo model we have Jane Jacobs and here are you feeling? What are they doing And they develop their own tradition that allows you to understand cities and space use and how cities are growing and that’s really important because again what we’re doing is bringing together the economics, and the spatial in a way that allows us to understand the financialization of the housing market, ie the economic effects within a spatial system

And there’s an additional piece that’s really important how cities grow and are functioning in terms economically, which is about how they grow and this is why we bring in agglomerulation effects to the model because what we see is that city scale with population city productivity scales in cities with population super linearly right this is something that has become a key stylized fact in economics whatever you wanna say.

You add in some any extra things about growth you wanna add here like how has Ben looked at? This has been looked at this way and this is where we’re drawing from like maybea little bit of the moth like fairly simple but like the key salient pieces about growth.

And then you say, finally before we go into the model, There’s one last piece to introduce so far everything that we’ve talked about has been very important to the base model …  building this relationship between tween models of economic productivity and space these go into the base model in a way that allows us to look at how financial is ation. The capture of value within the housing market is going to affect ownership patterns because it Hass to do with how expensive it is to live near the core of the city i.e. spatial rents but we’re doing that with this model that has equilibrium in it as well and on urban theory whatever but then when we want to approach the question, how do these things affect productivity i.e. feedback into productivity in the city or feed into productivity in the city? We have this additional thing because that’s not part of this initial system, so we’ve build an extension onto the model, which will describe soon or I’ll describe soon but to do that they we have to understand that there sum possible various possible channels in which financial ation and the shift ownership and the shift from owners to tenants might affect urban productivity so we identify a few or these number of possible channels, which include list the channels the potential channels which we can explore through the extension to the model




Into now we’re gonna move into explaining the model now

We actually did the model then explain a little bit about the approaches very briefly you can add stuff. I’ll tell you how long it can be like this may be too long anyways, I guess.

I am not gonna try to summarize the model at this point I think you’re pretty good at being like this is what we built you will want a summary statement which is like it has these it is an agent based model With whatever it based on this these are the sections like kind of map out the overall and then go into the details I would suggest they try to do a high-level map and then go into the details… You’re gonna introduce the details I can’t do that for you
Then you’re gonna go into OK so with this model we looked at first of all our first hypothesis which is that financial ation would affect ownership and here’s the results we got around ownership

And then you do hear the results we got around ownership and then hear the results with the extension

And then we get to the park you really want to talk about which is future work and you can say so this model has these things these basic results it can be used for policy it can be used for these things here some questions you could look at and there’s a number of extensions both to things that like you extend the model or you do this and then you go through the thing, right


And then in conclusion what we’ve done here is fundamentally a contribution to bringing space and economic, modelling back together with with new tools right so if you look at classical economic space was built in because of the economy, but space is always part of the economy and part of what we’re saying is that space really matters And this is helpful for understanding I think, saying in the conclusion like this is really helpful for understanding the housing crisis. It’s also helpful for understanding cities and how economies of cities work because they are cities are spatial entities and if we only understand them like this is the place to say if we only to kind of like say why it matters I think if we only understand them as spaceless production, then we’re missing something key that helps us to understand number one what is actually feeding the productivity and how their functioning in the real world so bringing together the space in the economics allows us to understand in a deeper way And it also allows us to explore policy things in certain ways that otherwise we would miss it. It helps us understand. It helps us to understand what is actually happening in productivity and cities because you’re not missing you’re not missing if you only understand cities without the economy like if you wanna understand how do cities you have to understand maybe that’s
If you wanna understand how cities thrive that is fundamentally a question of how they are structured specially cause cities are spatial unit so if you wanna understand how they thrive, you have to understand them as cities but it’s a product it’s a economic question like you creation of wealth and growth are fundamental to how we understand the question of thriving, but we also have to understand that they are spatial entities And if we bring those elements together and understand how those things are feeding into each other both then we can under then we can approach this question in a more sophisticated and possibly accurate way
Which can ultimately inform better policy, as well as better academic understanding




\section{Notes}
Financialization capture value/flows of surplus. and it is done through the creation of financial instruments. To financialize something involves creation of instruments that allows the capture of surplus value. 


The housing market is increasingly financialized. (page stacked newspapers) - #(buy. houses, stocks -  instruments, ..
 and so this thesis is what are the effects

So how does that affect ownership and how does that affect urban productivity?

To do this we’re going to start by going through the theoretical basis for understanding these things
1. financialization
2. spatial rents


what is - distance to city, to understand the value at the center of the city - how proximity is related, spatial rents
3. ways to understand space

because what we’re doing is bringing together ways of modeling economically. and ways of modelling spatial to understand the relationship between financialization and urbanization **

tools to understand productivity and ways to understand space. but this bring pre-existing approaches together to understand this specific question I've introduced. 

Quick summary
So what I'm going to do 
is introduce the key  ideas that are arranged into the foundation of this thesis, that is the theoretical background, then I'm going to describe how we've brought these things together in the actual model/how the model functions/and the detailed model. Then I'm going to talk about the results that we got from the model in particular around these key questions of ownership distribution and urban productivity.

We're going to talk about what financialization is  specifically  and how it is accomplished in housing markets. the specifics of what financialization looks like in housing market.

We'll go into classical economics to look at rent theory which illuminates the value/how spatial relationships are translated into value. Which make a key relationship between space and economics because econ is about money and productivity and space is .. here once you add transportation cost in, it gives you this connected between how far things away and what it costs.. rents are the key link between how far things are away from the center and the economic questions about productivity  that we're going to talk about

Then we'll talk about more commonly used economic models for productivity
rent will make the connection with the spatial. 
these are models of productivity-- making this connection between space and economics so we need these two things.
when you leave the rent they leave the spatial
Well use these elements from marginalism


then we'll go back into spatial to see how rents were developed into urban theory, which is farther from economics but still has important additions for the spatial element. 

Space chapter takes up urban rents and develops them through urban theory in ways that .. talk about how rents were developed further in urban theory.
Financialization
Urban rents

Then we'll go into theories of growth because productivity scale with productivity growth in cities, so we'll look at agglomeration effects which are key feature of the model- introduce that and why it's important

Then we'll add a finally little piece before we get  into the model,  which is about the relationship between financialiation of housing and productivity


So the base model gives us the first oof our questions which is the ownership piece, then we're going to be adding an extension which allows us to ask how it affects productivity

to do that we need this piece about how the fin of housing which is the transfer of value into financial actors, is affecting productivity, so we'll lay out some possible channels through which that effect may be happening.

Then we'll go into the model. 

The model is an agent based model. The results we get  are around ownership. then we have the extension result around productivity

and then we'll talk briefly about future work. 
