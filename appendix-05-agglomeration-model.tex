\renewcommand{\sfdefault}{phv}

\section{Initial values for  the agglomeration parameter}
Conclusion:

{\Large  $Prefactor = 1506.712$ based on width=height=10 and density=100} and agglomeration\_coefficient= 1.2,
We need to get the scales of the parameters and the population size roughly consistent.

I assume the 10X10 grid is full. 

\section{Computing the prefactor: details}
We have set the subsistence\_wage to \$40,000.

The urban wage premium is in the range of 13-20\%  Assume 20\% and we get \$8,000

Under the neoclassical assumption  \$40,000 is the  marginal productivity of rural worker

The urban production function is 
\[Y=AN^\beta\]
\[Y=prefactor*working\_population**scaling\_exponent\]

where $working\_population = width*height*density$ 

so 
\[Y=prefactorA*(width*height*density)^{\beta}\]

We have more conditions that this has to satisfy: The marginal product must be consistent with the urban wage and the distribution rule. The wage cannot add up to more than total output output. Workers cannot get 1.2Y/N, for example. With CRS they would get 0.8Y/N.    

\subsection{approach one: from agglomeration surplus}
\[urban\_wage= subsistence\_wage + wage\_share * agglomeration\_surlpus\]
The \textbf{agglomeration surplus} is the excess relative to the CRS case when $\beta=1$:
\[agglomeration\_surplus= A(N^{1.2} -N^1) \]
The urban wage premium is then the share for each worker:
\[\omega= wage\_share * \frac{agglomeration\_surlpus}{N}\]

and this becomes
\[\omega= wage\_share * A\left(\frac{N^{1.2}-N^1}{N}\right)=1 * A\left(N^{0.2}-1\right)\]

To see what this looks like, consider a population of 10,000 when  wage\_share=1
\[\omega= \$8,000 = A\left( 6.309-1 \right)\]

{\Large So $A = 1506.712$ based on width=height=10 and density=100}  

Smaller N makes A bigger

% Say width=height=15 and density= 200:
% \[\omega= \$8,000 = A\left(10000-1\right)\]

\subsection{Approach 2: from Marginal product and subsistance wage}
We want the marginal product of labour in the rural economy at the firm level to be at least close to \$40,000.

Firm employment $L$ is small relative to  urban employment {N}.  We can create a generic rural firm and then consider an urban firm with agglomeration effects to get the parameters we need. 

Assume that the rural p[rooduction function is 
\[Y^R=A^R K^\alpha L^\beta\]
where $\alpha=0.2$  and $\beta=0.8$. The marginal products are 
\[MPL=\beta Y/L=\$40,000\] and\[\ MPK=\alpha Y/K =0.05\]
From the first, 

\[Y=\frac{L*\$40,000}{0.8}=\$5\ million\]

This is firm revenue. From the MPK, 

\[ \frac{0.2 \$5\ million}{0.05}=K =\$20\ million \]

We now have the capital, labour and output for a model firm with a marginal product of labour  equal to the subsistence wage we have chosen.

We now consider that this firm operates in the city and enjoys  urban agglomeration benefits.  If the scale coefficient=1.12$,$ $\omega$ would be as a first appr Say that the marginal product rises to \$48,000. This supports an urban wage premium of $\omega\$8,000$.

We need to have a population size or a number of firms with a firm size. Assume the population is 10,000 and firms have 100 workers. All firms will have the same marginal product of labour in a competitive labour market, so size should not matter. 

\[Y^U=A^R N^\gamma K^\alpha L^\beta = N^\gamma Y^R\]
with a marginal product of 
\[MPL^u=\$48,000=N^\gamma \beta Y^R/L=N^\gamma *\$40,000\]
so $N^\gamma=1.2$ and $\gamma = .019$. 


This value is consistent with an empirical  value for $\beta$   1.02. The value is lower than empirical estimates. Furthermore, if we consider the same firm structure and a population of one million the value falls, which suggests that agglomeration effects are not scale-independent, but instead increase with urban size. $\beta$ may itself be a function of $N$.

A second interesting possible implication of our calculation is that only part of the agglomeration effect appears in wages. Urban rents  are large, but agglomeration effects are much larger.


 

%\[subsistance_wage= MPL(\beta=.= \]
