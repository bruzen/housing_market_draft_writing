\chapter{Gather for INTRO}


\cite{spenceUrbanizationGrowth2009} 
curve cost curve and savings rate.. huge chunk missing..
early papers in this book, also very good


This study is an exploration of the impact of financialization of the housing market, and what that means for urban productivity.

INTRO - SUMMARIZE 
This matters because, % thinking about housing markets and what was going on in Ontario, and around the world,
cities are going through a great transformation of ownership structures. 

link to productivity missing


\section{NEXT cities are central to productivity, housing ownership is central to wealth distribution, what's at stake is the future of the middle class and the productivity/success of cities}
- formal tools for analyzing resilience an interventions

- theire dynamics static tools are not enough. 

"Urbanization and growth go together: no country has ever reached middle income status without a significant population shift into cities." \cite{annezUrbanizationGrowthSetting2009} % (see book in \cite{spenceUrbanizationGrowth2009})

\subsection{Cities produce benefit and are central to productivity}
Cities produce enormous benefits and are central to productivity, they're also undergoing a massive change

the ownership tenure pumps wealth out'
however rich they get people are pushed to a kind of subsistence frontier. 
suggests also simple approaches to building of social wealth anchored in place. -- clear suggestions for productive interventions- clear policy implications. 
%Most of the economic theory talks about where people go, and it doesn't talk about the value they create in the city and where that goes. That's finacialization, capturing those benefits is what capitalists are doing now.

What does this mean - given the particular structure of urban wealth creation.

Who claims those benefits, and what does that mean for productivity and the future of cities?

Our analysis suggests  it actually has  implications for the success of cities. % and of socieities, the future of a middle class, and whether citiec can continue to generate value. 


Cities are essential - draw on bettencourt/west's case,
Geoffrey West has argued that ``Cities are the crucible of civilization, the hubs of innovation, the engines of wealth creation and centers of power, the magnets that attract creative individuals, and the stimulant for ideas, growth, and innovation.'' \cite{westScaleUniversalLaws2017} 
Jane Jacobs talks about the vibrancy of cities, and their central role as the source of productivity. 
 \cite{jacobsEconomyCities1969}.
 
 Cities are particularly central to productivity.
 The scaling of wealth in cities has strong theoretical and empirical support  \cite{bettencourtGrowthInnovationScaling2007, bettencourtOriginsScalingCities2013, dongUnderstandingMesoscopicScaling2020, loboUrbanScalingProduction2013} 


in particular they are key to productivity. Wealth scales superlinearly with density - every increase in productivity, more than increases the productivity of the place. 
The relationship goes back into prehistory, not tied to any approach to government or anything else, from smallest nomadic communities, to largest empires, city states, and modern metropolises.

they live forever - so unlike individuals, don't move, don't die, in the same location as in ancient cities. Tokyo grow back. 


part of the story is what benefits they create
the second is who gets it

Most of the economic theory talks about where people go, and it doesn't talk about the value they create in the city and where that goes. That's finacialization, capturing those benefits is what capitalists are doing now.

% can actually reduce the incentive to work- run the thing on debt.. not on profit..

% Canada faces a housing crisis.

A defining of this transition
beginning 

housing ownership has been the primary way 
under theorized why this happens - why is it that those with houses build wealth and those without don't - net wealth largely negative, not always but often

key to the racial wage gap - getting debt/resources for to acquire assets and skills at key moments - in particular a house and an education the central
key to how newcomers face systemic disadvantage
to how young people build or fail to build wealth




\subsection{Cities are undergoing a massive change in ownership structure}
% There's price changes, but there's this deeper story of the transformation of the ownership structures. 
 % I realized that this idea of financialization of the housing market that seemed to be gaining a certain amount of currency when I started to think about it had some

Cities matter and are central to productivity, but one of the big changes happening now is they are undergoing a massive change in ownership structure. 

CALIFORMA AFTER CMC... 
70 percent in ontario 
**NOTE FOR POLICY - CMHC LEANING INTO THIS- SUPPORTING BULT FOR RENTAL PROPERTIES. 

DEEP TRANSFORMATION OF OWNERSHIP STRUCTURE
what is the implication of it
- suburbs brought

- separate from development - owning suburban homes.. 
- this distinct rent profile.


CHANGE IN OPWNERSHIP STRUCTURE MATTERS- 
MAIN WAY PEOPLE BUILD WEALTH AND THE KEY TO THE MIDDLE CLASS
AND THE BOUNDARY -- THOSE OUTSEIDE HAVEN'T BUILT ANY WEALTH..

\subsection{The research gap}

WE STARTED WITH A SELF EVIDENT STAMENT

 but we don't have the tools to understand the implications of integrate it into formal theory.
 
 There's lots of work on causes.
 less on productivity 
 this is about making the link with productivity--

RESEARCH GAP..


yet, looking at formal models of the economy - missing

a kind of a mystery


\textbf{made what seemed like self evident statements given emerging analysis of what's happened WHAT WAS THE STATEMENT} LILIANA ETC.

i make models, i tried to find a model to explain formally what had happened
but it was like a blank spot. 

it was invisible. 

it was \textbf{kind of a mystery}, it seemed like such a simple thing, and yet, 
urban models of spatial 


\textbf{it turned out the actual categories we were talking about were the kind of things the formal model} are blind to
-parts there, can't talk because of how they're structure --- go deeply into the traditions to find the pieces that could be brough together

-add back rent
-adding space to finance/financialization.
..


\gls{middle class}
..Emergence..
class consistent with Roemoer.. 
to solve it ended up going deeply into theory

ACTUALLY RESEARCH GAP IS 2 FOLD
FOUND WE COULDN'T MODEL THIS LINK WITH PRODUCTIVITY.


speaks to how branches of econ don't talk with each other. classical economics - concerned with distribution.. models of finance spaceless, models of productivity without both, but with theses powerful tools to understand how communities build wealth.


RESEARCH GAP

The second sub-model incorporates  productivity of cities. Geoffrey West has argued that ``Cities are the crucible of civilization, the hubs of innovation, the engines of wealth creation and centers of power, the magnets that attract creative individuals, and the stimulant for ideas, growth, and innovation.'' \cite{westScaleUniversalLaws2017} 


There is a housing crisis ADD FACTS ABOUT WHAT WE MEAN BY THAT.



WHAT WE GET TO IS A MODEL THAT AIMS TO BE AS SIMPLE AS THE ALONZO MODEL, BUT LINKS THESE 3 BRANCHES OF THE STUDY OF THE ECONOMY AND SPACE TOGETHER

"The main goal of this research as been to add a financial system to the simple Alonzo model in order to examine the impact of financial cycles on urban productivity and distribution. The \gls{agent-based model} we develop has three major components: 

THAT IS EASILY EXTENSIBLE, CAN BE GROUNDED EMPIRICALLY, AND USED FOR POLICY MAKERS AND SYSTEMS CHANGE PEOPLE TO EXPLORE OTHER POSSIBILITIES.


While we leave much of the analysis



Resilience is also largely outside economic modelling
so we do one set of extensions showing a resilience/hysteresis analsyis. how this system of ownership pumps wealth out of communities on the up and down swing, a glaring gap in econ other

In addition to this domain contribution to understanding the implications for cities and urban productivity of shifts in tenure. 
systems design person.

There's also a methodological story 
ABMs in line with the frontier- what would happen - simplicity and clarity - these frontiers.. like sectional diagrams in engineering. 

and then the extension to hysteresis

in an age of endemic modelling, making these into little toys with sliders, the user interface of the model, a prototype

and the development of models that are interoperable, library elements and extensible. 


Most centrally the resilience dynamics.
the making of regimes which is what makes freedom possible in this kind of system. 
Methodological reflections on each stage of the development. 




\subsection{What we need to do}
% That sounds more simple than it is because to 
to achieve that, we need actually to 

1. build a model housing market for a city. 

2. And we have to, therefore have an urban model. 

3. And we have to figure out what it is that's at stake in the financialization which turns out to be land rents. And that means the city has to bring together essentially classical land rent theory, modern urban theory and the economics and bidding processes in a housing market. THIS IS THE MAIN PIECE, DO WE  NEED ALL IN THIS LIST?

4. And then it has to keep track of some macro variables that are significant for policy. 


\chapter{Cut from Introduction}

% MORE IS OWNED BY FINANCIAL....
% One of the great transformations, is that financial capital owns a great deal more property. % xx in california, xx in Ontario, 70\% of new builds in Waterloo region. % Check
% In the next section, we describe the core model, then in following sections sketch the position this in the literature, describe the core contributions, and then outline how the document will develop these ideas.

\section{The financialization of housing markets}
% This thesis is thus a contribution to developing a modern urban theory of rent.

%The effects of financialization on cities and economies has not been fully accounted for because the tools of the different, relevant disciplines have not been adequately integrated. The current approaches to describing the financialization of housing and its effects predict / explain /account for the housing crisis and effects on home ownership and access to housing, but our work shows that there are broader effects that have not been accounted for / predicted. 
% This fuller picture is made possible by bringing together classical rent theory, neo-classical XX and urban theory to create/and using/along with a agent-based model that allows us to .... (What the modelling technique enables) 
%This work is important for understanding the current policy context. 

\section{Position in the literature}

% We further link the model of urban rents to emerging concerns about the financialization of the housing market. 

%*** (another sentence on why this is great) --The space-less quality of the study of finance leaves out xyz GET PHRASING- CAN'T SEE- INTEGRATION OF SPACE NEGLECT GROWTH FACTOR. 
% Integrating classical and neoclassical economic approaches with standard urban theory,  allows us to identify 
% ..we can build a more comprehensive model of financialization and its effects. This makes it possible to trace the spatial distribution of the rents.

% Rent is important is all these traditions, but has been neglected in modern theory. We argue that to formally understand the processes behind financialization and the housing crisis, what's needed is a modern theory of urban rent. This thesis is a contribution to the development of that theory. REDUNDANCY?

% The focus of this thesis is a topic that falls in the overlap between three academic  disciplines, Economics, Urban Geography, and Planning. % ***E mAYBE SUMMARIZE THE FOCUS OF EACH? While Economics traditionally focuses on... Urban Geography looks at ..... and Planning is the study of....

 % ***E DO YOU MEAN GEOGRAPHIC SPACE AND PEOPLE SOMEHOW? i FEEL LIKE YOU DON'T MEAN THIS? MAYBE MORE LIKE: The central shared concern between the three discipline it the role of geographic space in shaping WHAT? social and economic systems? human systems? society??
% \begin{figure}
% \begin{tikzpicture}{scale=.5}
% % find color cotrol for ball. Tind way to stop line short of node
% \coordinate (planning) at (-5,1);%PREFACE
% \coordinate (economics) at (5,.75);%
%  \coordinate (geography) at (-.5,-2); %history
% \coordinate (finance) at (0,5); %

% \draw [line width=2mm, black!15, ] (planning)--(economics);
% \draw [line width=2mm, black!20, ] (geography)--(economics);
% \draw [line width=2mm, black!20, ] (geography)--(planning);

% %\draw [line width=2mm, black!25, ] (geography)--(finance);
% %\draw [line width=2mm, black!20, ] (planning)--(finance);
% %\draw [line width=2mm, black!20, ] (finance)--(economics);
% \node [circle,shading=ball, minimum width=2.1   cm, white, align=center] (ball) at (planning) {Planning};
% \node [circle,shading=ball, minimum width=2.2cm, white, align=center] (ball) at (economics) {Economics};
% \node [circle,shading=ball, minimum width=3cm, white, align=center] (ball) at (geography)[text width=2cm] {\large Urban\\ Geography};
% %\node [circle, shading=ball, minimum width=2.4cm, white, align=center] (ball) at (finance)[text width=2cm] {Finance};

% \node at (-.3,-.1) [red] {\Large \textbf{Space}};
% \end{tikzpicture}
% \caption{The common concern of three fields topic }
%     \label{fig-three-fields}
% \end{figure}

%This thesis relates to four major fields: classical rent theory, neoclassical production theory and growth theory, the scaling literature, and urban spatial models.
% economics and the study of cities. 
% Our model of the urban economy is based on work from those developed in 

%***E ADD:This thesis describes a model that draws together LIST HERE. 
% ***E OR list all??? ie. From Classical Rent to Neo-Classical Marginalist theory to XYZ urban theory and ...... etc. SET UP THE WHOLE SECTION.
%E ADD: The concept of rent illustrates WHAT in relation ot distribution, allows what kinds fo insights, which sets up this work.  We use the concept of rent to frame our consideration of how wealth is distributed in society by looking at WHAT?? how surplus is distributed? how locational value and location OR WHAT?? affects surplus distribution?? I dont know but please summarize how rent is used in this thesis. 

% Nearly contemporaneous thinker, Johann Heinrich  von Th\"unen (1783-1850) developed a planning model to guide the location of economic activities for an urban-agricultural society.  A version of that model  was reinvented in urban geography by XXX. Alonzo\footnote{We use a version of the well-established model of Alonso (1964), Muth (1969) and Mills (1967), and formalised by Wheaton (1974),} % ***E NEED MORE DETAIL HERE ABOUT ALONZO"S MODEL, 

% We link the Alonzo model with more recent work on growth theory starting with Robert Solo's XXX and with the endogenous growth models of Lucas () and draw on Jane Jacobs's insight that endogenous urban growth  is. now driving economic development. Jacobs's insight is empirically supported by recent work in the complexity literature on urban scaling by XXXX ()



% ***E NEED TO POSITION THIS WORK IN RELATION TO THE MARGINALIST ACCOUNT IN THIS SECTION BECAUSE IT IS ONE OF THE FEW THINGS YOU HAVE INTRODUCES. tHE URBAN MODELS YOU DESCRIBE HERE SHOULD BE FRAMED BY WHAT YOU ADD IN BACKGROUND ABOUT PLANNING/URBAN MODELS (AS PER MY SUGGESTION IN THAT SECTION) 
% ***E YOU ALSO NEED TO ADD A SUMMARIZING PARAGRAPH WHICH I CAN HELP WRITE WHEN THIS SECTION IS MORE FILLED OUT. MAYBE THIS:
% ***E ADD: By bringing these approaches into a coherent approach, we can better account for  WHAT... EXPLAIN THAT alone they don't give as complete a picture. This thesis show that certain things become clear when they are brought together that provide an better account of the current situation. Older economics models do not predict certain things that are currently happening. This is because they are based on out of date modes of production, fail to account for the importance of location value and the modelling tools available when they were developed required some simplification.   Updating the models for the changing times, using more complex modelling systems and incorporating space with the economic models allow us to create models and narratives that provide a more effective understanding of the situation as it stands today. 





% \begin{tikzpicture}{scale=.5}
% % find color cotrol for ball. Tind way to stop line short of node
% \coordinate (planning) at (-5,1);%PREFACE
% \coordinate (economics) at (5,.75);%
%  \coordinate (geography) at (-.5,-2); %history
% \coordinate (finance) at (0,5); %

% \draw [line width=2mm, black!15, ] (planning)--(economics);
% \draw [line width=2mm, black!20, ] (geography)--(economics);
% \draw [line width=2mm, black!20, ] (geography)--(planning);

% %\draw [line width=2mm, black!25, ] (geography)--(finance);
% %\draw [line width=2mm, black!20, ] (planning)--(finance);
% %\draw [line width=2mm, black!20, ] (finance)--(economics);

% \node [circle,shading=ball, minimum width=2.1   cm, white, align=center] (ball) at (planning) {Planning};
% \node [circle,shading=ball, minimum width=2.2cm, white, align=center] (ball) at (economics) {Economics};
% \node [circle,shading=ball, minimum width=3cm, white, align=center] (ball) at (geography)[text width=2cm] {\large Urban\\ Geography};

% %\node [circle, shading=ball, minimum width=2.4cm, white, align=center] (ball) at (finance)[text width=2cm] {Finance};
% \draw [line width=2mm, red!85, -latex ] (0, 7)--++(0,1.2)node[above=-.1] {\Large \textbf{FINANCE}};
% \draw [line width=2mm, red!85, -latex ] (0, 4.75)--++(0,1.2)node[above=-.1] {\Large \textbf{capitalization}};
% \draw [line width=2mm, red!85, -latex ] (0, 2.5)--++(0,1.2)node[above=-.1] {\Large \textbf{RENT}};
% \draw [line width=2mm, red!85, -latex ] (0, .25)--++(0,1.2)node[above=-.1] {\Large \textbf{value of location}};
% \node at (0,-.1) [red] {\Large \textbf{SPACE}};
% \end{tikzpicture}



% \vspace {2cm}
% Figure 4 with finance

% \begin{tikzpicture}{scale=.5}
% % find color cotrol for ball. Tind way to stop line short of node
% \coordinate (planning) at (-5,1);%PREFACE
% \coordinate (economics) at (5,.75);%
%  \coordinate (geography) at (-.5,-2); %history
% \coordinate (finance) at (0,5); %

% \draw [line width=2mm, black!15, ] (planning)--(economics);
% \draw [line width=2mm, black!20, ] (geography)--(economics);
% \draw [line width=2mm, black!20, ] (geography)--(planning);

% \node at (-.3,2) [red] {\huge \textbf{RENT}};

% \draw [line width=3mm,  black!50,opacity=.5 ] (geography)--(finance);
% \draw [line width=2mm, black!20, ] (planning)--(finance);
% \draw [line width=2mm, black!20, ] (finance)--(economics);

% \node [circle,shading=ball, minimum width=2.1   cm, white, align=center] (ball) at (planning) {Planning};
% \node [circle,shading=ball, minimum width=2.2cm, white, align=center] (ball) at (economics) {Economics};
% \node [circle,shading=ball, minimum width=3 . cm, white, align=center] (ball) at (geography)[text width=2cm] {\large Urban\\ Geography};

% \node [circle, shading=ball, minimum width=2.4cm, white, align=center] (ball) at (finance)[text width=2cm] {Finance};


% \end{tikzpicture}



% \vspace {2cm}
% Figure 4 with finance

% \begin{tikzpicture}{scale=.3}
% % find color cotrol for ball. Tind way to stop line short of node
% \coordinate (planning) at (-3,1.5);%PREFACE
% \coordinate (economics) at (5,.55);%
%  \coordinate (geography) at (-2.8,-2); %history
% \coordinate (finance) at (0,5); %

% \draw [line width=2mm, black!15, ] (planning)--(economics);
% \draw [line width=2mm, black!20, ] (geography)--(economics);
% \draw [line width=2mm, black!20, ] (geography)--(planning);

% \node at (.0,0) [red] {\huge \textbf{RENT}};

% \draw [line width=3mm,  black!50,opacity=.5 ] (geography)--(finance);
% \draw [line width=2mm, black!20, ] (planning)--(finance);
% \draw [line width=2mm, black!20, ] (finance)--(economics);

% \node [circle,shading=ball, minimum width=2.1   cm, white, align=center] (ball) at (planning) {Planning};
% \node [circle,shading=ball, minimum width=2.2cm, white, align=center] (ball) at (economics) {Economics};
% \node [circle,shading=ball, minimum width=3cm, white, align=center] (ball) at (geography)[text width=2cm] {\large Urban\\ Geography};

% \node [circle, shading=ball, minimum width=2.4cm, white, align=center] (ball) at (finance)[text width=2cm] {Finance};
% \draw [line width=4mm, red!85, -latex ] (0, .5)--(0,4);
% \end{tikzpicture}


\section{Modelling the financialization of an urban housing market}

%To capture urban productivity, 
% We introduce %rely on agglomeration effects, relating existing spatial and growth models to the scaling models from the study of complexity. 

% Productivity scales with density in cities, throughout history, this is a strikingly consistent result, stretching even into prehistory. And cities don't die, they are notoriously hard to move.
 %Geoffrey West has argued that ``Cities are the crucible of civilization, the hubs of innovation, the engines of wealth creation and centers of power, the magnets that attract creative individuals, and the stimulant for ideas, growth, and innovation.'' \cite{westScaleUniversalLaws2017} CHECK SOURCE. This urban aglomeration effect, scaling with density of networks and urban vibrancy, is % of course
 
 % MOVE THIS TO SPACE?
%Missing- transfer of money vs put in a spatial framework.
%Most of the economic theory talks about where people go, and it doesn't talk about the value they create in the city and where that goes. That's finacialization, capturing those benefits is what capitalists are doing now.
% rent is being in the house/what they pay, the transfer of money, vs what cities are, and how that produces value..

% Canada faces a housing crisis. The crisis now reaches far into the middle class, causing everything from declining home ownership rates to increasing poverty and displacement. 
%In the last few years, the need for affordable housing has come into focus as one of the most pressing issues facing Canadians. % ***E (ADD STRIKING STATISTIC - A NUMBER OR QUOTATION HERE.) 
% As more and more Canadians are finding 
%As more Canadians find housing unaffordable, the effects 
 % number of Canadians able to afford housing at all, leading to vacancies, poverty, and displaced workers. % (***E FILL THIS OUT A BIT - ADD SOME DETAILS HERE ABOUT THE EFFECTS OF THE CRISIS - THAT SET UP THE RESULTS OF THE THESIS)
% RATE OF INCREASE -VITAL SIGNS REPORT, CMHC
% There has been extensive debate about the drivers of the crisis. ADD FACTS. Proposed explanations include supply shortages, stagnating incomes, and the financialization of housing ownership. 
% (Centering on two dominant stories, a story of supply and demand and one of rights.) %FIX and cite

% There has been less work on the implications for productivity. 
% In this thesis we focus on the financialization of housing ownership and its impact on urban productivity in an economy driven by urban agglomeration effects. 

% The source of the rents captured and the broader social and economic effects have not been fully captured. 
 % This provides insight into both population and wealth distribution in cities. 
% Integrating financial markets into the the spatial urban model allows us to examine the effect of financialization on cities, specificaly on their role in growth, distribution, and productivity. % the growth and wealth distribution of cities, and more specifically on their productivity effects. 


% BUT THERE ARE OTHER THINGS MISSING - NO RENT, NO SPACE IN FINANCE, PRODUCTIVITY NOT RELATED TO SPACE OR LOCATION.


% Thus The productivity implications of the housing crisis are the focus of this thesis.

%(***E DEFINE TO SET UP THE DISTRIBUTIVE FEATURES OF ECONOMY). 
 % ***E ADD: the effects of housing affordability are pervasive / complex /run through the whole system / go far behind the obvious /direct effects / extend in non-obvious ways through the economy/whole system. What is at stake at a broader level is. 
% Yet, the economics is clear that what's at stake is the productivity of cities, the distributive features of the economy and the impact of the middle class % THIS IS A RESULT NOT AN INPUT. WHAT GOES HERE? ***E MAYBE ADD A CLARFIYING CONCLUDING SENTENCE HERE.... TO SAY SOMETHING LIKE THE EFFECTS OF HOUSING PRICES ARE NOT LIMITED TO 

% SAVE THIS ?? The greatest price increases have been in cities, where where people live and work, where  production is concentrated and where income is distributed. With humans becoming an increasingly an urban species, cities are a primary driver of technological development and increasing wealth. 

% (TIE BACK TO HOUSING CRISIS - EG. The affordability 
% \textbf{HOW WE ADD IT BACK IN}
%This thesis presents a spatial model of the city that incorporates distributional issues and financialization and allows us to examine the productivity implications of the housing crisis. The model that incorporates the scaling of productivity in cities within a standard urban model. 


%\textbf{WHY IT'S BEEN MISSING} EXPLAIN BETTER HERE WHY SPACE HAS BEEN LEFT OUT, AND WHY THAT LET'S US NEGLECTS SPATIAL RENTS AND MISS THE RELATION BETWEEN SPACE AND PRODUCTIVITY.
% We see urbanization and continuing and financialization accelerating. Financialization is driven by capital seeking profits, but what is the source of the rents they capture? The answer is in conventional urban theory, which allows us to identify the spatial distribution of those rents and traditional rent theory, which allows us to understand the social relations of those, those rents, the classical economists spend a great deal of time on that question. And we're very clear about it.
%We're talking about what is the puzzle? This is the teaser for this thesis and this thesis offers an answer to and I've just started to suggest that the teaser is given that 

% fig
% ***E I THINK THE PARAGRAPH ABOVE IS SAYING:  
% While financialization is usually understood as capital seeking prxfit, the source of the rents captured and the broader social and economic effects have not been fully captured.  XXXZ The current models for understanding financialization and it's effects don't predict the actual trends we are see. {\color{red} We argue this is because they miss the importance of space FIX.} 

%COMPELLING DESCRIPTION OF WHAT'S MISSING IN THE LITERATURE
%these share space, formalization in finance is spaceless



% Alternative phrasing 
%We integrate a labour market into a spatial urban model, set up to explore rent, and implications for the distribution of wealth.



 %calculate the urban surplus, and consider who gets it. 
 
% MOVE TO LIT REVIEW/CONTRIBUTION?
%: The work draws on the Alfonso/Von Thünen model of the concentric city and Dawn Parker and Filatova's work in agent-based modelling of housing markets (see http://jasss.soc.surrey.ac.uk/12/1/3.html 2009).% We begin with a simple model of a circular city with urban agglomeration effects. In subsequent sections we will use an agent-based model to relax assumptions to look at how the interaction between the production of social wealth in cities interacts with housing and the extraction of rent to drive patterns for individuals over space and time.
% Most of the analysis of urban systems has employed analytical models with roots that go back to von Thunen \cite{vonThunen} and more recently Alonzo \cite{ALONZO}. These models are extremely useful, but necessarily abstract from the concrete  and variable individual behaviour and  the details  of dynamics that make real cities path-dependent. XXX \cite{GET-Dawn} have shown that agent-based models can reproduce the features of the analytical models, at least in simple cases. TODO maybe divide chapter on theoretical core, followed by chapter on implementation (general for basic and resilience experiments).

\section{Contributions}

% STORY? We began with what seemed like an obvious statement, and looked for a model to do it, found the modelling tradition was blind to precisely the questions we were trying to explore, surprized, explain why?
% MOVE? The challenge is to develop a simple conceptually clear model that can be extended and used in a wide range so that the tools for thinking about ownership can be integrated and used in analysis of a wide range of ideas using conventional economic theory.
% \begin{enumerate}
%     \item we incorporate \gls{classical rent theory} into an agent-based urban model 
%     \item we allow the creation and distribution of rents to influence urban growth, productivity and  population structure. 
%     \item we incorporate current research on \gls{urban scaling} into the  core spatial urban model.   
%     \item we construct an   Urban \gls{ABM} that is consistent with \gls{neoclassical growth theory},
%     \item we integrate \gls{financial capital} into a standard spatial model of the urban system
%     \item we integrate financial capital into an population model of the urban system
%     \item we employ the ABM to examine how financial markets impact the urban land markets 
%     \item we test for \gls{hysteresis}  resulting from the business cycle  in the urban system 
%     \item we build a model that is easily extended to explore a wide range of issues
%     \item we provide a model that we believe can be used  to evaluate urban policies
% \end{enumerate}
% ADD TO EACH: THIS MEANS, THIS MATTERS BECAUSE, THEN FOLLOW WITH TO DO IT, WE NEED TO. IS THERE A PICTURE THAT COULD ILLUSTRATE THE CONTRIBUTIONS?
% CONSIDER MENTIONING WHAT ORDER THIS IS IN  % It's a kind of logical order, almost in the order of development: (types, order implemented, order theory is developed, order of importance?)
% methodological innovations% ROUGH LIST. TO SUMMARIZE AND SORT:
% Each of the items above requires us to  integrate models and concepts from different parts of the literature. 

% The core contribution is an exploration of the 


\section{Document overview}

%  Finally Part \ref{part-system} puts the analysis %model %and theory 
% in the context of a larger system, using methods of systems analysis and design, to discusses potential interventions and policy implications.
% \begin{enumerate}
%     \item Chapter \ref{chapter-interventions} considers the system, examines the potential for a range of interventions, and identifies policy implications. % with a diagram relating interventions.
%     \item TODO chapters .. may analyze particular interventions/cases in more detail: e.g. shared ownership models e.g. acquisition/land trusts, developer models, tax/zoning, and funds/financing.
% \end{enumerate}

% After we develop the mathematical description of the relationship among these will discuss in more detail, various relevant applications, and issues in the literature that draw on parts of this model and apply to the specific situation we're in why rent theory is related to discussions of exploitation why it might lead the inefficiencies, whether or not this links with other important models in the literature.

% Because we draw on a wide range of methods and literatures, we discuss the relevant literature and  methodologies in the chapters where they apply 


\chapter{COMBINE WITH RENT Background} \label{chapter-background}

% There are two dominant stories of \gls{distribution} in economics. The first and oldest is based on the classical concept of rent as explained  by David Ricardo \cite{ricardoEssayInfluenceLow1815}, in which owners of an asset are able to extract a value beyond what they contribute based on their ownership of a scarce resource. 
% The second is the marginalist approach, developed by Clark and others, in which workers and other factors  in competitive markets receive the \gls{marginal value-product} of their contribution to production. 

% Both developed in response to the specific social and economic conditions of the periods in which they emerged. Both attempt to explain where the output of society ended up within society. they are, at their heart, stories of who  claims what share of production.  Classical rent theory  emphasized the distribution of the social surplus, the part of production  over and above what was needed to reproduce society. This included only land rents initially, but was extended to the distribution of profit profits.   These are exactly the types of surplus income that are not explained by the neoclassical theory, which is  awkward because profits and rents form a substantial part of national income (20–25 percent) in the world the neoclassical model describes. In this thesis we identify classical rents intn the urban systems and exxamine their distribution.. 
% % ***E FIRSTNAME Ricardo was a INSERT NATIONALITY classical economist  in the WHICH Century. 

%Classical economics emerged in the 16$^{th}$ and 17$^{th}$ centuries at a time of exploding colonial wealth. In this period,  European economies were still structured largely around agricultural production in which farmers OR peasants work the land  owned by the descendants of feudal aristocrats. Urban economies based around factories were just beginning to emerge as significant drivers of wealth.\footnote{The Industrial Revolution is usually described as beginning around 1760 and having significantly transformed society by about 1820–1840.} % It was only after 1650 that living standards in the UK did start to increase for a sustained period. Before the modern era of economic growth, the economy worked very differently. Not technological progress, but the size of the population determined the standards of living.} 
In this economic context, the agricultural surplus is captured by feudal aristocrats. 

%As WHAT LED TO THE EXPLOSION... CHANGES IN UNIVERSITIES OR THE ECONOMY, there was an explosion of thinking about social and economic structures and systems.  This was a period where a range of thinkers explored many ideas to explain the shifting social and economic conditions. 

%Distribution, or who got what share of production was a central concern. This focus came to the forefront in the 1600s and 1700s as colonial expansion of land holdings, factories and supplied change concentrated wealth- funding investment in the arts and sciences, great fortunes - harsher conditions for farmers, pushed of their land with enclosures, and concentrated in urban factories, almost exclusively poor with - illness bad working conditions % 90 rural,-- population snap shot % (unprecedented in northern Europe, a relative backwater) --  %gt (new ineuqlity- jusxtabosition- fortuioned unpreciendet ont hat content to rivla hsotori great- wons- woth workers dying, kids in factories- Dickens period, pesantry.. - both were new- the eilsure- )
% and concerned with inequality, which suited a time in which the labourers both on the land, and in the emerging factories struggled with poverty, at a time of rapidly growing -
% -the intense interest spiked as wealth grew in colonial Europe. %(reformation another answer to this)
% NOTABLY THIS WAS AN URBAN TRANSITION- FARMS TO SUBURBS- POWER GROWTH, AN INFLECTION - LIKE COVID REVEALED THE NEW FORM WITH A RAPID MOBILITY CHANGE..
% The new form is the urban
% dense exchange - experience in person density can compete with screen experience -

%In this context, Ricardo refined the classical concept of rent.  

% ***E Today the word rent is usually used to refer to payments made by a tenant to a landlord to for the right to occupy a property, but this is not the traditional economic concept of  rent.  

% ***E As a concept, it's more closely related to profit / surplus than to rent paid for properties. 


% ***E NEED TO ADD HOW DOES THIS CONTRIBUTE OR EQUATE TO A THEORY OF DISTRIBUTION... WHAT DOES IT EXPLAIN...
%Ricardo's concept of rent accounted for the way that.... WHAT IT DESCRIBED. 
%This remains important to understanding distribution in general because.....


%The marginalist account describes how workers receive the marginal value of their contribution to production. THIS MEANS.....
%It formalizes what classical economist Adam Smith describes in the story of the pin factory. (EXPLAIN PIN FACTORY)

%The marginalist account of distribution gave a story of production that seemed to align with the rising fortunes of workers following WWII, in the 1950s and 1960s when it came to dominate. 

%This narrative dominates in neo-classical economics, particularly in the United States, and formed the basis of conventional micro-economics training. (EVEN NOW OR WHEN)
%It was also influential in shaping public discourse and policy in the early 20th century. For example, because it implies WHAT THAT MEANS MONOPOLY IS BAD... it provided the intellectual foundation for anti-monopoly political movements in the early 20th century.

\textbf{These two stories each reflect a distinct dominant mode of production}, period in society, and methodological set of tools available.From the Classical to Neo-Classical economic approaches, we also see development in the methodological approaches to describing economies. In Ricardo's era, economics was characterized by a free flowing descriptive approach characterized by wide debate, and many concepts explored. By the the time the marginalist approach emerged, new mathematical approaches were beginning to emerge OR dominate the economic discourse. 

SUMMARIZE THE DIFFERENCE IN WHAT EACH MODEL TELLS US ABOUT DISTRIBUTION. WHAT IS OVERALL DIFFERENT, WHAT DIFFERENT INSIGHTS EMERGE. 

\section{This Work}

\textbf{This thesis contributes to a third theory} % ***E IS THIS RIGHT?? I CHANGED IT TO THEORy OF DISTRIBUTION FROM THEORY OF MODERN URBAN RENT WHICH IS NOT THE CATEGORY WE ARE DISCUSSEING THEORIES FOR SO IT DOESNT MAKE SENES
of distribution by developing a theory of modern urban rent which integrates the classical descriptive work on rent and the neoclassical marginalist approach, with modern work on the scaling of wealth in cities. 

Just as the classical and neo-classical approaches \textbf{responded to and theorized the dominant mode of production in their time,} the theoretical framework emerging now, responds to the economic and social factors that are defining this period.  The early stories of production from thinkers like Ricardo centred on exploring who claimed the surplus from agricultural production Over time, the story moved to industrial production. Now the center of production is increasingly  WHAT KIND OF WORK (% ***E SAYING IT IS URBAN CHANGES THE SUBJECT TO LOCATION FROM TYPE OF WORK).
Industrialization shifted the centres of wealth to cities and since then the economic importance of cities has only grown. (%***E IS THIS SO?).  
% CUT : The social wealth of cities/human capacities developed in cities is central in new world where the production of value by people in cities 

NEED A CLEAR STATEMENT HERE ABOUT WHAT CITIES ARE ACTUALLY DOING. 
WHAT IS TANGIBLY HAPPENING IN CITIES THAT MAKES THEM SO IMPORTANT? WHAT KIND OF WORK? 
THIS IS THE PLACE TO SKETCH OUT A CLEAR AND RECOGNIZABLE PICTURE OF URBAN ECONOMIES IN THIS HISTORICAL MOMENT. ACCOUNT FOR TYPES OF WORK, COVID, ETC. 

Cities are the centres of social wealth and a wealth production. 
WHAT DOES WEALTH MEAN? 
WHAT DOES SOCIAL WEALTH MEAN?


A theory of production that doesn't center/involve the relation between cities and production/urban space and human capacity simply can't explain the creation of value in the modern world. It will miss WHAT SPECIFICALLY 

This is why is useful to draw in urban theory and spatial models (OR WHAT)

NOW SUMMARIZE IN BRIEF THE THREADS BEING BROUGHT TOGETHER. 
While econimists have focused on, URBAN THEORY has developed accoutns of WHAT????
WHAT ARE THE MAOR ACCOUNTS YOU ARE DRAWING ON? 
Meanwhile, while economists have focused on WHAT, Urban Theory has explored a different set of questions. 
WHAT DOES URBAN THEORY STUDY? WHAT KINDS OF QUESTIONS. 

This work weaves traditional economic approaches together with urban theory to contribute to a picture of distribution that accounts for how modern modern social and economic trends play out in space (FIX THIS SENTENCE)
ALSO ADD: 
key thinkers - is anyone else building the broader picture of this third story?? if not say something like "the key insights to this work have been developed separately in these different fields, this thesis pulls them into an singular account of distribution in the the current period. 


This approach also draws on the advancing\textbf{ methodological tools available}. The early classical work could reflect rich dynamic stories with many dimensions, using qualitative methods % ***E CAN YOU SAY QUALITATIVE METHODS??? I WANT TO EMPHASIZE THE DIFFERENCE IN METHODS MORE
to develop theories that responded to and referred to many individual stories. The neo-classical work was able to incorporate increasingly mathematical approaches% ***E (BE MORE SPECIFIC) MAYBE LIKE: THEY DID SOMETHING using the modelling systems available before large complex system modelling OR WHAT?? was possible.
Since then, computational capacity and approaches have progressed. This work uses ABM... COMPLEX SYSTEMS that enable us to work with large data sets and complex stories. % ***E BE SPECIFIC HERE ABOUT THE METHODS YOU USE) 
agent-based models and complex stories
-- big data sets, describe those concepts get back to..
cities - land wealth is the key stone
% #NEED TO FILL OUT HIS SUMMARY OF METHODS AND PROBABLY ADD A CONCLUDING PARAGRAPH THAT SUMMARIZES THE SHiFTS DESCRIBED ABOVE AND WHY THIS WorK IS IMPORTANT. CAN REWORK THE FOLLOWING:
Changes in social and economic structures are explains by different models and emerging methods that allow us to bring the complexity that could previously only be accounted for using narrative and qualitative approaches into math-based models. 
lassical Accountsfirst mostly farmers, very poor workers
then marginalist  'gave a story of production that seemed to align with the rising fortunes of workers following WWII, in the 1950s and 1960s when it came to dominate. - peoples fortunes were increasing, the new emerging production seemed central.
now inequality has risen sharply, it seems there is a need for a model that explains what has changed. This thesis contribute to developing that new understanding of how distribution functions in the world as it is today. 

% E CUT OR MOVE ALL THIS, I THINK: 
% They reflect

% WHAT IS THIS - what fits...
% neoclassical tradition emerged with high ..
% - elauition -- pro- vastly àmaterial realism. etc..

% we uses as a common threat to put these in the same language of formal function functions, tracing from early models of cobb doulgas fucntion

% we then tell the history..  
% descriptive, analytic and embedded in a complex syste

% The complexity - allows for tracing the paths of individual- what happens for whom under a far broader range of conditions

The clarity of pedagogical models- bottom up and top up both have illustrative cases e.g. edge worth box or the Schelling's/birds models.
But true theory integrates in something that moves between scales fluidly, makes it possible for the distinct scale based approaches to come together.


\section{CUT/EDIT Very rough notes on the three stories of distribution}

ABOVE WE DISCUSS THE CHANGE IN THE PRODUCTION SYSTEM, THE THREE STORIES ALSO REFLECTS METHODOLOGICAL EVOLUTION, AND THE POLITICAL AND SOCIAL CONTEXT.

They are also %These stories are, at their heart, 
stories of who claims what share of production. 

These stories evolved with changes in the system of production.
%They evolved within an evolving theory of production. 
The early stories of production thinkers like Ricardo focuses on were agricultural. Who claimed the surplus from agricultural production? Over time, the story moved to industrial production.
Now the center of production is increasingly urban- with the social wealth of cities/human capacities developed in cities dominating. 
new world where the production of value by people in cities is the center..
A theory of production that doesn't center/involve the relation between cities and production/urban space and human capacity can't explain the creation of value in the modern world. % Later thinkers including Smith and Marx%leaving aside purely inherited weath- as that becaumse caught in this same circuit of capital transforming from production, to money and back. 

They reflect
first mostly farmers, very poor workers
then marginalist  'gave a story of production that seemed to align with the rising fortunes of workers following WWII, in the 1950s and 1960s when it came to dominate. - people's fortunes were rising, the new emerging production seemed central.
now inequality has risen sharply, it seems there is a need for a model that explains what has changed.




*** METHOD ***
ADDING SOME ROUGH NOTES ON THE EVOLUTION OF METHOD AND POLITICAL CONTEXT FOR EACH OF THESE 3 THREADS (CLASSICAL, NEOCLASSICAL, AND WHATEVER WERE DOING WITH THE URBAN)..

A third thread is the advancing methodological tools available.
The early classical work could reflect rich dynamic stories with many dimensions - individual stories.

The stories also follow the methodological development of the discipline from the descriptive work of the classic descriptive work. ..
to central complex systems, large data sets
Methodological-- early descriptive theories told rich layered stories with different
The excitement concentrated  calculus.. in the classical distributional dynamic.

cutting edge technique focused on formalized.. 


The complexity of agent base models - allows for tracing the paths of individual- what happens for whom under a far broader range of conditions



*** SOCIAL AND POLITICAL CONTEXT ***

Early descriptive work
This explosion of formal rigour - focused attention.. 
And the political context..


Monopoly- political pressure real explosion of wealth creation-- economic success of political efforts to break up monopolies.
And a dynamic- lots of worker power- expanded equality-- workers seemed strong, 
As well as the political environment in the US during the cold war, older stories rooted- marx- repression, economists perhaps created an environment in which economists
a side of the economics

-- mythological center moved to the point where descriptive and historical approaches barely taught.

Samuelson-- successful exciting-- formal-
a generation
created micro, macro
-- at the moment of the baby boom- departments founded in this moment of exuberance. raised in it, taught according to this framework.
Polarized in the periodo of the cold war - the discussion of the market-- perhaps a tendency to avoid the distributionl.. Revolution and drama.

Computers took over from calculus -Brian Arthur
Cities took over from industries - concentrated value-- finance- and law main power centers.. - eigenvalue centrality.

Crisis in 2008 -- reintroduced descriptive methods and and an openness to new formal methods including computational and agent modelling beyond the 
Methodologial

Cities-- power law dist. rising debt and inequality. -- unstable and financialized
Increasing inequality, rising debt. - worker power expanding wages and equality, a story that explained- vs subsistence.


Exactly what those pattnersnew methods are so succesful are what was lef tout..

---


With Clark
A second great theory of distribution
The result is much of the theory of rent was lost. 
time

While Marx emphasized the tendency towards consolidation and exploitation in markets, Clark saw the tendency to increased competition. 

This allocation— dynamic quality of how wages evolved
They are bidding- and it will converge 
What share do workers get- subsistence wages- get 
But as output grows, and as firms compete for labour, particularly skilled labour, is that a sufficient experience.



Three drivers
Calculus had limits.
The political moment of expanding wages with a labour sector in a position to negotiate as the economy rebuilt following WWII and destruction of old wealth— dynamic time. 
Following WWII with growing demand for labour labour could bargain, 
Following WWII in the period— subsistence waves tending— when labour could bargain,
Following break up some of the largest monopolies like in steak— general steal


Also coincided with the political movement McCarthiesm perhaps led scholars to de-emphasize the connections of their work with the classical socialist literature.
Mathematical economics became an exciting and dynamic area.

Until this point the theory was largely descriptive..
xx Cobb working with Douglass developed a formulation — exponential, in economics their names have remained associated with the xyz formulation. 

Clark made a case it was just- became problematic.

OUR CASE IS THAT IT FALLS OFF A TOTALY DIFFERENT CLIFF



Economics had theories with rich dynamics, concerned 
Classical economics was concerted with ownership and wealth. But they were largely descriptive.
But the new calculus struggled to deal with stocks and with dynamics. 
(Came back with forester and other systems theory, as well as complexity etc.)

The French Engineers in the school of bridge end road used calculus early .. followed by xyz
Technical development and intellectual excitement aligned
Became very exciting dynamic, had many success - took over the discipline. 
Tied with political successes breaking up big monopolies — seemed to offer a path forward

US opposed soviet ideas and an intellectual environment that may have led academics to dephasize the aspects of their thinking connected with classical socialists thought. 

In this environment a particular approach became dominat— also at a moment when schools and departments were growing— the baby boom came to universities at the moment of Samuelson’s peace micro-macro divide gave a tool kit to a whole generation of economists— 

Embedded at the heart of micro- the satisfyingly precise formal structure of calculus.. the marginalize appraoche— 
Thus came to define a new disciplien— a formalization of Econ.. 
Extensions from that base became the defining advances of a generation of American economists..
Attracted math- a feedback loop.

Less emphasis on intellectual history, how changing- heterodox.. all the full range of thought

Including the much more exiting new techniques of complexity and systems- opening in 2007 an explosion of these techniques in the economies. 


CITIES

But cities matter more and more
Jacobs theory of wealth and value as fundamentally social.. 
Combined with xysz. Jacobs did

Now complexity and scaling theory revealing the universality of those principles advanced by Jacobs..

This requires a different formulation of rent… - and production wealth is inherently social what are the implicaitons— what does that mean.. 



In our model, land comes in implicitly through the demand for labour. 



