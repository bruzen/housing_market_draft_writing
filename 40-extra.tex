\chapter{Gather for INTRO}


\cite{spenceUrbanizationGrowth2009} 
curve cost curve and savings rate.. huge chunk missing..
early papers in this book, also very good


This study is an exploration of the impact of financialization of the housing market, and what that means for urban productivity.

INTRO - SUMMARIZE 
This matters because, % thinking about housing markets and what was going on in Ontario, and around the world,
cities are going through a great transformation of ownership structures. 

link to productivity missing


\section{NEXT cities are central to productivity, housing ownership is central to wealth distribution, what's at stake is the future of the middle class and the productivity/success of cities}
- formal tools for analyzing resilience an interventions

- theire dynamics static tools are not enough. 

"Urbanization and growth go together: no country has ever reached middle income status without a significant population shift into cities." \cite{annezUrbanizationGrowthSetting2009} % (see book in \cite{spenceUrbanizationGrowth2009})

\subsection{Cities produce benefit and are central to productivity}
Cities produce enormous benefits and are central to productivity, they're also undergoing a massive change

the ownership tenure pumps wealth out'
however rich they get people are pushed to a kind of subsistence frontier. 
suggests also simple approaches to building of social wealth anchored in place. -- clear suggestions for productive interventions- clear policy implications. 
%Most of the economic theory talks about where people go, and it doesn't talk about the value they create in the city and where that goes. That's finacialization, capturing those benefits is what capitalists are doing now.

What does this mean - given the particular structure of urban wealth creation.

Who claims those benefits, and what does that mean for productivity and the future of cities?

Our analysis suggests  it actually has  implications for the success of cities. % and of socieities, the future of a middle class, and whether citiec can continue to generate value. 


Cities are essential - draw on bettencourt/west's case,
Geoffrey West has argued that ``Cities are the crucible of civilization, the hubs of innovation, the engines of wealth creation and centers of power, the magnets that attract creative individuals, and the stimulant for ideas, growth, and innovation.'' \cite{westScaleUniversalLaws2017} 
Jane Jacobs talks about the vibrancy of cities, and their central role as the source of productivity. 
 \cite{jacobsEconomyCities1969}.
 
 Cities are particularly central to productivity.
 The scaling of wealth in cities has strong theoretical and empirical support  \cite{bettencourtGrowthInnovationScaling2007, bettencourtOriginsScalingCities2013, dongUnderstandingMesoscopicScaling2020, loboUrbanScalingProduction2013} 


in particular they are key to productivity. Wealth scales superlinearly with density - every increase in productivity, more than increases the productivity of the place. 
The relationship goes back into prehistory, not tied to any approach to government or anything else, from smallest nomadic communities, to largest empires, city states, and modern metropolises.

they live forever - so unlike individuals, don't move, don't die, in the same location as in ancient cities. Tokyo grow back. 


part of the story is what benefits they create
the second is who gets it

Most of the economic theory talks about where people go, and it doesn't talk about the value they create in the city and where that goes. That's finacialization, capturing those benefits is what capitalists are doing now.

% can actually reduce the incentive to work- run the thing on debt.. not on profit..

% Canada faces a housing crisis.

A defining of this transition
beginning 

housing ownership has been the primary way 
under theorized why this happens - why is it that those with houses build wealth and those without don't - net wealth largely negative, not always but often

key to the racial wage gap - getting debt/resources for to acquire assets and skills at key moments - in particular a house and an education the central
key to how newcomers face systemic disadvantage
to how young people build or fail to build wealth




\subsection{Cities are undergoing a massive change in ownership structure}
% There's price changes, but there's this deeper story of the transformation of the ownership structures. 
 % I realized that this idea of financialization of the housing market that seemed to be gaining a certain amount of currency when I started to think about it had some

Cities matter and are central to productivity, but one of the big changes happening now is they are undergoing a massive change in ownership structure. 

CALIFORMA AFTER CMC... 
70 percent in ontario 
**NOTE FOR POLICY - CMHC LEANING INTO THIS- SUPPORTING BULT FOR RENTAL PROPERTIES. 

DEEP TRANSFORMATION OF OWNERSHIP STRUCTURE
what is the implication of it
- suburbs brought

- separate from development - owning suburban homes.. 
- this distinct rent profile.


CHANGE IN OPWNERSHIP STRUCTURE MATTERS- 
MAIN WAY PEOPLE BUILD WEALTH AND THE KEY TO THE MIDDLE CLASS
AND THE BOUNDARY -- THOSE OUTSEIDE HAVEN'T BUILT ANY WEALTH..

\subsection{The research gap}

WE STARTED WITH A SELF EVIDENT STAMENT

 but we don't have the tools to understand the implications of integrate it into formal theory.
 
 There's lots of work on causes.
 less on productivity 
 this is about making the link with productivity--

RESEARCH GAP..


yet, looking at formal models of the economy - missing

a kind of a mystery


\textbf{made what seemed like self evident statements given emerging analysis of what's happened WHAT WAS THE STATEMENT} LILIANA ETC.

i make models, i tried to find a model to explain formally what had happened
but it was like a blank spot. 

it was invisible. 

it was \textbf{kind of a mystery}, it seemed like such a simple thing, and yet, 
urban models of spatial 


\textbf{it turned out the actual categories we were talking about were the kind of things the formal model} are blind to
-parts there, can't talk because of how they're structure --- go deeply into the traditions to find the pieces that could be brough together

-add back rent
-adding space to finance/financialization.
..


\gls{middle class}
..Emergence..
class consistent with Roemoer.. 
to solve it ended up going deeply into theory

ACTUALLY RESEARCH GAP IS 2 FOLD
FOUND WE COULDN'T MODEL THIS LINK WITH PRODUCTIVITY.


speaks to how branches of econ don't talk with each other. classical economics - concerned with distribution.. models of finance spaceless, models of productivity without both, but with theses powerful tools to understand how communities build wealth.


RESEARCH GAP

The second sub-model incorporates  productivity of cities. Geoffrey West has argued that ``Cities are the crucible of civilization, the hubs of innovation, the engines of wealth creation and centers of power, the magnets that attract creative individuals, and the stimulant for ideas, growth, and innovation.'' \cite{westScaleUniversalLaws2017} 


There is a housing crisis ADD FACTS ABOUT WHAT WE MEAN BY THAT.



WHAT WE GET TO IS A MODEL THAT AIMS TO BE AS SIMPLE AS THE ALONZO MODEL, BUT LINKS THESE 3 BRANCHES OF THE STUDY OF THE ECONOMY AND SPACE TOGETHER

"The main goal of this research as been to add a financial system to the simple Alonzo model in order to examine the impact of financial cycles on urban productivity and distribution. The \gls{agent-based model} we develop has three major components: 

THAT IS EASILY EXTENSIBLE, CAN BE GROUNDED EMPIRICALLY, AND USED FOR POLICY MAKERS AND SYSTEMS CHANGE PEOPLE TO EXPLORE OTHER POSSIBILITIES.


While we leave much of the analysis



Resilience is also largely outside economic modelling
so we do one set of extensions showing a resilience/hysteresis analsyis. how this system of ownership pumps wealth out of communities on the up and down swing, a glaring gap in econ other

In addition to this domain contribution to understanding the implications for cities and urban productivity of shifts in tenure. 
systems design person.

There's also a methodological story 
ABMs in line with the frontier- what would happen - simplicity and clarity - these frontiers.. like sectional diagrams in engineering. 

and then the extension to hysteresis

in an age of endemic modelling, making these into little toys with sliders, the user interface of the model, a prototype

and the development of models that are interoperable, library elements and extensible. 


Most centrally the resilience dynamics.
the making of regimes which is what makes freedom possible in this kind of system. 
Methodological reflections on each stage of the development. 




\subsection{What we need to do}
% That sounds more simple than it is because to 
to achieve that, we need actually to 

1. build a model housing market for a city. 

2. And we have to, therefore have an urban model. 

3. And we have to figure out what it is that's at stake in the financialization which turns out to be land rents. And that means the city has to bring together essentially classical land rent theory, modern urban theory and the economics and bidding processes in a housing market. THIS IS THE MAIN PIECE, DO WE  NEED ALL IN THIS LIST?

4. And then it has to keep track of some macro variables that are significant for policy. 


\chapter{Cut from Introduction}

% MORE IS OWNED BY FINANCIAL....
% One of the great transformations, is that financial capital owns a great deal more property. % xx in california, xx in Ontario, 70\% of new builds in Waterloo region. % Check
% In the next section, we describe the core model, then in following sections sketch the position this in the literature, describe the core contributions, and then outline how the document will develop these ideas.

\section{The financialization of housing markets}
% This thesis is thus a contribution to developing a modern urban theory of rent.

%The effects of financialization on cities and economies has not been fully accounted for because the tools of the different, relevant disciplines have not been adequately integrated. The current approaches to describing the financialization of housing and its effects predict / explain /account for the housing crisis and effects on home ownership and access to housing, but our work shows that there are broader effects that have not been accounted for / predicted. 
% This fuller picture is made possible by bringing together classical rent theory, neo-classical XX and urban theory to create/and using/along with a agent-based model that allows us to .... (What the modelling technique enables) 
%This work is important for understanding the current policy context. 

\section{Position in the literature}

% We further link the model of urban rents to emerging concerns about the financialization of the housing market. 

%*** (another sentence on why this is great) --The space-less quality of the study of finance leaves out xyz GET PHRASING- CAN'T SEE- INTEGRATION OF SPACE NEGLECT GROWTH FACTOR. 
% Integrating classical and neoclassical economic approaches with standard urban theory,  allows us to identify 
% ..we can build a more comprehensive model of financialization and its effects. This makes it possible to trace the spatial distribution of the rents.

% Rent is important is all these traditions, but has been neglected in modern theory. We argue that to formally understand the processes behind financialization and the housing crisis, what's needed is a modern theory of urban rent. This thesis is a contribution to the development of that theory. REDUNDANCY?

% The focus of this thesis is a topic that falls in the overlap between three academic  disciplines, Economics, Urban Geography, and Planning. % ***E mAYBE SUMMARIZE THE FOCUS OF EACH? While Economics traditionally focuses on... Urban Geography looks at ..... and Planning is the study of....

 % ***E DO YOU MEAN GEOGRAPHIC SPACE AND PEOPLE SOMEHOW? i FEEL LIKE YOU DON'T MEAN THIS? MAYBE MORE LIKE: The central shared concern between the three discipline it the role of geographic space in shaping WHAT? social and economic systems? human systems? society??
% \begin{figure}
% \begin{tikzpicture}{scale=.5}
% % find color cotrol for ball. Tind way to stop line short of node
% \coordinate (planning) at (-5,1);%PREFACE
% \coordinate (economics) at (5,.75);%
%  \coordinate (geography) at (-.5,-2); %history
% \coordinate (finance) at (0,5); %

% \draw [line width=2mm, black!15, ] (planning)--(economics);
% \draw [line width=2mm, black!20, ] (geography)--(economics);
% \draw [line width=2mm, black!20, ] (geography)--(planning);

% %\draw [line width=2mm, black!25, ] (geography)--(finance);
% %\draw [line width=2mm, black!20, ] (planning)--(finance);
% %\draw [line width=2mm, black!20, ] (finance)--(economics);
% \node [circle,shading=ball, minimum width=2.1   cm, white, align=center] (ball) at (planning) {Planning};
% \node [circle,shading=ball, minimum width=2.2cm, white, align=center] (ball) at (economics) {Economics};
% \node [circle,shading=ball, minimum width=3cm, white, align=center] (ball) at (geography)[text width=2cm] {\large Urban\\ Geography};
% %\node [circle, shading=ball, minimum width=2.4cm, white, align=center] (ball) at (finance)[text width=2cm] {Finance};

% \node at (-.3,-.1) [red] {\Large \textbf{Space}};
% \end{tikzpicture}
% \caption{The common concern of three fields topic }
%     \label{fig-three-fields}
% \end{figure}

%This thesis relates to four major fields: classical rent theory, neoclassical production theory and growth theory, the scaling literature, and urban spatial models.
% economics and the study of cities. 
% Our model of the urban economy is based on work from those developed in 

%***E ADD:This thesis describes a model that draws together LIST HERE. 
% ***E OR list all??? ie. From Classical Rent to Neo-Classical Marginalist theory to XYZ urban theory and ...... etc. SET UP THE WHOLE SECTION.
%E ADD: The concept of rent illustrates WHAT in relation ot distribution, allows what kinds fo insights, which sets up this work.  We use the concept of rent to frame our consideration of how wealth is distributed in society by looking at WHAT?? how surplus is distributed? how locational value and location OR WHAT?? affects surplus distribution?? I dont know but please summarize how rent is used in this thesis. 

% Nearly contemporaneous thinker, Johann Heinrich  von Th\"unen (1783-1850) developed a planning model to guide the location of economic activities for an urban-agricultural society.  A version of that model  was reinvented in urban geography by XXX. Alonzo\footnote{We use a version of the well-established model of Alonso (1964), Muth (1969) and Mills (1967), and formalised by Wheaton (1974),} % ***E NEED MORE DETAIL HERE ABOUT ALONZO"S MODEL, 

% We link the Alonzo model with more recent work on growth theory starting with Robert Solo's XXX and with the endogenous growth models of Lucas () and draw on Jane Jacobs's insight that endogenous urban growth  is. now driving economic development. Jacobs's insight is empirically supported by recent work in the complexity literature on urban scaling by XXXX ()



% ***E NEED TO POSITION THIS WORK IN RELATION TO THE MARGINALIST ACCOUNT IN THIS SECTION BECAUSE IT IS ONE OF THE FEW THINGS YOU HAVE INTRODUCES. tHE URBAN MODELS YOU DESCRIBE HERE SHOULD BE FRAMED BY WHAT YOU ADD IN BACKGROUND ABOUT PLANNING/URBAN MODELS (AS PER MY SUGGESTION IN THAT SECTION) 
% ***E YOU ALSO NEED TO ADD A SUMMARIZING PARAGRAPH WHICH I CAN HELP WRITE WHEN THIS SECTION IS MORE FILLED OUT. MAYBE THIS:
% ***E ADD: By bringing these approaches into a coherent approach, we can better account for  WHAT... EXPLAIN THAT alone they don't give as complete a picture. This thesis show that certain things become clear when they are brought together that provide an better account of the current situation. Older economics models do not predict certain things that are currently happening. This is because they are based on out of date modes of production, fail to account for the importance of location value and the modelling tools available when they were developed required some simplification.   Updating the models for the changing times, using more complex modelling systems and incorporating space with the economic models allow us to create models and narratives that provide a more effective understanding of the situation as it stands today. 





% \begin{tikzpicture}{scale=.5}
% % find color cotrol for ball. Tind way to stop line short of node
% \coordinate (planning) at (-5,1);%PREFACE
% \coordinate (economics) at (5,.75);%
%  \coordinate (geography) at (-.5,-2); %history
% \coordinate (finance) at (0,5); %

% \draw [line width=2mm, black!15, ] (planning)--(economics);
% \draw [line width=2mm, black!20, ] (geography)--(economics);
% \draw [line width=2mm, black!20, ] (geography)--(planning);

% %\draw [line width=2mm, black!25, ] (geography)--(finance);
% %\draw [line width=2mm, black!20, ] (planning)--(finance);
% %\draw [line width=2mm, black!20, ] (finance)--(economics);

% \node [circle,shading=ball, minimum width=2.1   cm, white, align=center] (ball) at (planning) {Planning};
% \node [circle,shading=ball, minimum width=2.2cm, white, align=center] (ball) at (economics) {Economics};
% \node [circle,shading=ball, minimum width=3cm, white, align=center] (ball) at (geography)[text width=2cm] {\large Urban\\ Geography};

% %\node [circle, shading=ball, minimum width=2.4cm, white, align=center] (ball) at (finance)[text width=2cm] {Finance};
% \draw [line width=2mm, red!85, -latex ] (0, 7)--++(0,1.2)node[above=-.1] {\Large \textbf{FINANCE}};
% \draw [line width=2mm, red!85, -latex ] (0, 4.75)--++(0,1.2)node[above=-.1] {\Large \textbf{capitalization}};
% \draw [line width=2mm, red!85, -latex ] (0, 2.5)--++(0,1.2)node[above=-.1] {\Large \textbf{RENT}};
% \draw [line width=2mm, red!85, -latex ] (0, .25)--++(0,1.2)node[above=-.1] {\Large \textbf{value of location}};
% \node at (0,-.1) [red] {\Large \textbf{SPACE}};
% \end{tikzpicture}



% \vspace {2cm}
% Figure 4 with finance

% \begin{tikzpicture}{scale=.5}
% % find color cotrol for ball. Tind way to stop line short of node
% \coordinate (planning) at (-5,1);%PREFACE
% \coordinate (economics) at (5,.75);%
%  \coordinate (geography) at (-.5,-2); %history
% \coordinate (finance) at (0,5); %

% \draw [line width=2mm, black!15, ] (planning)--(economics);
% \draw [line width=2mm, black!20, ] (geography)--(economics);
% \draw [line width=2mm, black!20, ] (geography)--(planning);

% \node at (-.3,2) [red] {\huge \textbf{RENT}};

% \draw [line width=3mm,  black!50,opacity=.5 ] (geography)--(finance);
% \draw [line width=2mm, black!20, ] (planning)--(finance);
% \draw [line width=2mm, black!20, ] (finance)--(economics);

% \node [circle,shading=ball, minimum width=2.1   cm, white, align=center] (ball) at (planning) {Planning};
% \node [circle,shading=ball, minimum width=2.2cm, white, align=center] (ball) at (economics) {Economics};
% \node [circle,shading=ball, minimum width=3 . cm, white, align=center] (ball) at (geography)[text width=2cm] {\large Urban\\ Geography};

% \node [circle, shading=ball, minimum width=2.4cm, white, align=center] (ball) at (finance)[text width=2cm] {Finance};


% \end{tikzpicture}



% \vspace {2cm}
% Figure 4 with finance

% \begin{tikzpicture}{scale=.3}
% % find color cotrol for ball. Tind way to stop line short of node
% \coordinate (planning) at (-3,1.5);%PREFACE
% \coordinate (economics) at (5,.55);%
%  \coordinate (geography) at (-2.8,-2); %history
% \coordinate (finance) at (0,5); %

% \draw [line width=2mm, black!15, ] (planning)--(economics);
% \draw [line width=2mm, black!20, ] (geography)--(economics);
% \draw [line width=2mm, black!20, ] (geography)--(planning);

% \node at (.0,0) [red] {\huge \textbf{RENT}};

% \draw [line width=3mm,  black!50,opacity=.5 ] (geography)--(finance);
% \draw [line width=2mm, black!20, ] (planning)--(finance);
% \draw [line width=2mm, black!20, ] (finance)--(economics);

% \node [circle,shading=ball, minimum width=2.1   cm, white, align=center] (ball) at (planning) {Planning};
% \node [circle,shading=ball, minimum width=2.2cm, white, align=center] (ball) at (economics) {Economics};
% \node [circle,shading=ball, minimum width=3cm, white, align=center] (ball) at (geography)[text width=2cm] {\large Urban\\ Geography};

% \node [circle, shading=ball, minimum width=2.4cm, white, align=center] (ball) at (finance)[text width=2cm] {Finance};
% \draw [line width=4mm, red!85, -latex ] (0, .5)--(0,4);
% \end{tikzpicture}


\section{Modelling the financialization of an urban housing market}

%To capture urban productivity, 
% We introduce %rely on agglomeration effects, relating existing spatial and growth models to the scaling models from the study of complexity. 

% Productivity scales with density in cities, throughout history, this is a strikingly consistent result, stretching even into prehistory. And cities don't die, they are notoriously hard to move.
 %Geoffrey West has argued that ``Cities are the crucible of civilization, the hubs of innovation, the engines of wealth creation and centers of power, the magnets that attract creative individuals, and the stimulant for ideas, growth, and innovation.'' \cite{westScaleUniversalLaws2017} CHECK SOURCE. This urban aglomeration effect, scaling with density of networks and urban vibrancy, is % of course
 
 % MOVE THIS TO SPACE?
%Missing- transfer of money vs put in a spatial framework.
%Most of the economic theory talks about where people go, and it doesn't talk about the value they create in the city and where that goes. That's finacialization, capturing those benefits is what capitalists are doing now.
% rent is being in the house/what they pay, the transfer of money, vs what cities are, and how that produces value..

% Canada faces a housing crisis. The crisis now reaches far into the middle class, causing everything from declining home ownership rates to increasing poverty and displacement. 
%In the last few years, the need for affordable housing has come into focus as one of the most pressing issues facing Canadians. % ***E (ADD STRIKING STATISTIC - A NUMBER OR QUOTATION HERE.) 
% As more and more Canadians are finding 
%As more Canadians find housing unaffordable, the effects 
 % number of Canadians able to afford housing at all, leading to vacancies, poverty, and displaced workers. % (***E FILL THIS OUT A BIT - ADD SOME DETAILS HERE ABOUT THE EFFECTS OF THE CRISIS - THAT SET UP THE RESULTS OF THE THESIS)
% RATE OF INCREASE -VITAL SIGNS REPORT, CMHC
% There has been extensive debate about the drivers of the crisis. ADD FACTS. Proposed explanations include supply shortages, stagnating incomes, and the financialization of housing ownership. 
% (Centering on two dominant stories, a story of supply and demand and one of rights.) %FIX and cite

% There has been less work on the implications for productivity. 
% In this thesis we focus on the financialization of housing ownership and its impact on urban productivity in an economy driven by urban agglomeration effects. 

% The source of the rents captured and the broader social and economic effects have not been fully captured. 
 % This provides insight into both population and wealth distribution in cities. 
% Integrating financial markets into the the spatial urban model allows us to examine the effect of financialization on cities, specificaly on their role in growth, distribution, and productivity. % the growth and wealth distribution of cities, and more specifically on their productivity effects. 


% BUT THERE ARE OTHER THINGS MISSING - NO RENT, NO SPACE IN FINANCE, PRODUCTIVITY NOT RELATED TO SPACE OR LOCATION.


% Thus The productivity implications of the housing crisis are the focus of this thesis.

%(***E DEFINE TO SET UP THE DISTRIBUTIVE FEATURES OF ECONOMY). 
 % ***E ADD: the effects of housing affordability are pervasive / complex /run through the whole system / go far behind the obvious /direct effects / extend in non-obvious ways through the economy/whole system. What is at stake at a broader level is. 
% Yet, the economics is clear that what's at stake is the productivity of cities, the distributive features of the economy and the impact of the middle class % THIS IS A RESULT NOT AN INPUT. WHAT GOES HERE? ***E MAYBE ADD A CLARFIYING CONCLUDING SENTENCE HERE.... TO SAY SOMETHING LIKE THE EFFECTS OF HOUSING PRICES ARE NOT LIMITED TO 

% SAVE THIS ?? The greatest price increases have been in cities, where where people live and work, where  production is concentrated and where income is distributed. With humans becoming an increasingly an urban species, cities are a primary driver of technological development and increasing wealth. 

% (TIE BACK TO HOUSING CRISIS - EG. The affordability 
% \textbf{HOW WE ADD IT BACK IN}
%This thesis presents a spatial model of the city that incorporates distributional issues and financialization and allows us to examine the productivity implications of the housing crisis. The model that incorporates the scaling of productivity in cities within a standard urban model. 


%\textbf{WHY IT'S BEEN MISSING} EXPLAIN BETTER HERE WHY SPACE HAS BEEN LEFT OUT, AND WHY THAT LET'S US NEGLECTS SPATIAL RENTS AND MISS THE RELATION BETWEEN SPACE AND PRODUCTIVITY.
% We see urbanization and continuing and financialization accelerating. Financialization is driven by capital seeking profits, but what is the source of the rents they capture? The answer is in conventional urban theory, which allows us to identify the spatial distribution of those rents and traditional rent theory, which allows us to understand the social relations of those, those rents, the classical economists spend a great deal of time on that question. And we're very clear about it.
%We're talking about what is the puzzle? This is the teaser for this thesis and this thesis offers an answer to and I've just started to suggest that the teaser is given that 

% fig
% ***E I THINK THE PARAGRAPH ABOVE IS SAYING:  
% While financialization is usually understood as capital seeking prxfit, the source of the rents captured and the broader social and economic effects have not been fully captured.  XXXZ The current models for understanding financialization and it's effects don't predict the actual trends we are see. {\color{red} We argue this is because they miss the importance of space FIX.} 

%COMPELLING DESCRIPTION OF WHAT'S MISSING IN THE LITERATURE
%these share space, formalization in finance is spaceless



% Alternative phrasing 
%We integrate a labour market into a spatial urban model, set up to explore rent, and implications for the distribution of wealth.



 %calculate the urban surplus, and consider who gets it. 
 
% MOVE TO LIT REVIEW/CONTRIBUTION?
%: The work draws on the Alfonso/Von Thünen model of the concentric city and Dawn Parker and Filatova's work in agent-based modelling of housing markets (see http://jasss.soc.surrey.ac.uk/12/1/3.html 2009).% We begin with a simple model of a circular city with urban agglomeration effects. In subsequent sections we will use an agent-based model to relax assumptions to look at how the interaction between the production of social wealth in cities interacts with housing and the extraction of rent to drive patterns for individuals over space and time.
% Most of the analysis of urban systems has employed analytical models with roots that go back to von Thunen \cite{vonThunen} and more recently Alonzo \cite{ALONZO}. These models are extremely useful, but necessarily abstract from the concrete  and variable individual behaviour and  the details  of dynamics that make real cities path-dependent. XXX \cite{GET-Dawn} have shown that agent-based models can reproduce the features of the analytical models, at least in simple cases. TODO maybe divide chapter on theoretical core, followed by chapter on implementation (general for basic and resilience experiments).

\section{Contributions}

% STORY? We began with what seemed like an obvious statement, and looked for a model to do it, found the modelling tradition was blind to precisely the questions we were trying to explore, surprized, explain why?
% MOVE? The challenge is to develop a simple conceptually clear model that can be extended and used in a wide range so that the tools for thinking about ownership can be integrated and used in analysis of a wide range of ideas using conventional economic theory.
% \begin{enumerate}
%     \item we incorporate \gls{classical rent theory} into an agent-based urban model 
%     \item we allow the creation and distribution of rents to influence urban growth, productivity and  population structure. 
%     \item we incorporate current research on \gls{urban scaling} into the  core spatial urban model.   
%     \item we construct an   Urban \gls{ABM} that is consistent with \gls{neoclassical growth theory},
%     \item we integrate \gls{financial capital} into a standard spatial model of the urban system
%     \item we integrate financial capital into an population model of the urban system
%     \item we employ the ABM to examine how financial markets impact the urban land markets 
%     \item we test for \gls{hysteresis}  resulting from the business cycle  in the urban system 
%     \item we build a model that is easily extended to explore a wide range of issues
%     \item we provide a model that we believe can be used  to evaluate urban policies
% \end{enumerate}
% ADD TO EACH: THIS MEANS, THIS MATTERS BECAUSE, THEN FOLLOW WITH TO DO IT, WE NEED TO. IS THERE A PICTURE THAT COULD ILLUSTRATE THE CONTRIBUTIONS?
% CONSIDER MENTIONING WHAT ORDER THIS IS IN  % It's a kind of logical order, almost in the order of development: (types, order implemented, order theory is developed, order of importance?)
% methodological innovations% ROUGH LIST. TO SUMMARIZE AND SORT:
% Each of the items above requires us to  integrate models and concepts from different parts of the literature. 

% The core contribution is an exploration of the 


\section{Document overview}

%  Finally Part \ref{part-system} puts the analysis %model %and theory 
% in the context of a larger system, using methods of systems analysis and design, to discusses potential interventions and policy implications.
% \begin{enumerate}
%     \item Chapter \ref{chapter-interventions} considers the system, examines the potential for a range of interventions, and identifies policy implications. % with a diagram relating interventions.
%     \item TODO chapters .. may analyze particular interventions/cases in more detail: e.g. shared ownership models e.g. acquisition/land trusts, developer models, tax/zoning, and funds/financing.
% \end{enumerate}

% After we develop the mathematical description of the relationship among these will discuss in more detail, various relevant applications, and issues in the literature that draw on parts of this model and apply to the specific situation we're in why rent theory is related to discussions of exploitation why it might lead the inefficiencies, whether or not this links with other important models in the literature.

% Because we draw on a wide range of methods and literatures, we discuss the relevant literature and  methodologies in the chapters where they apply 


\chapter{COMBINE WITH RENT Background} \label{chapter-background}

% There are two dominant stories of \gls{distribution} in economics. The first and oldest is based on the classical concept of rent as explained  by David Ricardo \cite{ricardoEssayInfluenceLow1815}, in which owners of an asset are able to extract a value beyond what they contribute based on their ownership of a scarce resource. 
% The second is the marginalist approach, developed by Clark and others, in which workers and other factors  in competitive markets receive the \gls{marginal value-product} of their contribution to production. 

% Both developed in response to the specific social and economic conditions of the periods in which they emerged. Both attempt to explain where the output of society ended up within society. they are, at their heart, stories of who  claims what share of production.  Classical rent theory  emphasized the distribution of the social surplus, the part of production  over and above what was needed to reproduce society. This included only land rents initially, but was extended to the distribution of profit profits.   These are exactly the types of surplus income that are not explained by the neoclassical theory, which is  awkward because profits and rents form a substantial part of national income (20–25 percent) in the world the neoclassical model describes. In this thesis we identify classical rents intn the urban systems and exxamine their distribution.. 
% % ***E FIRSTNAME Ricardo was a INSERT NATIONALITY classical economist  in the WHICH Century. 

%Classical economics emerged in the 16$^{th}$ and 17$^{th}$ centuries at a time of exploding colonial wealth. In this period,  European economies were still structured largely around agricultural production in which farmers OR peasants work the land  owned by the descendants of feudal aristocrats. Urban economies based around factories were just beginning to emerge as significant drivers of wealth.\footnote{The Industrial Revolution is usually described as beginning around 1760 and having significantly transformed society by about 1820–1840.} % It was only after 1650 that living standards in the UK did start to increase for a sustained period. Before the modern era of economic growth, the economy worked very differently. Not technological progress, but the size of the population determined the standards of living.} 
In this economic context, the agricultural surplus is captured by feudal aristocrats. 

%As WHAT LED TO THE EXPLOSION... CHANGES IN UNIVERSITIES OR THE ECONOMY, there was an explosion of thinking about social and economic structures and systems.  This was a period where a range of thinkers explored many ideas to explain the shifting social and economic conditions. 

%Distribution, or who got what share of production was a central concern. This focus came to the forefront in the 1600s and 1700s as colonial expansion of land holdings, factories and supplied change concentrated wealth- funding investment in the arts and sciences, great fortunes - harsher conditions for farmers, pushed of their land with enclosures, and concentrated in urban factories, almost exclusively poor with - illness bad working conditions % 90 rural,-- population snap shot % (unprecedented in northern Europe, a relative backwater) --  %gt (new ineuqlity- jusxtabosition- fortuioned unpreciendet ont hat content to rivla hsotori great- wons- woth workers dying, kids in factories- Dickens period, pesantry.. - both were new- the eilsure- )
% and concerned with inequality, which suited a time in which the labourers both on the land, and in the emerging factories struggled with poverty, at a time of rapidly growing -
% -the intense interest spiked as wealth grew in colonial Europe. %(reformation another answer to this)
% NOTABLY THIS WAS AN URBAN TRANSITION- FARMS TO SUBURBS- POWER GROWTH, AN INFLECTION - LIKE COVID REVEALED THE NEW FORM WITH A RAPID MOBILITY CHANGE..
% The new form is the urban
% dense exchange - experience in person density can compete with screen experience -

%In this context, Ricardo refined the classical concept of rent.  

% ***E Today the word rent is usually used to refer to payments made by a tenant to a landlord to for the right to occupy a property, but this is not the traditional economic concept of  rent.  

% ***E As a concept, it's more closely related to profit / surplus than to rent paid for properties. 


% ***E NEED TO ADD HOW DOES THIS CONTRIBUTE OR EQUATE TO A THEORY OF DISTRIBUTION... WHAT DOES IT EXPLAIN...
%Ricardo's concept of rent accounted for the way that.... WHAT IT DESCRIBED. 
%This remains important to understanding distribution in general because.....


%The marginalist account describes how workers receive the marginal value of their contribution to production. THIS MEANS.....
%It formalizes what classical economist Adam Smith describes in the story of the pin factory. (EXPLAIN PIN FACTORY)

%The marginalist account of distribution gave a story of production that seemed to align with the rising fortunes of workers following WWII, in the 1950s and 1960s when it came to dominate. 

%This narrative dominates in neo-classical economics, particularly in the United States, and formed the basis of conventional micro-economics training. (EVEN NOW OR WHEN)
%It was also influential in shaping public discourse and policy in the early 20th century. For example, because it implies WHAT THAT MEANS MONOPOLY IS BAD... it provided the intellectual foundation for anti-monopoly political movements in the early 20th century.

\textbf{These two stories each reflect a distinct dominant mode of production}, period in society, and methodological set of tools available.From the Classical to Neo-Classical economic approaches, we also see development in the methodological approaches to describing economies. In Ricardo's era, economics was characterized by a free flowing descriptive approach characterized by wide debate, and many concepts explored. By the the time the marginalist approach emerged, new mathematical approaches were beginning to emerge OR dominate the economic discourse. 

SUMMARIZE THE DIFFERENCE IN WHAT EACH MODEL TELLS US ABOUT DISTRIBUTION. WHAT IS OVERALL DIFFERENT, WHAT DIFFERENT INSIGHTS EMERGE. 

\section{This Work}

\textbf{This thesis contributes to a third theory} % ***E IS THIS RIGHT?? I CHANGED IT TO THEORy OF DISTRIBUTION FROM THEORY OF MODERN URBAN RENT WHICH IS NOT THE CATEGORY WE ARE DISCUSSEING THEORIES FOR SO IT DOESNT MAKE SENES
of distribution by developing a theory of modern urban rent which integrates the classical descriptive work on rent and the neoclassical marginalist approach, with modern work on the scaling of wealth in cities. 

Just as the classical and neo-classical approaches \textbf{responded to and theorized the dominant mode of production in their time,} the theoretical framework emerging now, responds to the economic and social factors that are defining this period.  The early stories of production from thinkers like Ricardo centred on exploring who claimed the surplus from agricultural production Over time, the story moved to industrial production. Now the center of production is increasingly  WHAT KIND OF WORK (% ***E SAYING IT IS URBAN CHANGES THE SUBJECT TO LOCATION FROM TYPE OF WORK).
Industrialization shifted the centres of wealth to cities and since then the economic importance of cities has only grown. (%***E IS THIS SO?).  
% CUT : The social wealth of cities/human capacities developed in cities is central in new world where the production of value by people in cities 

NEED A CLEAR STATEMENT HERE ABOUT WHAT CITIES ARE ACTUALLY DOING. 
WHAT IS TANGIBLY HAPPENING IN CITIES THAT MAKES THEM SO IMPORTANT? WHAT KIND OF WORK? 
THIS IS THE PLACE TO SKETCH OUT A CLEAR AND RECOGNIZABLE PICTURE OF URBAN ECONOMIES IN THIS HISTORICAL MOMENT. ACCOUNT FOR TYPES OF WORK, COVID, ETC. 

Cities are the centres of social wealth and a wealth production. 
WHAT DOES WEALTH MEAN? 
WHAT DOES SOCIAL WEALTH MEAN?


A theory of production that doesn't center/involve the relation between cities and production/urban space and human capacity simply can't explain the creation of value in the modern world. It will miss WHAT SPECIFICALLY 

This is why is useful to draw in urban theory and spatial models (OR WHAT)

NOW SUMMARIZE IN BRIEF THE THREADS BEING BROUGHT TOGETHER. 
While econimists have focused on, URBAN THEORY has developed accoutns of WHAT????
WHAT ARE THE MAOR ACCOUNTS YOU ARE DRAWING ON? 
Meanwhile, while economists have focused on WHAT, Urban Theory has explored a different set of questions. 
WHAT DOES URBAN THEORY STUDY? WHAT KINDS OF QUESTIONS. 

This work weaves traditional economic approaches together with urban theory to contribute to a picture of distribution that accounts for how modern modern social and economic trends play out in space (FIX THIS SENTENCE)
ALSO ADD: 
key thinkers - is anyone else building the broader picture of this third story?? if not say something like "the key insights to this work have been developed separately in these different fields, this thesis pulls them into an singular account of distribution in the the current period. 


This approach also draws on the advancing\textbf{ methodological tools available}. The early classical work could reflect rich dynamic stories with many dimensions, using qualitative methods % ***E CAN YOU SAY QUALITATIVE METHODS??? I WANT TO EMPHASIZE THE DIFFERENCE IN METHODS MORE
to develop theories that responded to and referred to many individual stories. The neo-classical work was able to incorporate increasingly mathematical approaches% ***E (BE MORE SPECIFIC) MAYBE LIKE: THEY DID SOMETHING using the modelling systems available before large complex system modelling OR WHAT?? was possible.
Since then, computational capacity and approaches have progressed. This work uses ABM... COMPLEX SYSTEMS that enable us to work with large data sets and complex stories. % ***E BE SPECIFIC HERE ABOUT THE METHODS YOU USE) 
agent-based models and complex stories
-- big data sets, describe those concepts get back to..
cities - land wealth is the key stone
% #NEED TO FILL OUT HIS SUMMARY OF METHODS AND PROBABLY ADD A CONCLUDING PARAGRAPH THAT SUMMARIZES THE SHiFTS DESCRIBED ABOVE AND WHY THIS WorK IS IMPORTANT. CAN REWORK THE FOLLOWING:
Changes in social and economic structures are explains by different models and emerging methods that allow us to bring the complexity that could previously only be accounted for using narrative and qualitative approaches into math-based models. 
lassical Accountsfirst mostly farmers, very poor workers
then marginalist  'gave a story of production that seemed to align with the rising fortunes of workers following WWII, in the 1950s and 1960s when it came to dominate. - peoples fortunes were increasing, the new emerging production seemed central.
now inequality has risen sharply, it seems there is a need for a model that explains what has changed. This thesis contribute to developing that new understanding of how distribution functions in the world as it is today. 

% E CUT OR MOVE ALL THIS, I THINK: 
% They reflect

% WHAT IS THIS - what fits...
% neoclassical tradition emerged with high ..
% - elauition -- pro- vastly àmaterial realism. etc..

% we uses as a common threat to put these in the same language of formal function functions, tracing from early models of cobb doulgas fucntion

% we then tell the history..  
% descriptive, analytic and embedded in a complex syste

% The complexity - allows for tracing the paths of individual- what happens for whom under a far broader range of conditions

The clarity of pedagogical models- bottom up and top up both have illustrative cases e.g. edge worth box or the Schelling's/birds models.
But true theory integrates in something that moves between scales fluidly, makes it possible for the distinct scale based approaches to come together.


\section{CUT/EDIT Very rough notes on the three stories of distribution}

ABOVE WE DISCUSS THE CHANGE IN THE PRODUCTION SYSTEM, THE THREE STORIES ALSO REFLECTS METHODOLOGICAL EVOLUTION, AND THE POLITICAL AND SOCIAL CONTEXT.

They are also %These stories are, at their heart, 
stories of who claims what share of production. 

These stories evolved with changes in the system of production.
%They evolved within an evolving theory of production. 
The early stories of production thinkers like Ricardo focuses on were agricultural. Who claimed the surplus from agricultural production? Over time, the story moved to industrial production.
Now the center of production is increasingly urban- with the social wealth of cities/human capacities developed in cities dominating. 
new world where the production of value by people in cities is the center..
A theory of production that doesn't center/involve the relation between cities and production/urban space and human capacity can't explain the creation of value in the modern world. % Later thinkers including Smith and Marx%leaving aside purely inherited weath- as that becaumse caught in this same circuit of capital transforming from production, to money and back. 

They reflect
first mostly farmers, very poor workers
then marginalist  'gave a story of production that seemed to align with the rising fortunes of workers following WWII, in the 1950s and 1960s when it came to dominate. - people's fortunes were rising, the new emerging production seemed central.
now inequality has risen sharply, it seems there is a need for a model that explains what has changed.




*** METHOD ***
ADDING SOME ROUGH NOTES ON THE EVOLUTION OF METHOD AND POLITICAL CONTEXT FOR EACH OF THESE 3 THREADS (CLASSICAL, NEOCLASSICAL, AND WHATEVER WERE DOING WITH THE URBAN)..

A third thread is the advancing methodological tools available.
The early classical work could reflect rich dynamic stories with many dimensions - individual stories.

The stories also follow the methodological development of the discipline from the descriptive work of the classic descriptive work. ..
to central complex systems, large data sets
Methodological-- early descriptive theories told rich layered stories with different
The excitement concentrated  calculus.. in the classical distributional dynamic.

cutting edge technique focused on formalized.. 


The complexity of agent base models - allows for tracing the paths of individual- what happens for whom under a far broader range of conditions



*** SOCIAL AND POLITICAL CONTEXT ***

Early descriptive work
This explosion of formal rigour - focused attention.. 
And the political context..


Monopoly- political pressure real explosion of wealth creation-- economic success of political efforts to break up monopolies.
And a dynamic- lots of worker power- expanded equality-- workers seemed strong, 
As well as the political environment in the US during the cold war, older stories rooted- marx- repression, economists perhaps created an environment in which economists
a side of the economics

-- mythological center moved to the point where descriptive and historical approaches barely taught.

Samuelson-- successful exciting-- formal-
a generation
created micro, macro
-- at the moment of the baby boom- departments founded in this moment of exuberance. raised in it, taught according to this framework.
Polarized in the periodo of the cold war - the discussion of the market-- perhaps a tendency to avoid the distributionl.. Revolution and drama.

Computers took over from calculus -Brian Arthur
Cities took over from industries - concentrated value-- finance- and law main power centers.. - eigenvalue centrality.

Crisis in 2008 -- reintroduced descriptive methods and and an openness to new formal methods including computational and agent modelling beyond the 
Methodologial

Cities-- power law dist. rising debt and inequality. -- unstable and financialized
Increasing inequality, rising debt. - worker power expanding wages and equality, a story that explained- vs subsistence.


Exactly what those pattnersnew methods are so succesful are what was lef tout..

---


With Clark
A second great theory of distribution
The result is much of the theory of rent was lost. 
time

While Marx emphasized the tendency towards consolidation and exploitation in markets, Clark saw the tendency to increased competition. 

This allocation— dynamic quality of how wages evolved
They are bidding- and it will converge 
What share do workers get- subsistence wages- get 
But as output grows, and as firms compete for labour, particularly skilled labour, is that a sufficient experience.



Three drivers
Calculus had limits.
The political moment of expanding wages with a labour sector in a position to negotiate as the economy rebuilt following WWII and destruction of old wealth— dynamic time. 
Following WWII with growing demand for labour labour could bargain, 
Following WWII in the period— subsistence waves tending— when labour could bargain,
Following break up some of the largest monopolies like in steak— general steal


Also coincided with the political movement McCarthiesm perhaps led scholars to de-emphasize the connections of their work with the classical socialist literature.
Mathematical economics became an exciting and dynamic area.

Until this point the theory was largely descriptive..
xx Cobb working with Douglass developed a formulation — exponential, in economics their names have remained associated with the xyz formulation. 

Clark made a case it was just- became problematic.

OUR CASE IS THAT IT FALLS OFF A TOTALY DIFFERENT CLIFF



Economics had theories with rich dynamics, concerned 
Classical economics was concerted with ownership and wealth. But they were largely descriptive.
But the new calculus struggled to deal with stocks and with dynamics. 
(Came back with forester and other systems theory, as well as complexity etc.)

The French Engineers in the school of bridge end road used calculus early .. followed by xyz
Technical development and intellectual excitement aligned
Became very exciting dynamic, had many success - took over the discipline. 
Tied with political successes breaking up big monopolies — seemed to offer a path forward

US opposed soviet ideas and an intellectual environment that may have led academics to dephasize the aspects of their thinking connected with classical socialists thought. 

In this environment a particular approach became dominat— also at a moment when schools and departments were growing— the baby boom came to universities at the moment of Samuelson’s peace micro-macro divide gave a tool kit to a whole generation of economists— 

Embedded at the heart of micro- the satisfyingly precise formal structure of calculus.. the marginalize appraoche— 
Thus came to define a new disciplien— a formalization of Econ.. 
Extensions from that base became the defining advances of a generation of American economists..
Attracted math- a feedback loop.

Less emphasis on intellectual history, how changing- heterodox.. all the full range of thought

Including the much more exiting new techniques of complexity and systems- opening in 2007 an explosion of these techniques in the economies. 


CITIES

But cities matter more and more
Jacobs theory of wealth and value as fundamentally social.. 
Combined with xysz. Jacobs did

Now complexity and scaling theory revealing the universality of those principles advanced by Jacobs..

This requires a different formulation of rent… - and production wealth is inherently social what are the implicaitons— what does that mean.. 



In our model, land comes in implicitly through the demand for labour. 



\chapter{Model}



\section{OLD Implementation}


Household agents have:
\begin{enumerate}
   \item A home. Agents live somewhere, inside or outside the city,
   \item finances: they have assets, a housing budget, income, and a spending pattern - family lending pattern  
   shaped by profession, demographics, family structure, etc.) - risk profile
   \item utility function: they have a utility function with location  preferences - amenity, open space, house size requirements, transportation costs - shaped by profession, demographics, family structure, etc
\end{enumerate}	

Buyers then consult with a financial agent to determine the maximum mortgage and interest rate they'd qualify for based on their income. This gives an upper bound to the range of homes they may consider. 

Finally agents valuate the homes offered and place bids. % For simplicity of implementation, they place bids on all homes they consider. 

%Calculate willingness to pay
%Consider options
%Place bids
%
%Calculate willingness to pay (urgency/position on the market)
%Assess need for housing
%- Urgency of need Unhoused, sold house or served notice? 
%- Family or demographic changes
%- Financial viability of current situation
%Assess financial situation
%Get Max mortgage and max carrying cost given income and wealth from a bank
%Get options from real estate agent
%Place bids based on xy
%Consider options
%Place bids
%
%BUYER
%Enter market to buy
%Decide level of urgency (or decide with prospect theory - functional form for optimism/urgency/time to choose)
%(income/wealth)
%Maximum mortgage 
%Maximum carrying cost
%Household attributes - household size, employment location, amenity
%Current housing
%
%Realtor gives list of houses to look (real estate search -e.g. price range)
%Place offers - low if can't afford, higher if market is tight
%If failed, consider renting or buying next time.

\begin{figure}[htb]
    \begin{center}
 \tikzstyle{decision} = [diamond, draw, fill=blue!20, 
     text width=4.5em, text badly centered, node distance=3cm, inner sep=0pt]
 \tikzstyle{block} = [rectangle, draw, fill=blue!20, 
     text width=5em, text centered, rounded corners, minimum height=4em]
 \tikzstyle{line} = [draw, -latex']
 \tikzstyle{cloud} = [draw, ellipse,fill=red!20, node distance=3cm,
     minimum height=2em]
%
 \begin{center}
 \begin{tikzpicture}[node distance = 2cm, auto]
     % Place nodes
     \node [block] (need) {Assess need for housing};
     \node [block, below of=need] (finance) {Assess financial situation};
     \node [block, below of=finance] (alternatives) {Select homes to consider};    
     \node [block, below of=alternatives] (bid) {Place bids on homes};    
     % Draw edges
     \path [line] (need) -- (finance);
     \path [line] (finance) -- (alternatives);
     \path [line] (alternatives) -- (bid);        
 \end{tikzpicture}   
 \end{center}
    \caption{}
    \label{fig:code_worker_choice}
    \end{center}
\end{figure}

\subsection{Initial conditions}
Initialize the model with grid. Each element contains 1 housing unit.

The model space is divided into a uniform grid of single family homes % Additions: apartment buildings, a mechanism to subdivide homes, rent bedrooms, accumulate adjacent land and build new higher density buildings, an urban land boundary, a mechanism for wealth agglomeration through density.

The social structure of the model begins with urban workers who commute and earn wages in the urban commuter-shed, and rural residents and landowners who may choose to move to the city. 
We imagine the initial set of workers in our city commuting daily, being paid monthly and residing in urban housing for their working lives, which are arbitrarily set. 

Initially all urban workers are also homeowners. When we allow the initial urban residents to retire, they may sell or rent out their properties to new workers. This introduces an additional \gls{class} of resident, the urban tenant. The final agent in the model is a `bank' that represents both the financial sector and the owners of financial capital. The bank provides mortgage funds can %may actively 
purchase land as a financial asset on behalf of investors.

\subsection{Model Steps - Housing market}
% Figure xyz traces the flow. 
In each time step agents firms update wages and job availability, agents decide whether to work and whether to buy and sell homes.
 % Schedule: Multi step by breed
 % Steps Labour
 % step - workers: market/production, enter market to buy, list properties real estate agent matches agents - has bids 
 % bidding - workers and firm consider properties and make bids (2nd step or spread over 2 steps)
 % negotiation - sellers consider and accept bids (or real estate agents manage negotiation)
%Buyers evaluate their need for housing.
% Agents decide whether to enter the housing market as a renter or a buyer.

at every time step 
    1) Some new tenants arrive into the city
    
    2) All unhoused tenants look for a housing unit 
    
        a) They have to decide whether to buy or rent. Decision depends on their income, and the availability of Corporation of Public Housing loans 


        b) If they buy, they seek apartment and evaluate, and place a bid (see ALMA)
        
        c) if they rent, they look for an apartment that matches xyz criteria, and place request to rent. Rental is given by the landlord to the first tenant that matches the asking rental price. 


Workers give up their search and leave the city if they do not find a housing unit 

%\chapter{Background Rought Notes - Rent History Etc}



\section{Agglomeration discussion}

The phenomenon of growing productivity was initially identified and estimated in the economics literature production at the national level. The estimated functions linked capital and labour inputs to output.  Soon after the  earliest econometric models of output  were estimated, it was found that equations were not stable over time. Productivity grew over time
(We can do the arithmetic with the cobb douglas to illustrate) 

Faced with this puzzle, Robert Solow introduced a term that was time dependent, and an entire literature developed to explain this term. One productive stream explained growing productivity in terms of agglomeration effects- more people, more workers more firm or more diversity of firms appeared to be associated with growing productivity. Two major schools emerge - roughly speaking,  the Marshallian explanation, which emphasized firm-level processes and the Jacobs model which focuses on the creative effect of agglomerations of people in cities. Both have receives empirical support.

(We can do the arithmetic with the Cobb Douglas to illustrate)

Louis M. A Bettancourt and others applied similar models at the level of cities, but rather than a time-dependent term, they introduced a population-dependent term and found evidence from cities around the world that productivity rose as population rose: The scale of the city has a positive effect. The result  was one of a wide range of scaling results identifies in a great variety of systems examined in the complexity literature 


\section{Background}

\subsection{Inequality}
this wasn't how capitalism was supposed to work
wages part, 
over 50 for rent

\subsection{Drivers of the housing crisis}
supply and demand, stagnant income, and finacialization of housing

Several explanations of the current situation are commonly proposed. The first is simply that the problem of housing is a supply and demand problem where supply is blocked by some features of urban regulation. The second explanation is that the distribution of income has changed in some way that mean a significant fraction of the population are unable to afford satisfactory housing, and therefore this is the problem that must be solved.  The third common explanation currently is that financialization of the housing market  is changing the way the city economy is working, redistributing income and potentially threatening the long term growth and wealth creating capacity of the city.


\subsection{How we do the resilience analysis}

- what will we do? *** 



\section{Rent, Production and the City: Who Gets the Wealth}


%The sources of economic growth and the distribution of income are themes that run through the history of economic thought. 

The story of rent is the story of 

two great theories of distribution
a methodological evolution from descriiption to calculus to complex systems and an evolution of the economy 
from agriculture to indsutrial produciton, to social scaling or wealht in cities. 



There are two great stories of distribution in economics. The first and oldest is rent, %the classical work on rent, going back to 
developed in Ricardo, in which owners of an asset are able to extract a value beyond what they contribute. 
The second is the marginalist approach, developed by Clark and others, looking at a scenario in which workers receive the marginal value of their contribution to production. This tradition dominate in neo-classical economics, particularly in the United States, and %formed the basis of conventional micro-economics training.
% it gave a story of production that seemed to align with the rising fortunes of ordinary people/workers following WWII, in the 1950s and 1960s when it came to dominate. 
% formed an intelectual foundation for anti-monopoly political movements in the early 20th century. 
This work contributes to a third theory of 

emergent complex systems methodology and  work on urban science, the power law concentration, and integrates/ to achieve a sysnthesis of  the clasical descriptive work on rent and the neoclassical marginalist appraoch

These stories are, at their heart, stories of who claims what share of production. They evolved within an evolving theory of production. The early stories of production thinkers like Ricardo focuses on were agricultural. Who claimed the surplus from agricultural production? Over time, the story moved to industrial production, and increasingly urban- with the social wealth of cities/human capacities developed in cities dominating. % Later thinkers including Smith and Marx%leaving aside purely inherited weath- as that becaumse caught in this same circuit of capital transforming from production, to money and back. 



Methodological-- early discririptive theories told rich layered stories with different
The excitiment concentrated  calculus.. in the classical distributional dynamic.

The complexity - allows for tracing the paths of individual- what happens for whom under a far broader range of conditions

The clarity of pedagogical methcs- bottom up and top up both have illustrative cases e.g. edgeworth box or the schellings/birds models.
But true theory integrates in something that moves between scales fluidly, makes itpossible for the distinct scale based approaches to come together.


Complexity

Early discriptive work
This explosion of formal rigour - focused attention.. 


And the political context..

Monopoly- political pressure real explosion ofwealth creation-- economic success of political efforts to break up monopolies.
And a dynamic- lots of worker power- expanded equality-- workers seemed strong, 
As well as the political environment in the US during the cold war, older stories rooted- marx- repression, ednomists perhaps created an envronment in which economists
a side fo the economcs

-- methological pdrived moved the point whre discriptive and historical appraoches baredly taguth.

Samuelson-- successful exciting-- formal-
a generation
created micro, macro
-- at the moment of the baby boom- departments founded in this moment of exuberance. raised in it, taught according to this framework.


Computers took over from calculus -Brian Arthur
Cities took over from industries - concentrated value-- finance- and law main power centers.. - eigen value centrality.

Crisis in 2008 -- reintroduced dscrptive
Methodologial
Cities-- power law dist. rising debt and inequality. -- unstable and financialize.d
Increasing inequality, risind debt. - worker power expanding wages and equality, a story that explained- vs subsistence.


Exactly what those pattnersnew methods are so succesful are what was lef tout..



Polarized in the periodo of the cold war - the discussion of the market-- perhaps a tendency to avoid the distributionl.. Revolution and drama.



Early theories were implicit. They have the same logic- but exist in words

mathematization was important to the simple centrality of marginalism.






With Clark
A second great theory of distribution
The result is much of the theory of rent was lost. 
time

While Marx emphasized the tendency towards consolidation and exploitation in markets, Clark saw the tendancy to increased competition. 

This allocation— dynamic quality of how wages evolved
They are bidding- and it will converge 
What share do workers get- subsistence wages- get 
But as output grows, and as firms compete for labour, particularly skilled labour, is that a sufficient experience.



Three drivers
Calculus had limits.
The political moment of expanding wages with a labour sector in a position to negotiate as the economy rebuilt following WWII and destruction of old wealth— dynamic time. 
Following WWII with growing demand for labour labour could bargain, 
Following WWII in the period— subsistence waves tending— when labour could bargain,
Following break up some of the largest monopolies like in steak— general steal


Also coincided with the political movement McCarthiesm perhaps led scholars to de-emphasize the connections of their work with the classical socialist litturatue.
Mathematical economics became an exciting and dynamic area.

Until this point the theory was largely descriptive..
xx Cobb working with Douglass developed a formulation — exponential, in economics their names have remained associated with the xyz formulation. 

Clark made a case it was just- became problematic.
Doesn't actually happen -- and not jsut

OUR CASE IS THAT IT FALLS OFF A TOTALY DIFFERENT CLIFF



Economics had theories with rich dynamics, concerned 
Classical economics was concerted with ownership and wealth. But they were largely descriptive.
But the new calculus struggled to deal with stocks and with dynamics. 
(Came back with forester and other systems theory, as well as complexity etc.)

The French Engineers in the school of bridge end road used calculus early .. followed by xyz
Technical development and intelleual excitement aligned
Became very exciting dynamic, had many success - took over the discipline. 
Tied with political successes breaking up big monopolies — seemed to offer a path forward

US opposed soviet ideas and an intellectual environment that may have led academics to dephasize the aspects of their thinking connected with classical socialists thought. 

In this environment a particular approach became dominat— also at a moment when schools and departments were growing— the baby boom came to universities at the moment of Samuelson’s peace micro-macro divide gave a tool kit to a whole generation of economists— 

Embedded at the heart of micro- the satisfyingly precise formal structure of calculus.. the marginalize appraoche— 
Thus came to define a new disciplien— a formalization of Econ.. 
Extensions from that base became the defining advances of a generation of American economists..
Attracted math- a feedback loop.

Less emphasis on intellectual history, how changing- heterodox.. all the full range of thought

Including the much more exiting new techniques of complexity and systems- opening in 2007 an explosion of these techniques in the economies. 




CITIES

But cities matter more and more
Jacobs theory of wealth and value as fundamentally social.. 
Combined with xysz. Jacobs did

Now complexity and scaling theory revealing the universality of those principles advanced by Jacobs..

This requires a different formulation of rent… - and production wealth is inherently social what are the implicaitons— what does that mean.. 






In our model, land comes in implicitly through the demand for labour. 


\section{Chapter: Draft Literature Review and Background}

\section{Background}
\label{Sec:Background}
\newpage
Our approach/model is constructed, drawing together pieces from a number of research areas from economics and the study of cities, including rent theory, production functions, the standard urban model, growth theory, urban growth theories, financialization and the theory of distribution. 
We relate this to the scaling models from the study of complexity. This gives a deeper look at distribution in the cities, the effect of financialization, and effect of both of these on the growth and development of cities. 

The literature makes it clear that the cost of transportation is crucial, the cost of housing is crucial, and that there are strong pervasive agglomeration effects driving productivity and population growth. (City population is observed to follow a power law distribution.)

We are interested in agglomeration economies. The wage  structure would then be related to the population or industry  structure. Externalities driving agglomeration may be classified  into two types, the  or so-called ``Marshalian''  and ``Jacobs'' externalities


3 lines  
- production leading to Jane Jacobs
- cities leading to Jane Jacobs
- scaling factors and complexity leading to Jane Jacobs and empirics in a theory of cities

2 theories of production

Ricardo
Ricardo’s rent theory explained class and the distribution of income in terms of the the productivity and ownership of land in an agricultural society. Land was the scarce factor in production and control of land allowed landowners to extract any production in excess of the agricultural wage. Ricardo could assume that labour was in surplus and therefore the agricultural wage would approximate the subsistence wage. 

Marx
Marx adapted the Ricardian model to an industrial society in which surplus product could be used to create more capital and largely ignored land rents. He retained the subsistence wage, and explored the effect of reproducible capital.

Clark
Clark %\footnote{and Wicksteed} extended the theory of rents to produce 
elaborated a neoclassical distribution theory that tied income to the marginal product of each factor for a firm in a competitive economy rather than class ownership of capital. 
This marginal productivity theory became dominant SOME BENEFITS.. Although the marginal productivity theory became dominant in economic thinking, rent theory retained an important explanatory role in resource, agricultural, regional and later urban economics and even sports economics. 

Clark %\footnote{and Wicksteed} %extended the theory of rents to produce 
elaborated a neoclassical distribution theory that tied income to the marginal product of each factor for a firm in a competitive economy rather than class ownership of capital. Although the marginal productivity theory became dominant in economic thinking, rent theory retained an important explanatory role in resource, agricultural, regional and later urban economics and even sports economics. 

Rent theories have remained at the centre of economics despite the development by Clark (1894) of the more modern theory of distribution in which factors ideally receive the value of their marginal product. In modern welfare economics a measure of surplus that is the direct descendent of Ricardian land rent is at the core of the First Theorem of Welfare Economics, arguably the most significant theorem in the social sciences. With Alonso (1964), another another application of rent theory became the foundation of modern urban economics.

\section{Exploitation}
John Roemer’s 1982 Class Exploitation Correspondence Principle (CECP) states that producers who optimize by only selling labour are exploited at the economy’s equilibrium, and agents who optimize by hiring labour are exploiters. Exploitation and class structure are shown to arise from differential endowments in a manner consistent with both Ricardian explanation of class incomes and Marx’s conception of exploitation. We extend the argument to show that differential access to finance capital, urbanization, the growing importance of human capital in producing surplus and agglomeration economies endogenously generate a class structure based on the indirect capture of land rents, We illustrate the emergence of class structure within a simple agent-based model of the land market in a monocentric city. The model is consistent with the theories of Ricardo and Henry George in locating the ground of exploitation and class in the capacity to extract social surplus through land ownership, and differs from the standard Marxian analysis in its reliance on access to financial capital rather than control of productive physical capital.

The sources of economic growth and the distribution of income are themes that run through the history of economic thought. 

From Smith and Ricardo economist have understood that the net product of the economy is divided among functional classes.
 Ricardo is generally credited with providing the best early description for the division of the product of the land between labour and property owners. 
 Marx is generally credited with providing a convincing explanation based essentially on Ricardo's insights, of the distribution of the product of industrial capital with a class-monopoly on ownership the capital,  as well as insights about the evolution of a society based increasingly on produced rather than natural capital. 
 Henry George elucidated the role of land rent, particularly in the urban context as as a mechanism for extracting socially produced economic surplus.  

The division of the product of the earth among the classes of society has been a central issue in economics since at least Ricardo presented his theory of  rent, through Marx, adapted the concept of rent to an industrial and capitalist economy and Henry George, who applied it in the urban context. Land rent is also at the core of modern urban models.  John %\textcite{RoemerGT} 
in \textit{A General Theory of Exploitation and Class} 

%\textcite{RoemerGT} 
%\footnote{\cite{RoemerGT} p12} 

The distribution of rent, where it goes, and what the implications are. 

\section{Liturature Review}

3 lines  
- production leading to Jane Jacobs
- cities leading to Jane Jacobs
- scaling factors and complexity leading to Jane Jacobs and empirics in a theory of cities

2 distributional stories.
- class and rent -- synthesis of rent and class--

Then is the history of rent seperate?

O’Sullivan (2011) identified “five axioms of urban economics” that have emerged from a century of study: (a) location-specific costs and benefits balance to generate a locational equilibrium; (b) self-reinforcing effects induce concentration of activities and individuals; (c) externalities are prevalent; (d) production is subject to economies of scale, which favours agglomeration; and (e) competition generates zero economic profit. These features combine to produce dynamic urban system that will shape our future. 

We build a model that incorporates the five “axioms” to demonstrate how production externalities, in a class of models, can drive urbanization, class formation and the wealth distribution.

To understand the relationship requires integrating the distributional appraoch from clasical economic theory of rent with the modern marginalist model of distribution through wages. It does this by integrating the urban model with the model of production and including the cost of land and transportation in the urban wage in the labour costs. 
In this way the two factor model of projection reflects the clasical landowning extraction of rents by landowners, with a model of wages in competitive markets. 

\subsection{Rent}

 What Is Economic Rent?






neoclassical story.
Economic rent is an amount of money earned that exceeds that which is economically or socially necessary. This can occur, for example, when a buyer working to attain a good or service that is considered exclusive makes an offer prior to hearing what a seller considers an acceptable price. Market imperfections thus lead to the rise of economic rent; it would not exist if markets were perfect, since competitive pressures would drive down prices. 
%https://www.investopedia.com/terms/e/economicrent.asp
% https://www.wallstreetmojo.com/economic-rent/



Henry George brought the classical position to its logical conclusion: rent is an unearned increment. The Classical Base of Modern Rent Theory, Conway L. Lackman
“Whatever part of the produce or… of its price, is over and above this shame” (which pays for the capital advanced “together with the ordinary profits”), “he” (the landlord) “naturally endeavours to reserve to himself as the rent of his land” ([O.U.P., Vol. I, p. 163; Garnier,]  
l.c., p. 300). Theories of Surplus Value, Marx 1861. [Chapter XIV]  
 Adam Smith’s Theory of Rent [1.  Contradictions in Smith’s Formulation of the Problem of Rent]
This excess may “he considered as the natural rent of land” ([O.U.P., Vol. I, p. 163; Garnier,]
l.c., p. 300).


 \subsection{Ricardo}
 
 David Ricardo developed a theory of land rent.
Leading figure in classical economics



He modelled the agricultural economy.
Ricardo developed the idea of 

He was friends with James Mill, Jeremy Bentham and Thomas Malthus.

He theoriezed the agricultural economy.





\section{Drafting REVIEW}
This section traces the history of rent and production in economics.
 central issue in economics since at least Ricardo presented his theory of  rent, through Marx, adapted the concept of rent to an industrial and capitalist economy and Henry George, who applied it in the urban context. Land rent is also at the core of modern urban models.  


In economics, rent is a surplus value, i.e. the difference between the price at which an output from a resource can be sold and its respective extraction and production costs, including normal return (DFID, 2003; Luchsinger \& M\:uller, 2003; Sharp, 2003; Stoneham et al., 2005).

Chapter 24: Doctrine of Adam Smith concerning the Rent of Land
``Such parts only of the produce of land,” says Adam Smith, ``can commonly be brought to market, of which the ordinary price is sufficient to replace the stock which must be employed in bringing them thither, together with its ordinary profits. If the ordinary price is more than this, the surplus part of it will naturally go to the rent of land.

If it is not more, though the commodity can be brought to market, it can afford no rent to the landlord. Whether the price is, or is not more, depends upon the demand.''

More briefly, rent is a surplus value after all costs and normal returns have been accounted for. Normal costs include  payment of all the factors of production at their market rate.(Labour at the going wage, Capital at the interest rate, supplies at their normal price). The great social question at first was who gets the surplus.  

%The question was pressing because it appeared that landlords were capturing the surplus without contributing to production while may peasants were very poor. 

Ricardo

% Born in the late 1700s, was a British poltical economist. The son of a stockbroker, he built a fortune by investing.
The law of rent was formulated by David Ricardo around 1809, and presented in its most developed form in his magnum opus, On the Principles of Political Economy and Taxation. This is the origin of the term Ricardian rent. Ricardo's formulation of the law was the first clear exposition of the source and magnitude of rent, and is among the most important and firmly established principles of economics.

The landlord would rent out all the land which generated at least enough to pay all the costs. Anything in excess of the costs could be charged as land rent to a tenant farmer.

This excess, or surplus, he identified as the income of the landlord. The landlord captures the surplus by ownership of the natural resource land. 

Ricardo, did not write down a production function his, but his analysis can be understood as implying one.

Clearly in his model there are two basic productive factors, land and labour. The landlord  receives the surplus generated by the land and the rest of the value of production goes to labour. Ricardo essentially assumes that there wage is  just sufficient to reproduce the labouring class.\footnote{``In the natural advance of society, the wages of labour will have a tendency to fall, as far as they are regulated by supply and demand; for the supply of labourers will continue to increase at the same rate, while the demand for them will increase at a slower rate.''  This is  basically Malthus.} He has explained the distribution of the fruits of the land among the main classes of the economy.

The implicit production function is
\[Y=F(N, L)\]

Where the output $Y$ is a function of $N$, the number of workers and $L$, land.

His analysis included a concept of diminishing marginal return, the rate at which production grows declines. 
This shows in his use of the terms ``extensive margin'' and ``intensive margin'' to explain the income of the landowner. He focused on the difference between the cost of production on a unit of land and the revenue generated. 



employers enjoy a bargaining advantage over workers and can coerce them to accept worse terms, because they need individual workers less than individual workers need employment. It is no surprise Marx was an admirer. Wages are not the simple product of supply and demand in Smith; bargaining asymmetries are key.

Ricardo included concept of diminishing marginal product, which means


His analysis can be understood in terms of a production function. 
% Factors of production are things that play a role in creating the output. There are several factors of production, things required for production/to create wealth/value (using capital and labour). Most obviously capital and labour. One of the factors of production is labour. In principle the rate at which hiring changes output can take any form. If hiring one more worker increases output, the marginal product of labour is positive. The marginal product can be positive and increasing, or positive and decreasing. If the marginal product of labour is decreasing, the curve is slopes downward, there are decreasing returns to scale, and each additional worker adds less to output than the last. % DIAGRAM  In general hiring more people increases production. In general employers choose workers who would increase production. Otherwise they would not hire. Returns to scale determins how much hiring one more worker increases output. With increasing returns to scale, each new worker increases output more than the last one did, and companies tend to grow big. With decreasing returns to scale, each new worker increases output by less than the last one did and so more, smaller companies may form % (TODO: clarify explanation of implications). To connect with the tradition of analytic economic modelling, ensure there is an equilibrium, by making the marginal product of labour monotonically declining, ors declining over the whole function. This equilibrium condition ensures the curve in the diagram is slopped downward and the curves intersect. % (TODO: discuss/justify assumption, ground empirically)


 2 factors is typical since it's enough for most kinds of analysis. Another factor typically used to introduce another constraint on production, or potentially consumption. Eg. labour is mobile, capital takes time to accumulate but you can put it anywhere, but forest is a flow from the ground-- puts a limit on the region of the city. - land, flow from the land, natural resources.
 
 The factors are usually chosen so xyz
 To keep it tractable- one which is slow to adjust and one which we can adjust quickly - one which is world/embodied work and one which is current work, the rest falls between.

%and is replaced with capital. -- What you'd do an analysis if you put in 30 factors, in the analysis you hold 28 and let 2 move to see what's going on, and what you get is - you have a production function with certain mathematical factors, you can expand as much as you want. you can expand as much as you like-- everything you deduce about 1 is true about any 1 and all the rest if they hold the same functional relation to one another.This is a mathematical trick that gives you xyz.. things you get from it are things like there's a good reason to assume growth till you get 0 profit, then you get competitive markets, all that drops out of the math. - profit maximization you can impose, expansion to 0 profit, expansion till marginal product of labour equals the wage, value of marginal product of capital equals the interest rate, marginal product of whatever is equal to the price/unit of whatever it is - 


Historic evolution is land and labour.

Marx


, as we move from agriculture, 





John Roemer’s 1982 Class Exploitation Correspondence Principle (CECP) states that producers who optimize by only selling labour are exploited at the economy’s equilibrium, and agents who optimize by hiring labour are exploiters.

Roemer
RoemerGT demonstrates that in equilibrium there are classes that are exploited and classes that are exploiters, as well as intermediate cases, using  a definition a definition of exploitation 

that is essentially Marxian and is consistent with Ricardo's rent theory and that of Henry George: 
%\begin{quotation}
%
An agent is exploited  if and only if the value of the labour the agent sells plus the value of own production plus wage earnings is less than the maximum value of their consumption bundle.
%\vspace{.25cm}
%
%and\vspace{.25cm}
%
%An agent is an exploiter  if and only if the value of the labour the agent sells plus the value of own production is less than the minimum value of their consumption bundle.
%\end{quotation}

A General Theory of Exploitation and Class examined a General equilibrium linear economy in which all individuals rationally choose their  activities given their initial endowments and demonstrated  the endogenous emergence of a class structure in a purely neoclassical model. 

Marx
distribution of the product of industrial capital with a class-monopoly on ownership the capital,  as well as insights about the evolution of a society based increasingly on produced rather than natural capital. 



CITY
Rent howerer has remained central in the study of the city
Began everywhere.

Johann Heinrich von Th\"unen was influential in developing the spatial analysis of rents, which highlighted the importance of centrality and transport. Simply put, it was density of population, increasing the profitability of commerce and providing for the division and specialization of labor, that commanded higher municipal rents. These high rents determined that land in a central city would not be allocated to farming but be allocated instead to more profitable residential or commercial uses. 


 Henry George elucidated the role of land rent, particularly in the urban context as as a mechanism for extracting socially produced economic surplus.  

OLD?
 
Ricardo developed a theory of land rent. He did not write down a production function, but he quite clearly understood and used the concept of diminishing marginal product. This shows in his use fo the terms ``extensive margin'' and ``intensive margin'' to explain the income of the landowner. He focussed on the difference between the cost of production on a unit of land and the revenue generated. The landlord would rent out all the land which generated at least enough to pay all the costs. Anything in excess of the costs could be charged as land rent to a tenant farmer.

This excess, or surplus, he identified as the income of the landlord. The landlord captures the surplus by ownership of the natural resource land. 

Clearly in his model there are two basic productive factors, land and labour. The landlord  receives the surplus generated by the land and the rest of the value of production goes to labour. Ricardo essentially assumes that there wage is  just sufficient to reproduce the labouring class.\footnote{ ``In the natural advance of society, the wages of labour will have a tendency to fall, as far as they are regulated by supply and demand; for the supply of labourers will continue to increase at the same rate, while the demand for them will increase at a slower rate.''  This is  basically Malthus.} He has explained the distribution of the fruits of the land among the main classes of the economy.

The implicit production function is

\[Y=F(N, L)\]

Where $N$ is the number of workerrs

\subsection{Marx}
 Marx examined a developing manufacturing economy. In this economy the owners contributed the machinery, buildings, and even working capita to fund the workers until the product can be sold. This contribution must be accumulated from their profits in the preceding cycle of production,  and has to be reinvested once the revenues of the current round have come in and the bills have been paid. Marx actually describes a circuit of capital from its for as money to its form as physical capital. 
 
 
The implicit production function is

\[Y=F(N, K)\]
where $K$ stands for the productive capital stock. 

As in Ricardo, labour is in surplus and capital is scarce. As in Ricardo the scarce factor owned by a special class - now the capitalists, is able to appropriate the is able to capture the surplus value. Like Ricardo,  marx saw the appropriation of surplus as without morel justification - 


Marx also pointed to a new dynamic in capitalist systems - that productive capital is not fixed as land is, but does and must expand as surplus is reinvested. The expansion will eventually outrun the expansion of demand and the rate of return will fall, leading capitalists unwilling to invest and creating a crisis,.


 
\subsection{Henry George} 
  Henry George returned to land rent with a new insight based on the emergence of the capitalist city. Since land rent is unearned income he argued that it should be seen a social income - that it could be used to pay for all the needs of the community. This is the basis of the `single tax' movement. He cleasrly looks back to Ricardo and the early rent theory, but also forward to urban models. His analysis would be recovered in urban models with the proof of. the `Henry George Theorem" in... by .... It demonstrated that if it was some ;public good that attracted people to a city, the optimal level of the good was jus the amount that could be paid for from the increment in land value.\footnote{Progress and Poverty: An Inquiry into the Cause of Industrial Depressions and of Increase of Want with Increase of Wealth: The Remedy is an 1879 book by social theorist and economist Henry George.}
  
  Wikipedia expresses the dynamics this way: ``The tendency of speculators to increase the price of land faster than wealth can be produced to pay has the result of lowering the amount of wealth left over for labor to claim in wages, and finally leads to the collapse of enterprises at the margin, with a ripple effect that becomes a serious business depression entailing widespread unemployment, foreclosures, etc. ''
  
  In George land includes all natural resources, everything ``that is freely supplied by nature.''  
  \footnote{Analysis of the locational rents generated in this class of models has resulted in several authors demonstrating the validity of variants of the Henry George Theorem (\cite{Arnott-Stiglitz79, Arnott04, BehrensKanemoto14, JohnM.Hartwick1980THGR}). These analyses  show that the land rents can exactly equal the cost of the public good that draws individuals to the city or the production services that draws firms. In our model these rents are extracted by land-owning financial capital. They are not invested in a public good or in expanded production capacity.}
  
  \subsection{John Bates Clark}
  Another socialist like George, he was also one of the pioneers of marginalism. By 1986 he was praising the dynamical process of competition partly in opposition to the single tax movement George had initiated.  His (1891). ``Distribution as Determined by a Law of Rent,'' argued that, given  competition and homogeneous factors of production labor and capital, the division of the social product will be according to the productivity of the last physical input of units of labor and capital.\footnote{Responding to the "indictment that hangs over society" that it involves "exploiting labor," Clark wrote:

    It is the purpose of this work (his 1899 'Distribution of Wealth) to show that the distribution of the income of society is controlled by a natural law, and that this law, if it worked without friction, would give to every agent of production the amount of wealth which that agent creates. However wages may be adjusted by bargains freely made between individual men (i.e., without labor unions and other "market imperfections"0, the rates of pay that result from such transactions tend, it is here claimed, to equal that part of the product of industry which is traceable to the labor itself; and however interest (i.e., profit) may be adjusted by similarly free bargaining, it naturally tends to equal the fractional product that is separately traceable to capital.} 
  
 \subsection{Cobb and Douglas}
 The neoclassical revolution opened the use of formal functional mathematics and calculus. Cobb and Douglas (notably Cobb) came up with a specific and very convenient functional form that captured much of what economists were talking about:
 
 \[Y=AK^\alpha L^\beta\]
 
 Where $A$ is a constant scale factor\footnote {apparently previously used by Knut Wicksell, Philip Wicksteed, and L\'eon Walras. I didn't know that!}. The Cobb–Douglas form was developed and tested against statistical evidence by Charles Cobb and Paul Douglas between 1927–1947. It was  the widely circulated empirical work seems to have permanently associated the rather familiar function with the two names for economists.
 
 A 2021 meta-analysis of 3186 estimates concludes that "the weight of evidence accumulated in the empirical literature emphatically rejects the Cobb-Douglas specification."\footnote{Gechert, Havranek, Irsova, Kolcunova (2021), "Measuring capital-labor substitution: The importance of method choices and publication bias", Review of Economic Dynamics, doi:10.1016/j.red.2021.05.003, S2CID 236400765}
 
 The form captured  important regularities in the data but these drifted over time. 
 
 %COBB DOUGLASS additive property-- multiplicative.. additive property-- expandibile in the sense you can add more in-- some functions which are copb douglas which is log linear so seperable, and so additive in an important sense. BUT MVP is true for any firm successfully multipled function even if pruduction function is not seperable..
 
% Write profit using input, profit, - differentiate, get first order conditions which are a peak in the profit function, take those and manipulate to get rules, features of the optimal behaviour-- set marginal value equal to the price, set quantity to the point where profit goes to zero.. approach to standard economics..
 
 \subsection{Solow}
To deal with the drift, Solow introduced a refinement, opening the field for a further series of refinements  in an enterprise that became known as ``growth theory.'' \footnote{A Contribution to the Theory of Economic Growth,  Robert M. Solow, The Quarterly Journal of Economics, Vol. 70, No. 1 (Feb., 1956), pp. 65-94. Stable URL: http://www.jstor.org/stable/1884513}

Solow argued ``As a result of exogenous population growth the labor force increases at a constant relative rate n,'' so
  \[L(t)= L_0e^{nt}\]


 \[Y=A(t)K^\alpha L^{1-\alpha}\]
 where $A$  explains the change in factor productivity as a function of time. It is no surprise that adding a variable allowed the model to track the data better. Solo went farther and described the dynamics of the model using an explicit time dependence: ``As a result of exogenous population growth the labor force increases at a constant relative rate n,'' so
  \[L(t)= L_0e^{nt}\]
  
  
 As a result, if we stick this into the production function 
 \begin{eqnarray}
 Y&=cK^\alpha (L_0e^{nt})^{1-\alpha}\\
    &=c(e^{nt})^{1-\alpha}K^\alpha L^{1-\alpha}\\
    &=A(t)K^\alpha L^{1-\alpha} \label{Eq:Solow}
 \end{eqnarray}
 where
 \[A(t)=c(e^{nt})^{1-\alpha}\]
 
 N. Gregory Mankiw, David Romer, and David Weil created a human capital augmented version of the Solow–Swan model that can explain the failure of international investment to flow to poor countries.

    \[Y(t)=(A(t)K(t)^\alpha H(t)^\beta L(t))^{1-\alpha -\beta} \]
    
    From the Solow example we can see that if all the time functions are exponential we end up with equation~\ref{Eq:Solow} again.
    
\subsection{How the Solow model performed}    
The estimated model explained 78\% of variation in income across countries, the estimates of $\beta$ implied tha t\textbf{ human capital's external effects on national income are greater than its direct effect on workers' salaries.}%(\url{https://en.wikipedia.org/wiki/Solow\%E2\%80\%93Swan_model)}.  
    
This is interesting for me. I got a similar result. Theodore Breton provided an insight that reconciled the large effect of human capital from schooling in the Mankiw, Romer and Weil model with the smaller effect of schooling on workers' salaries. He demonstrated that the mathematical properties of the model include significant external effects between the factors of production, because human capital and physical capital are multiplicative factors of production.[20] The external effect of human capital on the productivity of physical capital is evident in the marginal product of physical capital:

    \[ MPK={\frac {\partial Y}{\partial K}}=\frac {\alpha A^{1-\alpha }(H/L)^{\beta }}{(K/L)^{1-\alpha} }\]
 
 \subsection{Endogenous growth theories}  
 Endogenous growth theories make the increase in factor productivity depend on an optimizing decisions about human capital investment, invention, investment in technology improvement.  
 
  Productivity growth results from an active search process for innovations in
which the ability to appropriate profits determines the resources devoted to
innovative activity (OECD, 1992, Crafts, 1996). Growth depends on the incentives to in-
vest in improving technology.% https://link.springer.com/chapter/10.1007%2F978-1-349-26732-3_13
 
  I don't think these models are much help in understanding cities, which appear to be come more productive and population rises. this is an agglomeration effect.



\subsection{Solow MORE K}

The firm maximizes profit by setting the marginal value of the product of each factor equal to the unit cost per factor. 

This ensures that the marginal rate of technical transformation equals the price ratio. 

*** WHAT IS THE PRICE RATIO

the marginal rate of technical transformation is the (neg) slope of the indiference curve, 
- technical since it's a technology, the production function representws a technology, and you have inputs-- you can maintain one output by substutiting one input for another, you ocan see them as transfromation (substitution)
- changing technology changes the shape of the curve.. ---note growth theories takes technology out of the term-- growth is a term up front.. on avg it's bringing the whole thing up. consistent with the scaling term-- there is this pre factor that changes a lot- e.g. china to US-- 

NOTE AALSO  the pre factor is by nation not just time-- policy matters weath matters, geopolitical context matters.. a. lot. - US is 8x larger than china 27x larger than nigeria (CHECK NUMBERS..) .. bus driver paid x less.


the indifference curve is the curve along which output stays the same as you supstitute labour for capital or vice versa. 
the optimum input mix, it turns out, is where the isoquant (equal quantity - indifference to quantities curve vs indifference to utilities curve) of the production fuction equals output line is tangent to a cost constraint..

curve tangent to a straight line at the optimum-- at the straight line is the price  ratio

the wage over the interest rate. 

ALTERNATIVE FUNCTIONAL FORMS FOR UTILITY FUNCTIONS AND PRODUCTION FUNCTIONS
isoquant measure how things you eat make you happy-- the production function that has that isoquant is measuring equal outputs
vs indifference curve. .. that has the utility-- is measuring happiness, but they are both the same kind of geometric mean of the inputs. 

Examples whrere the isoquant would not euqal the indifference curve include: -- leiontief for instance- right angle corners with 0 substitution . 2 shoes in a pair. need a left and a right one, two right one doesn't help at all. in that case they're perfect complents not subsitutes
if they're straight line, you have perfect substittes a.. 




\subsection{COBB DOUGLASS PROPERTIES}

We use a Cobb Douglass function for production because it has convenient properties -- you can control the degree of degree to return to scale simply by varying alpha and beta. secondly alpha turns out to be under the marginal productivity interpretation of income and optomixation, it ends up being the share of income that goes to capital and to labour- ends up being the elasticity of capital with respect to labour or elasticity of capital with respect to labour.. very much a funcitonal form in line with the scaling  appraoch and they come to the same.. -let's you talk about your returns to scale naturally and ascribe them as you will. it also -- really convenient form used throughout econ for illustration, it has been estimated, and it can be derived alternatively as the form you can look for from the scaling research..]
*** XYZ properties,

cobb douglass has another trait, which is it's a constant elasticity of substitution fuction.
elasticity of subsstittuion combines the slope and the amt of inputs. . easy to slow graphically CEF - constant elasticity of substitution.




Growth theories have several components
the amt of each you have

e.g. if an economy increases the labour or capital it has, it can move to a higher isoquant. if it increases 1 it can move to a higher isoquant-- that doesn't change the shape-- you can move since you have more inputs.

if you have technical change that makes 1 factor more productive.. imagine a nice smooth isoquant.. and it takes some labour that puts you on that isoquant.. if the labour got more producteive, the same amt of labour would move you to a higher isoquant.. cou..


Solow swan puts tech to the front-- the prefactor containts tech growth

we actually put the factor of production into the prefactor.. ** 



% Note on functional form: The analytical approach looks at marginal product and curvature of these functions -- so all we need is something with a curvature and a marginal product in each factor that you put in -time capital, labour 1, labour 2, labour 3.. our focus is on the local curvature and the slope.
%(\cite{Solow56, Swan}), 


VARIATIONS

There are all kinds of other things that could be included in Solow-Swann. For instance, the Mankiw–Romer–Weil version of the model adds a term for human capital.

The general form can be extended in many ways 0 
 In principle it could be restrictred offer time, different workers, firms, sectors, neighbourhooods, evolving over time.


\subsection{Exploitation}\label{Sec:Exploitation: A Note}
The division of the product of the earth among the classes of society has been a central issue in economics since at least Ricardo presented his theory of  rent, through Marx, adapted the concept of rent to an industrial and capitalist economy and Henry George, who applied it in the urban context. Land rent is also at the core of modern urban models. 
%John \textcite{RoemerGT} in \textit{A General Theory of Exploitation and Class} examined a General equilibrium linear economy in which all individuals rationally choose their  activities given their initial endowments and demonstrated  the endogenous emergence of a class structure in a purely neoclassical model. 

%\textcite{RoemerGT} demonstrates that in equilibrium there are classes that are exploited and classes that are exploiters, as well as intermediate cases, using  a definition a definition of exploitation\footnote{\cite{RoemerGT} p12} that is essentially Marxian and is consistent with Ricardo's rent theory and that of Henry George: 
\begin{quotation}

An agent is exploited  if and only if the value of the labour the agent sells plus the value of own production plus wage earnings is less than the maximum value of their consumption bundle.\vspace{.25cm}

and\vspace{.25cm}

An agent is an exploiter  if and only if the value of the labour the agent sells plus the value of own production is less than the minimum value of their consumption bundle.
\end{quotation}

Class position is shown to depend on initial endowments. 
Our model also reveals classes that capture surplus generated by the labour of others. In our model the process is driven by agglomeration economies and urbanization. The analysis does not depend on whether we accept the notion of exploitation presented by Roemer, Marx or any other theorist. We simply note that the model generates classes in a well defined sense, and a division of that surplus among those classes that can be understood  as exploitative in the  classical sense. 







\subsection{Facts justifying our model}

The New Geography of Jobs - what's scarce is tallent. There is catastrophic agglomeration and a growing divide
and those in the good parts cant even afford to own that future, and those outside have no claim even on the income, joy or the chance at a family it offers.

It is debt farmed for ideas till it sheds its' need of you. 
The. actuall needs of humans could sputter out. but what is happening now is simply the draining

the continued logic of extraction.
Tbe logic- it can go somewhere else so ti flows around till place is destroyed. 

the other logic is investment. it looks amost the same. still uses prices, choices, the adjustment so it has alittle more humanity than the naked logic of the individual. -- still spend alone or together. still tu

but the more that stays local the better. and a certain amount stll builds up== the principle builds- the human expression of capacity--

-** all value is craeted for itself. except alienated value. Marx believed the alienation was endemic, but there's a more basic shift in what is needed-- we are needed, at least a few of us as full humans.

--- it is unpredictable

Winner take all-- means everyone gets the best thing. 

The problem of distribution has become the big thing
wealth is connection-- it is the productivity of the dense clusters.. 
Every urban boundary iatrs a potential. we could simply cross the line and install a new logic

1. you'd need a mechanism to drive density- to fight nimbyissm -- the inexorable drive of the markt to unlock an unwillign strivign 
and 
2. you'd need

can we choose a better world ourselves or must we be trappe din it.

of course allow as many other worlds to flourish. most of the world will be free for other worlds so that is no problem.. not even a contracdiction. a new wildness will be on offer, an enw pastoralsim an solw travel airships. 

I have dreamed a few times and cast my webs. played in the eddies of the dancing shimering world as it comes into being. I was at blackberry when it was everywehre, walked in the washinton where it was thew first thing each reached for in the morning and dreamed of the clean lines of the screen ot the edge, the ocmputer. and then held that very thing I'd dreamed of in my hands,

the edge of the inexorable world. anyone can go there and dream and some rutheless one can claim a little share of it.
most will be harvested by the little spiede3rs from law school hwo sit in the corners and and pounce with words never met from it
reality- they are the matrix keepers-- they keep oru constraints in place, they ony thing that takes out enough of our free choice that wecan be predicted and better managed a little

-- it is only the unexpected that is not claimed.
the ghosts of bubbles past.
the hangofers
the feelings never felt

if death iddn't exist we'd invent it
if place didn't exist we'd invent it

we just wiped out something we now have to cojure up but it is just place. a simple thintg not like the dead dancing spiders who well never come back on the odle ear= who will never danc again.
not like the spiders who will never dance in their colours again

come to the museum and see it
be hungry for the flavors adn spices of the world-- see the world -- meet people all over the world killthem.
what can you see, what can you take.

the flowering of the colision
and how few survive it.

but we in the wake of that exploseion. the attom unlocked bust now invent place.

cities were privates.

they are only freed for a moment when they realize that the free mind creates. but it is in a box.

free us all and you can only imagine what we'd make

the pale impovrished publics. the mall is is just a mockery of the public world-- fo the genuine quality of the backyard gathering. 

'Global inequality is not natural, inevitable, or accidental' Jason Hickel The Divide: Global Inequality from Conquest to Free Markets - give aid, they'll get institutions and get rich, but it's a lie of course.
The division is structural.

Have your diversity clubs, but they still cut to the bone, and drive you till your dead. You already adopt an alienated language- runing things up the flag pose. Cliches you'd never dream of in scohol. And so it is just one more thing to control your eyes when the races flash before you on the 'race' tests, just one more false alienated language to adopt. 
And it is all helpsless when it -
and you run farther and farther and still own nothing. Just another day older anddeeper in deblt
But we've sung labour songs for generations. And what we sang is in our bones, it will take more than these false words to cut it off of us. 
I don't want words they say.
I want a way out. I want hope. But even hope smarts, too fresh a promise broken. I cannot vote for hope again, though I cry every time it wins. Another neo-liberal lie

liberalism is this grand dream of liberty and hope. 
But if you don't have the guts
the only dream that cant be co-opted is one that wins.
And it never even tried to win. It never even cared. 


So yes dream of liberty. Drean of truth, and the skyline of forever. But don't dry to me if walking on dreams takes you no-where real.
but don't hit the dreamers.
Hit the liars and the theifs who never gave it a chance.
The ideal was so magical-- so dense, that the worst among it hid in its vabours. 

so densse, lofty diaphanous. Who tood up it's words.
Words. Words. 
The wordy will take them first.

The anti-racist neoliberal beast.
The xx family was the only one to ever embrase the entreprenru.

Great ideas have been peformed. 
And who knows if that's all constatintoble did when he saw the x in the sky and united gentile and jew in the greatest army him history, and binding our god in the. 
But she's free now. The Volcano brought her out. 
And so maybe god is dead. I tried to read him and could get nothing but an empty space.
IT took some time. 
But it is easiery to predict history
But it is easier to predict the future than to say when it will happen

Maybe we can wipe clean our failures now. 
Maybe that slow adoption is ready to speed up.

An emergency is not done by giving up before our time has come. 
Can you make this an experience. 




\subsection{There is an urban rent premium/scaling effect}
There is an urban wage premium %\textcite{HirschJahn} observe that, ``Following \textcite{GlaeserMare}   a  large  empirical  literature  has  investigated differences in wages across labor markets of different sizes. The general finding of this literature is that a significant urban wage premium exists. and that this premium consists both of a level effect and a growth effect that arises as workers gain urban work experience''.

applies most in tech economies like waterloo (maybe also with univerity,   info econ)

applies most with strong urban boundary -- functions essentially as a potential-- could grow with density- urban grows s ti grow the potentieal as well s the wealt-- adds a resilience benefit-- that's why exploiters sees to suc out

scaling laws- socioeconomic outputs scale for the reasons we say.


\subsection{Henry George Results}
Analysis of the locational rents generated in this class of models has resulted in several authors demonstrating the validity of variants of the Henry George Theorem (\cite{Arnott-Stiglitz79, Arnott04, BehrensKanemoto14, JohnM.Hartwick1980THGR}). These analyses  show that the land rents can exactly equal the cost of the public good that draws individuals to the city or the production services that draws firms. In our model these rents are extracted by land-owning financial capital. They are not invested in a public good or in expanded production capacity.

\subsection{Modelling Notes}


simple model justified when there are unknowns % - consider largest uncertainty not just largest data volume - highly available data in part of a model can tempt to complicate a model.. model simplicity should be constriande by the least of data rather than the most. . tend to use all the data we have can unballance

\subsection{How our model compares with other models}

- an origin story, and a test bed-- a grounding that ties it rigorously with the neoclassical model-- 


contrib - a novel intervention, based on a novel theory, recognizing an unusual opportunity- within this region.

contrib - It appears that the analysis of  agglomeration effects has not explored what the endogenous growth literature has to offer.

Production is by firms at the center. The production economy is what generates a surplus.  % This means we have a production/production economy. %, while many models of wealth distibution look at capital and financial markets, 
- vs lots of econo physics models looking at trades/markets rather than production.

Land plays a role in production, but not as a formal factor in the production function, but turns out to be a limit on output. The transportation cost ties labour to land. 

With agglomeration, firms produce more by being near more people.

We put the agglomeration factor in the bracket with labour because agglomeration scales the produtivity of workers.
we actually put the factor of production into the prefactor

The scale litturature comes to the same model with a different derivation, the relationship is traced in Section \ref{Sec:Scale} on Scale. 

 CUT? This differs from  the Slow-Swan, model in which labour augmenting technical change increases according to an exogenous (exponential) - but it's equivalent to the scaling law.

Notice that this model ascribes the agglomeration effects to labour rather than capital. Deepening  and widening of the labour pool was one of Marshall's explanations of the formation of industrial districts. 
The model can therefor  be seen as incorporating a Jacobs/Marshall externality (\cite{Beaudry:2009ua, Panne:2004vb}) of the sort often invoked as an explanation of industrial clusters. 
These externalities  are not a product of any firm or individual, they come from the social interaction of many people.

\subsection{Pictures of the model}

% In Figure~\ref{Fig:Rent1} 
the blue line is a conventional urban rent profile. $A(0)$ represents the effect of including a consumption externality as we do later.  For  $A(0)=0$ the orange line and orange block disappear. 
The social surplus generated by agglomeration effects in production appear as the white triangle below the blue line. The social surplus generated by agglomeration effects in consumption appear as the difference between the blue line and the orange one.
%\begin{figure}[htbp]
%\begin{center}
%\input{SA_RentProfileClasses.tex}
%\caption{Rent profile and population segregation with amenities}
%\label{Fig:Rent1}
%\end{center}
%\end{figure}




\subsection{***Initialization}

% We explore the wealth forming dynamics of the urban agglomeration effect by modelling % a city in which 
Agents work in the city and leave their jobs when they reach retirement age.
New agents enter the city to work.

% There is an outside world in two respects. NEAT 
 
 If there is a housing market, agents can move. %In the analytic case above, the population stays in place, and travels to work if it is worthwhile given the transportation costs. 
%Those who come to the city will be those for whom the benefits the city offers make it worthwhile to  whether that's building their network, accessing markets, accessing amenity, learning, finding specialized employment, or increased wages. In this model 
The demand for labour drives urban growth. % The housing market depends on how many people from the periphery are completing to claim places in the city. TODO IF ANY RURAL AGENT COULD MOVE IF THE CITY HAS ADDITIONAL DEMAND FOR LABOUR, HOW DO WE DECIDE WHICH DO? Could use a parameter for immigration (or how 'hot' the market is) and in the simplest case (corresponding to how many agents from outside are looking at the housing/rental market), have the inflow match. % Agents have debt and there is an undifferentiated labour market


Workers can leave the workforce and retire, and new people can come into the city. 

Initially, all homes are owner-occupied. This is of interest as a starting point because we are interested in the evolution of a society with widespread home ownership in the face of financialization pressures. 

We also start with owners who have a long run before retirement. The housing market is  during the initialisation runs to allow the wage and population to converge. The all-ownership, no-turnover case provides a basis for expectation-formation and examining the basic population-productivity dynamic. 

When agents reach the end of their set working lives they retire. %It is easy to introduce random life events, differentiated households and other variety, but while this kind of variation is likely to generate  complex and interesting life stories it is not what we choose to focus on at this stage.  
Owners may sell their homes and either leave the city or move to cheaper housing. the latter would put pressure on housing supply. New entrants to the labour force come from outside of the city at various stages in their life cycle and either buy or rent homes. 

\section{ANALYSIS - MOVE}
\epigraph{“An economist is someone who says, when an idea works in practice, ‘let’s see if it works in theory.'”
— Walter W. Heller, former chairman of the Council of Economic Advisers.}{ Walter W. Heller, former chairman of the Council of Economic Advisers. (\href{https://quoteinvestigator.com/2015/08/30/practice/}{1979, The Washington Star)}}


The model shows how land rent is captured by landowners and how that affects wealth creation and the development of the city. 
Rents go to landowners. %the owners of a given property. 
Landowners therefore capture a fraction of the wage premium generated by agglomeration.
If workers own their own homes, rents go to them. If others own the land, they capture them.

Land value rises as the city grows, so newcomers pay more for housing near the centre.

% Agglomeration benefits get extracted by landowners. Labour gets only their marginal value they don't get any of the surplus. They don't even keep always their marginal value.
The dynamic story is that the class of landowners eventually becomes financial capital.
PLOT RENTS HERE
The value of land increases over time. Those who purchase land earlier claim a share of the growing value of the city. % As the city grows, they own an increasingly valuable asset.
 
%In this model, workers are the initial owners, but they build this wealth which becomes a source of capital that can support them.
EQUATION FOR THE SHARE THEY CAN CAPTURE

In the case in which individual workers purchase houses and then sell them on retirement, the housing market drives the creation of classes on its own. A strictly random process in which agents have a range of ages and sell at retirement creates a structural advantage where workers who arrive earlier in the city and own land, benefit from their own labour but also get to claim a share of the productive output of the city as it grows. % those who begin work later. % to a division in wealth
%the emergence of a class of those who came early and those who came late.
%Early agents may also rent out their land. Could it be though of as a pyramid scheme?

%With financialization, in the case where 
If financialized buyers can access a better interest rate, they can consolidate ownership, capture rents, drive class differentials, and amplify wealth inequality. % This appears to be the case as lenders offer wealthier and larger entities lower interest rates. % We expect to observe in this class of models larger, likely power law-distributed, wealth effects.

%There is a supply of money- if there's too much for other investments, some will flow here- e.g. excess liquidity.

\subsection{Implications}
- totaly tax maintained
Germany higher natural redistribution..

This fits empirical results of increasing financialzaiton 70+ financialized
Fr\'ed\'erick Demers found that the response of housing investment to interest rates has become more pronounced over time \cite{demersModellingForecastingHousing2005}. %Modelling and Forecasting Housing Investment: The Case of Canada,  Research Department, Bank of Canada, Ottawa, Ontario, Canada K1A 0G9 fdemers@bankofcanada.ca



Friedman’s 1953 essay, ``The Case for Flexible Exchange Rates'' \cite{friedmanEssaysPositiveEconomics1953} argued that speculation is stabilizing. Market prices in this view  are set on the basis of economic fundamentals- what we call the warranted price. When prices diverge from those fundamentals that creates a profitable opportunity. Speculators then step in and buy or sell, driving prices back to the level warranted by fundamentals.

Increasing the number of traders and volume of trading is also regarded as improving financial market outcomes. Increased trade volume increases market liquidity 



\section{Intro conclusions/summary}
The model has a Solow-Swan style production model with agglomeration effects using a \gls{Cobb-Douglas} production function that incorporates Jacobs-style labour-augmenting agglomeration economies %(Beaudry and Schiauerova 2009, Panne 2004, J. Jacobs 1969), 
in the way neoclassical growth theory incorporates labour-augmenting technical change.
It integrates the production function with an Alonso-style urban model of a city economy \cite{alonzoTheoryUrbanLand1960}.


\chapter{GATHER}
\section{Housing crisis in popular media - From wikipedia article on housing bubble March 7}


Haber, Bob. "Canadian Real Estate Bubble Blowing Up North." Forbes, Forbes Magazine, April 3, 2018, www.forbes.com/sites/bobhaber/2018/04/02/canadian-real-estate-bubble-blowing-up-north/#1b74d3871d5e.
Tencer, Daniel (October 3, 2018). Canada At Risk As 'First Cracks' Appear In Global Housing Bubbles: UBS., HuffPost (Canada edition)
"So Where Is the Next House Price Bubble Brewing?". Bloomberg.com. July 30, 2019. Archived from the original on December 14, 2019. Retrieved June 25, 2021.
"World's Bubbliest Housing Markets Flash 2008 Style Warnings". Bloomberg.com. June 15, 2021. Archived from the original on June 15, 2021. Retrieved June 25, 2021.
"Speculation by Canadians 'absolutely' playing a role in red-hot home prices: expert - National | Globalnews.ca". Global News. Retrieved April 27, 2022.
Castaldo, Joe. "How Canada's Real Estate Market Went Completely Insane." Canadian Business - Your Source For Business News, July 10, 2017, www.canadianbusiness.com/economy/how-canadas-real-estate-market-went-completely-insane/.
Andrews, Jeff. "Canada's Housing Bubble Is Starting to Burst." Curbed, Curbed, March 7, 2018, www.curbed.com/2018/3/7/17085794/canada-housing-market-collapse.
Tencer, Daniel. "Canada At Risk As 'First Cracks' Appear In Global Housing Bubbles: UBS." HuffPost Canada, HuffPost Canada, October 3, 2018, www.huffingtonpost.ca/2018/09/29/toronto-vancouver-have-world-s-3rd-and-4th-largest-housing-bubbles-ubs_a_23544956/.
"Ontario's Fair Housing Plan." News.ontario.ca, news.ontario.ca/mof/en/2017/04/ontarios-fair-housing-plan.html.
"Canada's Housing Market Still 'Highly Vulnerable' despite Easing Prices, CMHC Warns." Financial Post, October 25, 2018, business.financialpost.com/real-estate/prices-easing-but-canadas-housing-market-still-highly-vulnerable-cmhc.
Alini, Erica. "Will It Crash? Here's What to Expect from the Canadian Housing Market in 2019." Global News, Global News, December 2, 2018, globalnews.ca/news/4688308/canada-housing-market-outlook-2019/.
Khan, Mikael; Webley, Taylor (April 2019). "Disentangling the Factors Driving Housing Resales" (PDF). Bank of Canada. p. 8. Retrieved May 13, 2019.
"Investors account for a fifth of home purchases in Canada. Are they driving up housing prices in a booming market?". Retrieved June 23, 2021.
"Toronto condo, apartment rental prices drop again amid ongoing coronavirus pandemic". Global News.
"Canada Tried To Stop Real Estate Prices From Falling, And Created A Bigger Bubble". Better Dwelling. March 5, 2021.
"CMHC Warns Canadians: "Support For Homeownership Cannot Be Unlimited"". Better Dwelling. May 20, 2020.
"Bank of Canada Governor Tiff Macklem sees "some signs of excess exuberance"" – via www.youtube.com.
"BOC Governor Says Canada Will Lean On Real Estate Because "We Need The Growth"". Better Dwelling. February 25, 2021.
"Bank of Canada Wants A Housing Bubble, While Other Central Banks Try To Pop Them". Better Dwelling. March 9, 2021.
"Bank of Canada governor strongly hints that benchmark rate will stay for now". Radio Canada International. February 24, 2021.
https://money.usnews.com/investing/news/articles/2021-03-08/bank-of-canada-expected-to-resist-investor-expectations-of-early-rate-hike[bare URL]
Smith, Fergal (June 20, 2021). "Bank of Canada to break sequence of lower terminal rates as governments splurge". CTVNews. Retrieved June 25, 2021.
"Best communities in Canada: Why Atlantic Canada comes out on top". April 8, 2021.
"Rising interest rates will be 'No. 1 issue' for Canada's housing market, economists say". Global News. Retrieved June 25, 2021.
"Canada Housing Market Crash | Fast-rising Borrowing Costs Predicted to Deepen Slump".
Hogue, Robert (February 15, 2023). "Home prices still dropping across Canada". RBC Thought Leadership. Retrieved February 22, 2023.
https://edition.cnn.com/2023/01/01/business/canada-bans-home-purchases-foreigners/index.html
"How immigration and an aging population will affect Canada's housing market". financialpost. Retrieved February 22, 2023.
Barcelo, Yan. "Canadians: Don't Rush To Buy a House". Morningstar CA. Retrieved February 22, 2023.
"What $500,000 buys in today's Canadian real estate market". vancouversun. Retrieved July 18, 2021.
"Liberals need a plan to tackle housing - or they risk alienating young Canadians". Newsrooms. May 25, 2021. Retrieved May 25, 2021.
McNutt, Lydia (December 1, 2020). "Canadian Housing Market Outlook (2021)".
"Vancouver housing prices stay hot even as market cools: Royal LePage report". vancouversun. Retrieved July 18, 2021.
"Cullen Commission". cullencommission.ca. Retrieved December 20, 2020.
"Vancouver model for money laundering unprecedented in Canada, B.C. inquiry hears". CBC. Retrieved December 20, 2020.
"Canadian Real Estate Still Opaque, But Light is Creeping In". Transparency International Canada. Retrieved December 20, 2020.
"Billions In Toronto Real Estate Bought Anonymously, With Funds of Unknown Origin". Better Dwelling. March 21, 2019. Retrieved December 20, 2020.
Carmichael, Kevin (February 1, 2021). "While the world is in the midst of a tech revolution, Canadians bet on real estate". Financial Post. Retrieved April 27, 2022.
Evans, Pete (June 18, 2021). "Canadian mortgage debt grew by $18 billion in April, biggest monthly gain ever, StatsCan says". CBC News. Archived from the original on June 18, 2021.

    Kartashova, Katya; Zhou, Xiaoqing. "How Do Mortgage Rate Resets Affect Consumer Spending and Debt Repayment? Evidence from Canadian Consumers" (PDF). Bank of Canada. Retrieved July 18, 2021.

    vte

Financial bubbles
Categories:

    Economic history of CanadaReal estate bubbles of the 2010sEconomy of CanadaImmigration to CanadaMoney launderingHousing in CanadaInflation in CanadaInfrastructure in CanadaReal estate in CanadaImpact of the COVID-19 pandemic in Canada

