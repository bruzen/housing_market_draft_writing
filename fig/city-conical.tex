
\begin{tikzpicture}[scale=.5]
   %%%%%%%%%%%%%%%%%%%%%%%%%%%%%%%%%%%%%%%%%%%%%%%%
% definitions for schematic
\def\bndmax{5}        %https://tex.stackexchange.com/questions/68462/filling-a-complex-region-with-tikz
\def\bndmin{0.2}
\def \n {10}  % height of y axis
\def \d {12}  % length  of x axis
\def \t {.75}  %  cost of transportation per unit x
\def \th {1}   % theta?
\def \w {7}    %  wage premium
\def \om{1.5}%  omega =rural wage Zero for urban population
\def \azero{2}
\def \aprime {-.0}	
\tikzset{func/.style={thick,color=blue!90}}	

    %%%%%%%%%%%%%%%%%%%%%%%%%%%%%%%%%%%%%%%%%%%%%%%%
% definitions for Cone3
%\node at (0, 2.5){\input{SA_Cone3.tex}};
     \pgfmathsetmacro{\radiush}{9.7};%Cone base radius was 9.6
        \pgfmathsetmacro{\theight}{7.1}%Cone height (negative if you want a inverse cone)
        \pgfmathsetmacro{\cheightp}{.03}%Cut height in percent of cone height

        %Calculating coordinates
        \coordinate (center) at (0,0);
        \pgfmathsetmacro{\radiusv}{.2 * \radiush}; %HORIZONTAL RADIUS
        \coordinate (peak) at ($(center) + (0,\theight)$);     
        \pgfmathsetmacro{\sradiush}{\radiush * (1 - \cheightp)};%ADJUST FOR COVERAGE AT CORNERS
        \pgfmathsetmacro{\sradiusv}{.2 * \sradiush};
   %     \pgfmathsetmacro{\sradiusv} {\sradiusv -.1 };

\coordinate (antipeak) at ($(center) + (0,-\theight)$);  %thanks  %I added this
\coordinate (vert1) at ($(center)+(\radiush,-.2)$);
\coordinate (vert2) at ($(center)-(\radiush,.2)$);
%problem
   
\coordinate (svert1) at ($(vert1)!\cheightp!(peak) +(0.1,.75)$);
\coordinate (svert2) at ($(vert2)!\cheightp!(peak)+(.5,.75)$);  
    % \coordinate (svert3) at ($svert1+(0,\w)$);
    % \coordinate (svert4) at ($vert2)+(0,\w)$);  
    %  \coordinate (svert3) at ($svert1+(0,7)$ );  % Shifting up by W
    % \coordinate (svert4) at ($svert2 + (0,\w)$0;
   %%%%%%%%%%%%%%%%%%%%%%%%%%%%%%%%%%%%%%%%%%%%%%%%


 
%\draw[step=.5,black,thin] (-9.6,0) grid (9.6,7);
 
% Cone Drawing    
 \fill[ left color=red!70, right color=red!70,  opacity=20,middle color=red!20,shading=axis] (svert1) -- (peak) -- (svert2) arc (170:370:\sradiush cm and \sradiusv cm);

    % FAT GREEN BAR
 \draw [fill=green,opacity=80] (-.2, 0) rectangle(.2, \w);
 \node[above] at (0,\w){$\omega$};
 
%Uncomment this for top of cylinder
      \fill[inner color=gray!2,outer color=gray!40,shading=radial,opacity=.5] ($(center) + (.35,\theight)$ ) circle (9.4 cm and 1.55 cm );
      
        % \draw [thick]($(svert1) +(.3,-.3)$)-- ++ (90:\w-.2);
        % \draw [thick]($(svert2)-(.2,.3)$)-- ++ (90:\w-.2);
        %Lines, \h in percent of cone height
 def \sradiusv2 \sradiusv cm -.1 cm)
% Cylinder drawing
  \fill[ left color=black!50, right color=red!30,  middle color=red!30,shading=axis,opacity=.2]  (-9.05,.5) 
  arc (180:360:\sradiush cm and \sradiusv cm)-- ++(90:\w-.2) 
  arc (360:180:\sradiush cm and \sradiusv2 cm -.1 cm)--(-9.05,.5);  

   \node[above] at (0,\w){\Large $\omega$};
% TRY TO Make a cylinder
%\draw ($svert2 + (0,\theight)$) [arc (180:360:\sradiush cm and \sradiusv cm)]; 
%     \fill[left color=gray!70,right color=gray!70,middle color=gray!30,shading=axis] (vert1) -- (svert1) arc (0:-180:\sradiush cm and \sradiusv cm) -- (vert2) arc (180:360:\radiush cm and \radiusv cm);

% DASHED LINE AT BACK OF CONE
\foreach \h in {0.03}{   %.38,.34,.30, .7
            \pgfmathsetmacro{\rh}{-\radiush * (1 - \h)}
            \pgfmathsetmacro{\rv}{.2 * \rh}
            \draw[black!70,densely dashed] ($(svert2)!\h!(peak)-(.3,.9)$) arc (370:170:\rh cm and \rv cm);%$(vert2)!\h!(peak)$)
        }
  %      \draw[opacity=.90, line width=.05cm, green] (0,0)--(0,{\theight - .05});
%     \foreach \h in {0, .38,.34,.30, .7}{
%            \pgfmathsetmacro{\rh}{\radiush * (1 - \h)} %            \pgfmathsetmacro{\rv}{.2 * \rh}
%            \draw[black!70,densely dashed] ($(antipeak)!\h!(vert2)$) arc (180:360:\rh cm and \rv cm);
%   }
%  \draw[red] (antipeak) arc (30:60:3);
%  \draw[dashed, thick] arc (0:-180:\sradiush cm and \sradiusv cm) -- (vert2) arc (180:360:\radiush cm and \radiusv cm);
%%%%%%%%%%%%%%%%%%%%%%%%%%%%%%%%%

% %\foreach \xi in {0,..., \n} \draw (\xi,0)--(\xi,-.1)node[below=1]{\small$\xi$};
% %\foreach \yi in {1,...,\n} \draw (0,\yi)--(-.1,\yi)node[left]{$\yi$};
% %\foreach \i in {1,4,9,16} {
% %\node at (7,-\om/2){people scattered uniformly across the land  };

% %SECOND FIGURE WITH AGGLOMERATION WAGE
% %  add urban production and net wage
% %\draw[fill=white, white] (0.1,-0.1) rectangle (14,-\om+.1);

% \node[right, text width=4cm] at  (3, \w+1){Added Productivity due to agglomeration};
% %\node[right, text width = 3cm] at  (10,9){Where does the increase in productivity come from?};
 \draw [ thick, ->](0,0)--(2.5, 0)node [right] {\Large $d$};


% \draw[thick] (0,0) -- ++ (50:2.6cm);  %   diagonal for perspective
% \draw[thick] (0,0) -- ++ (230:2.35cm); 

% %  THIRD FIGURE  add RENT profile in blue

% %\node[right, white, fill=white,  text width = 3cm] at  (10,9){Where does the increase in productivity come from?};
% \draw[func, domain=0:\w/\t+1,ultra thick] plot [samples=200] (\x,{\w-\t*\x}); %Net wageprofile  for 
% %\node[right, white, fill=white] at  (.25, \w/2){Added Productivity};
% %\node[right, fill=white, text width =3.5cm ] at  (1, \w/2){Declining wage  net \\of transportation\\ costs $T(d)$ };
% %\node[right, fill=white, text width =3.5cm ] at  (4,9){Declining wage  net \\of transportation\\ costs  };
% %
% %\node at (0, 1.5){\includegraphics{\input{SA_Cone3.tex}} };
% %\node at (0, 2.5){\input{SA_Cone3.tex}};

% %   FOURTH FIGURE     commuters
% %\pause
% %\draw[fill=blue!40] (0.1,-0.1) rectangle (9.2,-\om+.1);
% \node at (4.5,.4*\om){commuters};


\end{tikzpicture}