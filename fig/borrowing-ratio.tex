\begin{tikzpicture}[scale=.5]  % Individual borrowing ratio mi as a function of wealth (in tenths)
% \def\bndmax{5}        %https://tex.stackexchange.com/questions/68462/filling-a-complex-region-with-tikz
% \def\bndmin{0.2}
\def\Y{10}  % height of y axis pecent
\def\y{1} 
\def\W{20}  % length  of x axis
\def\Wbar{4}
\def\scale{0,...,1}
\def\rbar{8}% this is the prime rate
% Equation   \[ r_i = (A + .5 \frac{\bar{W}}{W_i})\omega\]
% \def\Wmin{.63}  %This sets the lower limit fo the 
\def\Wmin{(\B*\Wbar)/(\Y/\rbar-\A)} %function to keep in in bounds
\tikzset{func/.style={thick,color=blue!90}}	
% BOX
\draw [](\W,\Y)-- (0,\Y)node[above=.5cm, left]{$m_i$} -- (0,0)--(\W,0)node[below]{Wealth}--cycle;  	% Axes box

\draw [dashed, thick] (0,\rbar) -- (\W,\rbar);  	% Axes
\draw [thick,dashed] ( \Wbar,0)node[below=.5cm]{$\bar{W}$} -- (\Wbar,\Y);  	% Axes

%\foreach \i\j in {1/A,2/B,3/C,4/D,5/E,6/F}
\foreach \yi\s in {0/0.0, 1/0.1,2/0.2,3/0.3,4/0.4,5/0.5, 6/0.6, 7/0.7, 8/0.8, 9 /0.9, 10/1.0} \draw (0,\yi)--(-.1,\yi)node[left]{\footnotesize$\s$};
% \foreach \yi in {0,2,4,6,8,10} \draw (0,\yi)--(-.1,\yi));
% node[left]{\small$\yi$};
% \foreach \yi in {0,2,4,6,8,10}node at (-.1,yi) {{10*yi}} ;
\draw[func,domain=0:\W] plot [samples=200] (\x,(9-\x^.5/2);
\end{tikzpicture}