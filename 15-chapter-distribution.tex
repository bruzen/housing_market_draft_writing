\chapter{Distribution and Growth} \label{chapter-distribution}

\epigraph{A large body of literature documents the existence of agglomeration economies in developed economies (see Rosenthal and Strange 2004 for a review). The main conclusion of this literature is the finding of scale economies of 3–8 percent (that is, a 10 percent increase in the size of an activity in a city raises productivity in this activity by 0.3– 0.8 percent).}{Gilles Duranton  \cite{durantonAreCitiesEngines2009}} 

The notion that your labour force is on average more productive when there are more people around is pretty dramatic and it's very much not part of the basic model that we use. Our starting point is that's the fundamental feature of cities, and what does that do with financial capital and what does that do to distribution and that's not been explored. 

The concentration of wealth is associated with slower econ growth.  There's a possibility that somehow the city will grow less.
{\color{red} We can speculate on how the distribution. of wealth directly affects the scaling parameter $\beta$. This probably calls of a discussion of the paths through which agglomeration works } 



A growing body of research on cities  has demonstrated  increasing returns to  population. In our chapter on growth theory we showed that neoclassical growth theory associated these increasing returns to scale at the urban or national level with increasing effective human capital, which grows faster than the labour supply as a result of increased education or agglomeration effects. The two lines of research have settled on what might be termed a  common ``stylized  fact'': population and output are related according to 
\begin{equation}
    Y=AN^\beta,\qquad \beta>1 \label{equation-the-fact}
\end{equation}
% can be expressed as Y (\lambda N)~Z(\lambda,N)Y (N). When the scale factor Z depends only on \lambda, i.e. Z(\lambda,N)~Z(\lambda), equation (2) can be solved uniquely to give the scale-invariant result of equation (1), with Z(l\lambda)~\lambda^\beta.
Empirical estimates of $\beta$ vary considerably\cite{rosenthalEvidenceNatureSources2004}\cite{bettencourtIntroductionUrbanScience2021} \cite{loboUrbanScalingProduction2013} 
% from LOBO et al:
%18. Carlino GA (1979) Increasing returns to scale in metropolitan manufacturing. J Reg Sci 19: 343–351. 
% % 19. Moomaw RL (1988) Agglomeration economies: localization or urbanization? Urban Stud 25: 150–161. 
% 20. Beaudry C, Schiffauerova A (2009) Who’s right, Marshall or Jacobs? The localization versus urbanization debate. Res Policy 38: 318–337. 
% 21. Duranton G, Puga D (2004) Micro-foundations of urban agglomeration economies. In: Henderson, J.V Thisse, J.F., editors. Handbook of Regional and Urban Economics 4: 2063–2117. 
% 22. Puga D (2010) The magnitude and causes of agglomeration economies. J Urban Econ 50: 203–219. 
% 23. Knudsen B, Florida R, Stolarick K, Gates G (2008) Density and creativity in U.S. regions. Ann Assoc Am Geogr 98: 461–478. 
% 24. Rauch JE (1993) Productivity gains from geographic concentration of human capital: evidence from cities. J Urban Eco 34: 380–400. 
% 25. Rosenthal SS, Strange WC (2001) The determinants of agglomeration. J Urban Econ 50: 191–229. 
% 26. Segal D (1976) Are there returns to scale in city size? Rev Econ Stat 58: 339350. 
% 27. Shefer D (1973) Localization economies in SMSAs: a production function approach. J Reg Sci 13: 55–64. 
% 28. Sveikauskas L (1975) The productivity of cities. Q J Econ 89: 393–413. 
% 29. Moomaw RL (1981) Productivity and city Size? A critique of the evidence. Quarterly Journal of Economics 96: 675–688.
% 
and the literature has not yet offered an explanation  of the variation.\cite{loboUrbanScalingProduction2013} The source of the variation is a matter of considerable policy importance, however. If $\beta$ can be affected positively by policy choices, or perhaps negatively by trends that might be affected by policy, governments may be able to significantly increase social wealth and well being.\footnote{Interestingly, the residuals, or unexplained components for smaller cities are much larger than for large cities, suggesting that clues about potential policy will bee found by examining smaller and mid-sized cities and that potential policy impacts may be greater for these cities.}
%\footnote{Lobo et al are carful not to  claim that there is a causal relation between urban scaling and urban productivity. Causality, they say, stems from the ways in which being embedded inside larger agglomerations fundamentally affects how individuals interact with each other. Such micro-processses have not been demonstrated.}

Equation~\ref{equation-the-fact} is a long term relationship. Urban productivity can only evolve very slowly. The cross sectional studies to date are simply not able to identify the variable that might explain the varied results for  $\beta$. \textbf{In this chapter we examine a number potential  links between urban population and productivity that have been proposed and show how we  incorporate specific policy-relevant linkages into our model.}  Empirical verification will wait on the development of extended time series on urban output. The analysis of those time series, however, will require theoretical models that make $\beta$ depend on policy variables of interest,  

The literature has identified many possible channels that affect overall productivity, but has not, as far as we have been able to determine, settled on which  of these dominate or how to incorporate the multiple channels that have been identified. This gap  is not surprising. The literature on urban scaling is relatively new and, while neoclassical growth theory provides a convincing general explanation, it does not provide empirical evidence on precisely how the effects of agglomeration are transmitted. 

Our interest in this thesis is on how financialization might affect cities, and in this chapter specifically on the channels through which financializaition might affect the productivity and growth of cities.

 Lobo et al show that Equation~\ref{equation-the-fact} can be written   as an explicit function of population size, $N_i$, and specific local deviations, $\xi_i$ that they  call a Scale-Adjusted Metropolitan Indicator (SAMI). The SAMI they derive depends on local wages and capital costs. The  SAMI accounts for the residuals when $\beta$ is estimated. Their preferred form is
 \[Y_i =A N^\beta_ i e^{\xi_i^Y} \]
 ``As a consequence,''they conclude, ``any additional urban property proposed to explain a higher or lower productivity of specific cities not tied to their size (see below) must be expressed in terms of its contribution to the SAMIs for W, L, R and K.''  This appears to be a 
useful frame for our  search for of potential policy levers.  

(We immediately suspect that financialization make less local capital available and raises the cost of capital in the city.  Is this correct?)

(We also wonder i f we need to incorporate effective labour, which raises $L_0$. Cutting transportation costs will cut the necessary wage and increase the effective wage and productivity)



Tomaskovic-Devey points to
a shift in behavior of non-finance firms away from production and non financial services and toward financial investments and services. This shift, he argues  has reduced the bargaining power of labor and led to lower employment, income transfers to executives and capital owners, and increased inequality among workers.\cite{tomaskovic-deveyFinancializationCausesInequality2013}. This suggests to us that one specific channel  through which financialization will affect long term growth of cities is through reducing the effective wage. We set the long-term wage at the inclusive marginal product of labour. We could  mimic the effects of financialization by introducing a a gap between this value  and the wage paid. 

%\cite{rosenthalEvidenceNatureSources2004}
This gives 
A language to talk about a whole class of problems that are important/of rising importance in the literature but that this class of formal models has lacked the infrastructure/language/conceptual toolkit to explore. ** MAYBE ALSO MENTION THIS IN DOC INTRO/CONCLUSION
NEED TO SHOW A CONNECTION BETWEEN GROWTH RATE AND DISTRIBUTION - IN THE GROWTH CHAPTER.. -- OR RIGHT AFTER AS A NEW CHAPTER..

``Urbanization and growth go together: no country has ever reached middle- income status without a significant population shift into cities ''\cite{annezUrbanizationGrowthSetting2009}


\epigraph{Widespread urbanization is a recent phenomenon. In 1900 just 15 percent of the world’s population lived in cities. The 20th century transformed this picture, as the pace of urban population growth accelerated very rapidly in about 1950. Sixty years later, it is estimated that half of the world’s people lives in cities.}{Annez and Buckley\cite{annezUrbanizationGrowthSetting2009}}


Urban infrastructure has high social but low private returns. It is unlikely private capital will flow into urban infrastructure projects with substantial public support. That is the motive for ``\gls{Public-Private Partnerships}'' (PPPs) that involve private capital financing government projects and services up-front, and then subsidizing the private investors  from taxpayers out of the expected social returns over the course of the PPP
Annez and Buckley\cite{annezUrbanizationGrowthSetting2009} observe that 
\begin{quotation}
Britain’s cities were cleaned up only when the central government stepped in to alleviate the binding financial constraint in cities. In this story lies an important lesson about building urban infrastructure, especially those lumpy discrete investments in networks that expand the limits at which congestion costs outweigh agglomeration benefits. Neither the municipal finance systems that worked before the urban transition nor those suitable for cities in a demographic steady state will necessarily generate finance for investments in local public goods that more than pay for themselves in economic terms. 
\end{quotation}




%``As Quigley points out in chapter 4, the fundamental question in urban economics is why people voluntarily live in close proximity to one another when there are costs to competing for land. The simple answer has two parts: efficiency gains and consumption benefits. Recent theoretical and empirical work provides a sense of the nature and significance of these gains.'' Annez and Buckley\cite{annezUrbanizationGrowthSetting2009}% P  13 %see theie list