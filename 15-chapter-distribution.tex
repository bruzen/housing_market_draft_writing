\chapter{Distribution and Growth} \label{chapter-distribution}

\epigraph{A large body of literature documents the existence of agglomeration economies in developed economies (see Rosenthal and Strange 2004 for a review). The main conclusion of this literature is the finding of scale economies of 3–8 percent (that is, a 10 percent increase in the size of an activity in a city raises productivity in this activity by 0.3– 0.8 percent).}{Gilles Duranton \cite{durantonAreCitiesEngines2009}} 


%Cities are the engines of economic growth (Jacobs, 1969; Bairoch, 1988). It is in cities that a large share of the innova- tions and entrepreneurship takes place that fosters economic growth in the long run. Spontaneous Orders and the Emergence of Economically Powerful Cities. JOHANNA PALMBERG

A growing body of research on cities has demonstrated increasing returns to population. In our chapter on growth theory we showed that neoclassical growth theory associated these increasing returns to scale at the urban or national level with increasing effective human capital, which grows faster than the labour supply as a result of increased education or agglomeration effects. The two lines of research have settled on what might be termed a  common ``stylized  fact'': population and output are related according to 
\begin{equation}
    Y=AN^\beta,\qquad \beta>1 \label{equation-the-fact}
\end{equation}
% can be expressed as Y (\lambda N)~Z(\lambda,N)Y (N). When the scale factor Z depends only on \lambda, i.e. Z(\lambda,N)~Z(\lambda), equation (2) can be solved uniquely to give the scale-invariant result of equation (1), with Z(l\lambda)~\lambda^\beta.
Empirical estimates of $\beta$ vary considerably\cite{rosenthalEvidenceNatureSources2004}\cite{bettencourtIntroductionUrbanScience2021} \cite{loboUrbanScalingProduction2013} 
% from LOBO et al:
%18. Carlino GA (1979) Increasing returns to scale in metropolitan manufacturing. J Reg Sci 19: 343–351. 
% % 19. Moomaw RL (1988) Agglomeration economies: localization or urbanization? Urban Stud 25: 150–161. 
% 20. Beaudry C, Schiffauerova A (2009) Who’s right, Marshall or Jacobs? The localization versus urbanization debate. Res Policy 38: 318–337. 
% 21. Duranton G, Puga D (2004) Micro-foundations of urban agglomeration economies. In: Henderson, J.V Thisse, J.F., editors. Handbook of Regional and Urban Economics 4: 2063–2117. 
% 22. Puga D (2010) The magnitude and causes of agglomeration economies. J Urban Econ 50: 203–219. 
% 23. Knudsen B, Florida R, Stolarick K, Gates G (2008) Density and creativity in U.S. regions. Ann Assoc Am Geogr 98: 461–478. 
% 24. Rauch JE (1993) Productivity gains from geographic concentration of human capital: evidence from cities. J Urban Eco 34: 380–400. 
% 25. Rosenthal SS, Strange WC (2001) The determinants of agglomeration. J Urban Econ 50: 191–229. 
% 26. Segal D (1976) Are there returns to scale in city size? Rev Econ Stat 58: 339350. 
% 27. Shefer D (1973) Localization economies in SMSAs: a production function approach. J Reg Sci 13: 55–64. 
% 28. Sveikauskas L (1975) The productivity of cities. Q J Econ 89: 393–413. 
% 29. Moomaw RL (1981) Productivity and city Size? A critique of the evidence. Quarterly Journal of Economics 96: 675–688.
% 
and the literature has not yet offered an explanation  of the variation.\cite{loboUrbanScalingProduction2013} The source of the variation is a matter of considerable policy importance, however. If $\beta$ can be affected positively by policy choices, or perhaps negatively by trends such as financialization, governments may be able to significantly increase social wealth and wellbeing.\footnote{Interestingly, the residuals or unexplained components for smaller cities are much larger than for large cities, suggesting that clues about potential policies will be found by examining smaller and mid-sized cities and that potential policy impacts may be greater for these cities.}

\begin{figure}
    \centering
    \includegraphics[scale=0.25]{fig/Residuals-Lobo.png}
    \caption{Residuals from regressing ln(total wages) on ln(population) using data for all 943 urban areas of the United States smoothed over the 2009–2011 period. Source: Lobo et al \cite{loboUrbanScalingProduction2013}}
    \label{fig:Residuals-Lobo}
    
\end{figure}
%\footnote{Lobo et al are careful not to  claim that there is a causal relation between urban scaling and urban productivity. Causality, they say, stems from the ways in which being embedded inside larger agglomerations fundamentally affects how individuals interact with each other. Such micro-processses have not been demonstrated.}

Equation~\ref{equation-the-fact} is a long-term relationship. Urban productivity can only evolve very slowly. The cross-sectional studies to date are simply not able to identify the long-term effects of allk of the variables that might explain the varied results for  $\beta$. The literature has identified many possible channels that affect overall productivity. Neoclassical growth theory, for example, provides one convincing general explanation in terms of human capital, but it does not provide empirical evidence on precisely how the effects of agglomeration are transmitted. Researchers have not settled on which  of the possible influences dominate, what channels they work through, or what policies might improve the transmission of positive effects. These gaps are  not surprising. The literature on urban scaling is relatively new.  Empirical verification will wait on the development of extended time series on urban output.


The analysis of those time series, however, will require theoretical models that make $\beta$ depend on policy variables of interest,  In this chapter, we examine a number of potential  links between urban population and productivity that have been proposed, and discuss how we  might incorporate specific policy-relevant linkages into our model.  Our interest is in how financialization might affect cities, and in this chapter specifically on the channels through which financialization might affect the productivity and growth of cities.


\subsection{The Scale-Adjusted Metropolitan Indicator}
 Lobo et al \cite{loboUrbanScalingProduction2013} provide an  analysis that is a useful starting point for our  search for potential policy levers. They show that Equation~\ref{equation-the-fact} can be written as an explicit function of population size, $N_i$, and specific local deviations, $\xi_i$ that they  call a Scale-Adjusted Metropolitan Indicator (SAMI). The SAMI they derive depends on local wages and capital costs. The  SAMI accounts for the residuals when $\beta$ is estimated using population alone. 
 
 The model treats Equation~\ref{equation-the-fact} as a production function, as we do in Chapter~\ref{chapter-growth}.  Their preferred form is
 \[Y_i =A N^\beta_ i e^{\xi_i^Y} \]
 They work backward from this expression to a Cobb-Douglas production function that depends only on labour and capital inputs. ``As a consequence,'' they conclude, ``any additional urban property proposed to explain a higher or lower productivity of specific cities not tied to their size (see below) must be expressed in terms of its contribution to the SAMIs for W, L, R and K,''   where W and R are the wage and capital shares of income and  L and K are the wage and capital inputs.

(We immediately suspect that financialization make less local capital available and raises the cost of capital in the city.  Is this correct?)

 
 

(We also wonder i f we need to incorporate effective labour, which raises $L_0$. Cutting transportation costs will cut the necessary wage and increase the effective wage and productivity)

``\dots high productivity cities show invariably high wages and high levels of employment relative to their size expectation. Conversely, low productivity cities show both low wages and employment.''

Tomaskovic-Devey points to
a shift in behavior of non-finance firms away from production and non financial services and toward financial investments and services. This shift, he argues  has reduced the bargaining power of labor and led to lower employment, income transfers to executives and capital owners, and increased inequality among workers.\cite{tomaskovic-deveyFinancializationCausesInequality2013}. This suggests to us that one specific channel  through which financialization will affect long term growth of cities is through reducing the effective wage. We set the long-term wage at the inclusive marginal product of labour. We could  mimic the effects of financialization by introducing a a gap between this value  and the wage paid. 

%\cite{rosenthalEvidenceNatureSources2004}
This gives 
A language to talk about a whole class of problems that are important/of rising importance in the literature but that this class of formal models has lacked the infrastructure/language/conceptual toolkit to explore. ** MAYBE ALSO MENTION THIS IN DOC INTRO/CONCLUSION
NEED TO SHOW A CONNECTION BETWEEN GROWTH RATE AND DISTRIBUTION - IN THE GROWTH CHAPTER.. -- OR RIGHT AFTER AS A NEW CHAPTER..

``Urbanization and growth go together: no country has ever reached middle- income status without a significant population shift into cities ''\cite{annezUrbanizationGrowthSetting2009}


\epigraph{Widespread urbanization is a recent phenomenon. In 1900 just 15 percent of the world’s population lived in cities. The 20th century transformed this picture, as the pace of urban population growth accelerated very rapidly in about 1950. Sixty years later, it is estimated that half of the world’s people lives in cities.}{Annez and Buckley\cite{annezUrbanizationGrowthSetting2009}}


Urban infrastructure has high social but low private returns. It is unlikely private capital will flow into urban infrastructure projects with substantial public support. That is the motive for ``\gls{Public-Private Partnerships}'' (PPPs) that involve private capital financing government projects and services up-front, and then subsidizing the private investors  from taxpayers out of the expected social returns over the course of the PPP
Annez and Buckley\cite{annezUrbanizationGrowthSetting2009} observe that 
\begin{quotation}
Britain’s cities were cleaned up only when the central government stepped in to alleviate the binding financial constraint in cities. In this story lies an important lesson about building urban infrastructure, especially those lumpy discrete investments in networks that expand the limits at which congestion costs outweigh agglomeration benefits. Neither the municipal finance systems that worked before the urban transition nor those suitable for cities in a demographic steady state will necessarily generate finance for investments in local public goods that more than pay for themselves in economic terms. 
\end{quotation}




%``As Quigley points out in chapter 4, the fundamental question in urban economics is why people voluntarily live in close proximity to one another when there are costs to competing for land. The simple answer has two parts: efficiency gains and consumption benefits. Recent theoretical and empirical work provides a sense of the nature and significance of these gains.'' Annez and Buckley\cite{annezUrbanizationGrowthSetting2009}% P  13 %see theie list



Evidence from a broad panel of countries shows little overall relation between income inequality and rates of growth and investment. However, for growth, higher inequality tends to retard growth in poor countries and encourage growth in richer places. The Kuznets curve-whereby inequality first increases and later decreases during the process of economic development-emerges as a clear empirical regularity.


\subsection{Effects through innovation}

%Belderbwhich ideas How quickly trom person to person.  %%% ???


Jacobs (1969, 1984) argued that interactions between people in cities help them get ideas and innovate, a view of cities that fits nicely with the work (Romer 1986; Lucas 1988)on economic growth that views externalities  externalities associated with knowledge spillovers as driving growth. Griliches (1979) surveyed the empirical literature on the role of knowledge spillovers. 

Three theories have dominated the discussion. The Marshall-Arrow-Romer (MAR) theory concerns knowledge spillovers between firms in an industry.  Marshall (1890) described how the concentration of an industry in a city helps knowledge
spillovers between firms and, therefore, the growth of that industry and of that city. Porter (1990) argued that local competition in concentration of industry drive innovation and growth. Both assume that technological spillovers occur within a=n industry. In the Jacobs-Rosenberg\cite{rosenbergTechnologicalChangeMachine1963}-Bairoch \cite{bairochCitiesEconomicDevelopment1988} model,\footnote{Glaeser et al \cite{glaeserGrowthCities1991} refer to the Jacobs model as the Jacobs-Rosenberg\cite{rosenbergTechnologicalChangeMachine1963}-Bairoch \cite{bairochCitiesEconomicDevelopment1988} model. } unlike MAR and Porter, believes that the most important knowledge transfers come from outside the core industry through cross-fertilization among a variety  of  industries.

%Bairoch, P. (1988)\cite{Condit1990CitiesAE}. Cities and Economic Development: From the Dawn of History to the Present. Chicago, IL: University of Chicago Press.

Scherer (1982) presented systematic evidence indicating that around 70 percent of inventions in a given industry are used outside that industry. Much evidence thus suggests that knowledge spills over across industries. Because cities bring together people from different walks of life, they foster transmission of ideas.


Glaeser, Kallal, Sheinkman and Shleifer \cite{glaeserGrowthCities1991} used data on 170 of the largest US cities which industries in which cities grew fastest between 1956 and 1987 and why. They found that  industries grow slower in cities in which
they are more heavily over-represented and faster where the firms in the industry are smaller than the national average  and when the city is less specialized. This evidence is
thus negative on MAR, mixed on Porter, and consistent with Jacobs.



Henderson 1986 presented evidence that output per labour hour is higher i when firms in the same industry are clustered.

Henderson also described "urbanization externalities" that lead different firms to locate together. In this view urbanization increases final market demand and diveristy of products, but Glaeser et al suggest that this effect does not drive growth

Lobos et. al examine the simultaneous effects of spillovers due to research and development by universities and by firms \cite{belderbosWhatSpilloversUniversities2022}.

Glaeser et al \cite{glaeserGrowthCities1991} tested a model  of employment growth (not productivity growth) in an industry in a city as a function of the specialization of that industry in that city, local competition in the city-industry, and city diversity. None of the their5 results  support the importance
of within-industry knowledge spillovers for growth. If such spillovers. %This may be usefull for us because it rules out a veryd ifficult channel to examine.

\subsection{Effects through inequality}
Barro \cite{barroInequalityGrowthInvestment1999} found that inequality had a negative effect on grwoth in poorer countries but no significant effect for the wichre countires
Grigoli et al find  that the effect of income inequality on economic growth can be either positive or negative, and that at levels  of inequality  represented by a Gini coefficient below about 27  to be exact inequality hurts economic development. \footnote{Canada's Gini Coefficient Index was 66.7 in 2017.}\cite{grigoliInequalityGrowthHeterogeneous2016} 

 ``On the one hand, a higher concentration of income in the hands of a few is reflected in reduced demand by a larger share of poorer individuals, which would invest less in education and health and grow a sense of social and political discontent, jeopardizing human capital and stability. Moreover, more inequality can exacerbate households’ leverage to compensate for the erosion in relative income, empower the influence of the richer population on the legislative and regulatory processes, and motivate redistribution policies that are often blamed for slowing growth, especially when aggressive. On the other hand, a certain level of inequality endows the richer population with the means to start businesses, as well as creates incentives for individuals to increase their productivity and invest their saving, hence promoting economic growth.';'Across income levels, only the findings for emerging markets indicate that more inequality slows economic growth. The only country groups for which we find evidence of a significant negative effect are the Middle East and Central Asia, the Western Hemisphere, and emerging markets (3/8).

 But Canada is in the western hepsphere with inequality more like that of europe 
 
 Other theories propose a positive relationship. These are based on the argument that inequality can rather provide incentives for innovation and higher productivity (Lazear and Rosen, 1981; Okun, 2015), foster saving and investment to the extent that rich people have a higher propen- sity to save (Kaldor, 1957), and endow richer individuals with the minimum capital and education needed to start some economic activity (Barro, 2000)
Bivens, reporting on the USA, argues that inequalliity isw reducing grwoth by reducing the aggregate demand of the population below the 90$^{th}$ percentile in income.\cite{bivensInequalitySlowingUS2017} 



----


The notion that your labour force is on average more productive when there are more people around is pretty dramatic and it's very much not part of the basic model that we use. Our starting point is that's the fundamental feature of cities, and what does that do with financial capital and what does that do to distribution and that's not been explored. 

The concentration of wealth is associated with slower econ growth.  There's a possibility that somehow the city will grow less.
%{\color{red} We can speculate on how the distribution. of wealth directly affects the scaling parameter $\beta$. This probably calls of a discussion of the paths through which agglomeration works } 
