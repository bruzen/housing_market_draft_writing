\chapter{Introduction}

% \vspace*{\fill}
\epigraph{Begin at the beginning, the King said gravely, ``and go on till you come to the end: then stop.''}{Lewis Carroll, \textit{Alice in Wonderland}}

%Missing- transfer of money vs put in a spatial framework.
%Most of the economic theory talks about where people go, and it doesn't talk about the value they create in the city and where that goes. That's finacialization, capturing those benefits is what capitalists are doing now.
% rent is being in the house/what they pay, the transfer of money, vs what cities are, and how that produces value..

In Canada, there is a housing crisis. In the last few years, the need for affordable housing has come into focus as one of the most pressing issues facing Canadians. As more and more Canadians are finding housing unaffordable, the effects are being seen in everything from declining homeownership rates to an increasing number of Canadians unable to afford housing at all.

The effects are pushing substantially into the middle class, driving vacancies, poverty, and displacing a whole class of workers.

There has been extensive debate about the drivers of the crisis.  Proposed explanations include supply shortages, stagnating incomes, and the financialization of housing ownership.
(Centering on two dominant stories, a story of supply and demand and one of rights.) %FIX and cite

There has been less work on the implications for productivity (DEFINE). Yet, the economics is clear that this is what's at stake is the productivity of cities, the distributive features of the economy and the impact of the middle class % THIS IS A RESULT NOT AN INPUT. WHAT GOES HERE? 

The housing crisis raises the question of whether Canadian cities can continue to attract people and accumulate wealth for its residents and industries, whether in fact they can even sustain their growth.

Cities are where people live and work, where a great deal of production is concentrated, in addition to being where wealth is created and accumulated, cities are also where income is actually distributed. 

Human beings are increasingly an urban species. Cities are one of the primary sources of technological development and increasing wealth. Behind these observations is a fundamental feature demonstrated in the recent literature on scaling laws: the productivity of cities increases super-linearly in population. Cities are the locus of a positive feedback loop: rising populations raises productivity, rising productivity attracts more people and resource.

Cities are where people live and work, where a great deal of production is concentrated, in addition to being where wealth is created and accumulated, cities are also where income is actually distributed. 

\textbf{HOW WE ADD IT BACK IN}

This thesis presents a spatial model of the city that incorporates distributional issues and financialization and allows us to examine the productivity implications of the housing crisis. The model that incorporates the scaling of productivity in cities within a standard urban model. 

Our approach/model is constructed, drawing together pieces from %a number of sub-literatures from 
economics and the study of cities, including rent theory, production functions, the standard urban model, growth theory, urban growth theories, financialization and the theory of distribution.
We relate this to the scaling models from the study of complexity. This gives a deeper look at distribution in the cities, the effect of financialization, and effect of both of these on the growth and development of cities. 

The urban model is based on those developed in geography, planning and urban economics. The organizing principle in  the spatial models of all three disciplines is an economic variable, land rent, which is the link to distribution, financialization and continuing productivity. *** (another sentence on why this is great)
--The space-less quality of the study of finance leaves out xyz GET PHRASING- CAN'T SEE- INTEGRATION OF SPACE NEGLECT GROWTH FACTOR. 

\textbf{WHY IT'S BEEN MISSING}
EXPLAIN BETTER HERE WHY SPACE HAS BEEN LEFT OUT, AND WHY THAT LET'S US NEGLECTS SPATIAL RENTS AND MISS THE RELATION BETWEEN SPACE AND PRODUCTIVITY.

We're talking about what is the puzzle? This is the teaser for this thesis and this thesis offers an answer to and I've just started to suggest that the teaser is given that we see urbanization and continuing and financialization. accelerating. Although we understand that financialization is really just capital seeking profits, where what is the source of the profit of the rents that they're capturing? And the answer is in conventional urban theory, which allows us to identify the spatial distribution of those rents and traditional rent theory, which allows us to understand the social relations of those, those rents, the classical economists spend a great deal of time on that question. And we're very clear about it.

Our focus is land rents, but in the context of an urban economy. 

% \textbf{WHICH GIVES US THESE CONTRIBUTIONS}
% \section{Contributions}
%PROBABLY INTEGRATE WITH THE ABOVE, MAYBE MAP/HIGHLIGHT CONTRIBUTIONS IN A FIGURE/TABLE

In addition to the core contribution - linking housing and productivity, 
(rent is key to financialization, main urban models don't observe the distribution of rents)

There are three methodological lines of contribution, and there are policy implications SUMMARIZE METHODOLOGICAL CONTRIBUTIONS

%This work is important for understanding the current policy context. 
The analysis makes clear that in addition to the recognized distributional consequences, the housing crisis has productivity impacts that should be considered in developing urban and housing policy. Particularly it centers concern with implication for urban development of growing rent extraction by the financial sector. 


\section{Document overview}

There are three parts, Part \ref{part-background} gives the background and develops the theory for our analysis, Part \ref{part-model} develops the model and analyzes the results, and Part \ref{part-system}, puts the model and theory in the context of a larger system, using methods of systems analysis and design, then discusses the potential for interventions and policy.

Chapter \ref{chapter-background} sketches how this thesis relates to four major fields ..., and the role of space as a unifying factor across three of the fields.

In Chapter \ref{chapter-rent}, we link classical rent theory, neoclassical production theory, neoclassical growth theory, the scaling literature, and urban spatial models.

Chapter \ref{chapter-growth} is on growth. To show how our model is directly connected with this broad collection of linked theories, we use the Cobb-Douglas function, which is used across this entire range of literature 

% After we develop the mathematical description of the relationship among these will discuss  in more detail, rent theory and our contribution, scaling laws, ......  and other issues in the literature that draw on parts of this model and 

 % ???  apply to the specific situation we're in why rent theory is related to discussions of exploitation why it might lead the inefficiencies, whether or not this links with other important models in the literature.

Chapter \ref{chapter-space} is on space and the Alonzo model.

In Chapter \ref{chapter-financialization} we  provide a description of finacialization and show it is a a form of rent-seeking in the housing market and ?? the potential consequences of fiancialization in the housing market. 

In Chapter \ref{chapter-model} we  describe an illustrative agent-based model of the urban system. Most of the analysis of urban systems has employed analytical models with roots that go back to von Thunen () and more recently Alonzo. These models are extremely useful, but necessarily abstract from the concrete  and variable individual behaviour and  the details  of dynamics that make real cities path-dependent. XXX (Dawn) have shown that agent-based models can reproduce the features of the analytical models, at least in simple cases. 


In the context of this we position the results in the larger system, analyze the potential for a range of interventions, and suggest policy implications.

% After we develop the mathematical description of the relationship among these will discuss in more detail, various relevant applications, and issues in the literature that draw on parts of this model and apply to the specific situation we're in why rent theory is related to discussions of exploitation why it might lead the inefficiencies, whether or not this links with other important models in the literature.

% Because we draw on a wide range of methods and literatures, we discuss the relevant literature and  methodologies in the chapters where they apply 