\chapter[Parameters]{Parameter Values} \label{appendix-paremeters}

\section{Model}

\subsection{Time}
 The computational cycle is a year, so all time-dependent variables, such as wage, transportation cost, and interest rates, are specified for the yearly interval. The mortgage period is set arbitrarily 
 
\subsection{Wage premium} \label{section-wage-premium}

HirschWage \cite{hirschUrbanWagePremium2019} observe that, ``Following Glaeser and Maré \cite{glaeserCitiesSkills2001},  a  large  empirical  literature  has  investigated differences in wages across labor markets of different sizes. The general finding of this literature is that a significant urban wage premium exists. and that this premium consists both of a level effect and a growth effect that arises as workers gain urban work experience''. 

Almeida et al \cite{almeidaUrbanWagePremium2022} found for Brazil, that the female urban wage premium is on average 11.3\%, almost double the average male premium of 5.7\% and that higher in formal and informal jobs and across various agglomeration levels. The premium is larger in denser areas.

\subsection{Spillover effects} \label{section-spillover}

Spillover effects can be large. Irwin and Klenow  studied learning in chip production focusing  on the key issue of spillovers. They found learning rates of 10 to 27 per cent, averaging 20 per cent. They indicated that a good part of learning is internal, and that national spillovers were no greater than international spillovers. "... a firm learns three times as much from an additional unit of its own cumulative output as from another firm's cumulative output, regardless of the other firm's country of location. However, rest-of-world cumulative production is typically more than three times any given firm's cumulative production. This means that the absolute contribution of world cumulative production to each firm's experience outweighs the absolute contribution of its own cumulative production. In this sense, spillovers are substantial." (pp. 1217-1218).



\section{Transportation Parameter Values}
In the following sections we and present illustrative values for parameter values.


%==========================EXAMPLE=========================== https://www.kaggle.com/code/prateekmaj21/basic-financial-calculations-using-python/notebook
  
% def compound_interest(p,r,t):  %EXAMPLE
    
%     print('Amount: ', p)
%     print("Rate of Interest (Per Annum)", r)
%     print("Time (In Years): ",t)
    
%     a= p*((1+r/100)**t)
    
%     ci= a-p
%     print("Final Amount: ", a)
%     print("Compound Interest: ", ci)
 

\section{Transportation costs}
Transport costs have two parts:
1) fuel and vehicle costs per km
2) time costs per km

\subsection{Vehicle related costs}
Use one year as the wage period, converting transportation costs per km to annual cost for consideration in the household budget. Starting with the cost per km, calculate the cost per year:

\textbf{cost per km =$\textit{t}$}:. \$0.59   (from  Ontario data, 2021). sensitive to congestion, use of subways (\$5 /day?), 

 \textbf{work trips per year} 2 way * 5 days/week * 50 weeks work days = 500. [range: 450-550]

\textbf{cost per km-year} = work trips per year*cost per km

=\$0.59/km*500 trips/year  =  \$295/km year 



\subsection{Time costs}
\textbf{time per km}. range: 20km/hr -> 3min/km, 40km/hr -> (1.5min/km - 3min/ km)per trip 

(New York rush hour is much slower:  4-9km/hr ->6-15 min/km)

\textbf{time  per km-year} = work trips per year*time t per trip = 500* 3min  = 1500 min/km year = 25 hours= 3-3.5 days/km
 
\textbf{time cost per km-year} =  (days per km-year /work days/year)*wage premium per year  = 3/250 = 0.012 years/km year. ?

\textbf{money cost of time per km year} 

=time cost per km-year* wage(including subsistence) 

= 0.012 year* wage per year

\subsection{Total cost per km year of commuting for one agent}
\textbf{money cost of time per km year + \$295/km year * distance} \\
= (0.012 w+ \$295)/km year 
    \begin{quotation}
    \textbf{Example}
    To get a sense of the required wage if we have this annual cost structure, assume city\_extent $d^*$ is 30 km. At this point the transport cost is equal to the wage

\[(0.012 w+ \$295)/km year)*30 =  w\] 
\[.36w+ 8850=w\]
\[w=13828.12\]
        \begin{quotation}
        \textbf{PLAUSIBILITY CHECK}
This is plausible land rent, but does not include building rent. 
Capitalized at 5\% this house is worth \$ 276,562, a fairly cheap house 30 miles from city centre
        \end{quotation}
    \end{quotation}



\subsection{Value of transportation price to use in model}
\[ {c}=(0.012 w+ \$295)/km year \]



