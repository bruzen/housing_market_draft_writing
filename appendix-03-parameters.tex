\chapter[Parameters]{Parameter Values}
\label{AppendixParemeters}

FIX 



\renewcommand{\arraystretch}{1.5}
\begin{tabular}{rlrr}\
Symbol         & Name                                 & Value      & Formula  \\ \hline
$m_i$          & Individual borrowing-ratio           & 0.75-0.85  & $M/P_{ask?}$ \\
$M^{max}_Yi$.  & Maximum mortgage based on income     &            & $\frac{0.28(\omega+w)}{r_i}$ \\
 $M^{max}_P$   & Maximum mortgage based on the price  &            & $0.8*P_0$ \\
$IS$           & Income share for housing debt        & 0.25-0.35  & Missing? \\
$\rho$         & Rent ratio                           &            & $\frac{\omega-tau*d_i}{P_0}$ \\
$\kappa $      & Operations ratio                     & 0.1-0.3    & e.g. $ 0.2\frac{\omega-tau*d_i}{P_0}$ \\
$\sigma$       & Tax ratio                            & 0.25-0.35  & e.g. $ 0.3\frac{\omega-tau*d_i}{P_0}$ \\
$\dot P $      & Price growth                         &[]          & $\frac{P_t-P_{t-1}}{P_{t-1}}$\\
$P^e_T$        & Expected price in T years            &            & $P_0(1+\dot P)^T$ \\
$r^{disc}_i$   & Individual discount rate             &            & To assign \\
$\bar r$       & Prime interest rate                  &            & \\
$r_{target}$   & Investor/bank target rate            &            & $\bar r + margin$ \\
$r_i$          & Individual borrowing-rate            &            & \\
$\delta_i$     & Discount factor for T                &            & $\left(\frac{1}{1+r^{disc}_i}\right)^T$ \\
\end{tabular}
\renewcommand{\arraystretch}{1.0}


%==========================EXAMPLE=========================== https://www.kaggle.com/code/prateekmaj21/basic-financial-calculations-using-python/notebook
  
% def compound_interest(p,r,t):  %EXAMPLE
    
%     print('Amount: ', p)
%     print("Rate of Interest (Per Annum)", r)
%     print("Time (In Years): ",t)
    
%     a= p*((1+r/100)**t)
    
%     ci= a-p
%     print("Final Amount: ", a)
%     print("Compound Interest: ", ci)
 

\section{Transportation costs}
Transport costs have two parts:
1) fuel and vehicle costs per km
2) time costs per km

\subsection{Vehicle related costs}
Use one year as the wage period, converting transportation costs per km to annual cost for consideration in the household budget. Starting with the cost per km, calculate the cost per year:

\textbf{cost per km =$\textit{t}$}:. \$0.59   (from  Ontario data, 2021). sensitive to congestion, use of subways (\$5 /day?), 

 \textbf{work trips per year} 2 way * 5 days/week * 50 weeks work days = 500. [range: 450-550]

\textbf{cost per km-year} = work trips per year*cost per km

=\$0.59/km*500 trips/year  =  \$295/km year 



\subsection{Time costs}
\textbf{time per km}. range: 20km/hr -> 3min/km, 40km/hr -> (1.5min/km - 3min/ km)per trip 

(New York rush hour is much slower:  4-9km/hr ->6-15 min/km)

\textbf{time  per km-year} = work trips per year*time t per trip = 500* 3min  = 1500 min/km year = 25 hours= 3-3.5 days/km
 
\textbf{time cost per km-year} =  (days per km-year /work days/year)*wage premium per year  = 3/250 = 0.012 years/km year. ?

\textbf{money cost of time per km year} 

=time cost per km-year* wage(including subsistence) 

= 0.012 year* wage per year

\subsection{Total cost per km year of commuting for one agent}
\textbf{money cost of time per km year + \$295/km year * distance} \\
= (0.012 w+ \$295)/km year 
    \begin{quotation}
    \textbf{Example}
    To get a sense of the required wage if we have this annual cost structure, assume city\_extent $d^*$ is 30 km. At this point the transport cost is equal to the wage

\[(0.012 w+ \$295)/km year)*30 =  w\] 
\[.36w+ 8850=w\]
\[w=13828.12\]
        \begin{quotation}
        \textbf{PLAUSIBILITY CHECK}
This is plausible land rent, but does not include building rent. 
Capitalized at 5\% this house is worth \$ 276,562, a fairly cheap house 30 miles from city centre
        \end{quotation}
    \end{quotation}

{\color{red}
\subsection{? Value of $t$ to use in model}}
\[ \tau=(0.012 w+ \$295)/km year \]


\newpage
\section{Notation Table}

% \documentclass{standalone}
% \usepackage[dvipsnames]{xcolor}       \usepackage{calc}     
% \usepackage{tikz}
% \usetikzlibrary{shadings, shadows, shapes, arrows, calc, positioning, shapes.geometric}
% \usepackage{pgfplots}
% \pgfplotsset{compat=1.16}
% \usepackage{mathtools,amssymb}

% \begin{document} 
% \section{Notation for Urban and Production Sectors}

% $t$     & Time \\ 

t is time
TODO - uncomment the commented out ones with fraction- unidentified control sequence

delta is density ** - - infitesmal density increase as the city moves out. [[adjustment speed for wage N]] imagine a density function over the city. 
using the geometry for the derivative of that with respect - how does it change if you increase the amount under the integral, wehre does it go..  - in cont terms.  

N is population, n
d is diameter %(https://www.overleaf.com/project/606a6b286ae1c9f203fadab5 ). \\
omega is wage premium - which gives  the parented population
tau is linear transport cost per unit distance


\newpage
\begin{longtable}{lp{10cm}}
\caption{Notation}                \\

\hline           &  \textbf{Productivity} \\ \hline
$K$              &  Capital               \\ 
$L$              &  Labour                \\
$N$              &  Population, which equals labour, $L$ \\ 
% $n_i$  &  Number of workers employed by firm $i$ \\
% n=\sum_i n_i$  &  Number of workers, the urban population in the model \\
% $\#f=\frac{n}{n_i}$&number of identical firms \\ %not used
% $f$  &  Number of firms =1 \\
% $n =f n_i$  &  Aggregate labour \\
% $\Lambda(n)$    &  Labour-augmenting agglomeration effect \\
% $n^\gamma$ & The labour-augmenting agglomeration effect,  modelled as an exponential function of the number of people \\
% $\Lambda(n)n_i$ &  Effective labour for firm $i$ \\
% $\Lambda'=\die{\Lambda(n)}{n} $ & Derivative of the labour-augmenting agglomeration effect\\
$\alpha$         &  Elasticity of output with respect to capital          \\
$\beta$          &  Elasticity of output with respect to effective labour \\
$\gamma$         &  Elasticity of the urban agglomeration effect          \\ % , $\Lambda(n)$, for illustration \\
%%$Y_i=K_i^{\alpha }(\Lambda(n)n_i)^{\beta }$  &  Urban firm $i$'s output \\
$Y=N^\gamma K^{\alpha }N^{\beta }$  &  Urban output                \\
%%$Y=\frac{n}{n_i}K_i^{\alpha }(\Lambda(\sum_i n_i)n_i)^{\beta }$  &  Aggregate output of all firms in the city \\
% $\die{Y}{n}=\beta\frac{1}{n} Y  \left( 1+ \frac{n\Lambda'}{\Lambda} \right)$  &  Social marginal product of labour \\
% $Y_i=K_i^{\alpha }(\Lambda(n)n_i)^{\beta }$    &  Urban firm $i$'s output \\
% $\die{Y_i}{K_i}	=\alpha \frac{1}{K_i} Y_i $  & Marginal product of capital for firm $i$ \\
% $\die{Y_i}{n_i}	=  \beta\frac{1}{n_i} Y_i $  &  Marginal product of labour for firm $i$ \\
%%$\eta=\frac{n_i\Lambda'}{\Lambda}$  &   Marginal agglomeration effect on a firm's output of increasing it's own labour stock \\
% \hline
	% &\textbf{Amenity}\\ \hline
% $A(d, n)$   &  Agglomeration amenity   \\

\hline           &  \textbf{Labour market}  \\ \hline %and urban stucture??
$P$              &  Property price          \\
$P^e_T$          & Expected price in T years    \\
$\dot p$         &  Rate of price growth    \\
$\psi$           &  Rural wage              \\
% $\psi$  &  ?Per-period cost of a unit of productive capital \\
$\omega$         &  Urban wage premium \\
% $\omega + \psi$  &  Urban wage including rural wage \\ %***
% $\textit{t}$ & {\color{red}transportation cost per km} \\%use   c?
$\tau$           &  Transportation cost per unit distance \\
% $w^n=w-\tau d$ & Wage  premium net of transportation costs \\
$\mathcal{R} = w-\tau d$ & Rent at distance $d$  \\ 
%% $\Omega=\frac{w+\psi}{\psi}$  &  Ratio of the urban wage to the  cost of capital \\
%% $\Pi$	   &  Profit \\
%% $ER$	   &  Excess return to capital \\ 
% \hline &\textbf{Spatial structure in the circular city} \\ \hline		
$d$              &  Distance of a residence from the centre of the city \\
$d^* = w/\tau$   &  Maximum distance commuters will travel \\ % to get the wage premium \\
$\zeta$          &  Population density at distance $d$ \\
%% $d^{max} = w/\tau$  &  Maximum distance commuters at which residents enjoy the urban amenity \\
%% $d^{**} = max(d^*, d^{max})$  &  radius of the city \\
$s$              &  Lot size     \\
%% $U$                     &  Worker utility **\\ %, a function of location and prices \\
%% $U^{urban}=U^{rural} $  &  Migration equilibrium assumption ** \\
% \hline & \textbf{Labour market} \\ 
$L$              &  Labour supply \\ %the number of workers, which, in the standard circular city model, equals the number of lots of size $s$  when workers live on identical individual lots. % Unless $d^{max}>d^*$ v  \frac{\pi}{s}(\frac{w}{\tau})^2 =

\hline           & \textbf{Financial market} \\ \hline
$\bar r$         &  Prime interest rate      \\
$r_{target}$     &  Investor or banks target interest rate, $\bar r + margin$ \\
$r =  r_i$       &  $i$'s personal borrowing rate  \\
$r_i^{disc}$     &  $i$'s subjective discount rate (possibly $r_i$)           \\
$\delta = \delta_i^T$ &  $i$'s discount factor for a period $T$           \\
$m = m_i(W_i)$   &  Wealth-based share of home price $i$ can mortgage     \\
$IS_i(\omega+w$  &  Income-based share of home price $i$ can mortgage     \\
$\rho$           &  Rent ratio             \\
$\kappa$         &  Operations ratio       \\
$\sigma$         &  Property tax share     \\ % (also considered $\chi \Gamma$  \rotatebox[origin=c]{180}{$t$} \reflectbox{$t$})
$t$              &  Time                   \\
\hline
\color{black}
\end{longtable}  

%%\item[$$]
%%\item[$$]
%%\item[$$]
%Rural producers pay a wage $\psi$. this covers a standard house, lot, entertainment, diet and consumption pattern. We  choose units so that per-period cost of a unit of productive capital is also $\psi$


\section{Additional Assumptions}
\begin{enumerate}
\item There is full employment, no frictional unemployment, and no labour adjustment costs.
\item Firms set output and factor inputs to maximize profits, so factors are paid the value of their marginal private product
\item Demand for output is perfectly elastic (constant price = 1)

\end{enumerate}

