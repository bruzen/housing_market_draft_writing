

\chapter{Urban Land and Land Rent}
\section{The Alonzo-Jacobs Model}
In 1964, William Alonso published \textbf{Location and Land Use}, in which he described a model that specifically linked the urban wage premium to urban land rents and  became the central model in modern urban economics. 
We use an \textbf{Alonzo-Jacobs model} to explore the distribution surplus value, where the reference to Jane Jacobs link the wage premium to Jacobs-style  agglomeration effects that generate urban productivity and the wage premium. 

Ricardo had described a model with a central market for corn, producing corn took land and transporting corn to market was costly. Because there is one market price for corn, land with low transportation costs near the central market earns a rent. More distant land has lower value. In Alonzo's model there is central market and a single price for labour, producing labour takes land, and transporting labour to the market is costly. Alonzo simply re-presents Ricardo's conception of rent  mathematically for a different social system and production technology.  

The logic of the model is illustrated in the following figure. The height of the green bar on the left illustrates the premium for urban labour at the centre of an Alonzo circular city. The height red triangle at the left is the rent earned on land at the centre, which has no transportation costs.\footnote{the model says nothing about who gets the rent. For classical economists it was obvious that the rents went to the class of land-owners.} Transportation to and from the center costs $t$ times the distance $d$ from the center. Fuel, capital, and time costs are  all included in $t$. 


\input{Simple_alonzo_fiigure.tex}
% %%%%%%%%%%%%%%%%%%%%%%%% PARTITIONING THE LABOUR SHARE
The entire rectangle, $\omega$ $\times$ $d^*$, is the surplus generated by urban agglomeration. Urban land rent, which is the residual when transport costs are deducted from the wage premium, declines  with distance $d$ until, at the very edge of the city, $d^*$, t the cost of transportation  consumes the entire wage. Property values are simply the the present discounted value of the rent at any point.

At the bottom of the figure we illustrate the conventional `subsistence wage'  earned by a worker whether in the city or outside of the city.   In most analyses of urban spaces this living wage is simply ignored, since it is the wage premium that generates rents.  It is also common to assume that the labour market and production at the centre takes no space.   


Workers are attracted to the city by the wage premium, $\omega$,  which represents the share of the surplus generated by the city that goes to labour.  The grey triangle represents the amount of the surplus dissipated in travel costs.  

The extent  of the city  $d^*$ is a simply the distance at which total $rt$ transportation cost  is equal to the wage premium
\[d^* t= \omega\]
where $t$ is the unit cost of transportation. In the figure, $-t$ is the slope of the diagonal line dividing rent from transportation expenditure

It is a beautifully simple model that accounts for many features of urban structure and urban history. From $w$, $t$ and population density we can derive population, wage bill, total rent, transportation costs.

 \section{Alonzo's Circular City}
The figure above suggest that  half of the urban surplus is spent on transportation, but because the city is circular,  rents can be represented as the volume of a cone:  
$ V=\frac{1}{3}\pi  r^2 \omega$. The total value of wage payments would appear as the volume of cylinder enclosing the cone\footnote{since the wage is the same for each unit of labour no matter where is t is produced (resides).} 
$V=\pi r^2 \omega$
and total transport costs are 
$\frac{2}{3}\pi  r^2 \omega)$

 


\begin{tikzpicture}[scale=.5]
   %%%%%%%%%%%%%%%%%%%%%%%%%%%%%%%%%%%%%%%%%%%%%%%%
% definitions for schematic
\def\bndmax{5}        %https://tex.stackexchange.com/questions/68462/filling-a-complex-region-with-tikz
\def\bndmin{0.2}
\def \n {10}  % height of y axis
\def \d {12}  % length  of x axis
\def \t {.75}  %  cost of transportation per unit x
\def \th {1}   % theta?
\def \w {7}    %  wage premium
\def \om{1.5}%  omega =rural wage Zero for urban population
\def \azero{2}
\def \aprime {-.0}	
\tikzset{func/.style={thick,color=blue!90}}	

    %%%%%%%%%%%%%%%%%%%%%%%%%%%%%%%%%%%%%%%%%%%%%%%%
% definitions for Cone3
%\node at (0, 2.5){\input{SA_Cone3.tex}};
     \pgfmathsetmacro{\radiush}{9.6};%Cone base radius
        \pgfmathsetmacro{\theight}{7.1}%Cone height (negative if you want a inverse cone)
        \pgfmathsetmacro{\cheightp}{.03}%Cut height in percent of cone height

        %Calculating coordinates
        \coordinate (center) at (0,0);
        \pgfmathsetmacro{\radiusv}{.2 * \radiush};
        \pgfmathsetmacro{\sradiush}{\radiush * (1 - \cheightp)};%only for right circular cone
        \pgfmathsetmacro{\sradiusv}{.2 * \sradiush};
        \coordinate (peak) at ($(center) + (0,\theight)$);
\coordinate (antipeak) at ($(center) + (0,-\theight)$);  %thanks  %I added this
\coordinate (vert1) at ($(center)+(\radiush,-.2)$);
\coordinate (vert2) at ($(center)-(\radiush,.2)$);
%problem
     \coordinate (svert3) at ($svert1+(0,7)$ );  % Shifting up by W
    % \coordinate (svert4) at ($svert2 + (0,\w)$0;
\coordinate (svert1) at ($(vert1)!\cheightp!(peak)$);
\coordinate (svert2) at ($(vert2)!\cheightp!(peak)$);  
    % \coordinate (svert3) at ($svert1+(0,\w)$);
    % \coordinate (svert4) at ($vert2)+(0,\w)$);   
   %%%%%%%%%%%%%%%%%%%%%%%%%%%%%%%%%%%%%%%%%%%%%%%%
% Cone Drawing    
  \fill[ left color=red!70, right color=red!70,  opacity=20,middle color=red!20,shading=axis] (svert1) -- (peak) -- (svert2) arc (180:360:\sradiush cm and \sradiusv cm);
  
%Uncomment this for top of cylinder
        \fill[inner color=gray!5,outer color=gray!50,shading=radial,opacity=.5] (peak) circle (\sradiush cm and \sradiusv cm);
        \draw [thick](svert1)-- ++ (90:\w);
        \draw [thick](svert2)-- ++ (90:\w);
        %Lines, \h in percent of cone height
  
% Cylindar drawing
  \fill[ left color=black!50, right color=red!20,  opacity=20,middle color=red!20,shading=axis] (svert4)-- (svert2) arc (180:360:\sradiush cm and \sradiusv cm)--svert4 arc (360:180:\sradiush cm and \sradiusv cm) ;      
% TRY TO Make a cylinder
%\draw ($svert2 + (0,\theight)$) [arc (180:360:\sradiush cm and \sradiusv cm)]; 
%     \fill[left color=gray!70,right color=gray!70,middle color=gray!30,shading=axis] (vert1) -- (svert1) arc (0:-180:\sradiush cm and \sradiusv cm) -- (vert2) arc (180:360:\radiush cm and \radiusv cm);

\foreach \h in {0.03}{   %.38,.34,.30, .7
            \pgfmathsetmacro{\rh}{-\radiush * (1 - \h)}
            \pgfmathsetmacro{\rv}{.2 * \rh}
            \draw[black!70,densely dashed] ($(vert2)!\h!(peak)$) arc (360:180:\rh cm and \rv cm);%$(vert2)!\h!(peak)$)
        }
        \draw[opacity=.90, line width=.05cm, green] (0,0)--(0,{\theight - .05});
%     \foreach \h in {0, .38,.34,.30, .7}{
%            \pgfmathsetmacro{\rh}{\radiush * (1 - \h)} %            \pgfmathsetmacro{\rv}{.2 * \rh}
%            \draw[black!70,densely dashed] ($(antipeak)!\h!(vert2)$) arc (180:360:\rh cm and \rv cm);
%   }
%  \draw[red] (antipeak) arc (30:60:3);
%  \draw[dashed, thick] arc (0:-180:\sradiush cm and \sradiusv cm) -- (vert2) arc (180:360:\radiush cm and \radiusv cm);
%%%%%%%%%%%%%%%%%%%%%%%%%%%%%%%%%

% %\foreach \xi in {0,..., \n} \draw (\xi,0)--(\xi,-.1)node[below=1]{\small$\xi$};
% %\foreach \yi in {1,...,\n} \draw (0,\yi)--(-.1,\yi)node[left]{$\yi$};
% %\foreach \i in {1,4,9,16} {
% %\node at (7,-\om/2){people scattered uniformly across the land  };

% %SECOND FIGURE WITH AGGLOMERATION WAGE
% %  add urban production and net wage
% %\draw[fill=white, white] (0.1,-0.1) rectangle (14,-\om+.1);
% \draw [fill=green] (-.25, 0) rectangle(.25, \w);
% \node[right, text width=4cm] at  (3, \w+1){Added Productivity due to agglomeration};
% %\node[right, text width = 3cm] at  (10,9){Where does the increase in productivity come from?};
 \draw [ thick, ->](0,0)--(2.5, 0)node [right] {\Large $d$};


% \draw[thick] (0,0) -- ++ (50:2.6cm);  %   diagonal for perspective
% \draw[thick] (0,0) -- ++ (230:2.35cm); 

% %  THIRD FIGURE  add RENT profile in blue

% %\node[right, white, fill=white,  text width = 3cm] at  (10,9){Where does the increase in productivity come from?};
% \draw[func, domain=0:\w/\t+1,ultra thick] plot [samples=200] (\x,{\w-\t*\x}); %Net wageprofile  for 
% %\node[right, white, fill=white] at  (.25, \w/2){Added Productivity};
% %\node[right, fill=white, text width =3.5cm ] at  (1, \w/2){Declining wage  net \\of transportation\\ costs $T(d)$ };
% %\node[right, fill=white, text width =3.5cm ] at  (4,9){Declining wage  net \\of transportation\\ costs  };
% %
% %\node at (0, 1.5){\includegraphics{\input{SA_Cone3.tex}} };
% %\node at (0, 2.5){\input{SA_Cone3.tex}};

% %   FOURTH FIGURE     commuters
% %\pause
% %\draw[fill=blue!40] (0.1,-0.1) rectangle (9.2,-\om+.1);
% \node at (4.5,.4*\om){commuters};


\end{tikzpicture}



 It is easy to see that the transportation revolution brought about by first street cars and later automobiles made much larger cities possible. In North America, with large amounts of land, it generated massive urban sprawl, but also made land available for a growing `middle class.' Ultimately it generated congestion and rising transportation cost that began to limit urban growth. 

Rising urban productivity will raise the wage, attracting more workers. If they are added in suburbs at the edge of the city -- Ricardo's extensive margin -- virtually all of the wage premium they receive is dissipated in transportation costs. Closer to the centre,  land rents rise. Owner-occupiers capture the increase as property value appreciation. Tenants are likely to be faced with higher rents.      

If agglomeration is the source of productivity gains, however, the new workers increase the urban premium, further increasing land values and attracting more workers. 
