\chapter{Conclusion} \label{chapter-conclusions}

The dissertation examines a pair of linked hypotheses: firstly, that financialization induces a shift towards tenancy among the urban workforce.  Secondly, that redirecting spatial rents is likely to result in decreased urban productivity. 

 We take a step beyond integrating labour markets in a city, to studying the distributional effects: who gets the surplus, what does that mean for the class structure, and ultimately the productivity of cities. 
We began with the fact that there is growing policy concern about the financialization of  housing. We have produced 


     We sketch sketch how this thesis relates to four major fields: classical rent theory, neoclassical production theory and growth theory, the scaling literature, and urban spatial models. % \dots , and the role of space as a unifying factor across three of the fields. % WITH FINANCE IS SPACELESS.
    *** ADD BACK? This work draws together sub-literatures including rent theory, production functions, the standard urban model, growth theory, urban growth theories, financialization, and the theory of distribution, so the chapters review those areas. % *** link the areas to the chapters better?  %theory for our analysis, 

    , using an approach similar to that developed in modern growth theory, that we discuss in Chapter~\ref{chapter-rent}.  In that chapter we ground the observation in  \gls{neoclassical growth theory} and recent empirical and theoretical work on \gls{urban scaling}. 


In this thesis we began with the broad question of how urbanization and financialization interact. To explore this question, we built a model of an urban system to explore financialization in the context of housing markets and we explored the systemic effects of the shifting ownership of property in cities on the distribution of wealth and the potential that this shift can actually change the ability of cities to grow, thrive, and produce wealth.'


THINGS TO INCLUDE
Past work did not yet bring totogether spatial insights with production insights in this way.... Because its a economic and urban quesiton, THIS WORK DRAWS TOGETHER - THEORETICAL FRAMEWORKS....

We built a model that brought these togterh SUMMARIZE...
Based on .... REVIEW TEHORETICAL BASIS FOR HYPOTHESIS... we hypothesized WHAT
SUMMARIZE CONCLUSIONS.... AS EXPECTED AND WhaT WAS SURPRISING>...

THIS WORK LAYS THE GROUND WORK FOR FUTURE STUDY OF ....





% \section{Rent and distribution}

% There's a gap, however, in the formal apparatus in standard economic theory for analyzing the distribution of the enormous value created. 

% The organizing principle in the spatial models of all three disciplines is an economic variable, land rent, % The three disciplines share a simple economic insight.

Financialization of the housing market is a flow of money into the housing market but not in to housing production, expanding financial ownership and tenancy, while reducing the share of owner-occupied housing. Its goal is the capture of spatial rents, which we will show has implications for both the productivity and the class structure of the city. 
In the urban context, financialization is fundamentally \gls{rent-seeking}, and we show it can have a profound impact on the system, including effects on both distribution and productivity. 

% A key insight is that the financialization of the housing sector is a form of \gls{rent-seeking} that must have detrimental effects on urban development and on the well-being of urban residents, that is neither constructive nor productive.

% CONCLUSIONS/SUMMARY?
We argue financialization %This dissertation makes the case that it
has worrying implications for the success of cities and the nature of our social fabric. 

We have used what we call the  \textbf{\gls{Alonso-Jacobs model}} to explore the source and distribution surplus value. We  work with an extension of the basic Alonso model that incorporates the \glspl{agglomeration effect} that Jane Jacobs  described in her book, The Economy of Cities \cite{jacobsEconomyCities1969}. In our model these effects generate the \gls{urban wage premium} central to urban growth. % and the wage premium. 

The result is a simple model in which marginal productivity determines the wage, the wage determines the size of the city, the size of the city determines the labour supply, and labour supply determines marginal productivity. 


There are two classes of results first on ownership, the second on productivity

The analysis suggests that in addition to the recognized distributional consequences, the housing crisis has productivity impacts that should be considered in developing urban and housing policy. 


The economics is clear that this is what's at stake is productivity of cities, the distributive features of the economy and the impact of the middle class. 
% Highlights the urgent need for more empirical work.


% Particularly, it centers concern with implication for urban development of growing rent extraction by the financial sector. 
The housing crisis raises the question of whether Canadian cities can continue to attract people and accumulate wealth for its residents and industries, and whether they can sustain their growth.
% Our focus is land rents, %but in the context of an urban economy. 


% This appears, at least part of it appears as locational rents. 
% Financialization, is about capturing the surplus generated by the city.  % To model the financialization of land markets.

% To model financialization we need rent because
% To develop a formal model of financialized urban land markets, we introduce rent because rent is % precisely   about extracting and allocating surplus value in a system. % and that is what financialization of the housing market is about/does. 
% The classical approach to rent is a core tool in the development because it brings the extraction of surplus into focus.
% GAP Nobody has linked the rents - linking rents to urban scaling. Beteencourt is talkign about a surplus in the system, wealth, but he hasn't linked to the market/land market for those locational rents.


% - hamstrings the whole thing.
% --> the whole system as a welfare producing system fails if the value gets sucked out---CONCLUSION TH THESIS---fails from a social point of view-
% these are averages---some are structurally below average so some are always behind simply because of the structure of the rents claimed.. that's built in FUTURE WORK- DIFFERENT INCOMES GETS YOU THAT. 


% We could run off a cliff and accidentally destroy the middle class, we should consider the implications, need a language to explore that.
% ---


% We examine the effect of housing on wealth inequality by looking at 

% Adding 2 things 1. rent extraction and 2. power law scaling of productivity, we find rent is the breaks on the engine of wealth creation

% If the links are correct. 



% There is a market for the urban product produced by firms, and a financial market that agents can invest in.

% ---
% WAS AT END OF SPACE CHAPTER 
% hey don't examine distribution or how the distribution of rents might affect the productivity of the city, which means they throw no light on the effects of the financialization of the property market on distribution or on urban productivity.  This is the limitation we seek to address in this thesis. 

We build the  standard urban model into our model of financialization by using the concept of rent as it appears in the urban system


We explore distributional consequences, how the distribution of rents feeds back into the productivity of cities, and how the urban economy is changed by the financialization of urban housing. % and relating them to neoclassical growth theory.  
% % This chapter introduces the background to the theory of the urban model. 
% E ADDED: 
By combining these pieces we can look at how ownership changes with finalization, how that can feedback into changing the process of value creation in the urban center.
% Rent provides a  neglected indicator or state variable in the urban system. It can be  seen as an indicator of the state of teh system that has been largely neglected.





----

The 3 parts of this do xyz
The dissertation is organized into three parts: background, methodology, and analysis. The background section provides theoretical foundations and reviews relevant literature, while the methodology section outlines the model framework. The analysis section presents results from simulations and discusses their implications.




\section{from intro}

% We have a model of productivity to change. 
%\textbf{We are in a housing crisis.  These are things that can illuminate this particular.. **}
%We examine how urbanization and financialization interact through urban housing markets. % The dissertation examines a pair of linked hypotheses about the evolution of an urban system: firstly, 
% ## Contributions:

%To test the 
%We built an agent based model to explain tenatization.
% models for each class of agent in the system and then allow agents to interact. 
% Firms hire workers if it is profitable. Firm productivity and wages rise with population due to agglomeration effects. 

% Rural workers near the city apply for urban jobs if the urban wage justifies travelling to the city center to work. Homes come up for sale when workers retire New entrants buy homes if they can afford homes, otherwise they become tenants. Investors purchase or sell properties if the net rent and capital gains exceed the return on alternative investments. % this paragraph describes an agent-based approach and establishes the independence of the agents 

%They don't live in the city, they can't reinvest, they don't belong to the class of those who own property and can save through homeownership. 

 %in a %spatially explicit urban land market. 
% The most complex part of the model is the financial bidding for property by different agents. % Firm technology and the price of output are held constant. Investors represent the supply of global capita and are by assumption non-resident.

% It also allows us to see how the locational rents generated by the city is distributed between investors or residents.  % This paragraph says that results emerge from the interaction of independent agents./
 % is captured through land rents. 
% By capturing rents, financialized actors are able to capture a share of urban productivity. 
%In the model, this affected the systemic effects of the shifting ownership of property in cities on the distribution of wealth and the potential that this shift can actually change the ability of cities to grow, thrive, and produce wealth. 
% Big chunks simplified but the guts are in financial behaviour. 

% It is helpful for economic policy in complex environment. 
% The model allows us to test qualitative effects of various policies.

0. Having understood what you mean by rent/by the implications of rent in an urban situation

1. First we put financialization into a model an agent-based urban land ownership model that is well-constrained and well-understood, and show that in a plausibly specified model, financialization drives a change in the class structure. 

2. The economic literature does not examine the class impact. Economists don't talk about. Sociologists talk about it, but they don't model it.

3. WE CAN We model boom bust dynamics/response. how it responds to changes in the outer world - fact you could get a change with an interest rate. what happens when the model is repeatedly struck. What are the impact effects and how do they play out?

4.  We ask the question about if there's a spillover to productivity. There is speculation about such a link in the left-wing urban literature but we haven't found any modelling.

This is what's done in the climate literature. The dynamics and resilience effects are not obvious from the equations and must be modelled. The modelling is important because empirical dynamics are really hard to establish. This is al1 off world. We can't run 50 versions. You can, however, model 50 versions. 

The basic fiancialization result is clear from looking at the model. The equations only make it obvious once you put it together, however. The way they are combined is not obvious.  It is important to note also that the results are not a consequence of building a model that will give any result, but only a consequence of building a model that behaves urban theory and the economic theory predicts. None of the equations were constructed to give a result. Each is our best about how it works. Each step was a best guess at how the process works and based on a knowledge of the literature and the economics. The base theory in neither literature clearly predicts the class effects (there is some work in the literature e.g. Jacobs, reinvesting in Henry George literature - a few places where the economists have gone in that direction) We implemented a version of the best of the urban growth theories.

% ## Hypotheses

1. The financial sector affects the ownership of housing and the class structure of society, 

2. There are dynamic/resilience features of this model that make the effects worse

- Boom bust - pump wealth on/out of city on the boom and on the bust. 'There are sharks in the water' 1. Higher bid can amplify up swings - have to compete with speculators 2. can't get it back on down swing since outside finance offers a stable floor- buys up on way down (we expect larger effects as people are 1. displaced on the way up and then 2. evicted on the way down - can't use their spaces)

- Hysteresis - perturb, doesn't come back - e.g. interest rates go up.

- The way it aligns with long run changes in the landscape e.g. tech changing local info changes vulnerability to these shocks -- Depth of the basin changes - erodes systemic resilience - together these changes the capacity to hold value in landscape. (links to the productivity feedback - much will/can they invest in increasing their productivity/supporting kids/good food to grow brains. Education to increase productivity is the feature that makes productivity increases resilient to de-industrialization). This has implications for landscape/system/class structure.

3. This shift in ownership may have implications for urban productivity. (can actually displace productive uses - empty store fronts) - who can/will enter, how , who can rent spaces, speculative value may keep it empty (work spaces or living space - lowering pop), reduces consumption


% 'Economic Policy in Complex Environments. computationally intensive methods for decision and policy analysis in complex and uncertain environments. Particular focus lies on the application of such methods in the domains of climate change and innovation.'
% INTRODUCE SPATIAL RENTS
% INTRODUCE CLASS
% CLEAR THERE IS NOT MULTIPLE FIRM.. MAYBE LAND MARKET FIRST, THEN SAY REP FIRM
% PRODUCTIVITY - NOT A HYPOTHESIS- WE LOOK AT EFFECT OF ADDING IT TO THE MODEL

%to explore financialization in the context of housing markets and 
% Consider the implications of the capture of spatial rents. 
% Ricardian rent.

% We're looking at financialization  via the housing market. 
% This is what we've put together to explore that. 
% These are the linked hypotheses 
% This is what we get 
