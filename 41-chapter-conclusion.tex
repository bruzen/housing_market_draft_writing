\chapter{Conclusion} \label{chapter-conclusions}

 We take a step beyond integrating labour markets in a city, to studying the distributional effects: who gets the surplus, what does that mean for the class structure, and ultimately the productivity of cities. 
We began with the fact that there is growing policy concern about the financialization of  housing. We have produced 


     We sketch sketch how this thesis relates to four major fields: classical rent theory, neoclassical production theory and growth theory, the scaling literature, and urban spatial models. % \dots , and the role of space as a unifying factor across three of the fields. % WITH FINANCE IS SPACELESS.
    *** ADD BACK? This work draws together sub-literatures including rent theory, production functions, the standard urban model, growth theory, urban growth theories, financialization, and the theory of distribution, so the chapters review those areas. % *** link the areas to the chapters better?  %theory for our analysis, 

    , using an approach similar to that developed in modern growth theory, that we discuss in Chapter~\ref{chapter-rent}.  In that chapter we ground the observation in  \gls{neoclassical growth theory} and recent empirical and theoretical work on \gls{urban scaling}. 


In this thesis we began with the broad question of how urbanization and financialization interact. To explore this question, we built a model of an urban system to explore financialization in the context of housing markets and we explored the systemic effects of the shifting ownership of property in cities on the distribution of wealth and the potential that this shift can actually change the ability of cities to grow, thrive, and produce wealth.'


THINGS TO INCLUDE
Past work did not yet bring totogether spatial insights with production insights in this way.... Because its a economic and urban quesiton, THIS WORK DRAWS TOGETHER - THEORETICAL FRAMEWORKS....

We built a model that brought these togterh SUMMARIZE...
Based on .... REVIEW TEHORETICAL BASIS FOR HYPOTHESIS... we hypothesized WHAT
SUMMARIZE CONCLUSIONS.... AS EXPECTED AND WhaT WAS SURPRISING>...

THIS WORK LAYS THE GROUND WORK FOR FUTURE STUDY OF ....





% \section{Rent and distribution}

% There's a gap, however, in the formal apparatus in standard economic theory for analyzing the distribution of the enormous value created. 

% The organizing principle in the spatial models of all three disciplines is an economic variable, land rent, % The three disciplines share a simple economic insight.

Financialization of the housing market is a flow of money into the housing market but not in to housing production, expanding financial ownership and tenancy, while reducing the share of owner-occupied housing. Its goal is the capture of spatial rents, which we will show has implications for both the productivity and the class structure of the city. 
In the urban context, financialization is fundamentally \gls{rent-seeking}, and we show it can have a profound impact on the system, including effects on both distribution and productivity. 

% A key insight is that the financialization of the housing sector is a form of \gls{rent-seeking} that must have detrimental effects on urban development and on the well-being of urban residents, that is neither constructive nor productive.

% CONCLUSIONS/SUMMARY?
We argue financialization %This dissertation makes the case that it
has worrying implications for the success of cities and the nature of our social fabric. 

We have used what we call the  \textbf{\gls{Alonso-Jacobs model}} to explore the source and distribution surplus value. We  work with an extension of the basic Alonso model that incorporates the \glspl{agglomeration effect} that Jane Jacobs  described in her book, The Economy of Cities \cite{jacobsEconomyCities1969}. In our model these effects generate the \gls{urban wage premium} central to urban growth. % and the wage premium. 

The result is a simple model in which marginal productivity determines the wage, the wage determines the size of the city, the size of the city determines the labour supply, and labour supply determines marginal productivity. 


There are two classes of results first on ownership, the second on productivity

The analysis suggests that in addition to the recognized distributional consequences, the housing crisis has productivity impacts that should be considered in developing urban and housing policy. 


The economics is clear that this is what's at stake is productivity of cities, the distributive features of the economy and the impact of the middle class. 
% Highlights the urgent need for more empirical work.


% Particularly, it centers concern with implication for urban development of growing rent extraction by the financial sector. 
The housing crisis raises the question of whether Canadian cities can continue to attract people and accumulate wealth for its residents and industries, and whether they can sustain their growth.
% Our focus is land rents, %but in the context of an urban economy. 


% This appears, at least part of it appears as locational rents. 
% Financialization, is about capturing the surplus generated by the city.  % To model the financialization of land markets.

% To model financialization we need rent because
% To develop a formal model of financialized urban land markets, we introduce rent because rent is % precisely   about extracting and allocating surplus value in a system. % and that is what financialization of the housing market is about/does. 
% The classical approach to rent is a core tool in the development because it brings the extraction of surplus into focus.
% GAP Nobody has linked the rents - linking rents to urban scaling. Beteencourt is talkign about a surplus in the system, wealth, but he hasn't linked to the market/land market for those locational rents.


% - hamstrings the whole thing.
% --> the whole system as a welfare producing system fails if the value gets sucked out---CONCLUSION TH THESIS---fails from a social point of view-
% these are averages---some are structurally below average so some are always behind simply because of the structure of the rents claimed.. that's built in FUTURE WORK- DIFFERENT INCOMES GETS YOU THAT. 


% We could run off a cliff and accidentally destroy the middle class, we should consider the implications, need a language to explore that.
% ---


% We examine the effect of housing on wealth inequality by looking at 

% Adding 2 things 1. rent extraction and 2. power law scaling of productivity, we find rent is the breaks on the engine of wealth creation

% If the links are correct. 



% There is a market for the urban product produced by firms, and a financial market that agents can invest in.

% ---
% WAS AT END OF SPACE CHAPTER 
% hey don't examine distribution or how the distribution of rents might affect the productivity of the city, which means they throw no light on the effects of the financialization of the property market on distribution or on urban productivity.  This is the limitation we seek to address in this thesis. 

We build the  standard urban model into our model of financialization by using the concept of rent as it appears in the urban system


We explore distributional consequences, how the distribution of rents feeds back into the productivity of cities, and how the urban economy is changed by the financialization of urban housing. % and relating them to neoclassical growth theory.  
% % This chapter introduces the background to the theory of the urban model. 
% E ADDED: 
By combining these pieces we can look at how ownership changes with finalization, how that can feedback into changing the process of value creation in the urban center.
% Rent provides a  neglected indicator or state variable in the urban system. It can be  seen as an indicator of the state of teh system that has been largely neglected.





----

The 3 parts of this do xyz
The dissertation is organized into three parts: background, methodology, and analysis. The background section provides theoretical foundations and reviews relevant literature, while the methodology section outlines the model framework. The analysis section presents results from simulations and discusses their implications.




