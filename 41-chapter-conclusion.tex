\chapter{Conclusion} \label{chapter-conclusions}


% INTRODUCE 1. WEALTH AND VALUE IS CREATED IN CITIES THROUGH the productivity of cities AND 2. PROPERTY IN HOUSE OWNERSHIP IS IMPORTANT TO DISTRIBUTION OF WEALTH.
%And ownership oofo the land and properties in cities is important to how that wealth is distributed.
We set out to determine whether financialization is likely to  shift the ownership of property in cities away from residents, affect the distribution of resources and change the ability of cities to grow, thrive, and produce wealth. 
% This work presents a model that allows us to explore the financialization of urban housing markets. 
We have produced a theoretical simulation that suggests that financialization of the housing market may indeed have worrying implications for the success of cities and the nature of our social fabric. 

Financialization expands financial ownership and tenancy, while reducing the share of owner-occupied housing. Its goal is the capture of spatial rents, which is fundamentally unproductive \gls{rent-seeking} that makes no positive contributions to the production sector or to residents of a city. 
It could, if it follows the pattern in the model, progress to the point where owner-occupiers are entirely squeezed out of the city. Currently home ownership in Canada exceeds 60\%, but that number is falling. 
% and our model suggest that it will continue to fall. 
If it continues to fall, the resulting tenantization would radically change Canadian social structure and have a profound negative impact on the distribution of wealth. Our analysis raises the question of whether Canadian cities, given the current housing crisis and the progression of financialization, can continue to attract people and accumulate wealth for residents and industries.

We also explore the possibility that financialization of the housing market may spill over to urban productivity, reducing urban growth and the urban wage. This is a worrying prospect, given the continued urbanization of the Canadian population and the widespread concern about falling Canadian productivity. We have identified a number of potential channels  for such spillover and suggest that identifying which, if any, are active should be high on the national % social-science
research agenda.

The purpose of our model is \gls{theoretical exposition}. It tells us about the relationship between variables in the model.  It makes no claims about the world outside the model, however it can suggest questions or directions for empirical work. With simple extensions and calibration the model is capable of generating testable empirical hypotheses on furthur policy issues.  % is a flow of money into the housing market but not into housing production, 

Contributions from this work, include that the model: 
\begin{enumerate}
    \item  Incorporates \gls{classical rent theory} into an \gls{agent-based} urban model. 
    
    \item Allows the creation and distribution of rents to influence urban growth. 
    
    \item Incorporates current research on \gls{urban scaling} into an explicitly spatial framework.
    
    \item Integrates an urban \gls{agent-based model} with {neoclassical growth theory}. 
    
    \item Integrates \gls{financial capital} into a standard spatial model.
    
    \item Integrates financial capital into an \gls{overlapping generations} population model of the urban system.
    
    \item Uses an agent-based model to examine how financial markets impact an urban \gls{land market}. 

    \item Can be easily extended to explore a range of issues and used to evaluate policy options. 
\end{enumerate}
This work thus contributes to understanding how financialization and productivity interact.   
{\color{red} NEED TO ADD A BIT MORE TO CONCLUDE}


% WHAT ARE THE 6 POLICY EXPERIMENTS
% % We built a model that brought these togterh SUMMARIZE...Based on .... REVIEW TEHORETICAL BASIS FOR HYPOTHESIS... we hypothesized WHAT
% SUMMARIZE CONCLUSIONS.... AS EXPECTED AND WhAT WAS SURPRISING>...
% There are two classes of results first on ownership, the second on productivity. ADD DETAIL

% THIS WORK LAYS THE GROUND WORK FOR FUTURE STUDY OF ....

% The analysis suggests that in addition to the recognized distributional consequences, the housing crisis has productivity impacts that should be considered in developing urban and housing policy. 


% \section{Consider}
% OR *** ADD BACK? any of this


% sketches how this thesis relates to four major fields: classical rent theory, neoclassical production theory and growth theory, the scaling literature, and urban spatial models. % \dots , and the role of space as a unifying factor across three of the fields. % WITH FINANCE IS SPACELESS.
%     This work draws together sub-literatures including rent theory, production functions, the standard urban model, growth theory, urban growth theories, financialization, and the theory of distribution, so the chapters review those areas. % *** link the areas to the chapters better?  %theory for our analysis, 

%     , using an approach similar to that developed in modern growth theory, that we discuss in Chapter~\ref{chapter-rent}.  In that chapter we ground the observation in  \gls{neoclassical growth theory} and recent empirical and theoretical work on \gls{urban scaling}. 
%  We take a step beyond integrating labour markets in a city, to studying the distributional effects: who gets the surplus, what does that mean for the class structure, and ultimately the productivity of cities. 
% We began with the fact that there is growing policy concern about the financialization of  housing. We have produced 


% \section{OTHER PIECES TO CONSIDER ADDING}

%  % We take a step beyond integrating labour markets in a city, to studying the distributional effects: who gets the surplus, what does that mean for the class structure, and ultimately the productivity of cities. 
% % We began with the fact that there is growing policy concern about the financialization of  housing. We have produced 

% % In this thesis we began with the broad question of how urbanization and financialization interact. To explore this question, 


% % THINGS TO INCLUDE


% % VS  This work draws together sub-literatures including rent theory, production functions, the standard urban model, growth theory, urban growth theories, financialization, and the theory of distribution. % , so the chapters review those areas. % *** link the areas to the chapters better?  %theory for our analysis, 

%     % , using an approach similar to that developed in modern growth theory, that we discuss in Chapter~\ref{chapter-rent}.  In that chapter

%  % THIS WORK DRAWS TOGETHER - THEORETICAL FRAMEWORKS....We sketch sketch how this thesis relates to four major fields: classical rent theory, neoclassical production theory and growth theory, the scaling literature, and urban spatial models. % 
%  % TODO add backThe organizing principle in the spatial models of all three disciplines is an economic variable, land rent.  The three disciplines share a simple economic insight.  ADD LINK and the role of space as a unifying factor across three of the fields. Past work did not  bring together spatial insights with production insights in this way.... Because its a economic and urban question FIX  -- WITH FINANCE IS SPACELESS.

% % We have used what we call the  \textbf{\gls{Alonso-Jacobs model}} to explore the source and distribution surplus value. We  work with an extension of the basic Alonso model that incorporates the \glspl{agglomeration effect} that Jane Jacobs  described in her book, The Economy of Cities \cite{jacobsEconomyCities1969}. In our model these effects generate the \gls{urban wage premium} central to urban growth. % and the wage premium. 

% % The result is a simple model in which marginal productivity determines the wage, the wage determines the size of the city, the size of the city determines the labour supply, and labour supply determines marginal productivity. 

% % The economics is clear that this is what's at stake is productivity of cities, the distributive features of the economy and the impact of the middle class. 
% % Highlights the urgent need for more empirical work.


% % Particularly, it centers concern with implication for urban development of growing rent extraction by the financial sector. 
% % Our focus is land rents, %but in the context of an urban economy. 


% % This appears, at least part of it appears as locational rents. 
% % Financialization, is about capturing the surplus generated by the city.  % To model the financialization of land markets.

% % To model financialization we need rent because
% % To develop a formal model of financialized urban land markets, we introduce rent because rent is % precisely   about extracting and allocating surplus value in a system. % and that is what financialization of the housing market is about/does. 
% % The classical approach to rent is a core tool in the development because it brings the extraction of surplus into focus.
% % GAP Nobody has linked the rents - linking rents to urban scaling. Beteencourt is talkign about a surplus in the system, wealth, but he hasn't linked to the market/land market for those locational rents.


% % - hamstrings the whole thing.
% % --> the whole system as a welfare producing system fails if the value gets sucked out---CONCLUSION TH THESIS---fails from a social point of view-
% % these are averages---some are structurally below average so some are always behind simply because of the structure of the rents claimed.. that's built in FUTURE WORK- DIFFERENT INCOMES GETS YOU THAT. 


% % We could run off a cliff and accidentally destroy the middle class, we should consider the implications, need a language to explore that.
% % ---


% % We examine the effect of housing on wealth inequality by looking at 

% % Adding 2 things 1. rent extraction and 2. power law scaling of productivity, we find rent is the breaks on the engine of wealth creation

% % If the links are correct. 



% % There is a market for the urban product produced by firms, and a financial market that agents can invest in.

% % ---


% % WAS AT END OF SPACE CHAPTER 
% EXPLAIN THE CONTRIBUTION MORE CLEARLY
% Other theories of space and transportation costs  don't examine distribution or how the distribution of rents might affect the productivity of the city, which means they throw no light on the effects of the financialization of the property market on distribution or on urban productivity.  This is the limitation we seek to address in this thesis. 

% We build the  standard urban model into our model of financialization by using the concept of rent as it appears in the urban system


% We explore distributional consequences, how the distribution of rents feeds back into the productivity of cities, and how the urban economy is changed by the financialization of urban housing. % and relating them to neoclassical growth theory.  
% % % This chapter introduces the background to the theory of the urban model. 
% % E ADDED: 
% By combining these pieces we can look at how ownership changes with finalization, how that can feedback into changing the process of value creation in the urban center.
% % Rent provides a  neglected indicator or state variable in the urban system. It can be  seen as an indicator of the state of teh system that has been largely neglected.





% ----

% % The 3 parts of this do xyz
% % The dissertation is organized into three parts: background, methodology, and analysis. The background section provides theoretical foundations and reviews relevant literature, while the methodology section outlines the model framework. The analysis section presents results from simulations and discusses their implications.






% % We have a model of productivity to change. 
% %\textbf{We are in a housing crisis.  These are things that can illuminate this particular.. **}
% We examine how urbanization and financialization interact through urban housing markets. %The dissertation examines a pair of linked hypotheses about the evolution of an urban system: firstly, 
% % ## Contributions:

% %To test the 
% %We built an agent based model to explain tenatization.
% % models for each class of agent in the system and then allow agents to interact. 
% % Firms hire workers if it is profitable. Firm productivity and wages rise with population due to agglomeration effects. 

% % Rural workers near the city apply for urban jobs if the urban wage justifies travelling to the city center to work. Homes come up for sale when workers retire New entrants buy homes if they can afford homes, otherwise they become tenants. Investors purchase or sell properties if the net rent and capital gains exceed the return on alternative investments. % this paragraph describes an agent-based approach and establishes the independence of the agents 

% %They don't live in the city, they can't reinvest, they don't belong to the class of those who own property and can save through homeownership. 

%  %in a %spatially explicit urban land market. 
% The most complex part of the model is the financial bidding for property by different agents. Firm technology and the price of output are held constant. Investors represent the supply of global capita and are by assumption non-resident.

% It also allows us to see how the locational rents generated by the city is distributed between investors or residents.  % This paragraph says that results emerge from the interaction of independent agents./
%  % is captured through land rents. 
% By capturing rents, financialized actors are able to capture a share of urban productivity.  
% In the model, this affected the systemic effects of the shifting ownership of property in cities on the distribution of wealth and the potential that this shift can actually change the ability of cities to grow, thrive, and produce wealth. 
% % Big chunks simplified but the guts are in financial behaviour. 

% % It is helpful for economic policy in complex environment. 
% The model allows us to test qualitative effects of various policies.

