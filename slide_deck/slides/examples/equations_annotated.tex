\begin{frame}
\[
\tikzmarknode{A}{f}(\tikzmarknode{B}{5}) = \tikzmarknode{C}{25}
\begin{tikzpicture}[overlay, remember picture,shorten <=1mm,
                    nodes={inner sep=1pt, align=center, font=\footnotesize}]
\draw (A.south) -- ++ (-1,-1) node[below] {The\\ machine};
\draw (B.south) -- ++ (.2,-1) node[below] {When\\ given 5};
\draw (C.south) -- ++ (.8,-1) node[below] {Gives\\ us 25};
\end{tikzpicture}
\vspace{3ex}
\]
\end{frame}

\begin{frame}
\[
\tikzmarknode{A}{f}(\tikzmarknode{B}{5}) = \tikzmarknode{C}{25}
\begin{tikzpicture}[overlay, remember picture,shorten <=1mm,
                    nodes={inner sep=1pt, align=center, font=\footnotesize},
                    every path/.style = {draw=red, Stealth-}] % <---
\draw (A.south) -- ++ (-1,-1) node[below] {The\\ machine};
\draw (B.south) -- ++ (.2,-1) node[below] {When\\ given 5};
\draw (C.south) -- ++ (.8,-1) node[below] {Gives\\ us 25};
\end{tikzpicture}
\vspace{3ex}
\]
\end{frame}


\begin{frame}{A slide with centered equation and explanation}
  \begin{center}
    \begingroup\larger[4]
    \[
    \tikzmarknode{n1}{\rho\frac{D\vec V}{Dt}}~=~\tikzmarknode{n2}{-\nabla p}
    + \tikzmarknode{n3}{\rho\vec g}+\tikzmarknode{n4}{\mu\nabla^2\vec V}
    \]
    \endgroup
  \end{center}
  
  \vspace{1ex}
  
  \begin{center}
    \begin{tikzpicture}[remember picture,shorten <=1mm,font=\footnotesize\sffamily]
      % Your TikZ code for the explanation here
    \end{tikzpicture}
  \end{center}
\end{frame}


\begin{frame}
\[\int\limits
  _{\tikzmarknode{a}{a}}
  ^{\tikzmarknode{b}{b}}\tikzmarknode{f}{f}(\tikzmarknode{x1}{x})
  \,\tikzmarknode{d}{\mathrm{d}}\tikzmarknode{x2}{x}
\begin{tikzpicture}[overlay,remember picture,cyan,>=stealth,shorten
 <=0.2ex,nodes={font=\tiny,align=left,inner ysep=1pt},<-]
  \draw ([xshift=-0.3ex]b.west|-x2) -- ++ (-1.5em,0) node[left] (l) {\textbf{Integral}};
  \draw (a.south) -- ++ (0,-1.5em) node[anchor=north west,align=left,xshift=-1.2ex]
    (u) {\textbf{Untergrenze}\\ Untere integrationsgrenze};
  \draw (b.north) -- ++ (0,1.5em) node[anchor=south west,align=left,xshift=-1.2ex]
    (o) {\textbf{Obergrenze}\\ Obere integrationsgrenze};
  \path (x1.north) ++ (0,1.5em) node[anchor=south west,xshift=-1.2ex] (il)
    {\textbf{Integrand}\\ Funktion \"uber die integriert werden soll};
  \draw (x1.north) |- ([xshift=0.3ex]il.south east);
  \path (d.south) ++ (0,-1.5em) node[anchor=north west] (diff)
    {\textbf{Differtential}};
  \draw (d.south) |- ([xshift=0.3ex]diff.south east);
  \draw ([xshift=0.3ex]x2.east) -- ++ (1.5em,0) node[right]  (r)
   {\textbf{Integrationsvariable}\\ Der Integrand wird \"uber $x$ integriert};
  \path let \p1=($(o.north)-(u.south)$),\p2=($(r.east)-(x2.east)$),
   \p3=($(a.west)-(l.west)$),\n1={\x2-\x3} in 
  \pgfextra{\xdef\tmpvspace{\y1}\xdef\tmphspace{\n1}};
\end{tikzpicture}\vcenter{\vspace{\tmpvspace}}
\hspace{\tmphspace} 
\]
\end{frame}



\begin{frame}{A slide with centered equation and annotations}
  \begin{center}
    \begin{relsize}{4}
    \begin{align*}
    \tikzmarknode{n1}{\rho\frac{D\vec V}{Dt}} &= \tikzmarknode{n2}{-\nabla p}
    + \tikzmarknode{n3}{\rho\vec g}+\tikzmarknode{n4}{\mu\nabla^2\vec V} \\
    \text{Total derivative} &= \text{Pressure gradient} + \text{Body force term} + \text{Diffusion term}
    \end{align*}
    \end{relsize}
  \end{center}
  
  \vspace{1ex}
  
  \begin{center}
    \begin{tikzpicture}[remember picture,shorten <=1mm,font=\footnotesize\sffamily]
      % Your TikZ code for the explanation here
    \end{tikzpicture}
  \end{center}
\end{frame}

\begin{frame}{Equation and Annotations}
  \begin{center}
    \begin{relsize}{4}
    \begin{align*}
    \tikzmarknode{n1}{\rho\frac{D\vec V}{Dt}} &= \tikzmarknode{n2}{-\nabla p}
    + \tikzmarknode{n3}{\rho\vec g}+\tikzmarknode{n4}{\mu\nabla^2\vec V} \\
    \text{Total derivative} &= \text{Pressure gradient} + \text{Body force term} + \text{Diffusion term}
    \end{align*}
    \end{relsize}
  \end{center}
  
  \vspace{1ex}
  
  \begin{center}
    \begin{tikzpicture}[remember picture,shorten <=1mm,font=\footnotesize\sffamily]
      % Your TikZ code for the explanation here
      \begin{scope}[nodes={text width=6.5em,align=left},node distance=1ex]
        \node (e1) {\underline{Total derivative}};
        \node[right=of e1.north east,anchor=north west] (e2) {\underline{Pressure gradient}\\[1em]
          Fluid flows in the direction\dots};
        \node[right=of e2.north east,anchor=north west] (e3) {\underline{Body force term}\\[1em]
          External forces\dots};
        \node[right=of e3.north east,anchor=north west] (e4) {\underline{Diffusion term}\\[1em]
          For a Newtonian fluid\dots};
      \end{scope}     
      \node[below=0.5ex of e1.south] (eq){$=$};
      \node[below=0.5ex of eq] (tot) {$\rho\left[\frac{\partial \vec V}{\partial t}
      +\left(\vec V\cdot\vec\nabla\right)\vec V\right]$};
      \path (tot.south west) -- (tot.south east) coordinate[pos=0.25] (tot1)
        coordinate[pos=0.75] (tot2);
      \node[below left=2ex and 0ex of tot.south,text width=4em,align=left] (c1) 
          {Change of velocity};
      \node[below right=2ex and 0ex of tot.south,text width=4em,align=left] (c2) {Convective term};
      \draw[cyan,-stealth] (c1) -- (tot1);
      \draw[cyan,-stealth] (c2) -- (tot2);
      \begin{scope}[overlay]
      \foreach \X in {1,...,4}
      {\draw[cyan,-stealth] (e\X) -- (n\X);} 
      \end{scope}
    \end{tikzpicture}
    \end{center}
\end{frame}



\begin{frame}{Equation and Annotations}
  \begin{center}
    \begin{relsize}{2} % Adjust the scaling factor as needed
    \begin{align*}
    \tikzmarknode{n1}{\rho\frac{D\vec V}{Dt}} &= \tikzmarknode{n2}{-\nabla p}
    + \tikzmarknode{n3}{\rho\vec g}+\tikzmarknode{n4}{\mu\nabla^2\vec V} \\
    \text{Total derivative} &= \text{Pressure gradient} + \text{Body force term} + \text{Diffusion term}
    \end{align*}
    \end{relsize}
  \end{center}
  
  \vspace{1ex}
  
  \begin{center}
    \begin{tikzpicture}[remember picture, shorten <=1mm, font=\small\sffamily] % Adjust font size here
      % Your TikZ code for the explanation here
      \begin{scope}[nodes={text width=6.5em,align=left},node distance=1ex]
        \node (e1) {\underline{Total derivative}};
        \node[right=of e1.north east,anchor=north west] (e2) {\underline{Pressure gradient}\\[1em]
          Fluid flows in the direction\dots};
        \node[right=of e2.north east,anchor=north west] (e3) {\underline{Body force term}\\[1em]
          External forces\dots};
        \node[right=of e3.north east,anchor=north west] (e4) {\underline{Diffusion term}\\[1em]
          For a Newtonian fluid\dots};
      \end{scope}     
      \node[below=0.5ex of e1.south] (eq){$=$};
      \node[below=0.5ex of eq] (tot) {$\rho\left[\frac{\partial \vec V}{\partial t}
      +\left(\vec V\cdot\vec\nabla\right)\vec V\right]$};
      \path (tot.south west) -- (tot.south east) coordinate[pos=0.25] (tot1)
        coordinate[pos=0.75] (tot2);
      \node[below left=2ex and 0ex of tot.south,text width=4em,align=left] (c1) 
          {Change of velocity};
      \node[below right=2ex and 0ex of tot.south,text width=4em,align=left] (c2) {Convective term};
      \draw[cyan,-stealth] (c1) -- (tot1);
      \draw[cyan,-stealth] (c2) -- (tot2);
      \begin{scope}[overlay]
      \foreach \X in {1,...,4}
      {\draw[cyan,-stealth] (e\X) -- (n\X);} 
      \end{scope}
    \end{tikzpicture}
    \end{center}
\end{frame}

