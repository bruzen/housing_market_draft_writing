% \section{Notation for Urban and Production Sectors}
\begin{longtable}{lp{10cm}}
\caption{Notation}                       \\

\hline           &  \textbf{Productivity} \\ \hline
$K$              &  Capital               \\ 
% $L$            &  Labour                \\
$N$              &  Population, equals labour \\ %, $L$                     
$Y=N^\gamma K^{\alpha }N^{\beta }$  &  Output \\ %Urban output            \\
$\alpha$         &  Elasticity of output with respect to capital          \\
$\beta$          &  Elasticity of output with respect to labour           \\ % vs effective labour
$\gamma$         &  Elasticity of agglomeration with respect to labour    \\ % , $\Lambda(n)$, for illustration \\

% $L$              &  Labour supply \\ %the number of workers, which, in the standard circular city model, equals the number of lots of size $s$  when workers live on identical individual lots. % Unless $d^{max}>d^*$ v  \frac{\pi}{s}(\frac{w}{\tau})^2 =
% $n_i$  &  Number of workers employed by firm $i$ \\
% n=\sum_i n_i$  &  Number of workers, the urban population in the model \\
% $\#f=\frac{n}{n_i}$&number of identical firms \\ %not used
% $f$  &  Number of firms =1 \\
% $n =f n_i$  &  Aggregate labour \\
% $\Lambda(n)$    &  Labour-augmenting agglomeration effect \\
% $n^\gamma$ & The labour-augmenting agglomeration effect,  modelled as an exponential function of the number of people \\
% $\Lambda(n)n_i$ &  Effective labour for firm $i$ \\
% $\Lambda'=\die{\Lambda(n)}{n} $ & Derivative of the labour-augmenting agglomeration effect\\

%%$Y_i=K_i^{\alpha }(\Lambda(n)n_i)^{\beta }$  &  Urban firm $i$'s output \\

%%$Y=\frac{n}{n_i}K_i^{\alpha }(\Lambda(\sum_i n_i)n_i)^{\beta }$  &  Aggregate output of all firms in the city \\
% $\die{Y}{n}=\beta\frac{1}{n} Y  \left( 1+ \frac{n\Lambda'}{\Lambda} \right)$  &  Social marginal product of labour \\
% $Y_i=K_i^{\alpha }(\Lambda(n)n_i)^{\beta }$    &  Urban firm $i$'s output \\
% $\die{Y_i}{K_i}	=\alpha \frac{1}{K_i} Y_i $  & Marginal product of capital for firm $i$ \\
% $\die{Y_i}{n_i}	=  \beta\frac{1}{n_i} Y_i $  &  Marginal product of labour for firm $i$ \\
%%$\eta=\frac{n_i\Lambda'}{\Lambda}$  &   Marginal agglomeration effect on a firm's output of increasing it's own labour stock \\
% \hline
	% &\textbf{Amenity}\\ \hline
% $A(d, n)$   &  Agglomeration amenity          \\

\hline           &  \textbf{Labour market}      \\ \hline %and urban stucture??
$\psi$           &  Rural wage                  \\ % 
$\omega$         &  Urban wage premium          \\
$\tau$ TODO change to $\gamma$?          &  Transportation cost per unit distance \\
$d$              &  Distance from city centre   \\
$d^* = w/\tau$   &  City extent \\ %, the maximum distance commuters will travel \\ % Maximum distance commuters will travel \\ % to get the wage premium \\
% $\mathcal{R} = w-\tau d$ &  Rent at distance $d$ \\ 
% $\zeta$          &  Population density at distance $d$     \\
% $s$              &  Lot size      \\
% $\psi$  &  ?Per-period cost of a unit of productive capital \\
% $\omega + \psi$  &  Urban wage including rural wage \\ %***
% $\textit{t}$ & {\color{red}transportation cost per km} \\%use   c?
% $w^n=w-\tau d$ & Wage  premium net of transportation costs \\
%% $\Omega=\frac{w+\psi}{\psi}$  &  Ratio of the urban wage to the  cost of capital \\
%% $\Pi$	   &  Profit \\
%% $ER$	   &  Excess return to capital \\ 
% \hline &\textbf{Spatial structure in the circular city} \\ \hline		
%% $d^{max} = w/\tau$  &  Maximum distance commuters at which residents enjoy the urban amenity \\
%% $d^{**} = max(d^*, d^{max})$  &  radius of the city \\
%% $U$                     &  Worker utility **\\ %, a function of location and prices \\
%% $U^{urban}=U^{rural} $  &  Migration equilibrium assumption ** \\
% \hline & \textbf{Labour market} \\ 

\hline           & \textbf{Financial market}          \\ \hline
$P$              &  Price of a property               \\ 
% $\dot P$       &  Rate of price growth              \\ % was $\dot p$  
$\mathcal{C}$    &  Capital gains                     \\ % was C
$\mathcal{C}_N$  &  Net capital gain, $C -$ net rent  \\
$M$              &  Mortgage                          \\ 
$m$              &  Mortgage share                    \\ 
$\mathcal{R}$    &  Rent                              \\
$\mathcal{R}_N$  &  Rent                              \\
$\rho$           &  Rent ratio                        \\
$\phi$           &  \Gls{rent share}                  \\
$\mathcal{O}$    &  Operational costs                 \\
$\theta$         &  Operations ratio                  \\ % was $\kappa$
$\mathcal{T}$    &  Taxes                             \\ % was $\Sigma, \Xi$  
$\tau$           &  Property tax share                \\ % was t then $\sigma, \xi$
$r$              &  Interest rate                     \\
$\delta$         &  \Gls{discount factor}             \\
% $W$            &  Wealth                            \\
% $\psi$         &  Fraction with rent/operating costs\\
$t$              &  Time                              \\
$T$              &  Time period                       \\
\hline
\color{black}
\end{longtable}  

% Sigma is tax share, what is the tax rate.

% delta is density ** - - infinitesmal density increase as the city moves out. [[adjustment speed for wage N]] imagine a density function over the city. 

\newpage

\begin{longtable}{lp{10cm}}
\caption{Rent}                                                            \\
\hline
$\omega-\tau d$               &  Warranted (economic) rent                \\
$\mathcal{R}=\omega-\tau d$   &  Equilibrium rent payment of tenant       \\
PDV                           &  Present discounted value                 \\  
$\mathcal{R}^T$               &  PDV of rent collected over period $T$    \\ 
$\mathcal{R}^T_N=(1-\kappa-\sigma)\mathcal{R}^T$  &  PDV of net rent collected over period $T$ \\
\hline
\end{longtable}

\begin{longtable}{lp{10cm}}
\caption{Bidding Mechanism Notation}                                          \\
\hline
$\mathcal{R}_N$  &  Net rent                                                  \\ % was NR
$P_0$            &  Purchase price for a property                             \\
$P^T_e$          &  Expected price at the end of period $T$                   \\
$\bar r$         &  Prime interest rate                                       \\
$r^{target}$     &  Investor or banks target interest rate, $\bar r + margin$ \\
$r_i$            &  Agent $i$'s personal borrowing rate                       \\
$r_i^T$          &  Agent $i$ interest rate compounded over a period $T$      \\
$r_i^{disc}$     &  Agent $i$'s subjective discount rate (which may equal $r_i$) \\
$r_\delta$       &  Discount rate                                             \\ % was $discr_i$
$\delta_i^T$     &  Discount factor for agent $i$ over period $T$             \\
$m^W$            &  Wealth-based share of home price a worker can mortgage    \\ % $= m_i(W_i)$
$m^\omega$       &  Income-based share of home price a worker can mortgage    \\ % IS_i   IS_i(\omega+\psi)$$
$m_i = min(m^W_i, m^\omega_i)$  & Mortgage, the share of home price worker $i$ can mortgage \\

\hline
\color{black}
\end{longtable}  

Replace $r_i^{disc}$ and $r_\delta$ with
$r^\delta$ or $r_i^\delta$

Problem - really the compounded rate over a period is an amount. Should it not be a capital then?

\section{Additional Assumptions}
\begin{enumerate}
\item There is full employment, no frictional unemployment, and no labour adjustment costs.
\item Firms set output and factor inputs to maximize profits, so factors are paid the value of their marginal private product
\item Demand for output is perfectly elastic (constant price = 1)
\end{enumerate}
%Rural producers pay a wage $\psi$. this covers a standard house, lot, entertainment, diet and consumption pattern. We  choose units so that per-period cost of a unit of productive capital is also $\psi$