\chapter{Myths}

\section{Emelya's magic pike}


``A village simpleton called Emelya catches a magic pike, which promises to fulfil any of his wishes if he spares it. Emelya uses the magic wishes to take care of everything that he is too lazy to do himself. Thus, "at the pike's behest", an axe chops wood by itself, buckets are filled with water, and so on. Emelya is too lazy to get up from his warm stove that he commands the stove to transport him anywhere he needs to go. In the end, he marries the tsar's daughter''
\footnote{``One of the most archetypal Russian fairy tales since its protagonist, its protagonist, albeit lazy and simple-minded, is kind-hearted and relies on luck and the untranslatable Russian avos'' %[avos'](https://www.rbth.com/education/332624-russian-avos). 
}

% ``The plot about a magic fish that makes your wishes come true was also used by Alexander Pushkin in his Tale of the Fisherman and the Fish. There, however, the main characters are an old man and his old wife, whose greed leads to a tragedy as the magic fish takes away everything that it had granted to the old couple, and they are left at their old "broken trough" where they started.'' % [](https://www.rbth.com/arts/334386-russian-folk-fairy-tales)


\section{Build a bigger teremok}

There was a small wooden tower house (teremok) standing in the field. A mouse was running past, stopped and asked: "Who lives in this little house?" No-one answered, so the mouse decided to settle there. Then it was joined in the house by a frog, a hare, a fox, and a wolf… They were all living very amicably together in the teremok, until a bear showed up. They invited it to join them too, but the bear was too big to squeeze into the teremok through the door, so it decided to climb onto the roof ... and crushed the teremok. Luckily, the animals managed to jump out in time. Together they began to build a new teremok, and it turned out even better than the previous one - and they all fit in it.

% ``This is yet another incredibly popular Russian fairy tale, which inspired many Russian writers to create their versions of it. Whereas the interpretation created by poet Samuil Marshak became a favorite play for staging in children's and home theaters.There was a small wooden tower house (_teremok_) standing in the field. A mouse was running past, stopped and asked: "Who lives in this little house?" No-one answered, so the mouse decided to settle there. Then it was joined in the house by a frog, a hare, a fox, and a wolf… They were all living very amicably together in the _teremok_, until a bear showed up. They invited it to join them too, but the bear was too big to squeeze into the _teremok_ through the door, so it decided to climb onto the roof ... and crushed the _teremok_. Luckily, the animals managed to jump out in time. ==Together they began to build a new _teremok_, and it turned out even better== than the previous one - and they all fit in it. ..This is yet another incredibly popular Russian fairy tale, which inspired many Russian writers to create their versions of it. Whereas the interpretation created by poet ==Samuil Marshak== became a favorite play for staging in children's and home theaters.  [](https://www.rbth.com/arts/334386-russian-folk-fairy-tales)