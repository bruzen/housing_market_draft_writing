\chapter{Methodology} \label{chapter-methodology}

In this chapter we discuss the methodological approach of the thesis, as well as a number of modelling decisions. We develop a model of a complex, dynamical, multi-agent economic system, drawing on techniques and insights from several fields. In doing so we are attentive to the strengths and weaknesses of the approaches that we are adopting. 

We are focused on the evolution of a city. Cities evolve continuously and may never be in equilibrium. At best the numerous people and organizations in the city are adjusting individually in sensible ways to the changing parameters of the city they inhabit. The system is complex because there are many agents making %unsynchonized, 
individual decisions independently. %One natural  way to explore such a system tool urban processes this sort is analogous to the way 
Agent-Based Models (ABMs) mimic this view of the city.   

%Our main simulation tool is an ABM. 
In an ABM Agents are defined as having adjustment rules or behaviours that respond to environmental variables.\footnote{This approach is not unfamiliar to economists - The Cournot duopoly model, for example, is analyzed using `reaction functions' which simply describe a firm's optimal response to as second firm's output choice.} Agent based models commit to computational methods from the beginning. A program is written that considers each agent sequentially and updates agent and system statuses as it goes. The program is allowed to iterate, and the values of any state variables of interest are recorded at each step. The model may or may not settle into a steady state. As with any computational model, we can explore the behaviour of the system by varying parameters, and can explore the parameter space using Monte Carlo methods.

In Economic modelling, on the other hand, the most familiar approach is the  analysis of systems in equilibrium. A set of necessary conditions describing a steady state of interest are imposed and then solved to find features of the equilibrium they generate. It is a highly productive methodology partly because it conveniently  bypasses the complex process of adjustment by focusing on the conditions that must be true if a particular situation is to persist.

The classic example is the ubiquitous supply and demand model. Each of the curves represents the plausible behaviours of a class of agents. A situation is not likely to persist if either of the classes of agents is not satisfied with the combination of price and quantity. The conclusion is that the only combinations that can persist for long are  those that satisfy the behavoural intentions of both classes i.e. are on both curves. If the two curves can be described mathematically, equilibrium prices and quantities can be derived solving the two-equation system.

The approach produces tractable models that can often be solved explicitly. To achieve tractability, it is often convenient, even necessary, to limit the number of independent decision-makers by employing a ``representative agent.'' To extend the analysis to dynamic systems, ``laws of motion'' (adjustment rules) can be added in the form of either difference or differential equations.  Models become intractable quickly when there are more agents or dynamic processes, and economists, like other modellers,  % physicists and weather forecasters, 
resort to simulation techniques  and numerical methods. 

The problem of model complexity remains a challenge with ABMs, however, and modellers introduce simplifying assumptions. These are not as a rule needed for the computational model but are very helpful in maintaining a focus on the key theme  of the model. 

For example, for this model, we draw heavily on economic analysis of cities, and employ three key economic equilibrium conditions drawn from the economic literature. 
%Some adjustments are slow and some are fast. Rates of adjustment can matter. 
Because our focus is financialization in the land market, we use the equilibrium conditions for competitive labour markets, to bypass the complex and partially understood wage-setting process. We assume, a core theoretical result from the theory of the firm, that workers are paid their marginal value-product. We ground this approach in recent research that has %shown that, in essence, 
derived and estimated an aggregate urban production function consistent with modern neoclassical growth theory.\footnote{we are not aware of other research that has taken this step yet.} We use the result to derive an urban wage premium that drives city growth. We would argue that, since labour markets can adjust more quickly than land markets, we can assume that their adjustment is occurring continuously in the background.  

A similar market-based argument allows us to assume that land rents are determined by the classic urban locational equilibrium. The theory is based on the insight that if individual utilities are higher at one location in the city, entrants or movers will choose that  location. This is an implication of consumer choice theory. In a market with any degree of churn market prices for land should converge on use values, and these will vary systematically with transportation costs. For simplicity, we present the model for the case where individuals have the same preferences, employment opportunities and transportation costs. Building on this, %In the computational model 
we can introduce site-specific and person-specific features. 

Finally, we employ \gls{equilibrium reasoning} in the housing and investment process. We assume investors behave rationally in that they estimate the potential returns on their investments and seek the highest returns they can get. They calculate an offer price in each transaction based on several factors: land rents, expected price changes, their individual price of capital and discounting rate, and the information available to them.

It is at this point we introduce an innovation to formal urban modelling. We draw on the literature for the relevant `\gls{stylized facts}' about differences in these values based on wealth and income. This opens the possibility of urban segregation and the endogenous emergence of class distinction based on capital access, a process explored by John Roemer in his neglected General Theory of Exploitation and Class \cite{roemerGeneralTheoryExploitation1982}.  

\Gls{expectations} have to take into account some amount of current and past information. Forecasts based on recent price movements can give rise to price bubbles through either speculative or precautionary motives. Expectations anchored in \gls{perfect foresight} have different dynamics. We employ backwards-looking expectation formation anchored to current rental returns. We derive an investment rule of this sort in Chapter~\ref{chapter-financialization}. 
          
In our model individuals have a working life, then retire. If they own homes they sell and move to the countryside where housing is cheaper. They are replaced by new entrants from the countryside. New entrants may become buyers or tenants. This is a variant of the overlapping generations  model in which savings are passed on to the next generation. In our case homes are passed on. While the model has multiple generations, it is not a `dynastic' model in which the generations are linked by family ties and inheritance maters in individual decision-making, because the core issue in the emergence of classes is whether the city continues to provide opportunity for newcomers and those without the benefit of generational wealth to share the  benefits of agglomeration effects. % to share in the . %We suspect that introducing families would complicate the model but add little to understanding  our  main  issue. 

A modelling device that plays a major role is our use of the \gls{urban wage premium}, the difference between a rural and an urban wage driven by positive agglomeration effects on productivity in the city. It cannot be competed away because transportation costs limit access to the productive centre. The wage premium is a partial measure of the agglomeration effect. It generates the rent premium on urban land. We want to call attention to the distribution of that rent premium.

To simplify the analysis, we assume that we can partition actual wages between the wage premium, which is available for accumulation or spent on transportation, and all other expenses. We assume urban and non-urban populations all receive a subsistence wage which covers  the cost of buildings, food and other living costs and a base cost of land. This base cost is usually described as the cost of agricultural land. It is more precisely described as the opportunity cost of urban land, and we have simply extended the technique to include the opportunity cost of urban labour. 

There are advantages and disadvantages to this technique. The motivating advantage is that it lets us isolate the social surplus generate by agglomeration. One disadvantage is that it suppresses many issues that are of interest in other contexts - such as choice of home size and quality. In the computational model it is easy to allow for different building types, but we make not provision for choice. We do foresee adding a layer to the computation  model that incorporates redevelopment and construction and does allow for decisions about scale, type and location. 

Another disadvantage of technique in the analytical version is that it does not allow us to make speculation about the buildings part of our market process. It is a problem that can be dealt with in the computation model, but we believe the added complexity will not add much to our central argument.


For example, from the Alonzo model we implicitly impose the locational equilibrium condition,
\[U_i(d,\dots)=U_j(d, /dots)\]
This says that identical individuals must get the same utility living at at whatever distance from the city centre. If that were not the case, individuals would move to a location where they get higher utility.

different location would take a long time to work through the system. Prices, however, can adjust much more quickly. 

 It follows that 
\[\frac{\partial U_i(d, d(dots)}{\partial t}=\frac{\partial U_j(d, /dots)}{\partial t}\]
With identical transportation cost, 
land rents are determined by transportation costs  

We use  equilibrium arguments in our model to describe hiring decisions by producers and wage demands by workers. Equilibrium locational by commuters determines the extent and even population of the city. They also determine the pattern of land rents. 

% More recent work has looked at agent-based modeling, looking at the dynamics, space and individual work. 

\textbf{Some limitations of preceding work}%We offer three developments/extensions building on that work.}

Any adjustment process that requires people to move to a 
At the same time, because our ..

providing tractable models.  equilibrium analysis of marginal effects, and representative agents which hid distributional effects, as well as spaceless economic models of markets made it difficult to capture the richer spacial dynamics of urban rents, and the details of the ways economic forces play out for individual actors.

First they've not tended to build in, in a sophisticated way classical economic theory,
In general, classical economic theory has not been developed in agent based modeling work
Agent based models have begun with simple models, using and relaxing neoclassical assumptions, and building from first principles. This is absolutely the place to start. As the field matures, it makes sense to introduce theory in a more nuanced way, that connects with classical theory/the history of thought, etc

Second, relatively little agent based modelling work integrates with the neoclassical economic work in a way that makes the relation clear/holds the advantages. ABM work tends to both reject neoclassical approaches and rely on neoclassical assumptions.
More generally only a few agent models (e.g. spruce budworm) connect the analytic and agent models in a clear rigorous way. We focus on holding in addition to the relation with classical theory, a close connection with the many advances made withing neoclassical modelling

- this connection will make it easier to incorporate in teaching and for mainstream economists to engage on and build with.

This work builds, first, a simple conceptually clear model tightly integrated with the core economic modelling traditions, that builds on the theory of rent.

% ALSO (Agent modelling also tents to model individuals- we also take some steps to agent based modelling beyond the individual, and to the work developing model in a mode ideas)

Third, econ lacks resilience analysis and models, yet hysteresis clearly present in the relation between the built environment and econ activity. Although there's been work on dynamics and individual effects, there has been little work looking at the resilience dynamics in economic models, we take that approach looking at the resilience of community and individual wealth, and the relationship between that wealth and productivity. 

- This puts resilience dynamics at the center of economic analysis.

The resilience analysis looks at the dynamics of rent in economic boom and bust cycles.
There is a ratchet effect, achieved through hysteresis in the system, in which sucks wealth out of communities on the way up and on the way down. % DETAIL ONCE DRAFTED.


\section{Other notes - to sort}

% TEMP - here are some other notes we may want to reference or bring in. [[non individualistic modeling of agents]] [[generalizability in agent vs classical econ models]]

The emphasis is on clarity and connecting with the equilibrium in economics, and systematically relaxing each, to connect with the analytic tradition of economic modelling
The clarity of intuition of the neoclassical tradition with the deeper root of distribution theory rooted in classical economics and the breath and rigor possible with new tools from the study of complexity and statistical physics.

Methodological questions: 

    - agent models (integrating theory more completely into agent models)
    
    - rent theory

Core model

    - static version
    
    - dynamic version

Simulations
Result - hysteresis,
Policy
