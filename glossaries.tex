% Main glossary entries---definitions of relevant terminology
% \glsdisp{main}{varriation}

% \newglossaryentry{}
% {
% name=,
% description={}
% }

% \newglossaryentry{}
% {
% name=,
% description={}
% }

% \newglossaryentry{}
% {
% name=,
% description={}
% }

% \newglossaryentry{}
% {
% name=,
% description={}
% }

% \newglossaryentry{}
% {
% name=,
% description={}
% }

% \newglossaryentry{}
% {
% name=,
% description={}
% }

% \newglossaryentry{}
% {
% name=,
% description={}
% }

% \newglossaryentry{}
% {
% name=,
% description={}
% }


% \newglossaryentry{}
% {
% name=,
% description={}
% }

% \newglossaryentry{}
% {
% name=,
% description={}
% }

% \newglossaryentry{}
% {
% name=,
% description={}
% }

% \newglossaryentry{}
% {
% name=,
% description={}
% }

\newglossaryentry{impact channels}
{
name=impact channels,
description={For the purposes of this work...}
}

\newglossaryentry{the transmission puzzle}
{
name=the transmission puzzle,
description={}
}

\newglossaryentry{net present value}
{
name=net present value,
description={net of \gls{present value}}
}

\newglossaryentry{internal rate of return}
{
name=internal rate of return,
description={from the document 'in general the solution to a polynomial and does not guarantee a single-valued result \cite{robinsonOptimalTerminationIRR1996}. Multiple real-valued  IRRs may arise;  complex-valued IRRs may arise;  the IRR is, in general, incompatible with the net present value (NPV) in accept/reject decisions; the IRR ranking is, in general, different from the NPV ranking; the IRR criterion is not applicable with variable costs of capital. Ways to salvage the IRR as a usable criterion have been proposed that are consistent with our approach \cite{magniAverageInternalRate2010}, and are worth exploring.'}
}

\newglossaryentry{classical}
{
name=classical,
description={See \gls{classical economics}.}
}

\newglossaryentry{classical economic theory}
{
name=classical economic theory,
description={See \gls{classical economics}.}
}


% \newglossaryentry{classical economics}
% {
% name=classical economics,
% description={An approach to the study of the economy and economic behaviour that attempts to explain the production, pricing, and the consumption of goods and services through ownership of the factors of production, the cost-of-production theory from classical economics with a consumer demand theory based on utility maximization.}
% }
\newglossaryentry{classical economics}
{
name=classical economics,
description={Also called classical \gls{political economy}. %Referring to the period of economic theorizing, primarily in northern Europe, roughly between 1750 and 1870, prior to the neoclassical period in economics. 
A school of thought in political economy that flourished, primarily in Britain, in the late 18th and early-to-mid 19th century. It was part of the intellectual development of  Western liberal democracies in the 18th and 19th centuries and was brought into the mainstream by Scottish economist Adam Smith. Its main thinkers include Adam Smith, Jean-Baptiste Say, David Ricardo, Thomas Robert Malthus, and John Stuart Mill.  The term classical was coined by Karl Marx %(1847) 
as a description of David Ricardo's formal economics \cite{colanderDeathNeoclassicalEconomics2000}. After 1850, key features of the classical approach were carried forward by Marx, and his followers, and by Henry George. Classical economics provided the foundation for the development of \gls{neoclassical economics}. See also \gls{classical}.  Classical economic theory explains the cost-of-production, pricing, and the consumption of goods and services as a consequence of ownership of the factors of production.}
}


\newglossaryentry{capital gains tax}
{
name=capital gains tax,
description={a tax on the increase in value of a property that results form an increase in the price. The capital gains tax is a tax on speculative return on housing as an investment. In our model captures some or all of the expected gain from rising locational rents.}
}

\newglossaryentry{Alonzo model}
{
name=Alonzo model,
description={An circular city model developed in the 1960s in which transportation costs combine with residential mobility to determine land rents, and hence urban land use.}
}

\newglossaryentry{Alonzo-Jacobs cycle}
{
name=Alonzo-Jacobs cycle,
description={a feedback loop between an \Gls{Alonzo model} determining land rents Jacobs-style \gls{agglomeration} model determining productivity and population.}
}

\newglossaryentry{Alonzo-Jacobs model}
{
name= Alonzo-Jacobs model,
description={An urban model combining the Alonzo \cite{alonsoModelUrbanLand1960}model determining land rents with a Jacobs-style population-based \gls{agglomeration} effect \cite{jacobsEconomyCities1969} driving population size through its influence on productivity.}
}



\newglossaryentry{effective labour}
{
name=effective labour,
description={nominal labour augmented by \gls{human capital}. One skilled worker might be as productive as two unskilled workers. In that case,  the skilled worker would be represented as two units of effective labour in the  \gls{production function}. Similarly, when we introduce \gls{agglomeration} economies, labour effectiveness depends on the size  of the enterprise or community.}
}

% \newglossaryentry{prefactor}
% {
% name=,
% description={}
% }

\newglossaryentry{human capital}
{
name=human capital,
description={refers to the economic value of a worker's experience and skills. Human capital includes assets like education, training, intelligence, skills, health, and other things employers value such as loyalty and punctuality. It is a produced but intangible asset embodied in workers that affects firm productivity but can't be listed on a company's balance sheet. It is however a measurable asset of communities and even nations.}
}

\newglossaryentry{scaling law}
{
name=scaling law,
description={Scaling laws are the mathematical expressions of the \glsdisp{extensive property}{extensive properties} of a system. The laws are of the form  $B=c*A^\beta$, tying the size of $B$ tightly to the size of $A$ but allowing them to grow at different rates.}
}

\newglossaryentry{extensive property}
{
name=extensive property,
description={In classical  physics, an extensive property is a physical quantity whose value is proportional to the size of the system it describes. In more recent times, in statistical mechanics, or urban studies, for example, the expression ``extensive property'' has  been extended from  the ``scale-invariant'' case in which  doubling quantity A exactly  doubles property B (doubling the size doubles the weight) to allow for scaling laws of the form $B=c*A^\beta$. These \glspl{scaling law} still deal with extensive properties, but the classical usage is seen to be a special case: $B=c*A^1$.}
plural={extensive properties}
}



\newglossaryentry{discount rate}
{
name=discount rate,
description={A rate used in estimating the current value of future returns. Discount rates vary across time and individuals. It is used in calculating  the \gls{discount factor} for any given future period
} 
}

%A rate is a number like an interest rate.. a factor is what that adds up to over some period.. Every project can be described as a sequence of payments or costs in periods. Simplest is eveyr period- certain period away so you discount ot that time. Have a series of payments. Pay student fees, pay residents, income comes in positive. For each year, you need a separate discount factor, but they can all be based on the same discount rate. So you have a 3 year discount factor.. 1/(1+rate) to the power of 4. - factor what you get to tell you what it's worth today. considering a project, bring everything to present value. add up value to discounted at different factors.
% https://gocardless.com/guides/posts/how-to-calculate-discount-factor/#:~:text=The%20discount%20factor%20and%20discount,or%20its%20net%20future%20value.
 % ** TODO factor not a rate. distingiush value vs rate.



\newglossaryentry{discount factor}
{
name=discount factor,
description={The present value of one dollar received at particular point in the future, given the date of receipt and the \gls{discount rate}. It is the  factor by which a benefit at a particular point in the future is converted to present values. It is smaller when the wait for the future benefit is longer.  The most commonly used discount function is the exponential discount function, although that function fails to match several empirical regularities. The inverse of the continuously compounded \gls{discount rate}. }
}
% The theory of discounted utility is the most widely used framework for analysing intertemporal choices. The most commonly used discount function is the exponential discount function. The discount factor is the inverse of the continuously compounded discount rate r(t). $D(\tau)=\delta^\tau$.
% Despite its many appealing properties, the exponential discount function fails to match several empirical regularities. Most importantly, a large body of research has found that measured discount functions decline at a higher rate in the short run than in the long run. 

% Patience appears to increase across the lifespan, with the young showing markedly less patience than middle-aged and older adults (Green, Fry and Myerson, 1994; Green et al. 1996; Green, Myerson and Ostazewski, 1999). Read and Read (2004) report that older adults (mean age=75) are the most patient age group when delay horizons are only one year. However, this study also finds that older adults are the least patient group when delay description={horizons are from three to ten years. This reversal probably reflects the fact that 75-year-olds face significant mortality/disability risk at horizons of three to ten years.(intertemporal_choice.pdf

\newglossaryentry{disagreement point}
{
name=disagreement point,
description={In a bargaining game the disagreement point is a pair of values which that present lowest value each player will accept. Niether player will accept an outcome which gives them less than their disagreement value. We ussually assume that they have an alternative activity or pruchase available that is as attractive as their disagreement value for the bargain under consideration.  }
}

% \newglossaryentry{prefactor}
% {
% name=,
% description={}
% }

\newglossaryentry{maximum mortgage share}
{
name=maximum mortgage share,
description={a parameter used in calculating the share of a home price $m_i$ that the bank will finance for individual $i$ given $i$'s level of wealth. It is specific to the functional form used. The use of this parameter is based on stylized facts from the literature.  No empirical estimates are available.}
}



\newglossaryentry{factor of production}
{
name=factor of production,
description={(such as labour, land, financial capital,  and human capital)}
}


\newglossaryentry{efficient market}
{
name=efficient market,
description={A technical term used in the economic literature for a market in which all possible gains from trade are achieved. Efficiency is affected by the number of market participants, information availability, and limits to trading. In the financial literature market efficiency refers to the degree to which market prices reflect all available, relevant information. See also \gls{efficient}} % .
}

% \newglossaryentry{supply demand curve}
% {
% name=supply demand curve,
% description={}
% }


\newglossaryentry{emergent}
{
name=emergent,
description={A system property that is not inherent in any of the individual parts of the system. % Life is emergent from non-life for example, and language is emergent in certain social systems. Emergent properties are properties that are a result of processes within a system but are not features of the individual components themselves.
}
}

\newglossaryentry{critical parameter}
{
name=critical parameter,
description={A functional variables that disproportionately controls the behaviour of the system. In particular, criticality is linked to the parameter's effect on any critical quality attribute. %Used in pharmaceutical production for process variables which have an impact on a critical quality attribute. ** fix
}
}

\newglossaryentry{feedback loop}
{
name=feedback loop,
description={%Some portion or all of the 
Where an output from a systems is an input to a future cycle. Feedback loops may damp or reinforce change.}
}

\newglossaryentry{open system}
{
name=open system,
description={A %term form systems dynamics for a 
system that has flows into or out of the system, across the system's boundary. % to or from entities outside of the system. 
Such flows may affect stocks or flows in the system. Income from trade, or population flows, for example, may cross the boundary of the system.}
}

\newglossaryentry{second-price auction}
{
name=second-price auction,
description={%A Vickrey auction or sealed-bid second-price auction  is a 
A type of sealed-bid auction. Bidders submit written bids without knowing the bid of the other people in the auction. The highest bidder wins but the price paid is the second-highest bid.}
}

\newglossaryentry{cheap talk}
{
name=cheap talk,
description={Communication between players that does not directly affect the payoffs of the game. Providing and receiving information is free, non-binding, and unverifiable. This is in contrast to signaling in which sending certain messages may be costly for the sender, depending on the state of the world, and may affect payoffs.}
}

\newglossaryentry{sellers' bargaining power}
{
name=sellers' bargaining power,
description={A measure of the degree to which bargaining over price favours the seller. Sellers have an advantage when prices are expected to rise, a common expectation when prices have been rising.}
}

 \newglossaryentry{regional science}
{
name=regional science,
description={A field of the social sciences concerned with analytical approaches to problems that are specifically urban, rural, or regional. Its a discipline that is traditionally allied to economics in methodology.}
}

\newglossaryentry{system dynamics}
{
name=system dynamics,
description={System dynamics a computer simulation modeling methodology that is used to analyze complex nonlinear dynamic feedback systems for the purposes of generating insight and designing policies that will improve system performance. It was originally created in 1957 by Jay W. Forrester of the Massachusetts Institute of Technology as a methodology for building computer simulation models of problematic behavior within corporations.}
}

\newglossaryentry{efficient}
{
name=efficient,
description={The arrangement where it is impossible more nearly achieve a goal without adding more resources. Efficiency is increases when inputs are reduced with out losing the output of interest or output is increased without using more resources.}
}

\newglossaryentry{reproducible capital}
{
name=reproducible capital,
description={any tangible asset that can be duplicated (reproduced), such as a building or machinery.}
}


\newglossaryentry{consumer surplus}
{
name=consumer surplus,
description={%For an individual
At the individual level, consumer surplus is the net benefit from the purchase of any commodity, i.e. it is the total benefit to the person minus the cost paid. At the level of a market, consumer surplus is the sum of the individual consumer surpluses. In supply and demand diagrams it it shown as a triangle above the price line and below the demand curve.}
}

\newglossaryentry{excess profits}
{
name=excess profits,
description={}
}

\newglossaryentry{free entry}
{
name=free entry,
description={}
}

\newglossaryentry{the second circuit of capital}
{
name=the second circuit of capital,
description={}
}

\newglossaryentry{producer surplus}
{
name=producer surplus,
description={}
}

\newglossaryentry{scarcity}
{
name=scarcity,
description={}
}

\newglossaryentry{spatial rent}
{
name=spatial rent,
description={ See \gls{locational rent}}
}

\newglossaryentry{quasi-rent}
{
name=quasi-rent,
description={}
}


\newglossaryentry{margin}
{
name=margin,
description={edge or boundary. In economics the term has an expanded metaphorically supported technical usage. Ricardo referred to the \gls{extensive margin} as the geographical limit of production and emphasised that that limit was the limit of profitable cultivation. It was where a sensible person would stop expanding the area of cultivation for economic reasons. Later economists extended the notion to the stopping point for all kinds of decisions. Using calculus they identified the conditions under which going farther adding more const more than it added.  Margin   appeared as a metaphor in the adjective  \gls{marginal} and in compound terms like \gls{marginal product}, where it refers to the effect of a small change on some variable  such as a small increase in output  from a small increase fertilizer or labour employed. Focus on such quantities is the main feature of the \gls{marginalist} approach. }
}

\newglossaryentry{marginalism}
{
name=marginalism,
description={see \gls{marginalist}}
}

\newglossaryentry{produit net}
{
name=produit net,
description={Profit from a sale after having deducted the costs and charges related to the manufacture and marketing of a product.}
}

\newglossaryentry{competitive}
{
name=competitive,
description={See \gls{perfect competition}.}
}

\newglossaryentry{commodity}
{
name=commodity,
description={1) Any product  made for exchange on the market; 2) a basic good used in commerce that is interchangeable with other goods of the same type, usually  as inputs to the production of other goods or services; 3) raw materials or primary agricultural products.}
}

\newglossaryentry{marginalist}
{
name=marginalist,
description={A style of economic analysis that emphasizes marginal values as opposed to total values or average values. The significance of the distinction lies in the fact maximizing a function like the profit function or utility gives rise to expression in terms of marginal quantities like marginal cost, marginal revenue, and marginal utility. It is then possible to say a person wishing to maximize their profit or utility should want to satisfy the derived conditions on marginal values (prescription). Another step allows economists to assume that those conditions are likely to be satisfied (description). The style of argument generally relies on the use of calculus. The systematic shift to marginalist analysis is seen as the dividing line between classical and modern economics.}
}

\newglossaryentry{profit}
{
name=profit,
description={The amount retained from sale of a product after all costs including the normal cost of capital  been paid. This amount is the income of the enterprise. The normal cost of capital is thought of a price paid to the investors who lent their capital to the enterprise and must be paid as  much for its use as they would have received if they had lent to another project. }
}

\newglossaryentry{surplus value}
{
name=surplus value,
description={See \gls{surplus} and \dots value?}
}

\newglossaryentry{pseudo-rent}
{
name=pseudo-rent,
description={A term for profits used by Alfred Marshall to emphasize that profits are a form of rent, but differ in being subject to competitions  because capital, unlike land, is a produced input and is therefor only scarce in the short term. }
}

\newglossaryentry{Henry George Theorem}
{
name=Henry George Theorem,
description={A proof by Arnott and Stiglitz \cite{arnottAggregateLandRents1979} of the proposition that  that if economic activity is efficiently organized over a "large" space, aggregate land rents equal the aggregate losses from the decreasing returns to scale activities. In other words, consistent with the assertions of Henry George, under certain circumstances the \gls{single tax} would finance all the infrastructure costs of a city. }
}

\newglossaryentry{ground rents}
{
name=ground rents,
description={ all economic value accruing to owners of land, regardless of whether payments are explicitly made or the rents are imputed.}
}

\newglossaryentry{single tax}
{
name=single tax,
description={a tax on land and resources that Henry George and his followers suggest is capable of replacing all other taxes since, if properly implemented,it would capture all resource rents. }
}

\newglossaryentry{digitization}
{
name=digitization,
description={The process of converting stored information to digital form, or the process of replacing human mental and physical actions with digital processing and digitally mediated actions.}
}

\newglossaryentry{market}
{
name=market,
description={a means by which the exchange of goods and services takes place as a result of buyers and sellers being in contact with one another, either directly or through mediating agents or institutions.}%Britannica
}

\newglossaryentry{landowner}
{
name=landowner,
description={the class of people who receive income from their ownership of land. Usually reserved for those who receive all or most of their income from land ownership and who do not work their land themselves.}
}

\newglossaryentry{working class}
{
name=working class,
description={In Classical and Marxist theory, the category of people who had only their labour to sell were called working class}
}


\newglossaryentry{petite bourgeoisie}
{
name=petty bourgeoisie,
description={a French term that refers to a social class composed of semi-autonomous peasants and small-scale merchants. They are characterized by their ownership of small amounts of productive capital - land or property and equipment.}
}

\newglossaryentry{classical rent}
{
name=classical rent,
description={This term is sometimes used to refer to Ricardo's definition of rent and the value of the original newt productivity of land, as distinct from rental price for a property or the more general notion of economic rent. See \gls{classical rent theory}, or \gls{economic rent}.}
}

\newglossaryentry{economic rent}
{
name=economic rent,
description={Any payment to the owner of a factor of production in excess of the cost needed to bring that factor into production. In classical economics, economic rent is any payment or benefit received for non-produced inputs such as location and through creating official privilege over natural opportunities. See \gls{rent}}
}

\newglossaryentry{locational rent}
{
name=locational rent,
description={Income or payment for the use of location. Locational value is largely created by access to people not by landowners, hence any payment for locational advantages is a rent. See \gls{rent} or \gls{land rent}.}
}

\newglossaryentry{land rent}
{
name=land rent,
description={Ricardo payment for  the natural productivity of the land, but also considered proximity to markets (locational advantages) as a source of rent. }
}

\newglossaryentry{intensive margin}
{
name=intensive margin,
description={Distinguished from the \gls{extensive margin}, which is a locational concept. Intensive refers to  enhancing the productivity of  the land by adding labour, fertilizers or other inputs, i.e. by intensifying cultivation efforts.  The intensive margin refers to the maximum intensity of additional factors of production that makes sense economically. }
}

\newglossaryentry{extensive margin}
{
name=extensive margin,
description={A term from Ricardian rent theory that refers to either land at the greatest distance from the market, or land that has the minimum fertility to justify bringing it into commercial production. A feature of land at the margin is that it generates no \gls{economic rent}.}
}

\newglossaryentry{generalized arithmetic mean}
{
name=generalized arithmetic mean,
description={A family of functions for aggregating sets of numbers. One special case is the geometric mean,  and the Cobb-Douglas function is a special case of that.} % Wikipedia provides a \href{https://en.wikipedia.org/wiki/Generalized_mean}{useful discussion}. }
}

\newglossaryentry{transmission mechanism}
{
name=transmission mechanism,
description={A general term to describe the sequence of processes through which an action at one point in a system  affects a variable at another point in the system. It is commonly used when discussing  monetary policy and how  expanding the money supply eventually affects employment.}
}

\newglossaryentry{real asset}
{
name=real asset,
description={Real assets are physical assets that have an intrinsic worth due to their substance and properties. Real assets include precious metals, commodities, real estate, land, equipment, and natural resources. }
}

% % Do we want to just make this feedback loop and feedback cycle and make uses in glossary consistent?

\newglossaryentry{feedback}
{
name=feedback,
description={The result of a causal loop. A term used in cybernetics and systems theory referring to a situation in which a change in one variable affects a second variable that then affects the first one.}
}

\newglossaryentry{surplus}
{
name=surplus,
description={Any amount or production or value in excess of what  is needed to pay for all the required inputs. Profit or rent. }
}

\newglossaryentry{Ricardian rent theory}
{
name=Ricardian rent theory,
description={The version of classical rent theory propounded by David Ricardo in his essay on the corn laws and generally seen as the  canonical version of land rent theory.}
}

\newglossaryentry{land market}
{
name=land market,
description={The entire complex of institutions, agents, and rules involved in transferring ownership of land. A land market exists wherever it is possible to exchange rights in land for agreed amounts of money or services rendered.}
}

\newglossaryentry{Alonso-Jacobs cycle}
{
name=Alonso-Jacobs cycle,
description={A positive \gls{feedback} cycle that occurs when city population is increasing in the wage, as in the Alonso model, where the wage is increasing in city population, as implied by the Jacobs component of the \gls{Alonso-Jacobs model}.}
}

\newglossaryentry{Public-Private Partnerships}
{
name=Public-Private Partnerships,
description={A long-term arrangement between a government and private sector institutions, often  employed for building, equipping, operating and maintaining schools, hospitals, transport, water, and sewerage systems. PPPs are used for projects with high social but low private returns when government is unwillling or unable to provide the up-front capital cost. The private rate of return is often subsidized by a guarantee that the private investor will receive a share of the social return over the course of the project's operation.}
}

\newglossaryentry{rent-seeking}
{
name=rent-seeking,
description={An economic concept that refers to the activity, seeking to gain wealth without contributing to productivity. Gordon Tullock, who introduced  the term, identified it as a form of theft\cite{tullockWelfareCostsTariffs1967}.  %Rent-seeking is the act of growing one's existing wealth without creating new wealth by manipulating the social or political environment. 
\Gls{rent-seeking} activities have negative effects on the rest of society. They result in reduced economic efficiency through misallocation of resources, reduced wealth creation, lost government revenue, heightened income inequality.}
}

\newglossaryentry{middle class}
{
name=middle class,
description={A broad and fuzzy term used by sociologists to describe the members of the working classes who have equity and a standard of living above the subsistence level. The OECD includes anyone who earns between 75 per cent and 200 percent of median household income after tax. Based on the most recent data available from Statistics Canada, in this country that means anywhere from about \$45,000 to \$120,000. The middle class is usually defined in terms of income level. The middle class defined this way, once the economic stratum of a clear majority of North American adults, has steadily contracted in the past five decades according to
Rakesh Kochhar and  Stella Sechopoulos of the \href{https://www.pewresearch.org/fact-tank/2022/04/20/how-the-american-middle-class-has-changed-in-the-past-five-decades/}{Pew Research Centre}  in 2022.}
}

\newglossaryentry{rentier}
{
name=rentier,
description={A person living on income from property or investments rather than from current income. The term is from the  French \textit{rentier}, ``holder of rental properties or investments that pay income,'' from \textit{rente} `profit, income' \cite{GET_rentier_defn_quote}. %``Financial engineering has created a rentier class, a modern feudal system, and the biggest beneficiaries of all that extra debt have been the bankers.'' Times, Sunday Times (2016) 
}
}

\newglossaryentry{rate of return}
{
name=rate of return,
description={or return on investment: the money made or lost on an investment over some period of time. Expressed nominally as the change in dollar value of an investment over time or  as a percentage derived from the ratio of profit to investment. We compute the nominal return, convert it to a percentage and compare that to the investor's best alternative return or required return.}
}

\newglossaryentry{joint-stock company}
{
name=joint-stock company,
description={A joint-stock company is a business owned by its investors, with each investor owning a share of the company based on the amount that they've invested. It is a predecessor to the modern-day corporation and other types of registered companies. A joint-stock company is an artificial person; it has legal existence separate from persons composing it. It can sue and can be sued in its own name. The shareholders are usually not liable for any of the company debts that extend beyond the company's ability to pay up to the amount of them.}
}

\newglossaryentry{REIT}
{
name=REIT,
description={Real Estate Investment Trust. A REIT is a financial instrument that  makes it possible for individual investors to earn dividends from real estate investments without having to buy, manage, or finance properties themselves. Structured as a company that owns and sometimes operates income-producing real estate or related assets, REITs are modeled after mutual funds \cite[GET-reit-like-mortgages].} %cite REITs are modeled after mutual funds? 
}

\newglossaryentry{financial instrument}
{
name=financial instrument,
description={A monetary contract which confers a right or claim against some counterparty. It may involve a payment (for instance checks, bearer instruments), equity ownership or dividends (such as stocks), debt (such as deposit accounts, bonds, or loans), currency (such as foreign exchange), or derivatives (such as futures, forwards, options, and swaps). There are %\href{https://www.investopedia.com/terms/f/financialinstrument.asp}{many types} 
many types of financial instruments.} % \cite[WEB-investment-types].}
}
% \newglossaryentry{financial instrument}
% {
% name=financial instrument,
% description={a tradable paper asset representing the ownership of a stream of future benefits. Some examples of financial instruments are cheques, shares, stocks, bonds, futures, and options contracts. Financial instruments are defined in opposition to non-financial assets which are tangible or intangible properties.}
% }

\newglossaryentry{compound interest rate}
{
name=compound interest rate,
description={If the calculation of interest rate for a later period includes interest on the interest from earlier periods, the interest is said to `compound.' Where an interest rate is specified for a single term, such as a year, the rate for a longer, multi-period term is larger. Compound interest for a given period is calculated by multiplying the initial principal amount by one, plus the annual interest rate, raised to the number of compound periods, minus one.} % This is how interest is usually computed. 
}

\newglossaryentry{amortize}
{
name=amortize,
description={to reduce an amount gradually by making payment in installments: a to pay off (as a loan) gradually usually by periodic payments of principal and interest. }
}

\newglossaryentry{appraised value}
{
name=appraised value,
description={an evaluation of a property's value at a given point in time. The evaluation is typically performed by a professional appraiser during the mortgage origination process.}
}

\newglossaryentry{premium}
{
name=premium,
description={The difference between the base price and the price paid in a particular market or buy a particular buyer. In our model is is the difference between the wage of rural workers and the wage of urban workers required to induce workers to live in the city and incur commuting costs. See \gls{urban wage},\gls{urban wage premium}, \gls{rent premium}.}
}

\newglossaryentry{subsistence frontier}
{
name=subsistence frontier,
description={The minimum income or lowest standard of living that can sustain people in the economy. Rather than thinking of the limit as a single value---say the minimum survival income---it is more realistic to recognize that the limit can be achieved with different combinations of goods. For example, if clean water is freely available in a local stream, the subsistence income does not include the cost of bottled water. All the combinations can be seen as a \gls{frontier}. \newline In classical economics, the frontier was summarized as a subsistence wage. Subsistence theorists like Malthus argued that the market price of labour would not vary from the natural price for long: if wages rose above subsistence, the number of workers would increase and bring the wage rates down. The classical economist recognized that the limit was in part set by social convention, but it was analytically convenient to assume a subsistence wage, and it could be argued, following Malthus that a subsistence wage  represented a long-term limit or \gls{equilibrium}. As an analytical convenience in our model, we employ a subsistence wage that includes housing and a conventional standard of living.  }
}

\newglossaryentry{political economy}
{
name=political economy,
description={Political economy is a branch of social science that studies the relationship  between government and the economy. As a discipline, it dates back the  16$^{th}$ but is usually associated with the political economists of the mid-18$^{th}$ and  early 19$^{th}$  century like Adam Smith who began to explore the economic implications of free markets and industrialization. Departments of political economy persisted well into the mid 20$^{th}$ C before splitting into separate departments of economics politics \cite{helleiner20PoliticalEconomy2018}.}
}

\newglossaryentry{expected market price}
{
name=expected market price,
description={The price is expected to emerge in a \gls{market} at a future point in time as a result of the interaction of buyers and sellers. It may vary as the mixture of buyers and sellers changes or as their information changes.}
}

\newglossaryentry{market price}
{
name=market price,
description={The price that emerges in a \gls{market} as a result of the interaction of buyers and sellers. It may vary as the mixture of buyers and sellers changes or as their information changes.}
}



% \newglossaryentry{perfect}
% {
% name=perfect,
% description={}
% }

\newglossaryentry{total factor productivity}
{
name=total factor productivity,
description={total-factor productivity (TFP), also called multi-factor productivity, is usually measured as the ratio of aggregate output  to aggregate inputs. It is a scale factor used to explain why the same combination of inputs produces different quantities of output at different places or times. It appears the factor  $A$ discussed in Chapter~\ref{chapter-growth} and in cities is influenced by the size of the population.  }
}

\newglossaryentry{perfect competition}
{
name= perfect competition,
description={An imaginary but analytically useful ideal market condition with the following  characteristics: 1. Large numbers of buyers and sellers in each market so that no individual buyer or seller can affect the price. 2. Free entry and exit of firms in the market. 3. Firms in each market sell a homogeneous product. 4. Buyers and sellers possess complete knowledge of the market. 5. No price controls.\newline  Economists often compare the markets they study to the` idealized, perfectly competitive market structure.}
}

\newglossaryentry{frontier}
{
name=frontier,
description={In mathematical economics, the limit of what is possible. Like the frontier of a country, even if you can't cross it, you can move along it to find the best location  subject to that constraint. In elementary economics, the budget-line and the production possibilities frontier (PPF) are  frontiers. Tf you spend less than the budget are operating inside the PPF, you could do better. Your solution is inefficient. }
}

\newglossaryentry{attractor}
{
name=attractor,
description={In \gls{dynamical system} theory as described by difference or differential equations, an attractor is a point or orbit inside a region of the phase space. The phase space is a representation of all possible states of the system each corresponding to a unique point in the phase space. If there is an attractor in the region, if the system starts at any point in the region, it will eventually evolve to the attractor. Types of attractors include point attractors, limit cycle attractors, torus attractors, and strange attractors}
}

\newglossaryentry{dynamical system}
{
name=dynamical system,
description={A system that changes over time. Typically refers to % we mean 
a system, described by a set of equations, at least one of which is time-dependent.}
}

\newglossaryentry{price formation}
{
name=price formation,
description={The process of selecting a price based on the conditions in a system. The classic problem is the simple supply and demand model, in which sellers and buyers, each group with its own wants represented by an equation, interact to find a a price. The model  identifies a combination of price and quantity that would be acceptable to both at the same time, but doesn't say how they get to the price. It lacks a price formation mechanism.  \newline The fundamental problem is that the agents don't have complete information and may have limited computational ability, especially with multiple interacting markets. A theory of price formation has to describe the process of adjustment. This is usually represented as a set of individual adjustment rules, which makes any theory of price formation a dynamical system It may not always lead to a steady state equilibrium.}
}


\newglossaryentry{market rent}
{
name=market rent,
description={The amount a landlord charges a tenant for the use of a property in a competitive market. }
}

\newglossaryentry{mill rate}
{
name=mill rate,
description={The municipal tax rate: the amount per \$1,000 of the assessed value of a property which will be due as property tax.}
}
{}
\newglossaryentry{financialization}
{
name=financialization,
description={Financialization is process of creating and using  \glspl{financial instrument} to facilitate exchange of real or financial assets.  The process may permit real goods, services, and risks to be more easily exchanged, providing access to unrealized economic value. Something is financialized when a financial instrument representing it is created. The home mortgage market was financialized when financial institutions developed markets that let investors buy and sell mortgages between themselves. These secondary transactions gave investors ownership of the stream of income established by the mortgage contract. The transaction did not affect the mortgage conditions or the home: they simply added a new product for investors to speculate on. }
}

\newglossaryentry{amenity}
{
name=amenity,
description={a desirable or useful feature or facility of a building or place.}
}

\newglossaryentry{population}
{
name=population,
description={In our model, the number of city residents. }
}

\newglossaryentry{financialize}
{
name=financialize,
description={Something is financialized when a financial instrument representing it is created. For example, mortgages originated in England when people did not have the resources to purchase land in one transaction. Buyers would get loans directly from the seller---no banks or outside parties were involved. Home mortgages were financialized when financial institutions developed markets that let them buy and sell mortgages between themselves.  See \gls{financialization}}
}

\newglossaryentry{housing market}
{
name=housing market,
description={A market is defined as the sum total of all the buyers and sellers engaged in the transfer of ownership of an asset, good or service, plus all of the institutional machinery that supports the transactions. }
}

\newglossaryentry{urban center}
{
name=urban center,
description={In our model the urban centre is a point at the center of a population \gls{agglomeration} where all employment is located. More generally it is the area within an urban \gls{agglomeration} with the largest concentration of employment and or commercial activities.} 
}

\newglossaryentry{functional form}
{
name=functional form,
description={the algebraic form of a relationship between a dependent variable and explanatory variables.}
}

\newglossaryentry{production}
{
name=production,
description={The process of converting a set of \glspl{input} into a desired \gls{output}. See\gls{factor of production}.}
}

\newglossaryentry{perfectly elastic}
{
name=perfectly elastic,
description={Producing more won't affect the product's price. The term describes a  horizontal supply or demand curve. See \gls{elasticity}.} % On a \gls{supply demand curve} (is that the right name?) ***}
}

\newglossaryentry{elasticity}
{
name=elasticity,
description={The ratio of the percentage change in a quantity to the percentage change in another quantity. The price elasticity of demand, for  example, is the percentage change in the quantity demanded that accompanies a one-percent change in price. It is a local property of a demand curve and is typically  a negative number, such as $-0.3$ or $-1.5$, since demand typically slopes downward. See \gls{perfectly elastic}.}
}

\newglossaryentry{labour augmenting agglomeration}
{
name=labour augmenting agglomeration,
description={The situation in which bringing more workers together increases their average productivity.}
}

\newglossaryentry{present discounted value}
{
name=present discounted value,
description={The amount that someone should be willing to pay, in the present, for a stream of expected future payments.}
}

\newglossaryentry{wealth}
{
name=wealth,
description={In our model, wealth is the set of valuable economic resources owned, by an individual or organization as measured in either real goods or money value, that the bank considers in lending decisions. We model only housing and aggregate financial wealth (savings), but more generally, wealth includes stocks of human capital, equities, land, and other more subtle assets.}
}

\newglossaryentry{input}
{
name=input,
description={In production theory, an input is any good or service used to produce another another good or service. % anything  that is among the collections of goods and services that is used to produce a desire  product or service. 
For example, labour is a necessary input for producing food. }
}

\newglossaryentry{output}
{
name=output,
description={In production theory, an output anything produced.} % Often symbolized by $Y$ or $Q$  in relations like $Y= F(K,L,N)$.}
}

\newglossaryentry{subsistence wage}
{
name=subsistence wage,
description={In most urban models the subsistence wage is treated as base cost that is the opportunity cost of agricultural land. We have extended the technique to include the opportunity cost of urban labour. It is one of the simplifications which makes our model tractable and focuses it on the question of rents and the specifically urban productivity premium. In our model, the subsistence wage is a wage available inside and outside the urban area, which covers the cost of buildings, food, core living costs, and a base cost of land.}
}

\newglossaryentry{urban wage premium}
{
name=urban wage premium,
description={The wage premium is the premium above the \gls{subsistence wage} payed by employers to attract workers. An urban wage premium appears when workers in larger cities earn higher average wages than workers in smaller cities. In both the U.S. and Sweden a wage premium has been shown to follow a power-law relationship that scales superlinearly with city size. In other words, workers in larger cities not only earn higher average wages, they do so systematically as a power law function of the size of the city. Bettencourt  \cite{bettencourtIntroductionUrbanScience2021}, has demonstrated theoretically that a wage premium should manifest as a power law function and predicted the value of its exponent.}
}

\newglossaryentry{urban wage}
{
name=urban wage,
description={The \gls{urban wage} is the \gls{urban wage premium} plus the \gls{subsistence wage}.}
}

\newglossaryentry{product}
{
name=product,
description={A product is anything produced. It is an \gls{output} of a production process. % Our model has no specific products. 
Rather than specific products, the city in our model produces an aggregate output, which is not a variable in our analysis. Instead of producing explicit list of discrete products, output is defined by an implicit production function relating labour as an input to aggregate productivity and thus to wages. % as part of urban incomes.
}
}

\newglossaryentry{imperfect competition}
{
name=imperfect competition,
description={A market in which any of the conditions required for \gls{perfect competition} are not met.}
}

\newglossaryentry{demand function}
{
name=demand function,
description={An equation describing how much a potential buyer or group of buyers will purchase at any given price. A demand function can express price as a function of quantity or quantity as a function of price. In either case it will typically include other variables that are said to `shift' demand.}
}

\newglossaryentry{increasing returns to scale}
{
name=increasing returns to scale,
description={a property of a production process  such that when all the inputs are increased in the same proportion, the quantity of \gls{output} increases by a greater proportion.}
}

\newglossaryentry{decreasing returns to scale}
{
name=decreasing returns to scale,
description={A property of a production process such that that when all the inputs are increased in the same proportion, the quantity of \gls{output} increases by a lesser proportion.}
}

\newglossaryentry{constant returns to scale}
{
name=constant returns to scale \gls{CRS},
description={A property of a production process such that that when all
the \glspl{input} are increased in the same proportion, the quantity of \gls{output} increases by the same proportion.}
}

\newglossaryentry{equilibrium condition}
{
name=equilibrium condition,
description={A condition that must be satisfied if a resultant variables of the system are to remain constant. For instance, for an equilibrium of prices and quantities in normal free \gls{market}, supply must equal demand.} % TODO is this a check
}

\newglossaryentry{population equilibrium}
{
name=population equilibrium,
description={an \gls{equilibrium condition} that ensures population will not rise or fall  for the region under consideration. In urban model it is the condition that people cannot make themselves better off by moving to another location in the system. Formally it can be expressed by the requirement that the utility of people with the same  assets and tastes is equal at every location. }
}

\newglossaryentry{urban labour supply}
{
name=urban labour supply,
description={in our model this is simply the urban population, but more generally it is all those in the general population willing to work at the location or occupation.}
}

\newglossaryentry{stochastic}
{
name=stochastic,
description={Having a random probability distribution or pattern that may be analyzed statistically but may not be predicted precisely. Introducing even a small amount of random nose into even one variable in a model of a deterministic system converts the model into a stochastic model.}
}

\newglossaryentry{aggregate}
{
name=aggregate,
description={A total or derived variable, formed by the combination of many separate data points.}
}

\newglossaryentry{locational equilibrium}
{
name=locational equilibrium,
description={A situation in which no resident will make herself better off by moving to another location. A Nash equilibrium with housing efficiently allocated  given market prices. See \gls{migration equilibrium}.}
}

\newglossaryentry{agglomeration}
{
name=agglomeration,
description={A collection of similar items in one location. A city is an agglomeration of people, firms, and other entities. Agglomerations may have properties that individuals do not have, giving rise to \glspl{agglomeration effect} or \gls{agglomeration economies}, for instance in economics.}
}

\newglossaryentry{agglomeration effect}
{
name=agglomeration effect,
description={The result of \gls{agglomeration}. In economics, \glspl{agglomeration effect} refer to the external economies associated with size and concentration. The benefits of size and concentration vary for different cross-sections of the urban population. Three such groupings of benefits may be identified as: 1. consumer agglomeration economies; business agglomeration economies; and social agglomeration economies \cite{carlinoAgglomerationEconomiesSurvey1978}. In general, the effects of increasing the number of firms or workers in one place include larger, deeper, more specialized labour pool, which enables workers to better match their skills to the needs of firms or creates knowledge spillovers in which firms and workers learn from each other.}
}

% \newglossaryentry{agglomeration effects}
% {
% name=agglomeration effects,
% description={}
% }

\newglossaryentry{agglomeration economies}
{
name=agglomeration economies,
description={Economic advantages %efficiencies, and economies of scale 
resulting from \glspl{agglomeration effect}.}
}

\newglossaryentry{monopolistic competition}
{
name=monopolistic competition,
description={A type of  \gls{imperfect competition}. \gls{Perfect competition} is a description of a market with many seller, all of whom are price-takers. Monopoly is a market with a single seller, who therefore has the power to set the selling price. Monopolistic competition describes cases in between, with sellers that have some power to set prices within a segment of the market. It occurs when many companies offer competing products or services that are similar, but are not perfect, substitutes.}
}

\newglossaryentry{labour adjustment cost}
{
name=labour adjustment cost,
description={Costs associated with hiring, firing or training that prevent or slow the rate at which a firm will increase or decrease the number of workers it employs.}
}

\newglossaryentry{frictional unemployment}
{
name=frictional unemployment,
description={the part of total unemployment  due to people being in the process of voluntarily moving from one job to another.}
}

\newglossaryentry{marginal product}
{
name=marginal product,
description={the amount that the last unit of any factor  adds to output while holding all other factors constant. See \gls{marginal product of labour}.}
}

% \newglossaryentry{marginal product of labour}
% {
% name=marginal product of labour,
% description={Firms calculate what the next worker is worth to them. That's what they're willing to pay for labour. 
% This is the labour \gls{demand function} based on the \gls{marginal product} which is declining. When a firm has only a few workers, it is high on that demand function, and has to move down. It cuts workers. If it's too low, it expands and hires. %This says something about the geometry of what employers could pay. 
% % Firms can't pay workers more than they can earn in the long term, unless that money comes from somewhere, but they could push down wages and extract more profit, invest more in other factors of production, etc.
% }
% }

\newglossaryentry{marginal product of labour}
{
name=marginal product of labour (MPL),
description={the amount that the last worker  adds to output without changing the quantities of other inputs used. The firm's \gls{demand function} for labour in a competitive market is identically  the \gls{marginal product} of labour function, represented as a downward sloping curve due to the diminishing marginal productivity of labour. See \gls{marginal}.}
}

\newglossaryentry{monopoly}
{
name=monopoly,
description={ Market power means you can price above marginal costs. Need free entry to get rid of it---it doesn't drive out profit---profits can be sustained over longer. Monopolist can charge a higher price but pays a competitive price for all \glspl{input} including labour. If a firm also had a monopoly on offering jobs, they could drive down wages.}
}

\newglossaryentry{duopoly}
{
name=duopoly,
description={A market with two sellers. It can generate a range of outcomes. Under one set of assumptions the result will be the monopoly price, under others the situation will generate lower than monopoly prices, competitive pricing, or market instability.}
}

\newglossaryentry{monopsony}
{
name=monopsony,
description={A market with one buyer who therefore has some power to determine the purchase price by varying the quantity purchased. An example of \gls{imperfect competition}.}
}

\newglossaryentry{imperfect information}
{
name=imperfect information,
description={ the buyers and/or sellers do not have all the information necessary to make an informed decision.}
}

\newglossaryentry{externalities}
{
name=externalities,
description={Any indirect costs or benefits, to uninvolved third parties, that result from a decision-makers activity but are not included in the decision-maker's decision process cost-benefit calculations. For example lawn mowers may wake the neighbour, vehicles emissions may cause emphysema, burning fossil fuels may contribute to climate change, or painting one house may raise the value of a neighbour's house. In the computational model, when employers increase their workforce there is a positive effect on the productivity of all other workers in the city. This is an external effect.}
}

\newglossaryentry{competitive market}
{
name=competitive market,
description={In a competitive market, everybody is a price-taker. Price-takers  assume that that their output decisions do not affect other producers or suppliers, so they act in terms of their internal prices and costs as if the price will remain constant. Their decision-making process doesn't take into account anyone else's behaviour.
%? The easy way to see that is assume prices are fixed - all that's required to get the behaviour \dots  have 
Whether a market will be  competitive may depend on  whether conditions like free exit and entry, perfect information etc. are satisfied. Combined these features result in what is called an \gls{efficient market}.} % **FIX  (or to ensure price taking)} 
}

\newglossaryentry{spillover effects}
{
name=spillover effects,
description={ \Gls{externalities} are the most commonly discussed form of spillover effects but any economic event in one context that occurs because of something else in a seemingly unrelated context can be considered a spillover. It is a looser term than externality because an externality is a consequence, at least in economic theory, of rational optimizing behaviour.}
}

\newglossaryentry{substitutable}
{
name=substitutable,
description={One good may be substituted for another without loss of benefit. Two brands of motor oil are good substitutes for each other. Oranges are somewhat substitutable for apples, but not for screwdrivers.}
}

\newglossaryentry{neoclassical distribution theory}
{
name=neoclassical distribution theory,
description={A theory that states that in perfect competition the owner of every unit of every  \gls{factor of production} will be paid precisely the  value of the \gls{marginal product} of that factor for each unit they contribute to production.}
}

\newglossaryentry{Solow-Swan model}
{
name=Solow-Swan model,
description={a  model of macro-economy developed and analysed by Robert Solow and Trevor Swan independently to explain long-run economic growth \cite{dimandTrevorSwanNeoclassical2009}. It attempts to explain  growth in terms of the growth of three contributing factors, capital, labor (population), and  productivity. The productivity term is usually referred to as \gls{total factor productivity}, (TFP)}
}

\newglossaryentry{migration equilibrium}
{
name=migration equilibrium,
description={The theoretical situation in which no resident can  make themselves better off by moving to another location. It is a logical consequence of utility maximization and free mobility that results in a Pareto optimal allocation of housing. Technically it is a Nash equilibrium, While extremely useful in analysing urban systems, the concept does not closely describe real cities.
% A situation in which not resident will make herself better off by migrating to or between cities or countries. Similar to a migration equilibrium.
}
}

\newglossaryentry{commuter shed}
{
name=commuter shed,
description={For a city, the area over which people will travel to work in a city. In the \gls{Alonso-Jacobs model}, it is sharply defined by the maximum distance commuters can travel before transportation costs exceed the wage premium. In  practice, the duration of commutes is highly variable. It is greater in the case of men, singles, educated and foreign workers, persons living in rented housing, using public transport, living or working in large cities, or working in large firms,  and when the  unemployment rate is high\cite{axisaFactorsInfluencingCommute2012} .}
}

\newglossaryentry{circular city}
{
name=circular city,
description={In urban theory, an idealized city form predicted by models with uniform travel costs. It is the urban form that emerges in the basic \gls{Alonso model}. %in all directions 
%and a fixed household commuting budget. If a city is laid out on a rectangular grid, the same travel-cost logic yields a rectangular city. Recently the term is applied to cities committed to achieving a circular economy. 
}
}

\newglossaryentry{radial city}
{
name=radial city,
description={A radial concentric city plan is formed by streets that extend outward from a defined center and reach the outer edge of the city, together with concentrically arranged roads that connect the radial streets to the lots. it is an idealized pattern that traces back to ancient times and appears  today in planned cities and districts. See \gls{circular city}.}
}

% \newglossaryentry{surplus}
% {
% name=surplus,
% description={Or economic surplus. Any social product in excess of the minimum required to reproduce society. In value terms the surplus appears as profit or rent and accrues to the owner of a  scarce \gls{input} that varys in quality, such as land. In the mid-19th century, French engineer Jules Dupuit first extended the concept of economic surplus to what came to be called producer- and consumer-surplus.}
% }

\newglossaryentry{Alonso-Jacobs model}
{
name=Alonso-Jacobs model,
description={A model combining the \gls{Alonso model} of the urban land use  \cite{alonsoModelUrbanLand1960} with the \gls{agglomeration} theory of Jane Jacobs \cite{jacobsEconomyCities1969} which explains the productivity of cities.}
}

\newglossaryentry{monopsonist}
{
name=monopsonist,
description={A single buyer, usually in an input market. A monopsonist is not a price-taker, knowing that buying more well result in  higher prices. This rleads the monopsonist to purchase less than is socially efficient.}
}

\newglossaryentry{financial return}
{
name=financial return,
description={MAYBE ADD what is best definition? - there may be other returns. Assessed by comparing the net rent $\mathcal{R}_N$ to the costs of acquiring a property, in particular to the cost of borrowing money. CLARIFY}
}

\newglossaryentry{home services}
{
name=home services,
description={A property offers two kinds of services: home services and \gls{locational services}. Home services describes the value offered by living in a house: a place to sleep, to prepare food, the amenity of being in the home, etc. Since people require housing inside and outside the city, home services are modeled as paid for as a share of the subsistence wage ($a \psi$).}
}

\newglossaryentry{locational services}
{
name=locational services,
description={A property offers two kinds of services: \gls{home services} and locational services. Locational services are services accessed by right of location. They include access to the central city job, access to locational amenity, and the benefit of services and connections associated with a location. In the core model, Locational services are, on an annual basis, the rent premium $w$, minus the transportation costs $c$ for a property a given distance, $d$, from the center, $\omega- {dc}$.}
}

\newglossaryentry{rent share}
{
name=rent share,
description={}
}

\newglossaryentry{Pareto efficiency}
{
name=Pareto efficiency,
description={An economic state where resources cannot be reallocated to make one individual better off without making at least one individual worse off.}
}

\newglossaryentry{efficiency conditions}
{
name=efficiency conditions,
description={Conditions derived in neoclassical economic theory that must be satisfied if a system or activity is to achieve Pareto efficiency. Under somewhat reasonable conditions the efficiency conditions are achieved by agents acting in a decentralized manner to maximize their own profit or utility.}
}

\newglossaryentry{neoclassical economics}
{
name=neoclassical economics,
description={An approach to the study of the economy and economic behaviour that attempts to explain the production, pricing, and the consumption of goods and services through supply and demand, and to explain agent behaviour using a theory of rational agents who satisfy \gls{marginal} efficiency conditions. It integrates, within a mathematical framework, the cost-of-production theory from classical economics with a consumer demand theory based on utility maximization.}
}


\newglossaryentry{neoclassical}
{
name=neoclassical,
description={refers to a period in economic  theorizing beginning about 1870 that overlaps with  but largely follows classical economics. The neoclassical reuslts remain central to economic theory even today although interests have expanded far beyond the core neoclassical disccoveries. The term neoclassical was  coined by Thorstein Veblen (1900) The term neoclassical was initially coined by Thorstein Veblen (1900) as a negative description of Alfred Marshall's economics \cite{colanderDeathNeoclassicalEconomics2000}. See \gls{classical economics},  \gls{neoclassical economics}. }
}


\newglossaryentry{socioeconomic status}
{
name=socioeconomic status,
description={Socioeconomic status is typically broken into three levels, high, middle, and low,  commonly referred to as `upper class,' `middle class,' and `working or lower class,' it differs from `\gls{class}' in the more traditional sense, which is a functional classification. See \gls{Ricardian class}, being based on occupation, income, family wealth.}
}

\newglossaryentry{Ricardian class}
{
name=Ricardian class,
description={The conception of class in \gls{classical economics} including Marx, where class is based on the types and amounts of productive capital the individual owns. See \gls{class}.}
}

\newglossaryentry{rent profile}
{
name=rent profile,
description={see \gls{bid-rent function} or \gls{bid-rent curve}}
}

\newglossaryentry{class}
{
name=class,
description={%This term has a wide range of sometimes conflicting meanings.
In our usage, which is consistent with classical economics, % including Marx, 
class is based on the types and amounts of productive capital the individual owns. This is a functional definition distinct from socioeconomic status. %which is common in the current discussion. 
Our treatment, of the evolution of class structure with financialization, draws on \cite{roemerGeneralTheoryExploitation1982} work on class and exploitation. We allow people in different functional classes to own financial capital, producing intermediate classes like those appearing in Roemer's work. %. \`a la Roemer\cite{roemerGeneralTheoryExploitation1982}.
}
}

\newglossaryentry{capitalize}
{
name=capitalize,
description={% To capitalize a stream of expected income is to compute it's 
The capitalized value of a stream of expected income %. Capitalized value 
is the current worth of an asset, often real estate, based on a calculation of the present value of the total income expected over the course of the asset's economic lifespan.}
}

\newglossaryentry{price bubble}
{
name=price bubble,
description={The sustained rise in the price of an asset above its `normal' market value caused by agents (mainly speculators) forecasting further price increases base on previous increases, rather than on estimates on intrinsic value. Price bubbles are sustained by expectations of future increases in the price of an asset. They may end sharply, or crash, when expectations shift.}
}

\newglossaryentry{marginal value-product}
{
name=marginal value-product,
description={Also known as the marginal revenue product. The marginal revenue created due to an addition of one unit of productive resource, such as one more worker. Calculated by multiplying the marginal physical product by the price, or the marginal revenue in the case of a non-competitive market.}
}

\newglossaryentry{neoclassical growth theory}
{
name=neoclassical growth theory,
description={An economic theory that outlines how a steady economic growth rate results from a combination of three driving forces---labour, capital, and technology. Robert Solow and Trevor Swan developed and introduced the model of long-run economic growth in 1956. It is the  foundation of most empirical and theorical attempts to explain macroeconomic growth.}
}


\newglossaryentry{capital}
{
name= capital,
description={The word "capital" has many different meanings in economics and finance.  In economics, real capital is durable produced goods that are in turn used as productive inputs. \Gls{financial capital} is wealth that can be used to lend to others in exchange for interest payments or to purchase assets. Both these forms of capital are represented in the economy by extensive legal organizational structures that represent the interests of their owners. Other forms of capital recognized by economists are human capital, social capital natural capital, and intellectual capital. The distinctive  features of capital include that it takes time and energy to create (natural capital, however is  `a free gift of nature''), lasts a long time, deprecates, produces a stream of benefits over time,  and may be transferable as property}
}

\newglossaryentry{financial capital}
{
name=financial capital,
description={Lendable purchasing power. Owners of financial capital provide their liquidity to borrowers in exchange for a future return. Interest rates are the prices charged for the use of financial capital. It is generally based on, and  secured by, ownership of tradable assets. Anything can be a form of financial capital as long as it has a monetary value and can be used in the pursuit of future revenue. Marx distinguished financial capital, then called circulating capital or money capital, from fixed or real capital.}
}

\newglossaryentry{agent-based}
{
name=agent-based,
description={%a term for a model that is a  collection of autonomous decision-making entities called agents that %. In practice the agents are little sub-programs (automata, robots) that each separately
%use some information about their environment and follow some internal rules to choose a response in each model cycle. 
See \gls{agent-based model}.}
}

\newglossaryentry{agent-based model}
{
name=agent-based model,
description={Agent-based models are computer simulations. They are often stochastic
models, used to study the interactions between people, institutions and things, across places and time. The approach is to model from the `bottom up,' modeling individual agents and exploring the the effect of interactions, including macro level effects, and the effects on larger systems. %meaning by modelling individual agents (for instance people, institutions, etc). 
Agents are essentially sub-programs that respond to other agents and the environment in certain ways. %These interactions produce emergent effects that may differ from the results of traditional, regression-based methods in that, like systems dynamics modeling, 
Agent based models allow for the exploration of complex systems that display non-independence of individuals, heterogeneity, distributions,  and \gls{feedback} loops. % in causal relations. % mechanisms. % Core outcome is the ability to analyze macro level effects from micro level causes.
}
}

\newglossaryentry{urban scaling}
{
name=urban scaling,
description={Urban scaling laws reliably relate socio-economic, behavioural and physical variables to the population size of cities. They allow for approaches  to city planning and for an understanding of urban resilience and economics. In this thesis we use the well-established relationship between population and urban productivity \cite{GET_doi:10.1098/rsif.2020.0705}}.
}

\newglossaryentry{classical rent theory}
{
name=classical rent theory,
description={A theory developed in \gls{classical economic theory}, explaining how land generated surplus value for its owner, and how that surplus explained the wealth and income of the land-owning class. David Ricardo produced the classic description in 1815 based on extensive work  %prior analysis 
by others in the preceding century. The key notion in his analysis is the `marginal' unit of land. It is just barely worth putting this land into production because it just barely produces enough to justify the cost of production and transportation. More productive land or better-located land produces a surplus that the landowner collects in the form of land rent collected from tenant farmers. No tenant would pay to cultivate that unit of land. The theory employed the basic logic of later the later `marginalist' school of economic analysis. See also \gls{class} and \gls{rent}.}
}

\newglossaryentry{rent}
{
name=rent,
description={The economic  surplus generated in production as a result of differences in the quality of some \gls{factor of production}. Often described as the difference between the opportunity cost of a factor of production and the income it earns. In this thesis we focus on rents generated by \glspl{agglomeration effect}. According to \gls{classical rent theory}, rent is the price paid for the use of land. More generally it is the  surplus generated by any natural resource, up to and including the athletic talents of basketball stars \cite{lackmanClassicalBaseModern1976}. Land, talent, and mineral resources are seen as `the free gift of nature,' forms of capital which the owners do not create but do appropriate. Like the productivity of agricultural land in classical theory,  urban \glspl{agglomeration effect} produce land rents that are not created but are appropriated by the landowners. See \gls{class}.}
}

\newglossaryentry{maximum bid function}
{
name=maximum bid function,
description={A function that generates the maximum that an investor would bid for a property.  See \gls{bid-rent curve}, \gls{bid-rent function}.}
}

\newglossaryentry{bid-rent function}
{
name=bid-rent function ,
description={See \gls{bid-rent curve}.}
}

\newglossaryentry{reservation price}
{
name=reservation price,
description={Seller's minimum price of to accept a bid. If no offer is at least as large as the reservation price, the seller is effectively the buyer. It is lowest price that a prospective seller will accept, and is computed as seller's maximum bid price, which incorporates the net rent achievable.}
}

\newglossaryentry{bid-rent curve}
{
name=bid-rent curve,
description={The height of a graph showing distance from employment horizontally and the amount that residents will pay to rent land at that distance. Also called the \gls{rent profile}, and the \gls{bid-rent function}, it is the \gls{rent premium} for a particular distance from the urban center, and is at the heart of the \gls{Alonso model}. More generally, with varying agents and property attributes, the bid-rent curve can be seen a set of functions of location, each of which generates a bid price for one category of agent, representing what that agent is willing to pay for that property at that distance from a set of desired amenities and opportunities.}
}

\newglossaryentry{borrowing ratio}
{
name=borrowing ratio,
description={The maximum fraction of the price of a property that may be mortgaged. In the simulation, it is the value $m$, determined by the bank, the lender, based on the applicant's individual \gls{wealth} and income.} 
}

\newglossaryentry{rent premium}
{
name=rent premium,
description={or \gls{warranted economic rent} is the excess rent  that might be charge for the use of urban land relative the non-urban land. In our model the rent premium for an urban property is equal to the urban wage premium minus the transportation costs. }
}

\newglossaryentry{warranted rent}
{
name= warranted rent,
description={$\mathcal{R}_N$, at an  urban location  $d$ units from the centre, is the the value of the flow of services provided by the property, including the locational value, or \gls{warranted economic rent}. It is level of land rent that would be expected in equilibrium based on location and transportation costs.  (It may not be the rent actually charged to a tenant. The warranted rent is used as an initial value  for the market rent when running the model. }
}

\newglossaryentry{warranted price}
{
name= warranted price,
description={% in a computational model, at an  urban location  $d$ units from the centre,
The price that is economically warranted for a particular property. The \glsdisp{capitalize}{capitalized} value of the flow of services provided by the property, where the value of the flow of services is the \gls{warranted rent}. % It includes the locational value. %, or \gls{warranted economic rent}.
The warranted price may differ from the realized \gls{market price}. Represented by the variable $P_W$. }
}

\newglossaryentry{warranted economic rent}
{
name=warranted economic rent,
description={The locational value of an urban property. A surplus generated by \glspl{agglomeration effect}, equal to the urban (wage premium) minus transportation costs, $\omega-{c} d$). This is the the amount that an equilibrium market rent for a property would be expected to exceed the market rent for a similar non-urban property.}
}

\newglossaryentry{net rent}
{
name=net rent,
description={Or net market rent. The \gls{warranted rent} minus taxes and maintenance costs.}
}

% \newglossaryentry{rent share}
% {
% name=rent share,
% description={\dots}
% }

% rent paid
% economic rent
% locational rent

\newglossaryentry{marginal}
{
name=marginal,
description={relating to or situated at the edge or margin of something. In the marginalist approach to economics, it  refers to technique of focusing on the cost or benefit of the next unit or individual. See \gls{extensive margin}, \gls{intensive margin}, \gls{marginal product}, \gls{marginal value-product}, \gls{inframarginal}.}
}

\newglossaryentry{inframarginal}
{
name=inframarginal,
description={Coming before the margin is reached. For example, if the wage is $x$, all workers willing to work for less than $x$ are inframarginal. They are selling their time for more that it is worth to them. They come out ahead on the bargain. Workers who will work for $x$ but not a penny less are marginal. Similarly, with land, the most remote or the least productive land in use is `marginal' while  inframarginal land is more productive and generates a \gls{surplus}.}
}

\newglossaryentry{overlapping generations}
{
name=overlapping generations,
description={In the \gls{OLG} model individuals live a finite length of time, long enough to overlap with at least one period of another agent's life. The OLG model is the natural framework for the study of life-cycle behavior (investment in human capital, work and saving for retirement).}
}

\newglossaryentry{stylized facts}
{
name=stylized facts,
description={Economists use this term for observations that are widely understood to be empirical truths, to which theories must fit.  Also described as, `broad tendencies that aim to summarize the data, offering essential truths while ignoring individual details.' The term `stylized facts' was introduced by the economist Nicholas Kaldor in the context of a debate on economic growth theory in 1961 \cite{kaldorCapitalAccumulationEconomic1961}.}
}

\newglossaryentry{perfect foresight}
{
name=perfect foresight,
description={The correct prediction of future events. If agents have  all relevant information and  a correct model to use for prediction. When there is uncertainty it is not possible to have perfect foresight. In solving a complex intertemporal model, economists may assume agents have perfect foresight. This is called the rational expectations approach.}
}

\newglossaryentry{equilibrium reasoning}
{
name=equilibrium reasoning,
description={% In economics, analysis often focuses on variables identifies variable values of particular interest because they are likely to exhibit stability or capture the implications of the goals of agents. Using \gls{equilibrium} values, 
Exploring and how variables are likely to change in the region of an \gls{equilibrium}, and draw conclusions. % is `equilibrium reasoning' because 
%It bases the conclusions on assumptions about the behaviour of the variables near an equilibrium. 
Equilibrium reasoning implicitly assumes that the variables will tend to stay near and smoothly approach the equilibrium.}
}

\newglossaryentry{equilibrium}
{
name=equilibrium,
description={In economics and other sciences, an equilibrium is a situation in which forces such as supply and demand are balanced, and in the absence of external influences the equilibrium value of a variable will not change. In economics an equilibrium is often understood % usually understood behaviourally 
as a situation in which no agent has an incentive to change behaviour given what others are doing. Such a situation is called a Nash Equilibrium or a Cournot-Nash equilibrium.}
}

% \newglossaryentry{expectation}
% {
% name=expectation,
% description={in our model, an agent's estimate of an unobserved or future value of a variable such as price. In simple statistical analysis the expectation of a variable may be taken as the average of the previously observed values.  }
% }

\newglossaryentry{expectation}
{
name=expectation,
description={Predictions of future events or values, formed by agents for use in decision-making. Agents may form their expectations by looking backward at data on previous values, or by projecting forward using a mental model of how the system works. If agents are fully informed about the state of the system and how it works, their expectations are essentially the same as the predictions of the relevant economic theory and they are termed `rational expectations' \cite{muthRationalExpectationsTheory1961}. In probability theory, the expected value of a variable is the mean of its true distribution (the rational expectation), which is usually estimated using the observed realizations (a backward-looking estimate).}
}

\newglossaryentry{Alonso model}
{
name=Alonso model,
description={The model credited to William Alonso, also called the Alonso-Muth model. It is discussed in Chapter~\ref{chapter-space}, and a full development of the theory is presented in Alonso's doctoral dissertation \cite{alonsoTheoryUrbanLand1960}.} %, A MODEL OF THE URBAN LAND MARKET: LOCATIONS AND DENSITIES OF DWELLINGS AND BUSINESSES, University of Pennsylvania, 1960.}
}

\newglossaryentry{asking price}
{
name=asking price,
description={The price a seller initially posts on deciding to sell a property. It will be higher than the seller's maximum bid price.}
}

\newglossaryentry{bid price}
{
name=bid price,
description={The %In the computational model, any 
price that an agent offers to pay to purchase a property. Agents bid during the transaction process.} % In the computational model, the bid price is less than or equal to the agent's maximum bid price and less than or equal to the asking price.}
}

\newglossaryentry{maximum bid price}
{
name=maximum bid price,
description={The maximum price that investors will bid for a property. A bid that makes the expected return exactly the required or  target return.}
}

\newglossaryentry{bargaining}
{
name=bargaining,
description= {in the computational model during price setting for a particular property, there is a bargaining process that takes as arguments the highest bid price, reservation price, and asking price, returning a sale price for the property, as well as property transfer instructions.}
}
%The reservationn prices  is  the seller  own bid. If the max bid of the highest bid received is lower than the own bid the seller is the buyer- remains the owner. 
%Otherwisesimplest rule is  (reservation bid+maxbid)/2


\newglossaryentry{model}
{
name=model,
description={A system, A, which is useful for understanding another system, B. as the model we present is useful for understanding the effect of growing finacialization working through the system of urban land ownership.}
}

\newglossaryentry{specification}
{
name=specification,
description={With respect to a model or a theory, associating the theoretical constructs or relationships in a theory with a specific model, or associating specific model elements with observables.}
}

\newglossaryentry{distribution}
{
name=distribution,
description={The way some amount such as total \gls{output}, income, \gls{wealth} or assets are distributed among individuals, \glsdisp{class}{classes}, \glsdisp{factor of production}{factors of production}, etc.. The term may refer to a theoretical approach or to an empirical distribution of any of these. Theories of distribution are systematic attempts to account for the sharing of the national income.  Distributions across classes is known as a functional distribution distribution and  corresponds to the the approach of the \gls{classical economics}. Neoclassical economics examined distribution through the payment to factors of their \gls{marginal value-product}.}
}

\newglossaryentry{production function}
{
name=production function,
description={A representation of the technology of production, often a functional relationship between the \glspl{input} that enable production and the quantity of \gls{output}.}
}

\newglossaryentry{Cobb-Douglas}
{
name=Cobb-Douglas,
description={A specific production function, commonly used for illustration or estimation in economics. Essentially a form of the geometric mean.}
}

\newglossaryentry{productivity}
{
name=productivity,
description={The ratio of \gls{output} to \glspl{input}. Which outputs and inputs are considered varies. \gls{total factor productivity} refers to aggregate outputs and inputs in value terms. Marginal productivity refers to the addition to total output produced by one additional unit of input.} 
}

\newglossaryentry{growth}
{
name=growth,
description={The rate of increase in aggregate \gls{output} for a given production unit, such as a nation  or a city.}
}

\newglossaryentry{regime}
{
name=regime,
description={A distinct state of a system, a region of the system's phase space. In dynamical system theory, a phase space is a space in which all possible states of a system are represented, with qualitatively distinct  states corresponding to one region in the phase space.}
}

\newglossaryentry{resilience}
{
name=resilience,
description={The ability of a system to return to its original state when shocked by a change in its determining variables. May refer to smoothly or successfully adapting to a change in  determining variables.}
}

\newglossaryentry{hysteresis}
{
name=hysteresis,
description={An event in the economy that persists even after the factors that led to that event have been removed or otherwise run their course.}
}

\newglossaryentry{present value}
{
name=present value,
description={The value in cash today of a future sum of money or stream of cash flows, given a specified rate of return.}
}

\newglossaryentry{capital gain}
{
name=capital gain,
description={The difference between the future sale price and the current purchase price.}
}

\newglossaryentry{mortgage term}
{
name=mortgage term,
description={The length time after a house purchase until a sum for a house purchase, the mortgage, must be returned to the lender with interest.}
}

\newglossaryentry{use value}
{
name=use value,
description={The monetary value of being allowed to live at a certain location ignoring potential speculative gains or losses.}
}

\newglossaryentry{wealth trajectories}
{
name=wealth trajectories,
description={A term for the people's  asset portfolios change over time. As their \gls{wealth} changes, an individual's \gls{class} position may change. For example, if a household is able to make  a down payment on a house,  the person's wealth will generally increase more rapidly than t the wealth of a person who does not purchase a house. The wealth trajectories differ. Savings out of labour income may enable a member of the \gls{working class} to move into the \gls{middle class} and enjoy income from his or her financial assets. The collection  of wealth trajectories at any time characterizes a society. Changes in the ensemble of wealth trajectories reflect the evolution and class structure of society.   } 
}

% % Nomenclature glossary entries---New definitions, or unusual terminology
% \newglossary*{nomenclature}{Nomenclature}
% \newglossaryentry{dingledorf}
% {
% type=nomenclature,
% name=dingledorf,
% description={A person of supposed average intelligence who makes incredibly brainless misjudgments}
% }

% List of Abbreviations (abbreviations type is built in to the glossaries-extra package)

% \newabbreviation{}{}{}
% \newabbreviation{}{}{}
% \newabbreviation{}{}{}
% \newabbreviation{}{}{}
% \newabbreviation{}{}{}
% \newabbreviation{}{}{}

% \newabbreviation{REIT}{REIT}{real estate investment trust}

\newabbreviation{ABM}{ABM}{agent-based model}

\newabbreviation{CRS}{CRS}{constant returns to scale}

\newabbreviation{OLG}{OLG}{overlapping generations}

% List of Symbols
\newglossary*{symbols}{List of Symbols}
\newglossaryentry{rvec}
{
name={$\mathbf{v}$},
sort={label},
type=symbols,
description={Random vector: a location in n-dimensional Cartesian space, where each dimensional component is determined by a random process}
}