% Main glossary entries -- definitions of relevant terminology


\newglossaryentry{}
{
name= ,
description={}
}

% \newglossaryentry{}
% {
% name= ,
% description={}
% }

% \newglossaryentry{}
% {
% name= ,
% description={}
% }

% \newglossaryentry{}
% {
% name= ,
% description={}
% }

\newglossaryentry{agglomeration effects}
{
name=agglomeration effects,
description={..}
}

\newglossaryentry{capitalize}
{
name= capitalize,
description={To capitalize a stream of expected income is to compute it's capitalized value. Capitalized value is the current worth of an asset, usually real estate, based on a calculation of present value of expected income over the course of its economic lifespan. }
}

\newglossaryentry{agglomeration effect}
{
name= agglomeration effect,
description={refer to the external economies associated with size and concentration. The benefits of size and concentration vary for different cross-sections of the urban popula- tion. Three such groupings may be identified: 1. Consumer agglomeration economies; Business agglomeration economies; Social agglomeration economies.\cite{carlinoAgglomerationEconomiesSurvey1978}}
}

\newglossaryentry{price bubble}
{
name= price bubble,
description={The sustained rise in the price of an asset above its ``normal'' market value'' caused by agents (mainly speculators) forecasting further price increases base on previous increases, rather than on estimates on intrinsic value.  Price bubbles are sustained by expectations of future increases in the price of an asset. They may end sharplyt, or crash, when expectations shift.}
}

\newglossaryentry{marginal value-product}
{
name= marginal value-product,
description={also known as the marginal revenue product, is the marginal revenue created due to an addition of one unit of productive resource, such as one more worker. Calculated by multiplying the marginal physical product by the price, or the marginal revenue in the case of a non-competitive market. }
}


\newglossaryentry{neoclassical growth theory}
{
name= neoclassical growth theory,
description={an economic theory that outlines how a steady economic growth rate results from a combination of three driving forces—labor, capital, and technology. Robert Solow and Trevor Swan developed and introduced the model of long-run economic growth in 1956. It is the  foundation of most empirical and theorical attempts to explain macroeconomic growth.}
}

\newglossaryentry{financial capital}
{
name= financial capital,
description={The word "capital" has many different meanings in economics and finance. Financial capital is simply lendable purchasing power. Owners of financial capital provide their liquidity to borrowers in exchange for a future return. Interest rates are the prices charged for the use of financial capital. It is generally based on (secured by) ownership of tradable assets.  Anything can be a form of financial capital as long as it has a monetary value and can be  used in the pursuit of future revenue. Marx distinguished  financial capital (then called circulating capital or money capital) from fixed or real capital.   }
}


\newglossaryentry{Agent Based Model}
{
name= Agent Based Model,
description={Agent-based Models are computer simulations used to study the interactions between people, things, places, and time. They are usually stochastic models built from the `bottom up,' meaning by modelling individual agents (people, institutions, etc). Agents essentially sub-programs that respond to other agents and the environment in certain ways. These interactions produce emergent effects that may differ from the results of traditional, regression-based methods in that, like systems dynamics modeling, it allows for the exploration of complex systems that display non-independence of individuals and feedback loops in causal mechanisms..}
}


\newglossaryentry{ABM}
{
name= ABM,
description={See \gls{Agent Based Model}.}
}


\newglossaryentry{urban scaling}
{
name= urban scaling,
description={See \gls{}.}
}


\newglossaryentry{classical rent theory}
{
name= classical rent theory,
description={According to classical theory, rent is the price paid for the use of land. More generally it is the price surplus generated by any natural resource, up to and including the athletic talents of basketball stars.\cite{lackmanClassicalBaseModern1976} Land, talent and mineral resources are seen as ``the free gift of nature'', forms of capital which the owners do not create but do appropriate. Like the productivity of agricultural land in classical theory,  urban \gls{agglomeration effects} produce land rents that are not created but are appropriated by the landowners. See \gls{class}.}
}


\newglossaryentry{maximum bid function}
{
name= maximum bid function,
description={A function that generates the maximum that an investor would bid for a property.  See \gls{bid-rent curve}, \gls{bid-rent function}.}
}

\newglossaryentry{bid-rent function}
{
name=bid-rent function ,
description={See \gls{bid-rent curve}.}
}

\newglossaryentry{reservation price}
{
name=reservation price,
description={Seller's minimum price of to accept a bid. If no offer is at least as large as the reservation price, the seller is effectively the buyer. The lowest price that a prospective seller will accept, which is computed as  seller's maximum bid price, which would incorporate the net rent achievable.}
}


\newglossaryentry{bid-rent curve}
{
name=bid-rent curve,
description={also called a rent profile. The height of a graph showing distance from employment horizontally and the amount that residents will pay to rent land at that distance. With varying agents and property attributes can be seen a set of functions of location, each of which  generates a bid price for one category of agent.   See \gls{rent premium}, .}
}

\newglossaryentry{borrowing ratio}
{
name=borrowing ratio,
description={$m$. The maximum fraction of the price of a property that may be mortgaged. Determined by the bank (the lender) based on individual wealth and income. }
}

\newglossaryentry{rent}
{
name=rent,
description={The economic  surplus generated in production as a result of differences in the quality of some factor of production. Often described as the difference between the opportunity cost of a factor of production and the income it earns. }
}

\newglossaryentry{rent premium}
{
name=rent premium,
description={the excess rent  that might be charge for the use of urban land relative the non-urban land. In our model the rent premium is equal to the wage premium. }
}

\newglossaryentry{warranted market rent}
{
name= warranted market rent,
description={$\mathcal{R}_N$ at an  urban location  $d$ units from the centre, is the the level of land rent that would be expected in equilibrium based on location and transportation costs. It includes the locational value, or \gls{warranted economic rent}, (It may not be the rent actually charged to a tenant. }
}

\newglossaryentry{warranted economic rent}
{
name=warranted economic rent,
description={The urban rent premium (wage premium) minus transportation costs, $\omega-{c} d$). This is the the amount that and equilibrium market rent for a property would be expected to exceed the market rent for a similar non-urban property.}
}

\newglossaryentry{net rent}
{
name=net rent,
description={As used in this thesis, net rent is the warranted rents minus taxes and maintenance costs }
}

\newglossaryentry{rent share}
{
name=rent share,
description={..}
}

% rent paid
% economic rent
% locational rent

\newglossaryentry{class}
{
name=class,
description={This term has a wide range of sometimes conflicting meanings In our usage, which is consistent with classical economics including Marx class is based on the types and amounts of productive capital the individual owns. This is a functional definition quite distinct from socioeconomic status which is more common in current discussion. Our treatment of the evolution of class structure with financialization draws on \cite{roemerGeneralTheoryExploitation1982}.}
}

\newglossaryentry{marginal}
{
name=marginal,
description={relating to or situated at the edge or margin of something. In the marginalist approach to economics, it  refers to technique of focusing on the cost or benefit of the next unit or individual.}
}

\newglossaryentry{inframarginal}
{
name=inframarginal,
description={coming before the margin is reached. For example, if the wage is $x$, all workers willing to work for less than $x$ are inframarginal. They are selling their time for more that it is worth to them. The come out ahead on the bargain. Workers who will work for $x$ but not a penny less are marginal. }
}

\newglossaryentry{subsistence wage}
{
name=subsistence wage,
description={In our model, the wage which covers the cost of buildings, food and other living costs and a base cost of land. In most urban models this base cost is  the opportunity cost of agricultural land. We have extended the technique to include the opportunity cost of urban labour. }
}

\newglossaryentry{overlapping generations model}
{
name=overlapping generations model,
description={in the OLG model individuals live a finite length of time, long enough to overlap with at least one period of another agent's life. The OLG model is the natural framework for the study of life-cycle behavior (investment in human capital, work and saving for retirement).}
}

\newglossaryentry{stylized facts}
{
name=stylized facts,
description={Economists use this term for observations that are widely understood to be empirical truths, to which theories must fit.  Also described as, ``broad tendencies that aim to summarize the data, offering essential truths while ignoring individual details''. The term "stylized facts" was introduced by the economist Nicholas Kaldor in the context of a debate on economic growth theory in 1961.\cite{kaldorCapitalAccumulationEconomic1961}}
}

\newglossaryentry{perfect foresight}
{
name=perfect foresight,
description={The correct prediction of future events. If agents have  all relevant information and  a correct model to use for prediction. When there is uncertainty it is not possible to have perfect foresight. In solving a complex intertemporal model, economists may assume agents have perfect foresight. This is called the rational expectations approach. }
}

\newglossaryentry{equilibrium reasoning}
{
name=equilibrium reasoning,
description={Gls{equilibrium} analysis identifies variable values of particular interest because they are likely to exhibit stability or capture the implications of the goals of agents. Using these equilibrium values and how they are likely to change in the regions of an equilibrium draw conclusions to is `equilibrium reasoning' because it bases the conclusions on assumptions about the behaviour of the variables near an equilibrium. Equilibrium reasoning implicitly  assumes that the variables will tend to stay near and smoothly approach the equilibrium.}
}

\newglossaryentry{equilibrium}
{
name=equilibrium,
description={In economics and other sciences, an equilibrium is a situation in which forces such as supply and demand are balanced, and in the absence of external influences the (equilibrium) values of variables will not change. In economics an equilibrium is usually understood behaviourally as a situation in which no agent has an incentive to change behaviour given what others are doing. Such a situation is called a Nash Equilibrium or a Cournot-Nash equilibrium.}
}

\newglossaryentry{expectations}
{
name=expectations,
description={ expectations are predictions of future events or values formed by agents for use in decision-making. Agents may form their expectations by looking backward at data on previous values, or by projecting forward using a mental model of how the system works. If agents are fully informed about the state of the system and how it works, their expectations are essentially the same as the predictions of the relevant economic theory and they are termed `rational expectations'.\cite{muthRationalExpectationsTheory1961}. In probability theory, the expected value of a variable is the mean of its true distribution (the rational expectation), which is usually estimated using the observed realizations (a backward-looking estimate).}
}

\newglossaryentry{urban wage premium}
{
name=urban wage premium,
description={An urban wage premium exists when workers in larger cities earn higher average wages than workers in smaller cities. In both the U.S.  and Sweden a wage premium has been shown to follow a power-law relationship that scales superlinearly with city size. In other words, workers in larger cities not only earn higher average wages, they do so systematically as a power law function of city size. Bettencourt [7] demonstrated theoretically not only that a wage premium should manifest as a power law function, but predicted the value of its exponent. }
}

\newglossaryentry{Alonzo model}
{
name=Alonzo model,
description={also called the Alonso-Muth model. The model credited to William Alonzo, A full development of the theory is presented in his doctoral dissertation, A MODEL OF THE URBAN LAND MARKET: LOCATIONS AND DENSITIES OF DWELLINGS AND BUSINESSES, University of Pennsylvania, 1960. }
}

\newglossaryentry{asking price}
{
name=asking price,
description={the price a seller initially posts on deciding to sell a property. It will be higher than the seller's maximum bid price.}
}

\newglossaryentry{bid price}
{
name=,
description={in  the computational model, any price that an agent bids for a property in the transaction process. It will be less than or equal to the agent's maximum bid price and less than or equal to the asking price.}
}

\newglossaryentry{maximum bid price}
{
name= maximum bid price,
description={the maximum price that investors will bid for a property. A bid that makes the expected return exactly the required or  target return.}
}

\newglossaryentry{bargaining}
{
name=bargaining,
description= {in the computational model during price setting for a particular property, there is a bargaining process that takes as arguments the highest bid price, reservation price, and asking price, returning a sale price for the property, as well as property transfer instructions.}
}
%The reservationn prices  is  the seller  own bid. If the max bid of the highest bid received is lower than the own bid the seller is the buyer- remains the owner. 
%Otherwisesimplest rule is  (reservation bid+maxbid)/2


\newglossaryentry{model}
{
name=model,
description={A system, A, which is useful for understanding another system, B. as the model we present is useful for understanding the effect of growing finacialization working through the system of urban land ownership. }
}

\newglossaryentry{specification}
{
name=specification,
description={of a model or a theory: associating theoretical constructs or relationships in a theory with a specific model, or associating specific model elements with observables. }
}

\newglossaryentry{computational model}
{
name=computational model,
description={A program that receives input, performs operations, and provides output}
}

\newglossaryentry{distribution}
{
name=distribution,
description={The way total output, income, wealth or assets is distributed among individuals or among the factors of production (such as labour, land, and capital), or the empirical distribution of any of these across the population. The first is know as a functional distribution distribution and  corresponds to a distribution across the classes of owners  of capital, including land, financial capital, and human capital. }
}

\newglossaryentry{production-function}
{
name=production function,
description={A representation of the technology of production, often a functional relationship between the inputs that enable production and the quantity of output}
}

\newglossaryentry{cobb-douglas}
{
name=cobb-douglas,
description={A specific production function. commonly used for illustrative or estimation in economics. Essentially a form of geometric mean}
}

\newglossaryentry{productivity}
{
name=productivity,
description={The ratio of output to inputs. Which outputs and inputs are considered varies. Total Factor Productivity refers to aggregate outputs and inputs in value terms. Marginal productivity refers to the addition to total output produced by one additional unit of input.} 
}

\newglossaryentry{growth}
{
name=growth,
description={The rate of increase in aggregate output for a given production unit, such as a nation  or a city.}
}

\newglossaryentry{regime}
{
name=regime,
description={A distinct state of a system, a region of the system's phase space. (In dynamical system theory, a phase space is a space in which all possible states of a system are represented, with qualitatively distinct  states corresponding to one region in the phase space.)}
}

\newglossaryentry{resilience}
{
name=resilience,
description={The ability of a system to return to its original state when shocked by a change in its determining variables. May refer to smoothly or successfully adapting to a change in  determining variables. }
}

\newglossaryentry{hysteresis}
{
name=hysteresis,
description={An event in the economy that persists even after the factors that led to that event have been removed or otherwise run their course.}
}

\newglossaryentry{present value}
{
name=present value,
description={The value in cash today of a future sum of money or stream of cash flows, given a specified rate of return.}
}

\newglossaryentry{capital gain}
{
name=capital gain,
description={difference between the future sale price and the current purchase price}
}

\newglossaryentry{discount factor}
{
name=discount factor,
description={The present value of a dollar at a specified time in the future. It is a compounded value calculated using the individual discount rate.}
}

\newglossaryentry{mortgage term}
{
name=mortgage term,
description={The length time after a house purchase until a sum for a house purchase, the mortgage, must be returned to the lender with interest.}
}

\newglossaryentry{use value}
{
name=use value,
description={The monetary value of being allowed to live at a certain location ignoring potential speculative gains or losses. }
}

% % Nomenclature glossary entries -- New definitions, or unusual terminology
% \newglossary*{nomenclature}{Nomenclature}
% \newglossaryentry{dingledorf}
% {
% type=nomenclature,
% name=dingledorf,
% description={A person of supposed average intelligence who makes incredibly brainless misjudgments}
% }

% List of Abbreviations (abbreviations type is built in to the glossaries-extra package)
\newabbreviation{abm}{ABM}{Agent Based Model}

% List of Symbols
\newglossary*{symbols}{List of Symbols}
\newglossaryentry{rvec}
{
name={$\mathbf{v}$},
sort={label},
type=symbols,
description={Random vector: a location in n-dimensional Cartesian space, where each dimensional component is determined by a random process}
}