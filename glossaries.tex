% Main glossary entries -- definitions of relevant terminology
% \newglossaryentry{}
% {
% name=,
% description={ }
% }

% \newglossaryentry{}
% {
% name=,
% description={ }
% }

% \newglossaryentry{Alonzo model}
% {
% name=Alonzo model,
% description={also called the Alonso-Muth model. The model credited to William Alonzo, A full development of the theory is presented in his doctoral dissertation, A MODEL OF THE URBAN LAND MARKET: LOCATIONS AND DENSITIES OF DWELLINGS AND BUSINESSES, University of Pennsylvania, 1960. }
% }

\newglossaryentry{asking price}
{
name=asking price,
description={the price a seller initially posts on deciding to sell a property. It will be higher than the seller's maximum bid price.}
}

\newglossaryentry{reservation price}
{
name=reservation  price,
description={The lowest price that a prospective seller will accept, which is computed as  seller's maximum bid price, which would incorporate the net rent achievable.}
}

\newglossaryentry{bid price}
{
name=hysteresis,
description={in  the computational model, any price that an agent bids for a property in the transaction process. It will be less than or equal to the agent's maximum bid price and less than or equal to the asking price.}
}

\newglossaryentry{maximum bid price}
{
name=hysteresis,
description={the maximum price that investors will bid for a property. A bid that makes the expected return exactly the required or  target return.}
}

\newglossaryentry{bargaining}
{
name=bargaining,
description= {in the computational model during price setting for a particular property, there is a bargaining process that takes as arguments the highest bid price, reservation price, and asking price, returning a sale price for the property, as well as property transfer instructions.}
}
%The reservationn prices  is  the seller  own bid. If the max bid of the highest bid received is lower than the own bid the seller is the buyer- remains the owner. 
%Otherwisesimplest rule is  (reservation bid+maxbid)/2


\newglossaryentry{model}
{
name=model,
description={A system, A, which is useful for understanding another system, B. as the model we present is useful for understanding the effect of growing finacialization working through the system of urban land ownership. }
}

\newglossaryentry{specification}
{
name=specification,
description={of a model or a theory: associating theoretical constructs or relationships in a theory with a specific model, or associating specific model elements with observables. }
}

\newglossaryentry{computational model}
{
name=computational model,
description={A program that receives input, performs operations, and provides output}
}

\newglossaryentry{distribution}
{
name=distribution,
description={ the way total output, income, wealth or assets is distributed among individuals or among the factors of production (such as labour, land, and capital), or the empirical distribution of any of these across the population. The first is know as a functional distribution distribution and  corresponds to a distribution across the classes of owners  of capital, including land, financial capital, and human capital. }
}

\newglossaryentry{production-function}
{
name=production function,
description={A representation of the technology of production, often a functional relationship between the inputs that enable production and the quantity of output}
}

\newglossaryentry{cobb-douglas}
{
name=cobb-douglas,
description={A specific production function. commonly used for illustrative or estimation in economics. Essentially a form of geometric mean}
}

\newglossaryentry{rent}
{
name=rent,
description={the economic  surplus generated in production as a result of differences in the quality of some factor of production. Often described as the difference between the opportunity cost of a factor of production and the income it earns. }
}

\newglossaryentry{productivity}
{
name=productivity,
description={the ratio of output to inputs. Which outputs and inputs are considered varies. Total Factor Productivity refers to aggregate outputs and inputs in value terms. Marginal productivity refers to the addition to total output produced by one additional unit of input.} 
}

\newglossaryentry{growth}
{
name=growth,
description={the rate of increase in aggregate output for a given production unit, such as a nation  or a city.}
}

\newglossaryentry{regime}
{
name=regime,
description={a distinct state of a system, a region of the system's phase space. (In dynamical system theory, a phase space is a space in which all possible states of a system are represented, with qualitatively distinct  states corresponding to one region in the phase space.)}
}

\newglossaryentry{resilience}
{
name=resilience,
description={the ability of a system to return to its original state when shocked by a change in its determining variables. May refer to smoothly or successfully adapting to a change in  determining variables. }
}

\newglossaryentry{hysteresis}
{
name=hysteresis,
description={an event in the economy that persists even after the factors that led to that event have been removed or otherwise run their course.}
}

\newglossaryentry{warranted rents}
{
name=warranted rents,
description={the level of land rent that would be expected in equilibrium based on location and transportation costs. It may not be the rent charged to a tenant.}
}

\newglossaryentry{net rent}
{
name=net rent,
description={as used in this thesis, warranted rents minus taxes and maintenance costs }
}

\newglossaryentry{present value}
{
name=present value,
description={the value in cash today of a future sum of money or stream of cash flows, given a specified rate of return.}
}

\newglossaryentry{capital gain}
{
name=capital gain,
description={difference between the future sale price and the current purchase price}
}

\newglossaryentry{rent share}
{
name=rent share,
description={..}
}

\newglossaryentry{discount factor}
{
name=discount factor,
description={the present value of a dollar at a specified time in the future. It is a compounded value calculated using the individual discount rate.}
}

\newglossaryentry{mortgage term}
{
name=mortgage term,
description={the length time after a house purchase until a sum for a house purchase, the mortgage, must be returned to the lender with interest.}
}

\newglossaryentry{use value}
{
name=use value,
description={the monetary value of being allowed to live at a certain location ignoring potential speculative gains or losses. }
}


% Nomenclature glossary entries -- New definitions, or unusual terminology
\newglossary*{nomenclature}{Nomenclature}
\newglossaryentry{dingledorf}
{
type=nomenclature,
name=dingledorf,
description={A person of supposed average intelligence who makes incredibly brainless misjudgments}
}

% List of Abbreviations (abbreviations type is built in to the glossaries-extra package)
\newabbreviation{abm}{ABM}{Agent Based Model}


% List of Symbols
\newglossary*{symbols}{List of Symbols}
\newglossaryentry{rvec}
{
name={$\mathbf{v}$},
sort={label},
type=symbols,
description={Random vector: a location in n-dimensional Cartesian space, where each dimensional component is determined by a random process}
}