% Main glossary entries -- definitions of relevant terminology

% \newglossaryentry{}
% {
% name=,
% description={}
% }

% \newglossaryentry{}
% {
% name=,
% description={}
% }

% \newglossaryentry{}
% {
% name=,
% description={}
% }

\newglossaryentry{transmission mechanism}
{
name=transmission mechanism,
description={A general term to describe the sequence of processes through which an action at one point in a system  affects a variable at another point in the system. It is commonly used when discussing  monetary policy and how  expanding the money supply eventually affects employment.}
}

\newglossaryentry{real asset}
{
name=real asset,
description={Real assets are physical assets that have an intrinsic worth due to their substance and properties. Real assets include precious metals, commodities, real estate, land, equipment, and natural resources. }
}

% % Do we want to just make this feedback loop and feedback cycle and make uses in glossary consistent?

\newglossaryentry{feedback}
{
name=feedback,
description={The result of a causal loop. A term used in cybernetics and systems theory referring to a situation in which a change in one variable affects a second variable that then affects the first one.}
}

\newglossaryentry{surplus}
{
name=surplus,
description={Any amount or production or value in excess of what  is needed to pay for all the required inputs. Profit or rent. }
}

\newglossaryentry{Ricardian rent theory}
{
name=Ricardian rent theory,
description={The version of classical rent theory propounded by David Ricardo in his essay on the corn laws and generally seen as the  canonical version of land rent theory.}
}

\newglossaryentry{land market}
{
name=land market,
description={The entire complex of institutions, agents, and rules involved in transferring ownership of land. A land market exists wherever it is possible to exchange rights in land for agreed amounts of money or services rendered.}
}

\newglossaryentry{Alonzo-Jacobs cycle}
{
name=Alonzo-Jacobs cycle,
description={A positive \gls{feedback} cycle that occurs when city population is increasing in the wage, as in the Alonzo model, where the wage is increasing in city population, as implied by the Jacobs component of the \gls{Alonzo-Jacobs model}.}
}

\newglossaryentry{Public-Private Partnerships}
{
name=Public-Private Partnerships,
description={A long-term arrangement between a government and private sector institutions, often  employed for building, equipping, operating and maintaining schools, hospitals, transport, water, and sewerage systems. PPPs are used for projects with high social but low private returns when government is unwillling or unable to provide the up-front capital cost. The private rate of return is often subsidized by a guarantee that the private investor will receive a share of the social return over the course of the project's operation.}
}

\newglossaryentry{rent-seeking}
{
name=rent-seeking,
description={An economic concept that refers to the activity, seeking to gain wealth without contributing to productivity. Gordon Tullock, who introduced  the term, identified it as a form of theft\cite{tullockWelfareCostsTariffs1967}.}
}

\newglossaryentry{middle class}
{
name=middle class,
description={A broad term used by sociologists to describe the members of working classes who have equity and a standard of living above the subsistence level. The OECD includes anyone who earns between 75 per cent and 200 per cent of median household income after tax. Based on the most recent data available from Statistics Canada, in this country that means anywhere from about \$45,000 to \$120,000. The middle class is usually defined in terms of income level. The middle class defined this way, once the economic stratum of a clear majority of North American adults, has steadily contracted in the past five decades according to
Rakesh Kochhar and  Stella Sechopoulos of the \href{https://www.pewresearch.org/fact-tank/2022/04/20/how-the-american-middle-class-has-changed-in-the-past-five-decades/}{Pew Research Centre}  in 2022.}
}

\newglossaryentry{rentier}
{
name=rentier,
description={A person living on income from property or investments. The term is from the  French \textit{rentier}, ``holder of rental properties or investments that pay income,'' from \textit{rente} ``profit, income'' \cite{GET_rentier_defn_quote}. %``Financial engineering has created a rentier class, a modern feudal system, and the biggest beneficiaries of all that extra debt have been the bankers.'' Times, Sunday Times (2016) 
}
}

\newglossaryentry{rate of return}
{
name=rate of return,
description={or return on investment: the money made or lost on an investment over some period of time. Expressed nominally as the change in dollar value of an investment over time or  as a percentage derived from the ratio of profit to investment. We compute the nominal return, convert it to a percentage and compare that to the investor's best alternative return or required return.}
}

\newglossaryentry{joint-stock company}
{
name=joint-stock company,
description={A joint-stock company is a business owned by its investors, with each investor owning a share of the company based on the amount that they've invested. It is a predecessor to the modern-day corporation and other types of registered companies. A joint-stock company is an artificial person; it has legal existence separate from persons composing it. It can sue and can be sued in its own name. The shareholders are usually not liable for any of the company debts that extend beyond the company's ability to pay up to the amount of them.}
}

\newglossaryentry{REIT}
{
name=REIT,
description={Real Estate Investment Trust. A REIT is a financial instrument that  makes it possible for individual investors to earn dividends from real estate investments without having to buy, manage, or finance properties themselves. Structured as a company that owns and sometimes operates income-producing real estate or related assets, REITs are modeled after mutual funds \cite[GET-reit-like-mortgages].} %cite REITs are modeled after mutual funds? 
}

\newglossaryentry{financial instrument}
{
name=financial instrument,
description={A financial instrument is a monetary contract, which confers a right or claim against some counterparty in the form of a payment (checks, bearer instruments), equity ownership or dividends (stocks), debt (bonds, loans, deposit accounts), currency (foreign exchange or forex), or derivatives (futures, forwards, options, and swaps). There are %\href{https://www.investopedia.com/terms/f/financialinstrument.asp}{many types} 
many types of financial instrument \cite[WEB-investment-types].}
}

\newglossaryentry{compound interest rate}
{
name=compound interest rate,
description={Where an interest rate is specified for a single term, such as a year, the rate for a longer, multi-period term is larger. If the calculation for a later period includes interest on the interest from earlier periods, the interest is said to ``compound''. This is how interest is usually charged. Compound interest for a given period is calculated by multiplying the initial principal amount by one plus the annual interest rate raised to the number of compound periods minus one.}
}

\newglossaryentry{amortize}
{
name=amortize,
description={to reduce an amount gradually by making payment in installments: a to pay off (as a loan) gradually usually by periodic payments of principal and interest. }
}

\newglossaryentry{appraised value}
{
name=appraised value,
description={an evaluation of a property's value based on a given point in time. The evaluation is performed by a professional appraiser during the mortgage origination process.}
}

\newglossaryentry{premium}
{
name=premium,
description={}
}

\newglossaryentry{subsistence frontier}
{
name=subsistence frontier,
description={The minimum income or lowest standard of living that can sustain people in the economy. Rather than thinking of the limit as a single value -- say the minimum survival income -- it is more realistic to recognize that the limit can be achieved with different combinations of goods. For example, if clean water is freely available in a local stream, the subsistence income does not include the cost of bottled water. All the combinations can be seen as a \gls{frontier}. \newline In classical economics, the frontier was summarized as a subsistence wage. Subsistence theorists like Malthus argued that the market price of labour would not vary from the natural price for long: if wages rose above subsistence, the number of workers would increase and bring the wage rates down. The classical economist recognized that the limit was in part set by social convention, but it was analytically convenient to assume a subsistence wage, and it could be argued, following Malthus that a subsistence wage  represented a long-term limit or \gls{equilibrium}. As an analytical convenience in our model, we employ a subsistence wage that includes housing and a conventional standard of living.  }
}

\newglossaryentry{political economy}
{
name=political economy,
description={Political economy is a branch of social science that studies the relationship  between government and the economy. As a discipline, it dates back the  16$^{th}$ but is usually associated with the political economists of the mid-18$^{th}$ and  early 19$^{th}$  century like Adam Smith who began to explore the economic implications of free markets and industrialization. Departments of political economy persisted well into the mid 20$^{th}$ C before splitting into separate departments of economics politics.\cite{helleiner20PoliticalEconomy2018}}
}

\newglossaryentry{expected market price}
{
name=expected market price,
description={}
}

\newglossaryentry{market price}
{
name=market price,
description={}
}

\newglossaryentry{expectation}
{
name=expectation,
description={}
}

\newglossaryentry{perfect}
{
name=perfect,
description={}
}

\newglossaryentry{total factor productivity}
{
name=total factor productivity,
description={}
}

\newglossaryentry{factor of production}
{
name=factor of production,
description={(such as labour, land, financial capital,  and human capital)}
}

\newglossaryentry{perfect competition}
{
name= perfect competition,
description={An imaginary but analytically useful ideal market condition with the following  characteristics: 1. Large numbers of buyers and sellers in each market so that no individual buyer or seller can affect the price. 2. Free entry and exit of firms in the market. 3. Firms in each market sell a homogeneous product. 4. Buyers and sellers possess complete knowledge of the market. 5. No price controls.\newline  Economists often compare the markets they study to the` idealized, perfectly competitive market structure.}
}

\newglossaryentry{frontier}
{
name=frontier,
description={In mathematical economics, the limit of what is possible. Like the frontier of a country, even if you can't cross it, you can move along it to find the best location  subject to that constraint. In elementary economics, the budget-line and the production possibilities frontier (PPF) are  frontiers. Tf you spend less than the budget are operating inside the PPF, you could do better. Your solution is inefficient. }
}

\newglossaryentry{attractor}
{
name=attractor,
description={In \gls{dynamical system} theory as described by difference or differential equations, an attractor is a point or orbit inside a region of the phase space. The phase space is a representation of all possible states of the system each corresponding to a unique point in the phase space. If there is an attractor in the region, if the system starts at any point in the region,  it will eventually evolve to the attractor.}
}

\newglossaryentry{dynamical system}
{
name=dynamical system,
description={any system that changes over time. Typically we mean a  system that is described by a set of interrelated equations, one of which is time-dependent. }
}

\newglossaryentry{agent-based}
{
name=agent-based,
description={a term for a model that is a  collection of autonomous decision-making entities called agents. In practice the agents are little sub-programs (automata, robots) that each separately use some information about their environment and follow some internal rules to choose a response in each model cycle. See \gls{agent-based model}}
}

\newglossaryentry{price formation}
{
name=price formation,
description={The process of selecting a price based on the conditions in a system. The classic problem is the simple supply and demand model, in which sellers and buyers, each group with its own wants represented by an equation, interact to find a a price. The model  identifies a combination of price and quantity that would be acceptable to both at the same time, but doesn't say how they get to the price. It lacks a price formation mechanism.  \newline The fundamental problem is that the agents don't have complete information and may have limited computational ability, especially with multiple interacting markets. A theory of price formation has to describe the process of adjustment. This is usually represented as a set of individual adjustment rules, which makes any theory of price formation a dynamical system It may not always lead to a steady state equilibrium.}
}

\newglossaryentry{classical}
{
name=classical,
description={Referring to the period of economic theorizing primarily in Britain roughly between 1750 and 1870, prior to the neoclassical period in economics. See \gls{classical economics}.   }
}

\newglossaryentry{market rent}
{
name=market rent,
description={The amount a landlord charges a tenant for the use of a property in a competitive market. }
}

\newglossaryentry{mill rate}
{
name=mill rate,
description={The municipal tax rate: the amount per \$1,000 of the assessed value of a property which will be due as property tax.}
}

\newglossaryentry{financialization}
{
name=financialization,
description={Something is financialized when a financial instrument representing it is created. The  home mortgage market was financialized when financial institutions developed markets that let let investors buy and sell mortgages between themselves. These transactions gave investors ownership of the stream of income established by the mortgage contract. The transaction did not affect the mortgage conditions or the home: they simply added a new product for investors to speculate on. }
}

\newglossaryentry{amenity}
{
name=amenity,
description={a desirable or useful feature or facility of a building or place.}
}

\newglossaryentry{population}
{
name=population,
description={In our model, the number of city residents. }
}

\newglossaryentry{financialize}
{
name=financialize,
description={Something is financialized when a financial instrument representing it is created. For example, mortgages originated in England when people did not have the resources to purchase land in one transaction. Buyers would get loans directly from the seller – no banks or outside parties were involved. Home mortgages were financialized when financial institutions developed markets that let them buy and sell mortgages between themselves.  See \gls{financialization}}
}

\newglossaryentry{housing market}
{
name=housing market,
description={}
}

\newglossaryentry{urban center}
{
name=urban center,
description={}
}

\newglossaryentry{functional form}
{
name=functional form,
description={the algebraic form of a relationship between a dependent variable and explanatory variables.}
}

\newglossaryentry{production}
{
name=production,
description={The process of converting a set of \glspl{input} into a desired \gls{output}. See\gls{factor of production}.}
}

\newglossaryentry{perfectly elastic}
{
name=perfectly elastic,
description={Producing more won't affect the product's price. See \gls{elasticity}.} % On a \gls{supply demand curve} (is that the right name?) ***}
}

\newglossaryentry{elasticity}
{
name=elasticity,
description={The ratio of the percentage change in a quantity to the percentage change in another quantity. The price elasticity of demand, for  example, would be the percentage change in the quantity demanded that accompanies a one-percent change in price. It is a (local) property of a demand curve and would typically be a negative number like $-0.3$ or $-1.5$, since demand typically slopes downward. See \gls{perfectly elastic}.}
}

\newglossaryentry{labour augmenting agglomeration}
{
name=labour augmenting agglomeration,
description={The situation in which bringing more workers together increases their average productivity.}
}

\newglossaryentry{present discounted value}
{
name=present discounted value,
description={The amount that someone should be willing to pay, in the present, for a stream of expected future payments.}
}

\newglossaryentry{wealth}
{
name=wealth,
description={In our model, wealth is the set of valuable economic resources owned, by an individual or organization as measured in either real goods or money value, that the bank considers in lending decisions.}
}

\newglossaryentry{input}
{
name=input,
description={In production theory, an input is any good or service used to produce another another good or service. % anything  that is among the collections of goods and services that is used to produce a desire  product or service. 
For example, labour is a necessary input for producing food. }
}

\newglossaryentry{output}
{
name=output,
description={In production theory, an output anything produced.} % Often symbolized by $Y$ or $Q$  in relations like $Y= F(K,L,N)$.}
}

\newglossaryentry{subsistence wage}
{
name=subsistence wage,
description={In most urban models the subsistence wage is treated as base cost that is the opportunity cost of agricultural land. We have extended the technique to include the opportunity cost of urban labour. It is one of the simplifications which makes our model tractable and focuses it on the question of rents and the specifically urban productivity premium. In our model, the subsistence wage is a wage available inside and outside the urban area, which covers the cost of buildings, food, core living costs, and a base cost of land.}
}

\newglossaryentry{urban wage premium}
{
name=urban wage premium,
description={The wage premium is the premium above the \gls{subsistence wage} payed by employers to attract workers. An urban wage premium appears when workers in larger cities earn higher average wages than workers in smaller cities. In both the U.S. and Sweden a wage premium has been shown to follow a power-law relationship that scales superlinearly with city size. In other words, workers in larger cities not only earn higher average wages, they do so systematically as a power law function of the size of the city. Bettencourt, \cite{bettencourtIntroductionUrbanScience2021}, has demonstrated theoretically that a wage premium should manifest as a power law function and predicted the value of its exponent.}
}

\newglossaryentry{urban wage}
{
name=urban wage,
description={The \gls{urban wage} is the \gls{urban wage premium} plus the \gls{subsistence wage}.}
}

\newglossaryentry{product}
{
name=product,
description={A product is anything produced. It is an \gls{output} of a production process. % Our model has no specific products. 
Rather than specific products, the city in our model produces an aggregate output, which is not a variable in our analysis. Instead of producing explicit list of discrete products, output is defined by an implicit production function relating labour as an input to aggregate productivity and thus to wages. % as part of urban incomes.
}
}

\newglossaryentry{imperfect competition}
{
name=imperfect competition,
description={A market in which any of the conditions required for \gls{perfect competition} are not met.}
}











\newglossaryentry{demand function}
{
name=demand function,
description={An equation describing how much a potential buyer or group of buyers will purchase at any given price. It can express price as a function of quantity or quantity as a function of price. In either case it will usually include other variables that are said to "shift" demand.   }
}

\newglossaryentry{supply demand curve}
{
name=supply demand curve,
description={}
}

\newglossaryentry{increasing returns to scale}
{
name=increasing returns to scale,
description={}
}

\newglossaryentry{decreasing returns to scale}
{
name=decreasing returns to scale,
description={Each new worker increases \gls{output} by less than the previous worker did.}
}

\newglossaryentry{constant returns to scale}
{
name=constant returns to scale \gls{CRS}. ***,
description={When doubling all \glspl{input} results in exactly double the \gls{output}. }
}

\newglossaryentry{equilibrium condition}
{
name=equilibrium condition,
description={}
}

\newglossaryentry{population equilibrium}
{
name=population equilibrium,
description={an \gls{equilibrium condition} ADD}
}

\newglossaryentry{urban labour supply}
{
name=urban labour supply,
description={}
}

\newglossaryentry{stochastic}
{
name=stochastic,
description={Having a random probability distribution or pattern that may be analyzed statistically but may not be predicted precisely. Introducing even a small amount of random nose into even one variable in a model of a deterministic system converts the model into a stochastic model.}
}

\newglossaryentry{aggregate}
{
name=aggregate,
description={Formed or calculated by the combination of many separate units or items; a total.}
}

\newglossaryentry{agglomeration economies}
{
name=agglomeration economies,
description={Economic efficiencies resulting from \gls{agglomeration effects}.}
}

\newglossaryentry{agglomeration}
{
name=agglomeration,
description={A collection of similar items in one location. A city is an agglomeration of people and generally of firms. Agglomeration may have properties that individuals do not have, giving rise to \gls{agglomeration effects} or gls{agglomeration economies}.}
}

\newglossaryentry{locational equilibrium}
{
name=locational equilibrium,
description={A situation in which no resident will make herself better off by moving to another location. A Nash equilibrium with housing efficiently allocated  given market prices. See \gls{migration equilibrium}.}
}

\newglossaryentry{agglomeration effects}
{
name=agglomeration effects,
description={An effect of increasing the number of firms  or workers in one place. A larger, deeper, more specialized labour pool enables workers to better match their skills to the needs of firms or creates knowledge spillovers in which firms and workers learn from each other.}
}

\newglossaryentry{monopolistic competition}
{
name=monopolistic competition,
description={A type of  \gls{imperfect competition}. \gls{Perfect competition} is a description of a market with many seller, all of whom are price-takers. Monopoly is a market with a single seller, who therefore has the power to set the selling price. Monopolistic competition describes cases in between, with sellers that have some power to set prices within a segment of the market. It occurs when many companies offer competing products or services that are similar, but are not perfect, substitutes.}
}

\newglossaryentry{labour adjustment cost}
{
name=labour adjustment cost,
description={Costs associated with hiring, firing or training that prevent or slow the rate at which a firm will increase or decrease the number of workers it employs.}
}

\newglossaryentry{frictional unemployment}
{
name=frictional unemployment,
description={the part of total unemployment  due to people being in the process of voluntarily moving from one job to another.}
}

\newglossaryentry{marginal product}
{
name=marginal product,
description={the amount that the last unit of any factor  adds to output while holding all other factors constant. See \gls{marginal product of labour}.}
}


% \newglossaryentry{marginal product of labour}
% {
% name=marginal product of labour,
% description={the amount that the last worker  adds to output without changing the quantities of other inputs used. See \gls{marginal}.}
% }

\newglossaryentry{monopoly}
{
name=monopoly,
description={ Market power means you can price above marginal costs. Need free entry to get rid of it. -- it doesn't drive out profit - profits can be sustained over longer. Monopolist can charge a higher price but pays a competitive price for all \glspl{input} including labour. If a firm also had a monopoly on offering jobs, they could drive down wages.}
}

\newglossaryentry{duopoly}
{
name=duopoly,
description={A market with two sellers. Under one set of assumptions the result will be the monopoly price, under others, the situation will generate lower  than monopoly prices, ore even competitive pricing and may result in market instability.}
}

\newglossaryentry{monopsony}
{
name=monopsony,
description={A market with one buyer that therefore has market power.}
}

\newglossaryentry{imperfect information}
{
name=imperfect information,
description={ the buyers and/or sellers do not have all the information necessary to make an informed decision.}
}

\newglossaryentry{externalities}
{
name=externalities,
description={any indirect costs or benefit to uninvolved third parties that are an effect of a decision-makers activity but are not included in the decision-maker's cost-benefit calculations. Lawn mowers may wake the neighour, emissions from vehicles cause emphysema, burning fossil fuels may contribute to climate change, or painting your house may raise the value of the neighbour's house. In our model, when employers increase their workforce there is a positive effect on the productivity of all other workers in the city. This is an external effect}
}

\newglossaryentry{competitive market}
{
name=competitive market,
description={**FIX Everybody is a price taker. Price takers don't assume anything they do affects other producers or suppliers, so they act in terms of their internal prices and costs. 
This means their decision making process doesn't take into account any one else's behaviour.
? The easy way to see that is assume prices are fixed - all that's required to get the behaviour. ..  have a few other things like free exit and entry, perfect information etc -- to get the efficiency result. - (or to ensure price taking)}
}

\newglossaryentry{effective labour}
{
name=effective labour,
description={FIX - Effective labour is the productive \gls{output} from labour. As soon as you introduce agglomeration economies, labour becomes a more complex phenomena. There is the benefit of the single worker which should be perfectly declining on that nice concave production function and there is the diagonal movement as a result of increasing productivity because you keep adding people to the market. That means that your productivity of the worker isn't' just attached to the worker and your plant. It has this other component.. 'effective labour' -- the output including the A term.}
}

\newglossaryentry{spillover effects}
{
name=spillover effects,
description={ \Gls{externalities} are the most commonly discussed form of spillover effects but any economic event in one context that occurs because of something else in a seemingly unrelated context can be considered a spillover. It is a looser term than externality because an externality is a consequence, at least in economic theory, of rational optimizing behaviour.}
}

\newglossaryentry{substitutable}
{
name=substitutable,
description={One good may be substituted for another without loss of benefit. Two brands of motor oil are good substitutes for each other. Oranges are somewhat subsitutable for apples , but not for screwdrivers.}
}

\newglossaryentry{neoclassical distribution theory}
{
name=neoclassical distribution theory,
description={A theory that states that in perfect competition the owner of every unit of every  \gls{factor of production} will be paid precisely the  value of the \gls{marginal product} of that factor for each unit unit they contribute to production.}
}

\newglossaryentry{Solow-Swan model}
{
name=Solow-Swan model,
description={a  model of macro-economy developed and analysed by Robert Solow and Trevor Swan independently to explain long-run economic growth.\cite{dimandTrevorSwanNeoclassical2009} It attempts to explain  growth in terms of the growth of three contributing factors, capital, labor (population),  and  productivity.}
}

\newglossaryentry{marginal product of labour}
{
name=marginal product of labour,
description={Firms calculate what the next worker is worth to them. That's what they're willing to pay for labour. 
This is the labour \gls{demand function} based on the \gls{marginal product} which is declining. When a firm has only a few workers, it is high on that demand function, and has to move down. It cuts workers. If it's too low, it expands and hires. %This says something about the geometry of what employers could pay. 
% Firms can't pay workers more than they can earn in the long term, unless that money comes from somewhere, but they could push down wages and extract more profit, invest more in other factors of production, etc.
}
}

\newglossaryentry{migration equilibrium}
{
name=migration equilibrium,
description={The theoretical situation in which no resident can  make themselves better off by moving to another location. It is a logical consequence of utility maximization and free mobility that results in a Pareto optimal allocation of housing. Technically it is a Nash equilibrium, While extremely useful in analysing urban systems, the concept does not closely describe real cities.
% A situation in which not resident will make herself better off by migrating to or between cities or countries. Similar to a migration equilibrium.
}
}

\newglossaryentry{commuter shed}
{
name=commuter shed,
description={For a city, the area over which people will travel to work in a city. In the \gls{Alonzo-Jacobs model}, it is sharply defined by the maximum distance commuters can travel before transportation costs exceed the wage premium. In  practice, the duration of commutes is highly variable. It is greater in the case of men, singles, educated and foreign workers, persons living in rented housing, using public transport, living or working in large cities, or working in large firms,  and when the  unemployment rate is high\cite{axisaFactorsInfluencingCommute2012} .}
}

\newglossaryentry{circular city}
{
name=circular city,
description={In urban theory, an idealized city form predicted by models with uniform travel costs in all directions and a fixed household commuting budget. If a city is laid out on a rectangular grid, the same travel-cost logic yields a rectangular city. Recently the term is applied to cities committed to achieving a circular economy. }
}

\newglossaryentry{radial city}
{
name=radial city,
description={A radial concentric city plan is formed by streets that extend outward from a defined center and reach the outer edge of the city, together with concentrically arranged roads that connect the radial streets to the lots. it is an idealized pattern that traces back to ancient times and appears  today in planned cities and districts. See \gls{circular city}.}
}

% \newglossaryentry{surplus}
% {
% name=surplus,
% description={Or economic surplus. Any social product in excess of the minimum required to reproduce society. In value terms the surplus appears as profit or rent and accrues to the owner of a  scarce \gls{input} that varys in quality, such as land. In the mid-19th century, French engineer Jules Dupuit first extended the concept of economic surplus to what came to be called producer- and consumer-surplus.}
% }

\newglossaryentry{Alonzo-Jacobs model}
{
name=Alonzo-Jacobs model,
description={A model combining the \gls{Alonzo model} of the urban land use  \cite{alonsoModelUrbanLand1960} with the \gls{agglomeration} theory of Jane Jacobs \cite{jacobsEconomyCities1969} which explains the productivity of cities.}
}

\newglossaryentry{monopsonist}
{
name=monopsonist,
description={A single buyer, usually in an input market. A monopsonist is not a price-taker, knowing that buying more well result in  higher prices. This rleads the monopsonist to purchase less than is socially efficient.}
}

\newglossaryentry{financial return}
{
name=financial return,
description={MAYBE ADD what is best definition? - there may be other returns. Assessed by comparing the net rent $\mathcal{R}_N$ to the costs of acquiring a property, in particular to the cost of borrowing money. CLARIFY}
}

\newglossaryentry{home services}
{
name=home services,
description={A property offers two kinds of services: home services and \gls{locational services}. Home services describes the value offered by living in a house: a place to sleep, to prepare food, the amenity of being in the home, etc. Since people require housing inside and outside the city, home services are modeled as paid for as a share of the subsistence wage ($a \psi$).}
}

\newglossaryentry{locational services}
{
name=locational services,
description={A property offers two kinds of services: \gls{home services} and locational services. Locational services are services accessed by right of location. They include access to the central city job, access to locational amenity, and the benefit of services and connections associated with a location. In the core model, Locational services are, on an annual basis, the rent premium $w$, minus the transportation costs $c$ for a property a given distance, $d$, from the center, $\omega- {dc}$.}
}

\newglossaryentry{rent share}
{
name=rent share,
description={..}
}

\newglossaryentry{Pareto efficiency}
{
name=Pareto efficiency,
description={An economic state where resources cannot be reallocated to make one individual better off without making at least one individual worse off.}
}

\newglossaryentry{efficiency conditions}
{
name=efficiency conditions,
description={Conditions derived in neoclassical economic theory that must be satisfied if a system or activity is to achieve Pareto efficiency. Under somewhat reasonable conditions the efficiency conditions are achieved by agents acting in a decentralized manner to maximize their own profit or utility.}
}

\newglossaryentry{neoclassical economics}
{
name=neoclassical economics,
description={An approach to the study of the economy and economic behaviour that attempts to explain the production, pricing, and the consumption of goods and services through supply and demand, and to explain agent behaviour using a theory of rational agents who satisfy \gls{marginal} efficiency conditions. It integrates, within a mathematical framework, the cost-of-production theory from classical economics with a consumer demand theory based on utility maximization.}
}


\newglossaryentry{neoclassical}
{
name=neoclassical,
description={refers to a period in economic theorizing that overlaps with  but largely follows classical economics and dominates economic theory to  today. See \gls{classical economics}, gls{neoclassical economics}.}
}

\newglossaryentry{classical economics}
{
name=classical economics,
description={Also called classical \gls{political economy}. A school of thought in political economy that flourished, primarily in Britain, in the late 18th and early-to-mid 19th century. It was part of the intellectual  development of  Western liberal democracies in the 18th and 19th centuries and  brought into the mainstream by Scottish economist Adam Smith. Its main thinkers include Adam Smith, Jean-Baptiste Say, David Ricardo, Thomas Robert Malthus, and John Stuart Mill. After 1850, key features of the classical approach were carried forward by  Karl Marx and his followers, and by Henry George. Classical economics  provided the foundation for the development of \gls{neoclassical economics}.}
}

\newglossaryentry{socioeconomic status}
{
name=socioeconomic status,
description={Socioeconomic status is typically broken into three levels, high, middle, and low,  commonly referred to as ``upper class'', ``middle class'', and ``working or lower class'', it differs from `\gls{class}' in the more traditional sense, which is a functional classification. See \gls{Ricardian class}, being based on occupation, income, family wealth.}
}

\newglossaryentry{Ricardian class}
{
name=Ricardian class,
description={The conception of class in \gls{classical economics} including Marx, where class is based on the types and amounts of productive capital the individual owns. See \gls{class}.}
}

\newglossaryentry{rent profile}
{
name=rent profile,
description={see \gls{bid-rent function} or \gls{bid-rent curve}}
}

\newglossaryentry{class}
{
name=class,
description={This term has a wide range of sometimes conflicting meanings. In our usage, which is consistent with classical economics including Marx class is based on the types and amounts of productive capital the individual owns. This is a functional definition quite distinct from socioeconomic status which is more common in the current discussion. Our treatment of the evolution of class structure with financialization draws on We allow  people in different functional classes to own financial capital, producing intermediate classes \`a la Roemer\cite{roemerGeneralTheoryExploitation1982}.}
}
\newglossaryentry{capitalize}
{
name=capitalize,
description={To capitalize a stream of expected income is to compute it's capitalized value. Capitalized value is the current worth of an asset, usually real estate, based on a calculation of present value of expected income over the course of its economic lifespan.}
}

\newglossaryentry{agglomeration effect}
{
name=agglomeration effect,
description={The external economies associated with size and concentration. The benefits of size and concentration vary for different cross-sections of the urban population. Three such groupings may be identified: 1. Consumer agglomeration economies; Business agglomeration economies; Social agglomeration economies \cite{carlinoAgglomerationEconomiesSurvey1978}.}
}

\newglossaryentry{price bubble}
{
name=price bubble,
description={The sustained rise in the price of an asset above its ``normal'' market value'' caused by agents (mainly speculators) forecasting further price increases base on previous increases, rather than on estimates on intrinsic value.  Price bubbles are sustained by expectations of future increases in the price of an asset. They may end sharply, or crash, when expectations shift.}
}

\newglossaryentry{marginal value-product}
{
name=marginal value-product,
description={Also known as the marginal revenue product. The marginal revenue created due to an addition of one unit of productive resource, such as one more worker. Calculated by multiplying the marginal physical product by the price, or the marginal revenue in the case of a non-competitive market.}
}

\newglossaryentry{neoclassical growth theory}
{
name=neoclassical growth theory,
description={An economic theory that outlines how a steady economic growth rate results from a combination of three driving forces—labour, capital, and technology. Robert Solow and Trevor Swan developed and introduced the model of long-run economic growth in 1956. It is the  foundation of most empirical and theorical attempts to explain macroeconomic growth.}
}


\newglossaryentry{capital}
{
name= capital,
description={The word "capital" has many different meanings in economics and finance.  In economics, real capital is durable produced goods that are in turn used as productive inputs. \Gls{financial capital} is wealth that can be used to lend to others in exchange for interest payments or to purchase assets. Both these forms of capital are represented in the economy by extensive legal organizational structures that represent the interests of their owners. Other forms of capital recognized by economists are human capital, social capital natural capital, and intellectual capital. The distinctive  features of capital include that it takes time and energy to create (natural capital, however is  `a free gift of nature''), lasts a long time, deprecates, produces a stream of benefits over time,  and may be transferable as property}
}

\newglossaryentry{financial capital}
{
name=financial capital,
description={Financial capital is simply lendable purchasing power. Owners of financial capital provide their liquidity to borrowers in exchange for a future return. Interest rates are the prices charged for the use of financial capital. It is generally based on (secured by) ownership of tradable assets.  Anything can be a form of financial capital as long as it has a monetary value and can be  used in the pursuit of future revenue. Marx distinguished  financial capital (then called circulating capital or money capital) from fixed or real capital.}
}

\newglossaryentry{agent-based model}
{
name=agent-based model,
description={Agent-based models are computer simulations used to study the interactions between people, things, places, and time. They are usually stochastic models built from the `bottom up,' meaning by modelling individual agents (people, institutions, etc). Agents essentially sub-programs that respond to other agents and the environment in certain ways. These interactions produce emergent effects that may differ from the results of traditional, regression-based methods in that, like systems dynamics modeling, it allows for the exploration of complex systems that display non-independence of individuals and \gls{feedback} loops in causal mechanisms.}
}

\newglossaryentry{urban scaling}
{
name=urban scaling,
description={Urban scaling laws reliably relate socio-economic, behavioural and physical variables to the population size of cities. They allow for approaches  to city planning and for an understanding of urban resilience and economics. In this thesis we use the well-established relationship between population and urban productivity. \cite{doi:10.1098/rsif.2020.0705}}.
}

\newglossaryentry{classical rent theory}
{
name=classical rent theory,
description={explained how land generated surplus value for its owner and how this surplus explained the wealth and income of the land-owning class. David Ricardo produced the classic description in 1815 based on extensive prior analysis by others in the preceding century. The key notion is the ``marginal'' unit of land. It is just barely worth putting this land into production because it just barely produces enough to justify the cost of production, and transportation. More productive land or better located land produces a surplus that the landowner  collects in the form of land rent collected from tenant farmers. No tenant would pay to cultivate the  i unit of land. The theory employed the basic logic of later the later ``marginalist''  school of economic analysis. = See \gls{class}, \gls{rent}.}
}

\newglossaryentry{rent}
{
name=rent,
description={The economic  surplus generated in production as a result of differences in the quality of some \gls{factor of production}. Often described as the difference between the opportunity cost of a factor of production and the income it earns. In this thesis we focus on rents generated by \glspl{agglomeration effect}. According to \gls{classical rent theory}, rent is the price paid for the use of land. More generally it is the  surplus generated by any natural resource, up to and including the athletic talents of basketball stars.\cite{lackmanClassicalBaseModern1976} Land, talent, and mineral resources are seen as ``the free gift of nature'', forms of capital which the owners do not create but do appropriate. Like the productivity of agricultural land in classical theory,  urban \gls{agglomeration effect}s produce land rents that are not created but are appropriated by the landowners. See \gls{class}.}
}

\newglossaryentry{maximum bid function}
{
name=maximum bid function,
description={A function that generates the maximum that an investor would bid for a property.  See \gls{bid-rent curve}, \gls{bid-rent function}.}
}

\newglossaryentry{bid-rent function}
{
name=bid-rent function ,
description={See \gls{bid-rent curve}.}
}

\newglossaryentry{reservation price}
{
name=reservation price,
description={Seller's minimum price of to accept a bid. If no offer is at least as large as the reservation price, the seller is effectively the buyer. It is lowest price that a prospective seller will accept, and is computed as seller's maximum bid price, which incorporates the net rent achievable.}
}

\newglossaryentry{bid-rent curve}
{
name=bid-rent curve,
description={The height of a graph showing distance from employment horizontally and the amount that residents will pay to rent land at that distance. It is also also called a \gls{rent profile}. With varying agents and property attributes can be seen a set of functions of location, each of which  generates a bid price for one category of agent.   See \gls{rent premium}, .}
}

\newglossaryentry{borrowing ratio}
{
name=borrowing ratio,
description={$m$. The maximum fraction of the price of a property that may be mortgaged. Determined by the bank (the lender) based on individual wealth and income. }
}

\newglossaryentry{rent premium}
{
name=rent premium,
description={or \gls{warranted economic rent} is the excess rent  that might be charge for the use of urban land relative the non-urban land. In our model the rent premium for an urban property is equal to the urban wage premium minus the transportation costs. }
}

\newglossaryentry{warranted rent}
{
name= warranted rent,
description={$\mathcal{R}_N$ at an  urban location  $d$ units from the centre, is the the value of the flow of services provided by the property, including the locational value, or \gls{warranted economic rent}. It is level of land rent that would be expected in equilibrium based on location and transportation costs.  (It may not be the rent actually charged to a tenant.) }
}

\newglossaryentry{warranted price}
{
name= warranted price,
description={$\mathcal{R}_N$ at an  urban location  $d$ units from the centre, capitalized value of the flow of services provided by the property, the \gls{warranted rent}. 
which  includes locational value, or \gls{warranted economic rent}, (It may differ from the market price) }}

\newglossaryentry{warranted economic rent}
{
name=warranted economic rent,
description={The locational value of an urban property. A surplus generated by \glspl{agglomeration effect}, equal to the urban (wage premium) minus transportation costs, $\omega-{c} d$). This is the the amount that an equilibrium market rent for a property would be expected to exceed the market rent for a similar non-urban property.}
}

\newglossaryentry{net rent}
{
name=net rent,
description={Or net market rent. The warranted rents minus taxes and maintenance costs.}
}

% \newglossaryentry{rent share}
% {
% name=rent share,
% description={..}
% }

% rent paid
% economic rent
% locational rent

\newglossaryentry{marginal}
{
name=marginal,
description={relating to or situated at the edge or margin of something. In the marginalist approach to economics, it  refers to technique of focusing on the cost or benefit of the next unit or individual.}
}

\newglossaryentry{inframarginal}
{
name=inframarginal,
description={Coming before the margin is reached. For example, if the wage is $x$, all workers willing to work for less than $x$ are inframarginal. They are selling their time for more that it is worth to them. They come out ahead on the bargain. Workers who will work for $x$ but not a penny less are marginal. Similarly, with land, the most remote or the least productive land in use is ``marginal'' while  inframarginal land is more productive and generates a \gls{surplus}.}
}

\newglossaryentry{overlapping generations}
{
name=overlapping generations,
description={In the \gls{OLG} model individuals live a finite length of time, long enough to overlap with at least one period of another agent's life. The OLG model is the natural framework for the study of life-cycle behavior (investment in human capital, work and saving for retirement).}
}

\newglossaryentry{stylized facts}
{
name=stylized facts,
description={Economists use this term for observations that are widely understood to be empirical truths, to which theories must fit.  Also described as, ``broad tendencies that aim to summarize the data, offering essential truths while ignoring individual details''. The term "stylized facts" was introduced by the economist Nicholas Kaldor in the context of a debate on economic growth theory in 1961 \cite{kaldorCapitalAccumulationEconomic1961}.}
}

\newglossaryentry{perfect foresight}
{
name=perfect foresight,
description={The correct prediction of future events. If agents have  all relevant information and  a correct model to use for prediction. When there is uncertainty it is not possible to have perfect foresight. In solving a complex intertemporal model, economists may assume agents have perfect foresight. This is called the rational expectations approach.}
}

\newglossaryentry{equilibrium reasoning}
{
name=equilibrium reasoning,
description={Gls{equilibrium} analysis identifies variable values of particular interest because they are likely to exhibit stability or capture the implications of the goals of agents. Using these equilibrium values and how they are likely to change in the regions of an equilibrium draw conclusions to is `equilibrium reasoning' because it bases the conclusions on assumptions about the behaviour of the variables near an equilibrium. Equilibrium reasoning implicitly  assumes that the variables will tend to stay near and smoothly approach the equilibrium.}
}

\newglossaryentry{equilibrium}
{
name=equilibrium,
description={In economics and other sciences, an equilibrium is a situation in which forces such as supply and demand are balanced, and in the absence of external influences the (equilibrium) values of variables will not change. In economics an equilibrium is usually understood behaviourally as a situation in which no agent has an incentive to change behaviour given what others are doing. Such a situation is called a Nash Equilibrium or a Cournot-Nash equilibrium.}
}

\newglossaryentry{expectations}
{
name=expectations,
description={Predictions of future events or values, formed by agents for use in decision-making. Agents may form their expectations by looking backward at data on previous values, or by projecting forward using a mental model of how the system works. If agents are fully informed about the state of the system and how it works, their expectations are essentially the same as the predictions of the relevant economic theory and they are termed `rational expectations'.\cite{muthRationalExpectationsTheory1961}. In probability theory, the expected value of a variable is the mean of its true distribution (the rational expectation), which is usually estimated using the observed realizations (a backward-looking estimate).}
}

\newglossaryentry{Alonzo model}
{
name=Alonzo model,
description={The model credited to William Alonzo, also called the Alonso-Muth model. A full development of the theory is presented in Alonzo's doctoral dissertation \cite{alonzoTheoryUrbanLand1960}.} %, A MODEL OF THE URBAN LAND MARKET: LOCATIONS AND DENSITIES OF DWELLINGS AND BUSINESSES, University of Pennsylvania, 1960.}
}

\newglossaryentry{asking price}
{
name=asking price,
description={The price a seller initially posts on deciding to sell a property. It will be higher than the seller's maximum bid price.}
}

\newglossaryentry{bid price}
{
name=bid price,
description={In the computational model, any price that an agent bids for a property in the transaction process. It will be less than or equal to the agent's maximum bid price and less than or equal to the asking price.}
}

\newglossaryentry{maximum bid price}
{
name=maximum bid price,
description={The maximum price that investors will bid for a property. A bid that makes the expected return exactly the required or  target return.}
}

\newglossaryentry{bargaining}
{
name=bargaining,
description= {in the computational model during price setting for a particular property, there is a bargaining process that takes as arguments the highest bid price, reservation price, and asking price, returning a sale price for the property, as well as property transfer instructions.}
}
%The reservationn prices  is  the seller  own bid. If the max bid of the highest bid received is lower than the own bid the seller is the buyer- remains the owner. 
%Otherwisesimplest rule is  (reservation bid+maxbid)/2


\newglossaryentry{model}
{
name=model,
description={A system, A, which is useful for understanding another system, B. as the model we present is useful for understanding the effect of growing finacialization working through the system of urban land ownership.}
}

\newglossaryentry{specification}
{
name=specification,
description={With respect to a model or a theory, associating the theoretical constructs or relationships in a theory with a specific model, or associating specific model elements with observables.}
}

\newglossaryentry{distribution}
{
name=distribution,
description={The way total \gls{output}, income, \gls{wealth} or assets are distributed among individuals, each \gls{factor of production}, or the \glspl{class} of society. The term may refer to a theoretical approach  or to an empirical distribution of any of these. Theories of distribution are systematic attempts to account for the sharing of the national income.  Distributions across classes is known as a functional distribution distribution and  corresponds to the the approach of the \gls{classical economics}. Neoclassical economics examined distribution through the payment to factors of their \gls{marginal value-product}.}
}

\newglossaryentry{production function}
{
name=production function,
description={A representation of the technology of production, often a functional relationship between the \glspl{input} that enable production and the quantity of \gls{output}.}
}

\newglossaryentry{Cobb-Douglas}
{
name=Cobb-Douglas,
description={A specific production function. commonly used for illustrative or estimation in economics. Essentially a form of geometric mean.}
}

\newglossaryentry{productivity}
{
name=productivity,
description={The ratio of \gls{output} to \glspl{input}. Which outputs and inputs are considered varies. \gls{total factor productivity} refers to aggregate outputs and inputs in value terms. Marginal productivity refers to the addition to total output produced by one additional unit of input.} 
}

\newglossaryentry{growth}
{
name=growth,
description={The rate of increase in aggregate \gls{output} for a given production unit, such as a nation  or a city.}
}

\newglossaryentry{regime}
{
name=regime,
description={A distinct state of a system, a region of the system's phase space. In dynamical system theory, a phase space is a space in which all possible states of a system are represented, with qualitatively distinct  states corresponding to one region in the phase space.}
}

\newglossaryentry{resilience}
{
name=resilience,
description={The ability of a system to return to its original state when shocked by a change in its determining variables. May refer to smoothly or successfully adapting to a change in  determining variables.}
}

\newglossaryentry{hysteresis}
{
name=hysteresis,
description={An event in the economy that persists even after the factors that led to that event have been removed or otherwise run their course.}
}

\newglossaryentry{present value}
{
name=present value,
description={The value in cash today of a future sum of money or stream of cash flows, given a specified rate of return.}
}

\newglossaryentry{capital gain}
{
name=capital gain,
description={The difference between the future sale price and the current purchase price.}
}

\newglossaryentry{discount factor}
{
name=discount factor,
description={The present value of a dollar at a specified time in the future. It is a compounded value calculated using the individual discount rate.}
}

\newglossaryentry{mortgage term}
{
name=mortgage term,
description={The length time after a house purchase until a sum for a house purchase, the mortgage, must be returned to the lender with interest.}
}

\newglossaryentry{use value}
{
name=use value,
description={The monetary value of being allowed to live at a certain location ignoring potential speculative gains or losses.}
}

\newglossaryentry{wealth trajectories}
{
name=wealth trajectories,
description={**FIX add a term, part of linking resilience, class, and hysteresis.}
}

% % Nomenclature glossary entries -- New definitions, or unusual terminology
% \newglossary*{nomenclature}{Nomenclature}
% \newglossaryentry{dingledorf}
% {
% type=nomenclature,
% name=dingledorf,
% description={A person of supposed average intelligence who makes incredibly brainless misjudgments}
% }

% List of Abbreviations (abbreviations type is built in to the glossaries-extra package)

% \newabbreviation{}{}{}
% \newabbreviation{}{}{}
% \newabbreviation{}{}{}
% \newabbreviation{}{}{}
% \newabbreviation{}{}{}
% \newabbreviation{}{}{}

% \newabbreviation{REIT}{REIT}{real estate investment trust}

\newabbreviation{ABM}{ABM}{agent-based model}

\newabbreviation{CRS}{CRS}{constant returns to scale}

\newabbreviation{OLG}{OLG}{overlapping generations}

% List of Symbols
\newglossary*{symbols}{List of Symbols}
\newglossaryentry{rvec}
{
name={$\mathbf{v}$},
sort={label},
type=symbols,
description={Random vector: a location in n-dimensional Cartesian space, where each dimensional component is determined by a random process}
}