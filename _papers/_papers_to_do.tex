\documentclass[]{article}
\usepackage{tikz}
\usetikzlibrary{shadings, shadows, shapes, arrows, calc, positioning, shapes.geometric}
\usepackage{pgfplots}
\pgfplotsset{compat=1.16}
\usepackage{mathtools,amssymb}
%\input{Preamble}
%\input{(SpecialDistanceCourse}
%\input{FRAMES}
\begin{document} 
\begin{enumerate}
    \item Climate finance 
    \item Who will own  the city City  productive. Ownership is shifting.
    \item {\color{red!30}Financialization of the urban system; Three layers - Alozo-Jacobs, housing markets, and the end of an era? }
Lots of version here
    \item Modern Urban Rent Theory: 
    \item Spatial rent theory - Ricardo and the city
    \item {\color{red!30} Ubiquitous rent theory: it is everywhere. How political economy was hidden. mainstream ignores it while using it and this connects to Piketty}
    \item Urban Rent, ownership and Class 
    \item Ricardo, Rent and Roemer
    \item Growth theory and the city
    \item Growth theory and urban scaling
    \item DRS firms and IRS cities: one resolution.
So this refers to the difference between the firm level and the city level in decreasing returns at the firm level is a standard assumption although economist spent a lot of energy on increasing returns at the oil that might arise from networks or from consumer. Getting this clear is important it's not obvious to me that that economist in general think about the agglomeration affect the network at the city level that makes productivity and that's something to discuss I think to bring to the four because it is at a public productivity publicly produced productivity rather than simply him how would I put it some sort of private asset that should be probably should give rise to private income. So that what I meant by one resolution is showing off how you've done it in your model, and it wouldn't be terribly hard to introduce at two or three different ways. Those are simple additions to the modeling.
    \item Jacobs' city and neoclassical rent theory. This is this is the Alonso model that makes classical rent theory into accepted neo-classical rent theory.
    \item Financialization vs urban productivity: is there any evidence?
    \item A simple model of the distribution of urban rents urban 
    \item From one class of urban worker to two.
    \item Family and space in an urban model
    \item Equilibrium modelling for ABM modellers 
    \item Lots about lags and policies
    \item Overlapping generations in an urban ABM model 
    \item The Alonzo-Jacobs City - and simple model
    \item Do details of the production sector affect the city?
    \item The mortgage constraint
    \item Ranking Policy instruments that affect class structure
    \item Ranking Policy instruments that affect city size 
    \item Capital gains taxation and home-ownership - results from an ABM
    \item Capital gains taxation and urban growth - results from an ABM
    \item Density and housing prices
    \item Transportation cost and urban productivity
    \item The economics of urban amenities: Five comments
    \item More people or more money - lags and home building 
    \item Human wealth, urban wealth, and GDP
    \item Open cities and 
    \item Inter- and Intra-city transportation and housing costs
    \item is there a density-amenity frontier?
    \item Add a data set, do an optimization
    \item Book on  Modelling of Urban System with Chapters:
    \begin{enumerate}
        \item The productivity of cities - lit review 
        \item  Modelling the agglomeration effect 5 ways
        \item  Rent - with different structures.
        \item  Whose getting the rent with the markup model.
        \item  Can  you do anything about it, rules for policy/intervention
    \end{enumerate}  

    \item 
    \end{enumerate}  
\end{document}