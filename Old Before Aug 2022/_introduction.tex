\section{Introduction}
\label{Sec:Introduction}
%\ref{Sec:Introduction})


\note{This thesis examines the complex system consisting of  housing markets, institutional investment, production, and the wealth of households.}

\note{To explore this system, we build an agent based model of a housing market, integrated with a model of a production and employment. We then adapt analysis tools from the study of resilience to analyze the transitions between alternative regimes, or patterns of functioning, of the system. We apply  an emerging formulation of resilience as a systems property, and associated systems dynamics analysis techniques. A unique feature of this thesis is that we examine the wealth trajectories of individual households within a spatially explicit urban model.} % that includes both a housing market and a model of production and employment. 

In ``Order without Design,'' Bertaud makes the case that cities are primarily labour markets' because labour markets drive the productivity of cities. 
%the reason labour markets are important is because they really account for the productivity of cities
%Yet labour markets are not well integrated into urban models. 
%He argues further that economic analysis more generally has not been adequately integrated into the work of planners and so cities are often planned in ways that fail to take into account the underlying economics.
The literature makes it clear that there are strong and  pervasive agglomeration effects driving productivity growth in cities (CITE). %and population growth. (City population is observed to follow a power law distribution.) 
These agglomeration effects interact with the cost of transportation, speculative real estate investment, and the cost of housing. 

The notion that the labour force is on average more productive when there are more people around is pretty dramatic and is not part of the basic models of production or urbanization. Our starting point is that's the fundamental feature of cities. 
We integrate that dynamic within a model that brings together a model of a productive urban economy with agglomeration in an agent based land market model to explore the effect of financialized capital.\note{Maybe say "with ongoing financialization of land markets"? }
What does that do with financial capital and what does that do to distribution and that's not been explored.
\note{Replace with:?  The impact of financialization on the distribution of wealth has not been explored in urban models.  ?}
It's not enough to understand these forces individually.
%Existing models don't capture the feedbacks between financialized investment, housing markets, and labour markets. 
The future of cities and the productivity of human capital depends on how they interact to drive the growth of wealth and amenity in cities. 

%In this work, we model how land rent is captured by landowners and how that affects wealth creation and the development of the city, using an agent based model to simulate key elements of a housing market. 

%Housing market consists of these things, but this model adds an additional layer that is usually modelled separately: a growth model that models the growth or shrinking of the city. 
%Modelling a housing market, a production model, and speculative investment together lets you see the feedback and interaction and they way that they feed into each other. That feedback relationship create rich resilience dynamics within the modelled economy. 

%Integrating the classical and neoclassical  distributional stories, we are able to look at distributional effects and the way they feed back into productivity. The agent based spatially explicit model of distribution with both financialized capital, an agent based housing market, and a model of urban productivity makes it possible to look at how people are affected: who gets to live in the city, who contributes, and who benefits? (Need to el)
%who gets the rents/surplus what does that mean for the class structure, and ultimately the productivity of cities

% More compressed and technical
% We couple two simple, standard models. The first is a Cobb-Douglas production sector, the second a basic Alonzo style urban system with transportation costs. Coupling the two allows us to illustrate  in a simple way the endogenous distribution of wealth and to describe the endogenous development of a class system in an urban economy.  The model is constructed so that there is neither land rent nor capitalist exploitation in the rural economy. This simplification allows us to focus on the distribution of the social surplus generated by agglomeration economies. 

The next chapter provides the theoretical background for the model, looking the scaling of wealth, rent, production, and the city. 
It outlines the history of rent and distribution and brings together these two distributional stories in one model. \note{Land rent has been largely ignored in recent distributional analysis despite its centrality in urban models. (instead of  :Rent was lost from the distributional story."?} One contribution of this thesis is integrating these two distributional stories to provide a more nuanced analysis of the challenges of urbanization, the nature of social wealth and how rent interactions with competitive markets. %Classical economics
The following chapter introduces the model of production, connects it with the urban scaling literature and develops a simple formulation for use in studying the interaction between financialization and production in an urban agent based model.
%As in the standard circular city model the constraint on growth is provided by transportation costs, which limits the size of the commuter-shed and therefore the labour force at any wage.
%Individual firms have decreasing returns, but the presence of agglomeration economies external to firms but internal to the city gives the urban economy as a whole increasing returns to scale. Excess return for urban firms drive both firm expansion and firm entry. The result is continuous growth of the urban economy. % What are the regimes in which the economy grows or does not
The chapter after that introduces the agent based land market model with renters, buyers, financialized speculative buyers and a productive sector. 
%We integrate this with a housing market model ELABORATE, and do resilience analysis.
The following chapters examine resilience and distributional effects in the context of the model. % provides a resilience based analysis of the model. 
%We then look at the resilience effects with 3 interacting layers of hysteresis 1. the productive capacity of the city 2. speculative rent seeking investment and 3. built form, increasing density as city grows. 

We formulate a measure of systems functioning, map the boundaries of the basins of attraction in the model, given hysteresis, and explore the effects of different classes of interventions. %  for shaping the pattern of functioning in the model.. tenure etc.

We argue that there exist regimes in which housing tenure acts like a peristaltic pump, pumping wealth out of communities in both economic boom and bust cycles. There is also a distinct regime in which housing tenure plays a role in building local wealth that can act as a buffer against rises and falls in the larger economic system. Ultimately, we hope our work will provide a theoretical foundation for those %, like our partners at CMHC, 
developing policy for affordable housing that centre household wealth. 

% Also It contributes to one set of tools for analyzing agent based models that is suited to the study of social innovation
%There is a distinction between use value and investment value. There is an argument between people who argue for using prices as a mechanism to grow the housing supply and those who push for policy priority should be centring the use value so people can live the city. - the rights, the minimum standards. We argue instead that (price is useful, there is real value-- at the same time the wealth of the city is a social wealth created by those who come to the city - who subsidize those who extract)
