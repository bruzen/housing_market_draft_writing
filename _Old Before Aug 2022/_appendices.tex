\addcontentsline{toc}{chapter}{APPENDICES}
\chapter*{Appendix: Notation}
%\newpage

%omega is the slope of the budget life, a calculation variable, the ratio of capital costs to labour costs.

\begin{center}
\begin{longtable}{lp{10cm}}
\caption{Notation}\\\hline
		&\textbf{Productivity}\\ \hline
$K$  &  Capital\\
$n_i$  &  Number of workers employed by firm $i$\\
$n=\sum_i n_i$  &  Labour (number of workers)\\
%$f$  &  Number of firms\\
%$n =f n_i$  &  Aggregate labour \\
$\Lambda(n)$  &  Labour-augmenting agglomeration effect \\
% $n^\gamma$ & The labour-augmenting agglomeration effect,  modelled as an expontial function of the number of people \\
$\Lambda(n)n$  &  Effective labour \\
%$\Lambda'=\die{\Lambda(n)}{n} $ & Derivative of the labour-augmenting agglomeration effect\\
$\alpha$  &  Elasticity of output with respect to capital\\
$\beta$  &  Elasticity of output with respect to effective labour\\
$\gamma$  &  Elasticity of $\Lambda(n)$, labour-augmenting agglomeration\\
$Y=K^{\alpha }(\Lambda(n)n)^{\beta }$  &  Aggregate output of all firms in the city\\
$Y_i=K_i^{\alpha }(\Lambda(n)n_i)^{\beta }$  &  Urban firm $i$'s output\\
$\die{Y_i}{K_i}	=\alpha \frac{1}{K_i} Y_i $  & Marginal product of capital for firm $i$
\\
$\die{Y_i}{n_i}	=  \beta\frac{1}{n_i} Y_i $  &  
Marginal product of labour for firm $i$\\
$\die{Y}{n}=\beta\frac{1}{n_i} Y_i  \left( 1+ \frac{n\Lambda'}{\Lambda} \right) $  &  Social marginal product of labour\\
$\eta=\frac{n\Lambda'}{\Lambda}$  &   Marginal agglomeration effect on aggregate labour productivity\\
\hline
	&\textbf{Amenity}\\ \hline
$A(d, n)$   &  Agglomeration amenity\\
\hline
		& \textbf{Prices}\\ \hline
$\psi$  &  Rural wage\\
$\psi$  &  Per-period cost of a unit of productive capital\\
$w$  &  Urban wage premium\\
$\psi + w$  &  Urban wage\\
$\Omega=\frac{w+\psi}{\psi}$  &  Ratio of the urban wage to the  cost of capital\\
\hline
		&\textbf{Spatial structure in the circular city}\\ \hline		
$d$  &  Distance of a residence from the centre of the city\\
$\tau$  &  Linear transportation cost per unit distance\\
$c^{max} = w/\tau$  &  Maximum distance commuters will travel% for wage $w$
\\
$s$ & Lot size\\
$U$  &  Worker utility, a function of location and prices\\
$U^{urban}=U^{rural} $  &   Migration equilibrium assumption\\
\hline
		& \textbf{Labour market}\\ \hline
$L= \frac{\pi}{s}(\frac{w}{\tau})^2 = n$  &  
Labour supply, the number of workers, which equals the number of lots in the standard circular city model since workers live on identical individual lots\\
\end{longtable}  \end{center}



\chapter*{Appendix: Excess Return on Urban Capital}\label{Sec:ExcessProfit}

% At this point we may consider a contradiction. 
There is a problem. How to calculate how firms will respond?
We can calculate the labour-capital ratio and aggregate output based on the assumption that firms assume that the agglomeration effect in their production function is fixed. This assumption is reasonable. It seems likely that the agglomeration effect of adding a worker would be difficult to attribute to a new worker: it would be a lagged, unevenly distributed effect, and would appear exogenous\footnote{It is possible to interpret in other ways. Perhaps firms linearly extrapolate rewards. See xx for a review of approaches to modelling firm behaviour.}. 
If it is reasonable to assume that firms would not take the agglomeration term into account and the effect is thus not be compensated. This gives *** TODO labour-capital ratio and aggregate output
%The effect of adding a worker $Y_i \frac{\Lambda'}{\Lambda}$
%Assuming firms do not take the adjustment of the agglomeration term into account. 


Assuming firms take the wage as parametric and hire until the private myopic marginal product falls to the observed wage, % *** What does parametric mean here?

		\[  \beta\frac{1}{n_i} Y_i =    w+\phi , \]
%implying firms will hire fewer than the optimal number of workers.

Since urban firms pay each worker the private marginal product but gain from each worker the effective marginal product,   plus a share $n_i/n$ off the extra social marginal product, $\eta=\frac{n\Lambda'}{\Lambda}$.  There is a potential  inefficiency that arises from a failure to take into account the externality. 
%$The answer may depend on whether there is firm entry   !!!!

% *** WHAT IS THE UNDOING EFFECT? WHAT DO WE MEAN BY SYNCHRONIZED
%Will this information lead to undoing the effect?  - I don't think so - if hiring is NOT synchronized, breaking the link with own hiring future productivity would be tied only to expected population growth which will appear exogenous.
%

\begin{align} 
\die{Y}{n}=&\beta K^{\alpha }(\Lambda(n)n)^{\beta-1 }(T'n+T )  \nonumber \\
		=&\beta  \frac{K^{\alpha} (\Lambda(n)n)^{\beta}}  {\Lambda(n)n} (\Lambda'n+T)  \nonumber \\
		=&\beta  \frac{K^{\alpha} (T(n)n)^{\beta}}  {\Lambda(n)n} (\Lambda'n+\Lambda)  \nonumber \\
	Y_n	=&\beta \left( \frac{1}{n}+ \frac{\Lambda'}{\Lambda(n)} \right)	Y
\end{align}

Assuming all firms behave the same way, the social marginal product, taking into account the agglomeration effect is  

\[ \frac{n}{n_i} \die{Y_i}{n_i} \]


 marginal product of capital
\begin{align}
Y=&K^{\alpha }(\Lambda(n)n)^{\beta}   \nonumber  \\
\die{Y}{K}=&\alpha K^{1-\alpha }(\Lambda(n)n)^{\beta }  \nonumber \\
		=&\alpha \frac{K^{\alpha}(\Lambda(n)n)^{\beta }}{K}  \nonumber \\
	Y_K	=&\alpha\frac{1}{K} Y  \label{EQ:mpk}		
\end{align}
The aggregate marginal product of capital for the firm does not differ from the private marginal product of capital.


\hrule 

The marginal product of labour, however,  is complicated by the presence of the  agglomeration effect , $\Lambda(\sum n_j)$:
\begin{eqnarray} \die{Y_i}{n_i} &=   K_i^\alpha \left(  \beta\Lambda n_i^{\beta-1}  + n_i^{\beta}\Lambda'  \right)  \nonumber\\
&=   \beta\frac{1}{n_i} Y_i   +  Y_i \frac{\Lambda'}{\Lambda}  \nonumber\\
&=   Y_i  \left[  \beta\frac{1}{n_i}  + \frac{\Lambda' }{\Lambda} \right]   \label{Eqn:MPL}
\end{eqnarray}
The first term in the square bracket in Equation~\ref{Eqn:MPL} might be termed the \textbf{private myopic marginal product}. It is the addition to output directly attributable to an additional worker. This myopic MPL is smaller than the actual marginal product for the firm.  
The second term in the bracket   is the marginal agglomeration effect. It captures the effect on firm-wide labour productivity -- an increase in effective labour --   that results from adding a worker. We can expect this to be very small.\footnote{Using the specification in Footnote~\ref{Fn:PSI}, $Lambda(n)=n^\psi$, it would be $\frac{\psi}{n}$. If f $\psi=0.1$ and $n=250,000$ the number is $\frac{\Lambda' }{\Lambda} =4\times 10^{-7}$.}% REVIEW THIS 
% REVIEW THIS TOO
%\footnote{Firm size matters. 
%A monopoly employer would take into account the marginal agglomeration effect. 
%A firm that employed a large enough fraction of the urban workforce to notice agglomeration effects, say $\frac{1}{n}<k<1$, would enjoy that fraction of the external effect and set the wage at  
%\[w_k =  \beta\left( \frac{1}{n}+ \frac{k\Lambda'}{\Lambda(n)} \right)	Y\].} 

\subsubsection*{The Size of the Excess Return}
The surplus return to a firm's capital from adding one worker to the firm is the value of the marginal product of the increase in effective labour net of the additional wage payment.  Since workers are paid their myopic marginal product, the unexpected  excess return (ER) is 
\begin{eqnarray}
%\die{Y_{i}^a}{n_i} -(w +\psi) =& \frac{\beta}{n_i}  Y_i  \left[  \beta\frac{1}{n_i}  \\%+  \frac{\Lambda' }{\Lambda} Y_i    \beta\frac{1}{n_i} } \\
				ER	&=&   Y_i  \left[  \beta\frac{1}{n_i}  +  \frac{\Lambda' }{\Lambda} \right]   -Y_i  \left[  \beta\frac{1}{n_i} \right]\\
					&=&  Y_i   \frac{\Lambda' }{\Lambda}\\
\end{eqnarray}	
%The aggregate excess is simply $Y \frac{\Lambda' }{\Lambda}>0$. 
As a result, the urban economy should experience both firm expansion and firm entry. %This may be consistent with empirical research by Brown and Rigby (2013) showing that smaller and younger firms experience stronger productivity gains stemming from the localized pooling of workers with skills that match their needs and from knowledge spillovers than do larger firms. 

%\[w^t \Rightarrow n^t, k^t \Rightarrow  \pi^t>0  \Rightarrow w^{t+1} >w^t,  \#f^{t+1}> \#f^t \]
%A unit of urban capital therefor earns its own marginal product, $Y_K= \alpha Y/K$ 

The rate of excess return is the excess return divided by the size of the firm's capital stock, $K_i$. We can find the aggregate capital stock on the assumption firms maximize profits, which leads them to (myopically) set 


\begin{eqnarray}
\frac{ \Large{\die{Y_i}{n_i} }} { \Large \die{Y_i}{K_i} }&=\frac{ w+\psi }{ r }\nonumber \\
%	\frac{\cfrac{\partial Y _i}{\partial K_i}}{\partial K_i}=&\\%
%\frac{\frac{\beta}{n_i} Y_i }{ \frac{\alpha}{K_i} Y_i}&=\Omega\\%\frac{w+\psi}{\psi}\\
\frac{\beta}{\alpha}\frac{K_i}{n_i} &=\Omega \nonumber
\end{eqnarray}
so we can express the individual firm capital capital stock desired 
\begin{equation}
K_i =\frac{\alpha}{\beta}\Omega n_i\\
\end{equation}
Which in increasing in $n_i$.
 
%%%%%%%%%%%%%%%%%%%%%%%%%%%%%%%%%
%\textbf{\color{green}]Hidden section: the rate of excess return is always positive. }
so the rate of excess return is 
\begin{eqnarray}
\frac{ER}{K_i} &=\frac{Y_i   \frac{\Lambda' }{\Lambda}}  {\frac{\alpha}{\beta}\Omega n_i }   \nonumber\\
&=\frac{ 1}  {\alpha\Omega }\frac{\Lambda' }{\Lambda}\frac{\beta}{n_i }Y_i    \nonumber \\
&=\frac{ r}  {\alpha(w+\psi) }\frac{\Lambda' }{\Lambda}\frac{\beta}{n_i }Y_i 
\end{eqnarray}
which is always positive. %The last two terms in the expression are the myopic marginal product of labour.

 %\textbf{\color{green}Hidden section: wage bill is increasing with the cube of the wage}
% USEFUL   Notice that the wage bill is increasing with the cube of the wage, which could  be a powerful driver of demand for services and commercial and hence for urban commercial development. We will reexamine this observation below.

%We assume that demand for the urban product is perfectly elastic and the price per unit is 1. 




%
%We know the population, $n$, is simply the area of the city divided by the lot size, and the area is a function of the wage premium $w$.   The wage bill is 
% \[W	= (\psi + w)\frac{\pi }{s}  \left(\frac{w}{\tau}\right)^2\]
%
%
%
%With full employment, aggregate   demand for labour $\sum_i n_i,$  is equal to supply from  Equation~\ref{supply}, which gives us a way to calculate the aggregate capital desired
%
%\begin{eqnarray}
%\mathsmaller{\sum_i} K_i &=\frac{\alpha}{\beta}\Omega \sum_i n_i,\\ %note I have added package relsize 
%K &=\frac{\alpha}{\beta}\Omega n\\
% &=\frac{\alpha}{\beta}\Omega L,
%\end{eqnarray}

%
%Firm $i$'s output is 
%\begin{eqnarray} 
%Y_i	=&   \left( \frac{\alpha}{\beta}\Omega n_i \right)^\alpha  \left(\Lambda(n) n_i \right)^\beta  \nonumber\\
%	=& \left( \frac{\alpha}{\beta}\Omega\right)^\alpha n_i ^\alpha \Lambda(n)^\beta n_i ^\beta \nonumber \\
%	=& \left( \frac{\alpha}{\beta}\Omega\right)^\alpha \Lambda(n)^\beta n_i ^{\alpha + \beta}
%\end{eqnarray} 
%
%Firm profit is then
%
%\begin{eqnarray} 
% \Pi 	=&p \left( \frac{\alpha}{\beta}\Omega\right)^\alpha \Lambda(n)^\beta n_i ^{\alpha + \beta}
% -\Omega n_i -\psi \frac{\alpha}{\beta}\Omega n_i    \nonumber\\
%	=& p\left( \frac{\alpha}{\beta}\Omega\right)^\alpha \Lambda(n)^\beta n_i ^{\alpha + \beta} -\left(\Omega + \psi \frac{\alpha}{\beta}\Omega\right) n_i  \\
%	=& p\left( \frac{\alpha}{\beta}\Omega\right)^\alpha \Lambda(n)^\beta n_i ^{\alpha + \beta} -\left(1 + \psi \frac{\alpha}{\beta}\right) \Omega n_i  
%\end{eqnarray} 



% \textbf{\color{green}Hidden section: crosspartial  $\die{\frac{\alpha Y}{K}}{n}$}
%\begin{align} 
%\die{\frac{\alpha Y}{K}}{n} 	=&	\frac{\alpha}{K} \die{Y}{n}  \nonumber  \\
%					=&	\frac{\alpha}{K} \beta\left( \frac{1}{n}+ \frac{\Lambda'}{\Lambda(n)} \right)	Y \nonumber \\
%					=&\alpha(1-\alpha)\frac{K^{\alpha}(A(n)n)^{1-\alpha }  {KA(n)n)}(A'n+A) \nonumber \\
%					=&\alpha\beta\frac{Y}{K}\left( \frac{1}{n}+ \frac{\Lambda'}{\Lambda(n)} \right) \nonumber \\
%\end{align}

\subsubsection*{Method 2: Zero Conjectural Variation}

? Another way to consider how firms will respond is to assume they xyz- 0 conjectural variation. This is different from the above in that xyz....

Even taking into account firm-wide labor productivity gains, the private marginal product of labour is likely to be computed on the assumption that other firms do not expand  their workforce. This could be call the `zero conjectural variation' (0CV) case. If all firms  were expected to respond to the same signal the same way (1CV), the agglomeration effect  would be multiplied by the number of firms, $\frac{n }{n_i}$ . The marginal agglomeration effect for firm $i$ is then $\frac{n }{n_i}\frac{\Lambda' }{\Lambda}Y_i =\frac{\Lambda' }{\Lambda}Y$. 

Since the increase in $\Lambda$ affects all firms, the  social marginal product of all firms expanding their  workforces by one worker as $i$ does must be multiplied again by the number of firms. The value of the agglomeration effect  becomes large with a large number of firms. %(agents?) 


%\begin{figure}[tb]
%\begin{center}
%\input{SA_AmenityFigure.tex}
%\caption{Incorporating an consumption amenity}
%\label{default}
%\end{center}
%\end{figure}


\begin{eqnarray}
\die{Y}{n_i}=&   \beta\frac{1}{n_i}Y_i    + \frac{\Lambda' }{\Lambda}\sum_j Y_j\frac{n }{n_i} \\
%
		=&   \beta\frac{1}{n_i}Y_i    + \frac{\Lambda' }{\Lambda}Y\\
\end{eqnarray}		
 where $\frac{n }{n_i}$ is the number of firms. 




\subsubsection*{Underfunded Public Amenity}

The firm's assumption is incorrect however: the agglomeration effect is not fixed. It increases with the growing urban population, as more firms hire and attract more people. The firm thus experiences greater productivity growth than expected.% because it does not account for the agglomeration effects. 
Which means the firm's realized output, revenue, and profit are higher than expected. % for each firm, is higher than planned output, as are revenue and profits. %If urban firms capturing 
%If firms choose labour and capital to maximize profit given assumptions about the production function, and then enjoy unexpected high productivity productivity increases, which implies that the firm will find realized profit higher than expected  and, presumably, adjust its estimate of the marginal product of labour upward, and therefore demand more labour.

This excess return has the potential to drive firm expansion, firm entry, and the continual growth of the urban economy. %city. The result should be continuous growth of the urban economy. %(Will that growth converge or will it be unbounded?)

% because of the agglomeration effect, effective labour productivity will be greater than expected.

%We have calculated the labour-capital ratio and aggregate output based on the assumption that firms incorrectly assume that the agglomeration effect in their  production function is fixed. %This must hold in an equilibrium. 

In both scenarios agglomeration generates a public good that is underfunded privately.
The agglomeration effect is a public good in which individual firms under-invest. This raises a potential policy challenge. % that we leave for others.

%? a differential equation or difference equation in which $w_t \rightarrow n_{t+1} \rightarrow w_{t+1}$. How to write this? Do we  need to?)

These observations about a rising the rate of return on urban capital raise a set of very interesting issues.   
\begin{enumerate}
\item The effect of increasing the urban workforce is to increase the marginal products of  both capital and labour throughout the urban economy. This premium on production in cities is positive when $\Lambda'$ is greater than zero.
 \item  As long as $\Lambda'>0$ cities will  grow, potentially leading to the ``catastrophic agglomeration''  that has been noted in other models.  (\cite{FujitaKrugmanVenables, BaldwinMartin, Krugman1991, Gurwitz2019}).  City size could be bounded, in which case the city system would grow by increasing the number of cities. This appears to depend on the model features and parameters. 
 \item Even in competitive markets, until the wage rises, capital is capturing all of the socially produced benefit of agglomeration. (We will see below that the rest is captured by landowners.)
%\item it brings into question the standard assumption of an equilibrium  rate of return on capital.  
%\item It appears firms will capture all of the socially produced marginal product of agglomeration.

\item Cites will draw investment away from rural and remote areas.

\item Cites will draw labour away from rural and remote areas.
\end{enumerate}

These are features of the economy we observe.

\chapter*{Appendix: Amenity, Income, and Land Rent}

%Non monetary income well established for business literature, now extend to the city.
%Rent in the city can come from two places:
%1. generated through production function
%2. rent generated through some locally produced amenity. 
%There are papers that look at the Henry George rule for each case. The standard examination works on investment in amenity rather than production--public investment in either productivity or amenity.

If a city offers amenities, in addition to employment opportunities, their value appears in the rent profile. A simple way to incorporate agglomeration amenities is to include what might be called a `utility premium' for urban dwellers as non-monetary location income $A(d,n)$ where $A$ is amenity also located at the centre initially for convenience, $d$ is the distance from the amenity, and $n$ is the urban population. 

Amenity declines moving away from the centre and increases with the population so $\die{A}{d}\leq 0,\die{A}{n}\geq 0$. The logic for all kinds of urban amenities, from hospitals, lessons and research institutes, to clubs,  is similar to that for productive agglomeration. For example, with a hospital, in an urban centre, more people share the resource but a larger city can invest more so the hospital can purchase more equipment and offer more treatments. %Costs per person go down since it is shared. 
Specialty services, that might largely empty in a small hospital, get fuller utilization so the cost of operation can be lower. This allows more and cheaper services. This is compounded by the specialization of labour and the network of experts available nearby. There are transportation benefits if buses run to the centre and utilities scale sub linearly.

These amenities offer value that feeds directly into the rent profile. Those with the means are willing to pay more to live near amenities, 


 

Those which are available only to those very close. Some are free, some require fees. 


Amenities will affect peoples behaviour and be reflected in the rent profile. No amenities, it will be the wage, rental cost and transportation.
Throw in amenities. 



If $A=0$ %and $T=td$, \label{Eqn:U} 
 this model produces the standard result in the Alonzo model: land rent declines linearly with distance. 
 
If $A(0)>0$, willingness to pay and therefore the competitive rent profile is higher than the if there is only a wage premium.  
Migrants to the city will be a self-selected subset of all workers for whom the urban amenity is a substitute for something in the standard bundle purchased out or $\psi$. 
The rent profile is  
\begin{equation}
Rent(d)  =  w  + A(d) - \tau(d)	
\label{Eqn:Rent-at-d}\end{equation}
If we solve $w+A(d)-\tau d=0$ for $d$, we get the maximum  distance at which living near the city offers an advantage, $d^{max}$. 
For example, if  $A(d)=a_0 - b*d$ and transportation costs are  linear function as above

\begin{eqnarray}
w+ a_0 - b*d^{max} - \tau*d^{max}  	&=0		\nonumber \\
w+a_0 - (b+\tau)d^{max}  	&=0			\nonumber \\
d^{max}				&= \frac{w+a_0}{b+\tau}
\end{eqnarray}
For the case of a circular city, the area within this radius is $\pi \left(\frac{w+a_0}{b+t}\right)^2$. 
If the lot size is $s$ as before, we have the aggregate population for the city. 
 \[P= \frac{\pi}{s}  \left(\frac{w+a_0}{b+\tau}\right)^2,\]
Which is larger than the number of commuters if $a_0/b>w/\tau$% Commuters will only travel to work if the wage premium minus travel costs is positive. 
 To get total land rent with a consumption amenity, we integrate:

\begin{eqnarray}   %Total rent
R&=&  \int_0^{c^{max}} ( w-\tau d) \frac{2\pi d}{s}dd 	
	+ 	\int_0^{d^{max}} (a_0- bd	) \frac{2\pi d}{s}dd 	\nonumber \\	
	&=& \frac{\pi}{3}\left(\frac{w}{\tau}\right)^2w
	+	\frac{\pi}{3}\left(\frac{A_0}{b}\right)^2w
%	\nonumber \\
\end{eqnarray}




%The first term yields
%\begin{eqnarray}
%	&=&\frac{2\pi }{s} \int_0^{d^{max}} \left( ( w+a_0)d - (b+t)d^2\right)  dd  	 \nonumber \\
%	&=& \frac{2\pi }{s} \left[ \frac{1}{2}(w+a_0)(d^{max})^2 - \frac{1}{3}(b+t)(d^{max})^3\right]  
%\nonumber \\
%	&=& \frac{2\pi }{s} \left[ \frac{1}{2}(w+a_0) - \frac{1}{3}(b+t)d^{max}\right](d^{max})^2  \end{eqnarray}
%					&= &   2\pi\left[(w+a_0)d^{max} - \frac{1}{2}(b+t)d^{max}d^{max}\right]\\
%					&=&    2\pi\left[(w+a_0)d^{max} - \frac{1}{2}(w+a_0)d^{max}\right]\\
\begin{eqnarray}
					&= &  2\pi\frac{1}{2}(w+a_0)d^{max}\\
					&= &  \pi(w+a_0)d^{max}
\end{eqnarray}

\begin{eqnarray}Rent	&=&  \int_0^{d^{max}}( w+a_0 - (b+t)d) dd\\  % linear model
					&=&  (w+a_0)d^{max} - \frac{1}{2}(b+t)d^{max^2} \\
					&= &  (w+a_0)d^{max} - \frac{1}{2}(b+t)d^{max}d^{max}\\
					&=&   (w+a_0)d^{max} - \frac{1}{2}(w+a_0)d^{max}\\
					&= &  \frac{1}{2}(w+a_0)d^{max}
\end{eqnarray}

WHAT DOES THIS MEAN?
NOTE: a simpler version  with a two-sided linear city $\kappa$ lots wide and  one unit in depth is more ``intuitive'':
\[R	= \frac{\kappa}{2} \left[\frac{w^2}{\tau} +  \frac{a_0^2}{b}\right] \]
and the wage bill is 
\[W	=2 \kappa \frac{w^2}{\tau} \]
In this model, land rent absorbs roughly half of the aggregate wage premium,  falling to exactly half as the amenity goes to zero. 
The rest of the wage premium is spent on transportation. 
The presence of amenities in general increases the aggregate urban rent.

\chapter*{Appendix: A More Detailed Discussion of Financialized Returns}

To explore the implications  of the financialization of  the urban land market we need a function to calculate the return on a unit of land, h, which that reflects the actual gradient of opportunity in financial markets. We begin with the price appreciation, $\Delta P=P_T-P_0$. Transaction costs including real estate fees, take a fraction from the value of the final sale, leaving $\phi^sP_Th$ where $\phi^s<1$ is the share remaining for the seller. Moving costs, legal fees and land registration costs add to the price of purchase, so that the full cost of purchase is $\phi^bP_0h$, where $\phi^b>1$

%\input{SA_SpeculativeMotive.tex}
%%%%%%%%  VVVVVVVVVVVVVVVVVVVVVVV   This section May 18 to cut?  V

The net gain from appreciation with transaction costs is something like \[G^N=\phi^s P_T-\delta^T\phi^b P_0=P_0(\phi^s \frac{P_T}{P_0}-\delta^T\phi^b)\]%=P_0\psi(T)\]
where $\delta^T=\frac{1}{(1+r_i)^T}$ is the discount factor for $T$ periods. The expression $(\phi^s \frac{P_T}{P_0}-\delta^T\phi^b)=C^T$ summarizes transaction costs. $\phi^s$ and $\phi^b$ represent  fixed costs that  encourage longer ownership terms, decrease mobility and make the use of the housing stock less efficient. 

If  owners take on the maximum mortgage, $M=mP_0h$ and pay interest at rate $r_i$ %we can set the  un-discounted carrying cost of ownership is simply\footnote{
and do not pay down the mortgage, the value of the infinite sum of payments $ r_iP_0$  minus the discounted long tail of payments after T is 
\[C^C= r_iM\left(\frac{1}{r_i} - \frac{1}{r_i(1+r_i)^T}\right)= mP_0\left(1- \frac{1}{(1+r_i)^T}\right) \]%=\Gamma(r,T)\]
%\[C^O= rTP_0\]$
The required downpayment is $D=(1-m)P_0$. The buyer must forego a return of $r_i^d$ and will generally encounter some transaction cost $\phi^D$ as well. As a fraction of downpayment this is  $\phi^d=\frac{\phi^D}{(1-m)P_0}$.   so this amounts to

 \[C^D=  r_i^d(1-m)P_0\left(1- \frac{1}{(1+r_i)^T}  \right)  + \phi^D  \]

The return on the investment net of carrying costs and downpayment is 

\[R=G^N-C^C- C^D  \]
Writing $\frac{P_T}{P_0}= 1+r^p$, where $r^p$ is the compounding rate of growth of housing prices over the period $0-T$,  
%&=&P_0h\left(C^T-\right)\\
\begin{eqnarray}
%R&=&P_0\left[\left(  \phi^s (1+r^p) - \delta^T\phi^b\right)  -  m\left(1- \delta^T\right)  \right]  %add D here
%- r_i^d(1-m)P_0\left(1- \delta^T\right)   + \phi^D \nonumber \\
%%  simplify last term
%&=&P_0\left[\left(\phi^s  (1+r^p) - \delta^T\phi^b\right)  -  m\left(1- \delta^T\right) \right]  %add D here
%- P_0r_i^d\left(1-m) (1-\delta^T)\right )  + \phi^D   \nonumber \\
%%  combine terms
%&=&P_0 \left[\left(\phi^s  (1+r^p) - \delta^T\phi^b\right)  -  m\left(1- \delta^T\right)  %add D here
%-  r_i^d(1-m) (1-\delta^T ) \right]    + \phi^D   \nonumber \\
%%  simplify last term again
&=&P_0 \left[\left(\phi^s  (1+r^p) - \delta^T\phi^b\right)  -  m(1- \delta^T)  %add D here
-  r_i^d(1-\delta^T) +r_i^d m(1-\delta^T)  \right]    + \phi^D   \nonumber \\
%% combine terms in m
&=&P_0 \left[\left(\phi^s  (1+r^p) - \delta^T\phi^b\right)  -  m(1- \delta^T)+r_i^d\left(m(1-\delta^T) \right)  %add D here
-  r_i^d(1-\delta^T)  \right]    + \phi^D   \nonumber \\
% combine terms in m
&=&P_0 \left[\left(\phi^s  (1+r^p) - \delta^T\phi^b\right)  - (1+r_i^d) m(1- \delta^T)  %add D here
-  r_i^d(1-\delta^T)  \right]    + \phi^D   \nonumber \\
%
%&=&P_0 \left[     \phi^s \frac{P_T}{P_0}-\delta^T\phi^b      -m\  - \delta^Tm %add D here
 %\right] \nonumber %\\
%\frac{R}{P_0}&=&\left[     \phi^s \frac{P_T}{P_0}-\delta^T\phi^b      -m\   \delta^Tm \right] 
\end{eqnarray}
%
As a fraction of the downpayment  $\phi^d=\frac{\phi^D}{(1-m)P_0}$, so $P_0(1-m)\phi^d=\phi^D$
%
\[R=P_0 \left[\left(\phi^s  (1+r^p) - \delta^T\phi^b\right)  - (1+r_i^d) m(1- \delta^T)  %add D here
-  r_i^d(1-\delta^T)  + (1- m)\phi^d \right]  \]
%
The \textbf{effective rate of  return on the housing investment}, $r^h=\frac{R}{(1-m)P_0}$ satisfies 
\[ (1+r^h)^T =  \frac{1}{1-m} \left[\left(\phi^s  (1+r^p) - \delta^T\phi^b\right)  - (1+r_i^d) m(1- \delta^T)  
-  r_i^d(1-\delta^T)  + (1- m)\phi^d \right]  \]
%or
\[r^h   = \left(\frac{1}{1-m}\right)^{1/T} \left[\left(\phi^s  (1+r^p) - \delta^T\phi^b\right)  - (1+r_i^d) m(1- \delta^T)  -  r_i^d(1-\delta^T)  + (1- m)\phi^d \right] ^{1/T} -1  
        \]

and housing is a good financial investment if $r_i<r^h$,, i.e., if

%%%%%%%%%%%%%%   STOP HERE
\[r_i< \left[ \left(\phi^s  (1+r^p)-\frac{1}{(1+r_i)^T}\phi^b\right)      -m\left(1- \delta^T\right) \right]^{1/T}-1\]

%When will it be the case that 
%\[r<\left[\phi^s \frac{P_T}{P_0}-\phi^b - \frac{1}{r}\left( 1- \frac{1}{r(1+r)^T}\right)\right]^{1/T}?\]
%Writing $\frac{P_T}{P_0}= 1+r^p$, where $r^p$ is the compounding rate of growth of housing prices over the period $0-T$,  

\[r^p>\frac{r^T+ \frac{1}{r_i(1+r_i)^T}+\phi^b}{\phi^s}-1\]


\chapter*{Appendix: ODD + D Protocol}

The ODD + D protocol is a protocol for specifying agent based models. It is recomended that all agent based model specifications answer a set of questions discussed bellow CITE.

\section*{Purpose}

%Study Patterns Within Models % From Jangho Proposal 1 - Measuring of Resilience in Agent Based Models of Housing Markets - Kirsten.pdf https://app.gingkowriter.com/1zQLM
%Physics and economics both have a long history of using simple models to find insights into the structure of systems. 
%Method: Small/toy models and analysis of regimes/complex systems patterns
%Purpose: To understand patterns, methodology-- physics like simple models
% The purpose of this model is theoretical exploration.As computational power has increased, so has the complexity and richness of computational models. These models serve an increasingly diverse range of purposes. 
Epstein describes 17 distinct reasons to build computer models [2008]. However, Edmonds et al. suggest that models of social systems can be understood in terms of seven primary purposes, including prediction, explanation, description, theoretical exploration, illustration, analogy, and social interaction. These distinct purposes have to do with the way the model relates to external evidence, vs how it relates to the dynamics within the model itself. 

This model seeks to explore the impact on household wealth of the interaction between individual, and financialized investment in a productive urban centre with agglomeration driving growth. 
The analysis looks at the patterns of resilience given the model dynamics. It thus looks at patterns within models, so the purpose is theoretical exploration. In theoretical exploration, a set of hypotheses are formulated and then systematically tested. The goal is general insight, so the analysis examines the consequences of a set of theoretical assumptions. We do this by formulating and attempting to falsify hypotheses about the model's behaviour given the assumptions. 
%A challenge is that a given model run may not show the patterns of the model system, or even it's representative behaviour. The resilience analysis is used as a method to see the general patterns or behaviour in the model. To rigorously analyze the consequences of assumptions, of the model as a system, we use a resilience analysis to systematically map the patterns of behaviour reversibility, and irreversibility given three nested layers of hysteresis - the urban built form, the firm/agglomeration relationship, and the level of investment/size of the bubble relative to fundamental productive value. 
%We asses hypotheses about the general patterns in the behaviour of a set of assumptions, and evaluate those hypotheses. 
Are the hypotheses refuted with a set of experiments?

The results are useful to the extend that they may be generalized from the model to other circumstances. The relationship between claims on rents, production, and household wealth has general patterns, we aim to describe systematically, the patterns which emerge in this model. 

% We may also modify the mechanism, vary assumptions, or introduce interventions that mimic plausible policy interventions and examine the implications in the model to explore whether outcomes change significantly. 

Edmonds et al. highlight three risks to this approach. First there may be mistakes in the model code leading to misleading output. We mitigate this with testing. % (testing techniques - Galán et al. 2017) and by pair programming, ideally including a minimal reimplementation of the model to confirm two versions produce the same results. 
The code is also made available, and the purpose, design decisions and implementation is described using the ODD + D protocol, as well as a narrative account and explicit statement of the theoretical assumptions %and how they appear in code, as well as stating and illustrating the
and hypotheses. % making clear which hypotheses we're formulated in advance of testing, and what results came as a surprise. 

Second the model results may be brittle and apply only to a narrow set of assumptions, reducing the usefulness of the model. We mitigate this by doing a wide range of testing including with extreme conditions and noise to explore the models sensitivity. Because the analysis centres on a resilience analysis of alternative regimes, much of the analysis centres on exploring the generalizability of results and how they shift under different conditions. This method is particularly well suited to the theoretical exploration of simulation models to test against the brittleness of results. %, as well as mapping how assumptions may vary with future work.
% In addition to illustrating the pattern of behaviour, testing should include systematic efforts to refute hypotheses (Edmonds et al.)

Finally the analysis may over interpret the model, and say something about the external world. The model can imply a hypothesis about the world. The model does not, however, make it test that hypothesis empirically. This model draws on stylized facts and empirical regularities to suggest those hypothesizes may be realistic, but further analysis is required to confirm or refute those hypotheses. 

% Perhaps introduce a figure illustrating the role modelling thus plays in theory formation in the cycle of theory creation and testing in the social sciences. 

The model is designed as a contribution to urban economics.
Policy implications are perhaps relevant for policy makers, urban planners, and citizens.
The approach to resilience modelling and analysis may be of interest to other modellers and those working in systems design. 


\section*{Entities, State Variables, and Scales}

The agents in the model are workers/households who work and participate in the housing market, firms who employ workers, banks, and real estate agents. Space is modelled using a grid. Housing units are located on spatial grid locations in space. TABLE describes the attributes for each agent. Firm location determines the distance workers must travel to work and thus their transportation costs.
% TODO the grid size is - spatial and temporal extent. -parameter value table. Each time step is equvalent to one week (TODO check)

The attributes %(state variables and parameters)
 characterising each agent are:
%Model
%alpha, beta, gamma 
%immigration pressure/rate
%
%Worker - 
%wage, employer, housing unit, savings/assets
%
%Firm 
%location, list of employees, debt/savings, %(computes current wage current wage, production level, etc) %possibly production history
%
%Housing unit - location, condition/amenity value, operating costs, rent/sale price, rent/sale asking price
%
%Real estate agent - list of homes for sale, list of clients to sell homes to.
%
%Bank/Financial Institution -
%expected risk and return profile for alternative investments, 
%base/prime interest rate
%maybe information about the housing market
% lends to firms - they go out of business if needed, lends to individuals, lends to it's own investment arm which manages pensions.

The exogenous drivers of the model are the prices that goods can receive on the market, expected returns on alternative investments, %and realized risk and return profile of the best alternative investments, 
and immigration pressure. % TABLE describes the exogenous drivers of the market. 
% Additional policy drivers include operating expenses, taxes and fees, housing tenure, and tenant protections including minimum contract length.


\section*{Process, Overview, and Scheduling}

What entidy does what in what order? % Fig 3



\section*{Theoretical and Empirical Background}

%TODO - maybe this goes in background. ADD REFERENCES.
This work draws on several theoretical frameworks. 
The productive urban model draws on Jane Jacobs theory of urban productivty and recent complex systems work on scaling of socioeconomic outputs including urban productivity with population, as well as economic appraoches to distribution, production and growth. % In particular two stories of distribution, the classical approach to rents, and the marginalist appraoch to distribution through firm payments to workers.

For modelling we build from analytic neo-classical modelling tradition in economics, as well as more recent work on spatially explicity agent based moddeling, and in particular work on the modeling of housing markets. For the analysis, we draw on the theory of resilience, and methods for analyzing alternative regimes. 
This work feeds into research on the effect of the financialization of housing. 
The descicion making strategy may be described as boundedly rational. % We also explore variations in which agents are risk averse/loss averse, and calculate risk adjusted returns using prospect theory. This would bring in some non financial aspects of the discision making processess, in particular the aversion to losses. It also apears to align with some epirical evidence. 
The primary hypothesis is that there exists a regime in which the housing market acts as a peristaltic pump, pumping wealth out of a community on the up cycle and on the down cycle. 
%The goal is to explore the impact on housholde welath in different regimes. 

% TODO Parameters and data comes from ...- real estate association, CMHC, KWCF's Vital Signs, Financialization Lab data, stylized economic facts. 
% TODO At what level of agregation/disagregation is the data


\section*{Individual Decision Making}
For the decision model, workers maximize their utility given available information, selecting jobs, investments, and homes. Firms produce and hire to maximize their profits given their beliefs about returns. Real estate agents offer workers a selection of homes, given their budget and needs. Banks offer loans to individuals, firms, and pension fund investment, and invest thier own pension fund to maximize returns given the expected risk and return profile of alternative investments. % Real estate agents can also be modelled as profit seeking, recieve fees, and invest time in sales based on expected returns. 
The basic rationality behind the decision making is the agent's individual subjective utility, given available information and their attitute towards risk and loss. The exception is real estate agents and banks, which also also perform some services, like offering property listings and computing available interest rates, without explicity calculating their own return.

The sequence of decisions is:
Firms post jobs, at the marginal wage that would maximize their profit, given their expectations about the market. Firms fire workers if it is not profitable to employ them. If they go insolvent, firms close. 

If there is an opportunity, a new firm can enter. The larger the opportunity, the larger the probability a new firm will enter in any given time step. 

New workers, and laid offer workers apply for jobs if it is worthwhile for them to work given the wage premium and transportation costs. 

Each worker, listing a home, then decides whether to place a home for sale or rent, selects an asking price, and computes the minimum they could accept.

Each worker, looking for a home, asses their available budget by consulting with a financial institution, requests a list of properties to view from a real estate agent, and places offers on homes.

Banks managing pensions also have the chance to place offers on homes %They could advertise directly to individuals and agents, potentially getting some homes first.

Finally, selling/renting agents review offers, negotiate to get the price for sale or rental. % They take the maximum, randomize, or take offers in sequential order.
%At the end of each time step, the model records data.

TODO FIGURE gives the decision tree, tracing the logic agents use to make their decisions.

There are several ways in which agents adapt to changing state variables. 
Firms anticipate their profits myopically and so can get unexpected returns through the agglomeration effect. Under some circumstances this can led to continual growth of the urban center.
Agents adapt their asking price given market conditions, and past changes in property prices affect the expected return of the property market, relative to alternative investments. This expected return, then shapes the choice of agents to list their properties for rent or sale.   %Market conditions depend on growth, production %clarify.
Space shapes the decision process, and thus the rent profile, through transportation costs. People will work in the city if the urban wage premium is greater than the transportation costs. % Many extensions nuance the role of space.
Time plays a role in that past housing prices affect agents' expectations about future prices. % DETAIL 
Workers, firms, and pension funds also accumulate assets and debt over time, affecting their budget and thus their choices. % 3 time steps, weighting more recent time steps higher.
Investment decisions include risk. % and loss aversion in the prospect theory case. With prospect theory, agents weight probabilities of alternative outcomes, and thus include their uncertainty in their decision process.


\section*{Learning}

Agent's adjust their behaviour given the endogenous and exogenous state variables, but don't learn individually or collectively. % clarify definition.
%Learning comes in through firm's continual updating of production given returns, and though agent's need for housing which increases if they continue to look over time. %, %and through their optimism factor in the prospect theory choice formulation.
While there is not collective learning in the model explicitly, we do explore the effect of interventions such as taxes and rules, which might be understood as a collective response to guide the dynamics of the system in a direction that better serves the interests of agents. 
Social values and norms do not explicitly affect decision making. % Note prospect theory weightings could represent a proxy for how influenced agents are to anecdotal information about the market, thus implicitly bringing in social information if not social norms. We could also explicitly incorporate norms, perhaps through how people value amenity and include it in their utility decisions, perhaps as a kind of non-monetary income. 


\section*{Individual Sensing}
%Table of what information is available to what agent. at what time. Is the information precise/accurate, aggregate/disaggregation?
% what state variables of the system, what state variable of other agents? what scale? 

%All agents have access to xyz. Firms use xyz for xyz..
Agents can qualify for mortgages at a given interest rate, based on their income, from the bank. 
They also can get expected risks and returns for local property markets and alternative investments from the bank. % Note real estate agent could give them property market returns, or they cold compute it directly, but it makes little difference. The bank is the financial agent and gives them other financial information, so we have them provide expected returns on property based on market activity as well.
Agents get information about who is selling from real estate agents

There is a cost to looking at homes, so households have a limited capacity to compare homes. They look at a subset of the homes available to them, provided by a real estate agent. %The maximum number of homes to look at is a user set parameter.


\section*{Individual Prediction}
%\section{Interaction}
%\section{Collectives}
%\section{Heterogeneity}
%\section{Stochasticity}
%\section{Observation}
%\section{Implementation Details}
%\section{Initialization}
%\section{Input Data}
%\section{Submodels}




